%The index 0 represents the temporal component.
\section*{Notations and convention}
When talking about manifold with boundary, we consider a $(d+1)$-dimensional manifold $\mathcal{M}$ for the bulk with a $d$-dimensional boundary $\partial \mathcal{M}$ which is supposed to be static and we use the Minkowski space-time with signature $(+, -,\ldots, -)$.
We denote by $M$ an equal-time hypersurface of $\mathcal{M}$.
As we work with spinor fields, $\mathcal{M}$ and $\partial \mathcal{M}$ are required to be spin manifolds. 
A good review on the notions of spin structure can be find in~\cite{Trautman2007}.
\footnote{
Such a structure on a given manifold exists if and only if the second Stiefel-Whitney class of its bundle vanishes (see~\eg Chap. 2 of \cite{Lawson1989} or~\cite{Alagia1985} for extension to pseudo-Riemannian manifolds).}
%
The indice 0 will correspond to the time component.
We denote $\gamma^\mu$ for $\mu = 0, \ldots, d$ for the generators of the Clifford algebra\footnote{
For simplicity, we will sometimes call this generators gamma matrices in this report.
} in the bulk, which should satisfy
\begin{equation*}
\{ \gamma^\mu, \gamma^\nu \} = \eta^{\mu\nu}
\end{equation*} 
where $\eta = \mathrm{diag}(1, -1 ,\ldots, -1)$.
Also, we choose a Hermitian representation for $\gamma^0$, 
\ie
\begin{equation*}
(\gamma^0)^\dagger = \gamma^0
\end{equation*}
Like in many literatures, Greek letters are used for both spatial and temporal components, whereas Latin letters are only used for spatial ones.
With shorthand notations, the generators of the induced Clifford algebra on the boundary $\partial \mathcal{M}$ will also be denoted by $\gamma^\alpha$ for $\alpha = 0 ,\ldots, d-1$. 
However, we specify that the Greek letters $\mu$ and $\nu$ will be used for the indices of bulk terms (taking values in $\llbracket 0, d \rrbracket$) and the Greek letters $\alpha$ and $\beta$ will be used for the indices of boundary terms (taking values in $\llbracket 0, d-1 \rrbracket$).
Analogously, the Latin letters $i $ and $j$ will be used for bulk terms and $a$ and $b$ will be used for boundary terms. \\\\
%
We denote the spinor representation spaces of $\mathcal{M}$ and $\partial \mathcal{M}$ by $E$ and $F$ respectively.
For certain manifolds, we can define the Sobolev spaces on them (\cite{Hebey1996}, \cite{Eichhorn1996}).
We suppose that $\mathcal{M}$ and $\partial\mathcal{M}$ belong to these categories of manifold.\footnote{
We might encounter some difficulties defining the Sobolev spaces for the boundaries. Meanwhile, the boundaries of these two cases are in effect "open manifolds", \ie, without boundary and compact connected component. 
According to~\cite{Eichhorn1996}, it is possible to define Sobolev spaces which are also Banach spaces for them.
Then, when working with $L^2$-norm, we have Hilbert space structures since the inner products will be well-defined.  
}
In the following text, if not specified, the Sobolev space $W^{m,n}(\mathcal{M})$ on manifold $\mathcal{M}$ represents $W^{m,n}(\mathcal{M}, E)$.
On the other hand, $W^{m,n}(\partial \mathcal{M})$ represents $W^{m,n}(\partial \mathcal{M}, F)$.
This notation is also applicable to the $L^2$ spaces which appear in this section. \\\\
Finally, when we talk about Hamiltonian, we refer to the Hamiltonian of a dynamical system.
%
\paragraph{Hadamard parametrix}
We give hereunder the off-diagonal components\footnote{
In effect, since the gamma matrices that we choose here are off-diagonal, only the off-diagonal will be used when we calculate the vacuum current.
}
 of the Hadamard parametrix found in~\cite{Zahn2015} for the following representations of gamma matrices
\begin{equation*}
\gamma^0 = \begin{pmatrix}
0 & 1 \\
1 & 0 \end{pmatrix}  \quad  \gamma^1 = \begin{pmatrix}
0  & 1 \\
-1 & 0
\end{pmatrix}
\end{equation*}
which describes the case of a spin-$\frac 1 2$ massless particle with charge $+e$ in $(1+1)$-dimension in presence of a potential under static gauge $A^\mu(x) = (Ex^1, 0)$, where $E$ is a constant electric field and $x^1$ is the spatial coordinate
\begin{equation}\label{vacuum-hadamardparametrix}
\begin{split}
& H^\pm (x, y)^1_2 = \frac{-i}{2\pi}\frac{1-\frac i 2 e E(x^1 + y^1)(x^0-y^0) 
- \frac 1 8 (eE)^2(x^1 + y^1)^2(x^0 - y^0)^2}{x^0 - y^0 + x^1 - y^1 \mp i \epsilon}  + R^\pm(x,y)^{1}_2\\
& H^\pm (x, y)^2_1 = \frac{-i}{2\pi}\frac{1-\frac i 2 e E(x^1 + y^1)(x^0-y^0) 
- \frac 1 8 (eE)^2(x^1 + y^1)^2(x^0 - y^0)^2}{x^0 - y^0 - x^1 + y^1 \mp i \epsilon} + R^\pm(x,y)^{2}_1
\end{split}
\end{equation}
where $R^\pm(x,y)$ are smooth two-point functions vanishing when $y\rightarrow x$
