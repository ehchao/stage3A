
\begin{titlepage}

\newcommand{\HRule}{\rule{\linewidth}{0.5mm}} % Defines a new command for the horizontal lines, change thickness here

\center % Center everything on the page
 
%----------------------------------------------------------------------------------------
%	HEADING SECTIONS
%----------------------------------------------------------------------------------------

\textsc{\LARGE {\'E}cole Polytechnique  }\\[0.3cm] % Name of your university/college
%\textsc{\LARGE INSTITUTE OF TECHNOLOGY  }\\[0.3cm]
%\textsc{\Large JALANDHAR-144011, PUNJAB(INDIA) }\\[0.3cm]
\textsc{\Large Rapport de stage de recherche}\\[0.5cm] % Major heading such as course name
 % Minor heading such as course title

%----------------------------------------------------------------------------------------
%	TITLE SECTION
%----------------------------------------------------------------------------------------

\HRule \\[0.4cm]
{ \huge \bfseries Boundary and matching conditions \\ for quantized Dirac fields}\\[0.03cm] % Title of your document
\HRule \\[1.5cm]

 
%----------------------------------------------------------------------------------------
%	AUTHOR SECTION
%----------------------------------------------------------------------------------------

\begin{minipage}{0.4\textwidth}
\begin{flushleft} \large
\emph{Author:}\\
En-Hung CHAO (X2014) \\ {\'E}cole Polytechnique (France) \\D{\'e}partement de Physique\\ Option PHY591 % Your name
~
\end{flushleft}
\end{minipage}
\begin{minipage}{0.4\textwidth}
\begin{flushright} \large
\emph{Supervised by:} \\
Dr. Jochen ZHAN\\Institut f{\"u}r Theoretische Physik\\Universit{\"a}t Leipzig (Germany) % Supervisor's Name 
\end{flushright}
\end{minipage}\\[1cm]

% If you don't want a supervisor, uncomment the two lines below and remove the section above
%\Large \emph{Author:}\\
%John \textsc{Smith}\\[3cm] % Your name

%----------------------------------------------------------------------------------------
%	DATE SECTION
%----------------------------------------------------------------------------------------

{\large March 27th-July 14th, 2017 \\Research internship at \\ Institute of Theoretical Physics of University of Leipzig}\\[1cm] % Date, change the \today to a set date if you want to be precise

%----------------------------------------------------------------------------------------
%	LOGO SECTION
%----------------------------------------------------------------------------------------

\begin{minipage}{0.4\textwidth}
\begin{flushleft} \large
\includegraphics[scale=0.5]{logo_x}\\[1cm] % Include a department/university logo - this will require the graphicx package
  
\end{flushleft}
\end{minipage}\\[1cm]
\begin{minipage}{0.4\textwidth}
\begin{flushright} \large
\includegraphics[scale=0.2]{logo_leipzig}\\[1cm] % Include a department/university logo - this will require the graphicx package
  
\end{flushright}
\end{minipage}\\[1cm]
%----------------------------------------------------------------------------------------

\vfill % Fill the rest of the page with whitespace

\end{titlepage}


%--------------------------------------
\section*{Abstract}
This work consists of two parts.
In the first one, we investigate the vacuum polarization in 1+1 dimension in presence of a Kondo-type delta potential.
We apply the renormalization by subtraction of Hadamard parametrix to evaluate vacuum charge density and stress-energy tensor.
In the second part, we generalize the bag boundary condition for massless fermions.
After proving the well-posedness of dynamical problems under this generalized boundary condition, we look for solutions in some simple cases.
At the end, we study the vacuum polarization in using the same renormalization technique as in the first part.

\section*{Résumé}
Ce travail consiste en deux parties. 
Dans la première partie, nous nous intéressons à la polarisation du vide en dimension 1+1 en présence d'un poteniel singulier dit de "type Kondo".
Nous appliquons la renormalisation par soustraction du parametrix d'Hadamard afin d'évaluer la densité de charge du vide et le tensor énergie-impulsion.
Dans la seconde partie, nous généralisons la condition aux limites dite de "sac" pour les fermions de masse nulle.
Après avoir prouvé que les problèmes dynamiques sont bien posés sous cette condition aux limites généralisée, nous cherchons des solutions pour quelques exemples faciles.
Enfin, nous étudions la polarisation du vide en utilisant la même technique de renormalisation introduite dans la première partie.

\newpage
%-----------------------------------------
%acknowledgement
\section*{Acknowledgements}
I am very grateful to my supervisor, Dr. Jochen Zahn, for his constructive and patient mentoring during the entire period of internship, and also for his suggestions and reviews during the writing of this report afterwards.
The warm and friendly host of Elementary Particle Theory Group of the Insitute of Theoretical Physics of University of Leipzig (Germany), directed by Prof. Stefan Hollands, is greatly appreciated.
Besides, I would also like to thank Onirban Islam (University of Leipzig), Mojtaba Taslimitehrani (University of Leipzig) and Alexandre Efremov (École Polytechnique) for useful discussions and recommendations for literatures. 