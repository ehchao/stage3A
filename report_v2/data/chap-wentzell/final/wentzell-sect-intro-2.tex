%%%%%%%%%%%%%%%%%%
In this chapter, we are going to establish the so-called Wentzell boundary condition studied in~\cite{Zahn2016} for massless fermions.
As we will see, this boundary condition is a generalized version of the bag boundary condition~\cite{Chodos1974}.
We start by proving the well-posedness of dynamical problems with this boundary condition. 
Later on, we will prove the causal propagation property on certain types of manifold.
At the end of the chapter, 
we discuss the vacuum polarization and the stress-energy tensor under this boundary condition in the 1+1 dimensional case.
\\\\
We consider a $(d+1)$-dimensional manifold $\mathcal{M} = \mathbb{R}\times M$, where $M$ represents the spatial slices (always in Minkowski space-time), for the bulk with a static $d$-dimensional time-like boundary $\partial \mathcal{M}$.
We assume that we can define Sobolev spaces on $\mathcal{M}$ and $\partial \mathcal{M}$.
%As we work with spinor fields, $\mathcal{M}$ and $\partial \mathcal{M}$ are required to be spin manifolds. 
%A good review on the notions of spin structure can be find in~\cite{Trautman2007}.\footnote{
%Such a structure on a given manifold exists if and only if the second Stiefel-Whitney class of its bundle vanishes (see~\eg Chap. 2 of \cite{Lawson1989} or~\cite{Alagia1985} for extension to pseudo-Riemannian manifolds).}
%
With shorthand notations, the generators of the induced Clifford algebra on the boundary $\partial \mathcal{M}$ will also be denoted by $\gamma^\alpha$ for $\alpha = 0 ,\ldots, d-1$. 
However, we specify that the Greek letters $\mu$ and $\nu$ will be used for the bulk terms (taking values in $\llbracket 0, d \rrbracket$) and the Greek letters $\alpha$ and $\beta$ will be used for the boundary terms (taking values in $\llbracket 0, d-1 \rrbracket$).
Analogously, the Latin letters $i $ and $j$ will be used for the bulk terms and $a$ and $b$ will be used for the boundary terms. 
The spinor representation space of $\mathcal{M}$ will be denoted by $E$.
%We suppose that we can define Sobolev spaces on both $\mathcal{M}$ and $\partial\mathcal{M}$.\footnote{
%We might encounter some difficulties defining the Sobolev spaces for the boundaries. Meanwhile, the boundaries of the two special cases that we are going to study are in effect "open manifolds", \ie, without boundary and compact connected component. 
%According to~\cite{Eichhorn1996}, it is possible to define Sobolev spaces which are also Banach spaces for them.
%Then, when working with $L^2$-norm, we have Hilbert space structures since the inner products will be well-defined.  
%}
%In the following text, if not specified, the Sobolev space $W^{m,n}(\mathcal{M})$ on manifold $\mathcal{M}$ represents $W^{m,n}(\mathcal{M}, E)$.
%On the other hand, $W^{m,n}(\partial \mathcal{M})$ represents $W^{m,n}(\partial \mathcal{M}, F)$.
%This notation is also applicable to the $L^2$ spaces.
%
%%%%%%%%%%%%%%%%
\section{The action of the problem}
Among the literatures on AdS/CFT correspondence for Dirac fields,
\cite{Henningson1998} and~\cite{Contino2005} propose two different boundary actions that one could add to the total action.
It is of interest to study the physics behind the mixte of both boundary actions. 
For this purpose,
we consider the following action
\begin{equation}\label{wen-action}
\mathcal{S} = \frac{1}{2}i\int_{\mathcal{M}} \bar{\psi} \gamma^\mu \partial_\mu \psi - \partial_\mu \bar{\psi} \gamma^\mu \psi 
+ \frac{1}{2}\int_{\partial \mathcal{M}} ic \bar{\psi} \gamma^\alpha \partial_\alpha (1 - i \gamma^\bot) \psi
+ \bar{\psi} \psi
\end{equation}
for a certain constant $c >0$. 
$\gamma^\bot = n_j\gamma^j$ where $n$ represents the unit vector normal to $\partial \mathcal{M}$ in the incoming direction. 
As we will see later, $\frac 1 2 (1-i\gamma^\bot)$ acts as a projector on the boundary.
\\\\
We suppose $\partial(\partial \mathcal{M}) = \emptyset$.
By variational method, the equations of motion in the bulk and on the boundary obtained from this action are
\begin{equation}\label{wen-motion}
\begin{cases}
i \gamma^\mu \partial_\mu \psi = 0  \quad \textrm{in $\mathcal{M}$}\\
i \gamma^\alpha \partial_\alpha (1 - i\gamma^\bot) \psi = - c^{-1}(1 + i\gamma^{\bot}) \psi \quad \textrm{on $\partial \mathcal{M}$}
\end{cases}
\end{equation}
In terms of 
\begin{equation*}
\phi = \gamma^0 \psi
\end{equation*}
\cref{wen-motion} can be written as 
\begin{equation}\label{wen-maineq}
\begin{cases}
i \partial_0 \phi = i \gamma^0 \gamma^j \partial_j \phi   \quad \textrm{in $\mathcal{M}$}\\
i \partial_0(1 + i\gamma^\bot) \phi = i\gamma^0 \gamma^a \partial_a (1+ i\gamma^\bot)\phi - c^{-1} \gamma^0(1 - i \gamma^{\bot})\phi \quad \textrm{on $\partial \mathcal{M}$}
\end{cases}
\end{equation}
One can notice that the boundary condition implies constraints only on certain components of $\phi$. 
For instance, for $\dim \mathcal{M} = 3$, we can construct the following gamma matrices as suggested in~\cite{Polchinski1998}
\begin{equation*}
\gamma^0 = i\begin{pmatrix} 0 & 1 \\ -1 & 0 \end{pmatrix}  \quad
\gamma^1 = i\begin{pmatrix} 0 & 1 \\ 1 & 0 \end{pmatrix}  \quad
\gamma^2 = i\begin{pmatrix} 1 & 0 \\ 0 & -1 \end{pmatrix}  
\end{equation*}
Suppose that the inward normal vector of $\partial \mathcal{M}$ is $e_2$ at all point.
We have
\begin{equation*}
1 - i\gamma^\bot = 
\begin{pmatrix} 2 & 0 \\ 0 & 0\end{pmatrix} = 2 \mathcal{P}
\end{equation*}
where $\mathcal{P}$ is one of the chiral projectors on the boundary. 
More generally, for any space-like unit vector $n_j$,
\begin{equation*}
\mathcal{P}_\pm = \frac{1}{2}(1 \pm i n_j\gamma^j) 
\end{equation*}
are Hermitian projectors, since 
\begin{equation*}
\mathcal{P}_\pm^\dagger = 
\frac{1}{2}(1 \mp i (n_j \gamma^j)^\dagger)=
\frac{1}{2}(1 \pm i n_j \gamma^j)
\end{equation*}
and
\begin{equation*}
(\mathcal{P}_\pm)^{2} = \frac{1}{4}(2\pm 2i \gamma^\bot) = \mathcal{P}_\pm
\end{equation*}
Furthermore, they have the same rank since $\gamma^0$ is of maximal rank and
\begin{equation*}
\gamma^0\mathcal{P}_\pm = \mathcal{P}_\mp\gamma^0
\end{equation*}
In a more general way, 
we can rewrite~\cref{wen-maineq} as a time evolution problem with Cauchy data
$\Phi = \begin{pmatrix}
\phi, \phi_|
\end{pmatrix}$
\begin{equation*}
i\partial_0 \Phi = \Delta \Phi
\end{equation*}
where $\phi$ is the bulk field, $\phi_|$ is the boundary field and the operator $\Delta$ is defined as
\begin{equation}\label{wen-hamiltonian}
\Delta = \begin{pmatrix}
i \gamma^0 \slashed{\partial}  & 0 \\
-c^{-1} \gamma^0 \mathcal{P}_- \cdot \vert_{\partial M}&  i\gamma^0 \slashed{\partial}_|
\end{pmatrix}
\end{equation}
where $\slashed{\partial} = \gamma^j\partial_j$ for
$j \in \llbracket 1 , d \rrbracket$ and $\slashed{\partial}_| = h^{ab} \gamma_{a} \partial_{b}$ with $h$ being the induced metric on the boundary.
For simplicity, we will work on flat boundaries.
We choose a coordinate system such that 
%the boundary corresponds to the set $\{x\in(x^1,\ldots,x^d)\in M \enskip\vert\enskip x^d = 0\}$.
%Under this choice,
$\partial_|$ can be written in a simpler way
\begin{equation*}
\slashed \partial_| = \gamma^a\partial_a \quad\mathrm{for}\quad 
a \in \llbracket 1, \ldots, d-1 \rrbracket
\end{equation*}
We will take the above choice for the rest of the chapter. 
\begin{remark}
From~\cref{wen-motion}, we see that when $c \rightarrow 0$, the boundary condition becomes the bag boundary condition. 
\end{remark}




