\section{Well-posedness of the problem}\label{wen-sect-saw}
\subsection{Self-adjointness $\Delta$}
We are going to use tools of functional analysis to prove that the Cauchy problem is well-posed for certain types of initial data.
We define the Hilbert space 
\begin{equation*}
\mathcal{H} = L^{2}(M,E)\oplus L^{2}(\partial M, \mathcal{P}_+ E)
\end{equation*}
One should note the projection which is taken here on the boundary bundle.
This is used to ensure that we have a Hilbert space for the following computations.
,where $M$ is an equal-time hypersurface of $\mathcal{M}$ and $\partial M = M\cap \partial \mathcal{M}$,
with inner product
\begin{equation}\label{wen-innerpdt}
\langle \cdot, \cdot \rangle _\mathcal{H} = \langle \cdot, \cdot \rangle _{L^2(M)} + c \langle \cdot, \cdot \rangle _{L^2(\partial M)}
\end{equation}
In order to have a self-adjoint operator, we define $\Delta$ on the domain
\begin{equation*}
\dom( \Delta) = \{ \Phi = (\phi, \phi_|) \in W^{1,2}(M)\times W^{1,2}(
\partial M)\enskip | \enskip \mathcal{P}_+\phi \vert_{\partial M} - \phi_| = 0 \}
\end{equation*}
Let us find the domain of $\Delta^*$.  
Let $\Phi = (\phi, \phi_|) \in \dom(\Delta^*)$.
Then, for any $ \Psi = (\psi, \psi_|)\in\dom(\Delta)$
\begin{equation}\label{wentzell-proof-sa}
\begin{split}
\langle \Delta^*\Phi, \Psi \rangle_\mathcal{H} =
\langle \Phi, \Delta \Psi \rangle _\mathcal{H}
 = & \int_M \phi^\dagger i \gamma^0 \gamma^j \partial_j \psi 
+ \int_{\partial M} \phi^\dagger_|(  -\gamma^0\mathcal{P}_- \psi\vert_{\partial M} + ic \gamma^0 \gamma^a \partial_a\psi_|)   \\
 = & - \int_M \partial_j \phi^\dagger i \gamma^0 \gamma^j \psi 
+ \int_{\partial M} \phi^\dagger_|(-\gamma^0 \mathcal{P}_- \psi\vert_{\partial M} + ic \gamma^0 \gamma^a \partial_a  \psi_|) 
- i\phi\vert_{\partial M}^\dagger \gamma^0 \gamma^\bot \psi\vert_{\partial M}   \\
= &
- \int_M \partial_j \phi^\dagger i \gamma^0 \gamma^j \psi 
+ \int_{\partial M} - \phi^\dagger\vert_{\partial M}\mathcal{P}_- \gamma^0 \psi_| + ic \phi^\dagger_|\gamma^0 \gamma^a \partial_a  \psi_|  \\
& - i \phi\vert_{\partial M}^\dagger \gamma^0 \gamma^\bot \psi\vert_{\partial M} 
-\phi_|^\dagger \gamma^0 \mathcal{P}_- \psi\vert_{\partial M} 
+ \phi^\dagger\vert_{\partial M}\mathcal{P}_- \gamma^0 \psi_| \\
%--------------
%\underset{(\mathcal{P}_-)^\dagger = \mathcal{P}_-}{=} 
%& \langle \Delta\Phi, \Psi \rangle_\mathcal{H}
%+\phi^\dagger\vert_{\partial M} \gamma^0 (-i \gamma^\bot %\psi\vert_{\partial M} + \psi_|)
%- \phi_|^\dagger \gamma^0 \mathcal{P}_- \psi\vert_{\partial M} \\
%= & \langle \Delta\Phi, \Psi \rangle_\mathcal{H}
%------------------------
= &
- \int_M \partial_j \phi^\dagger i \gamma^0 \gamma^j \psi 
+ \int_{\partial M} - \phi^\dagger\vert_{\partial M}\mathcal{P}_- \gamma^0 \psi_| + ic \phi^\dagger_|\gamma^0 \gamma^a \partial_a  \psi_| \\
& + (\phi^\dagger\vert_{\partial M} - \phi_|^\dagger)\mathcal{P}_+ \gamma^0 \psi\vert_{\partial M}
\end{split}
\end{equation}
In the above calculation, we have implicitly assumed that $(\phi, \phi_|)\in W^{1,2}(M)\times W^{1,2}(\partial M)$.
$\Delta$ is symmetric because~\cref{wentzell-proof-sa} equals $\langle \Delta\Phi, \Psi\rangle_\mathcal{H}$ if $\Phi$ is furthermore an element in $\dom(\Delta)$.
In particular, the last line of~\cref{wentzell-proof-sa} can be rearranged as $\langle\Omega,\Phi\rangle_\mathcal{H}$ for some $\Omega\in\mathcal{H}$ if and only if the last term of~\cref{wentzell-proof-sa} disappears, \ie
$\mathcal{P}_+\phi\vert_{\partial M} - \phi_| = 0$.
Hence, the domain of $\Delta^*$ is contained in the domain of $\Delta$, which implies the self-adjointness of $\Delta$.\\\\
By the self-adjointness of $\Delta$, it follows that 
\begin{proposition}\label{wen-propwellposedness}
The Cauchy problem~\cref{wen-maineq} with initial data $\Phi_0\in\dom(\Delta)$ has a unique solution $\Phi(t) = e^{-i\Delta t}\Phi_0 $
\end{proposition}
%-----------------
\subsection{Examples}
%%%%%%%%
\subsection{Causal propagation}\label{wen-subsect-causal}
As a complement for~\cref{wen-propwellposedness},
we would like to show that on certain manifolds $\mathcal{M}$, 
the solution of~\cref{wen-maineq} propagates causally, \ie depending smoothly and causally on the initial data.
More explicitly, 
we would like to prove the following proposition.
\begin{proposition}
Let $k \in \mathbb{N}^*$.
Suppose that the initial data $\Phi_0 = (\phi, \phi_|)$ satisfies~\cref{wen-hilbertnorm}.
With the same notations and the same conditions as in \cref{wen-propcau}, 
the smooth solution $\Phi$ depends continuously on the initial data $(\phi, \phi_|)$ in the sense that
\begin{equation*}
\big\| \frac{\partial^k}{\partial t^k} \Phi\big\|_{\mathcal{H}(\mathcal{S}_1)}
\leq
\big\| \Phi\big\|_{\mathcal{H}^{k}(\mathcal{S}_0)}
\end{equation*}
\end{proposition}
Consider an element $\Phi \in \mathcal{K}$.
As the time evolution of the system is given by
\begin{equation*}
i \frac{\partial }{\partial t} \Phi = \Delta \Phi 
\end{equation*}
taking into account the fact that $\Phi\in\dom(\Delta^2)$, one gets
\begin{equation}\label{wen-tobebounded}
\begin{split}
\big\| \frac{\partial }{\partial t} \Phi \big\|^2_\mathcal{H} = \| \Delta \Phi \|^2_{\mathcal{H}}  = &
\langle \Phi, \Delta^2 \Phi \rangle_{\mathcal{H}}   \\ 
\underset{\textrm{Green's formula}}=
& - \int_M \partial^j \phi^\dagger \partial_j \phi 
 -  \int_{\partial M}\phi_|^\dagger \mathcal{P}_+(\partial_\bot \phi)\vert_{\partial M} 
 + \int_{\partial M} (\phi^\dagger \partial_\bot\phi)\vert_{\partial M}
- c\int_{\partial M} \partial^a \phi^\dagger_| \partial_a \phi_| \\
\underset{\Phi \in \dom(\Delta^2)}{=} &
\|\nabla \phi \|^2_{L^2 (M)} + c^{-1} \| \mathcal{P}_- \phi\vert_{\partial M} \|^2_{L^2(\partial M)}
+ c \| \mathcal{P}_+ \nabla \phi\vert_{\partial M} \|^2_{L^2(\partial M)}
\end{split}
\end{equation}
where $\nabla$ is the gradient.
We will prove the following proposition as in~\cite{Zahn2016}
\begin{proposition}\label{wen-propcau}
Let $\Sigma_0$ and $\Sigma_1$ two equal-time surfaces. 
Then, for $\Phi = (\phi, \phi_|) \in \mathcal{K}$ any smooth solution of \cref{wen-maineq} for smooth initial data $(\phi_0, \phi_{|0}) \in C^\infty(M) \times C^\infty(\partial M)$ on a Cauchy surface $\mathcal{S}_0 \subset \Sigma_0$,
the following relation holds
\begin{equation}\label{wen-causal}
\big\| \frac{\partial}{\partial t} \Phi \big\|_{\mathcal{H}(\mathcal{S}_1)}^2
\leq 
\big\| \frac{\partial}{\partial t} \Phi \big\|_{\mathcal{H}(\mathcal{S}_0)}^2
\end{equation}
for any $\mathcal{S}_1 \in D^+(\mathcal{S}_0)\cap\Sigma_1$, where $D^+(\mathcal{S}_0)$ is the future domain of dependence of $\mathcal{S}_0$.
\end{proposition}
\begin{proof}
As in~\cite{Zahn2016} we introduce the following quantities
\footnote{
These quantities are not tensors because the $\phi$s are spinor fields. 
In general, under a change of frame of reference
\begin{equation*}
T_{\mu'\nu'} \neq \Lambda^\mu_{\enskip \mu'}\Lambda^\nu_{\enskip \nu'}T_{\mu \nu} 
\end{equation*}
where $\Lambda$ is the element of the Lorentz group corresponding to the change of frame.
However, 
this will not affect our proof since we are not going to use tensor properties directly in the following.
}
\begin{equation*}
\begin{split}
& T_{\mu\nu} = \partial_{(\mu} \phi^\dagger \partial_{\nu)} \phi - \frac{1}{2}g_{\mu\nu} \partial_\lambda\phi^\dagger\partial^\lambda\phi  \\
& T_{|\alpha\beta} = c\Big( \partial_{(\alpha}\mathcal{P}_+\phi^\dagger_| \partial_{\beta)}\mathcal{P}_+\phi_| - 
\frac{1}{2}h_{\alpha\beta}\big( \partial_\gamma\mathcal{P}_+\phi^\dagger_| \partial^\gamma\mathcal{P}_+\phi_|
 - c^{-2}|\mathcal{P}_- \phi\vert_{\partial M}|^2 \big)\Big) 
\end{split}
\end{equation*}
As $\phi$ is a solution for the first equation of \cref{wen-maineq},
\begin{equation*}
(i\partial_0 + i\gamma^0\gamma^j\partial_j)( i\partial_0 -i\gamma^0\gamma^j\partial_j)\phi  = 
\Box \phi= 0
\end{equation*}
where $\Box = \partial^\mu\partial_\mu$ is the d'Alembertian.
On the other hand, 
\begin{equation*}
\begin{split}
&0 = (i\gamma^\mu\gamma^0\partial_\mu\phi)^\dagger
= - i\partial_j\phi^\dagger\gamma^j\gamma^0 - i\partial_0\phi^\dagger \\
\Rightarrow \quad &
\phi^\dagger(i\overleftarrow{\partial}_j\gamma^j\gamma^0 - i\overleftarrow{\partial}_0)
(i\overleftarrow{\partial}_j\gamma^j\gamma^0 + i\overleftarrow{\partial}_0)
= \phi^\dagger \overleftarrow{\Box} = 0
\end{split}
\end{equation*}
$T_{\mu\nu}$ is thus conserved on-shell. \\\\
From the expression of $\Delta^2$ \cref{wen-delta2}, 
\begin{equation*}
\Box_|\mathcal{P}_+ \phi_| = c^{-1}\mathcal{P}_+\partial_\bot \phi\vert_{\partial M}
\end{equation*}
which leads to
\begin{equation*}
\begin{split}
\partial^\alpha T_{|\alpha \beta} = & 
 \mathcal{P}_+(\partial_{(\bot} \phi^\dagger\vert_{\partial M})\partial_{\beta)}\mathcal{P}_+ \phi_| + 
c^{-1}(\mathcal{P}_-\phi^\dagger\vert_{\partial M}) \partial_\beta\mathcal{P}_- \phi\vert_{\partial M}
+c^{-1}(\mathcal{P}_-\partial_\beta\phi^\dagger\vert_{\partial M}) \mathcal{P}_- \phi\vert_{\partial M} \\ 
\underset{\Phi \in \dom(\Delta^2)}{=} & 
\partial_{(\bot}\phi^\dagger\vert_{\partial M} \partial_{\beta)}\phi\vert_{\partial M} \\
= & T\vert_{\partial M,{\bot\beta}}
\end{split}
\end{equation*}
on-shell. \\\\
One can easily verify that $T_{\mu\nu}$ and $T_{| \alpha\beta}$ satisfy the dominant energy condition, \ie for a time-like vector $\xi$ \footnote{
Even though these are not tensors, their positivitiy could still be ensured by the presence of the products of conjugated quantities.
}
\begin{equation*}
\begin{split}
T_{\mu\nu} \xi^\mu \xi^\nu \geq 0 \quad \mathrm{ and }\quad
T_{\alpha\beta} \xi^\alpha \xi^\beta \geq 0 
\end{split}
\end{equation*}
Let $\xi = e_0$, which is a Killing vector field in the static case.
Then we have $\partial^\mu T_{\mu\nu}\xi^\nu = 0$.
Using Stokes' theorem, the integral of this quantity over the zone contained between $\mathcal{S}_0$ and $\mathcal{S}_1$ which is in the future domain of dependance of $\mathcal{S}_0$ (cf~\cref{integratedzone})
%
\begin{figure}[!h]
  \centering
  %\captionsetup{width=0.8\textwidth}
  \includegraphics[height=0.4\textheight]{causal}
  \caption{Zone to integrate}\label{integratedzone}
\end{figure}
%
gives
\begin{equation*}
\begin{split}
0 \underset{\partial^\bot = -\partial_\bot}{=} &
- \int_{\mathcal{S}_0}T_{00} + \int_{\mathcal{S}_1} T_{00} + \int_{\mathcal{S}_2} n^\mu T_{\mu 0} + \int_{\partial \mathcal{M}} T \vert_{\partial \mathcal{M}, \bot 0}   \\
= & - \int_{\mathcal{S}_0} T_{00} + \int_{\mathcal{S}_1} T_{00} + 
 \int_{\mathcal{S}_2} n^\mu T_{\mu 0} -
  \int_{\partial \mathcal{M} \cap \mathcal{S}_0} T_{|00}  +\int_{\partial \mathcal{M} \cap \mathcal{S}_1} T_{|00}
 + \int_{\partial \mathcal{M} \cap \mathcal{S}_2} s^\alpha T_{\alpha 0}
\end{split}
\end{equation*}
where $n^\mu$ and $s^\alpha$ are future-directed unit vectors normal to $\mathcal{S}_2$ and $\partial M \cap \mathcal{S}_2$ respectively.
By the dominant energy condition, 
$n^\mu T_{\mu 0 }\geq 0$ and $s^\alpha T_{\alpha 0 }\geq 0$. 
We obtain \cref{wen-causal} by replacing $T_{00}$ and $T_{| 00}$ by their expression in terms of $\phi$
\end{proof}
%
The last step of proving causal propagation consists in constructing suitable Hilbert spaces such that we can make sense of "smooth dependence". 
We denote 
\begin{equation*}
\mathcal{H}^k(M) = \cap_{s=1}^{k} \dom(\Delta^s)
\end{equation*}
We note that, by the unitary time evolution, 
\begin{equation*}
\big\|\partial_t^m \Phi(t) \big\|= 
\big\|\Delta^m\Phi(t) \big\| =
\big\|\Delta^m \Phi(0) \big\|
\end{equation*} 
%we have
%\begin{equation*}
% \Big\|\frac{\partial^{m}}{\partial t^{m}} \Phi \Big\|^2 %= 
% \langle \Delta^{m-1}\Phi(0), \Delta^{2}(\Delta^{m-1} \Phi(0))\rangle_{\mathcal{H}(M)} 
%\end{equation*}
%If we can bound~\cref{wen-tobebounded} at instant $t$ by the Cauchy data at instant $0$, 
%the higher time-derivatives of $\Phi$ can also be bounded by the Cauchy data at instant $0$.\\\\
We consider the following map from $\mathcal{H}^k(M)$ to $\mathbb{R}_+$
\begin{equation}\label{wen-hilbertnorm}
\| \cdot \|_{\mathcal{H}^k(M)} = \| \Delta^k \cdot \|_{\mathcal{H}(M)}
\end{equation}
This map does not define a norm on $\mathcal{H}^k(M)$ because of the 0-modes of $\dom(\Delta)$ ($\ker \Delta \neq \{0\}$).
However, 
if 0 is an isolated point of the spectrum of $\Delta$, $\Delta^k \Phi$ does not vanish for all $k\in\mathbb{N}$.
In effect, 
if $\Delta^k \Phi = 0$ for a certain $k>0$,
then $\Delta^{k-1}\Phi\in\ker\Delta$, which is forbidden because $0$ is not in the spectrum of $\Delta$.
%
\begin{remark}
As we will see, $M = \mathbb{R}^{d-1}\times[-L,L]$ is an example where 0 is an isolated point of the sepctrum of $\Delta$. 
The problem has solutions which propagates causally in this case.
\end{remark}
%%%%%%%%%%%%%%%
%%%%%%%%%%%%%%%%%%%%
%\section{Plane wave solutions in two simple cases}
%We will discuss two simple cases, $M = \mathbb{R}^{d-1} \times \mathbb{R}_+$ at first and $M = \mathbb{R}^{d-1} \times [-L, L]$ later in this section. 
%%%%%%%
%\section{$M = \mathbb{R}^{d-1} \times \mathbb{R}_+$}\label{wen-subsect1}
\begin{proposition}
Let $M = \mathbb{R}^{d-1} \times \mathbb{R}_+$. $\Delta$ is a self-adjoint operator with spectrum $\mathbb{R}$
\end{proposition}
\begin{proof}
The self-adjointness being proven, let us find the eigenvalues of $\Delta$. 
Let $k$ be an eigenvalue of $\Delta$ and $\Phi_k = (\phi_k, \phi_{| k})$. Then
\begin{equation}\label{wen-motion2}
\begin{cases}
i \gamma^0 \gamma^j \partial_j \phi_k = k \phi_k \\
-c^{-1} \gamma^0 \mathcal{P}_- \phi\vert_{\partial M} + i \gamma^0 \gamma^a \partial_a \phi_{| k} = k \phi_{| k}
\end{cases}
\end{equation}
Since for all vector $\psi$,
\begin{equation*}
(i\gamma^0 \gamma^j\partial_j - k )(i\gamma^0 \gamma^j\partial_j + k )\psi = 
(- \partial^j\partial_j - k^2) \psi = 0
\end{equation*}
has plane wave solutions, 
all vectors of the form
\begin{equation*}
\begin{split}
\phi_n = & \Big((i\gamma^0\gamma^j\partial_j + k \mathbb{1}) e^{-ip_a x^a }(A\cos p_\bot x^\bot + B \sin p_\bot x^\bot) \psi_n \\
 = &\gamma^0\gamma^a p_a (A \cos p_\bot x^\bot + B \sin p_\bot x^\bot)e^{-ip_a x^a}
+ i\gamma^0\gamma^\bot p_\bot (-A \sin p_\bot x^\bot + B \cos p_\bot x^\bot) e^{-ip_a x^a} \\
& + k \mathbb{1} e^{-ip_a x^a}(A\cos p_\bot x^\bot + B \sin p_\bot x^\bot)\Big)\psi_n  \\
\end{split}
\end{equation*}
with $p_j$ satisfying
\begin{equation*}
- p^j p_j = k^2
\end{equation*}
are solutions of the bulk equation of~\cref{wen-motion}. 
The constants $A$ and $B$ will be determined by the boundary equation of \cref{wen-motion} and the boundary condition. 
The only constraint on $\psi_n$ is that $\psi_n$ should not be in $\ker( \gamma^0 \gamma^j p_j + k \mathbb{1})$. 
\footnote{
As 
$(\gamma^0\gamma^j p_j + k \mathbb{1})
(-\gamma^0\gamma^j p_j + k \mathbb{1})  
= k^2 - (p_j)^2= 0$, 
the kernel of $\gamma^0 \gamma^j p_j + k \mathbb{1}$ is not reduced to $\{ 0 \}$
} 
Since we have projectors $\mathcal{P}_\pm$, 
it is natural to take the basis of the total representation space as the union of an orthonormal basis $\mathfrak{B}_+$ of $\ran(\mathcal{P}_+)$ and an orthonormal basis $\mathfrak{B}_-$ of $\ran(\mathcal{P}_-)$. \\\\
Considering the boundary equation of~\cref{wen-motion} and the condition that an element of $\dom(\Delta)$ should satisfy, one gets
\begin{equation}\label{wen-boundary}
\begin{split}
& \mathcal{P}_+\phi\vert_{\partial M} =  \phi_| \\
& -c^{-1} \gamma^0 \mathcal{P}_- \phi\vert_{\partial M} + i\gamma^0\gamma^a\partial_a \mathcal{P}_+\phi\vert_{\partial M} = k \phi_| 
\end{split}
\end{equation}
which implies
\begin{equation}\label{wen-boundary2}
-c^{-1} \gamma^0 \mathcal{P}_-(\phi\vert_{\partial M}) = 
\mathcal{P}_+(k\phi\vert_{\partial M} - i\gamma^0\gamma^a\partial_a\mathcal{P}_+(\phi\vert_{\partial M})) = 
i\gamma^0\mathcal{P}_-(\gamma^\bot\partial_\bot \phi\vert_{\partial M})
\end{equation}
We compute
\begin{equation*}
\begin{split}
\partial_\bot \phi \vert_{\partial M} = 
\big((\gamma^0\gamma^a p_a + k\mathbb{1})e^{-i_a x^a} p_\bot B - i\gamma^0\gamma^\bot p_\bot^2 A \big) \psi
\end{split}
\end{equation*}
Hence, \cref{wen-boundary} implies
\begin{equation}\label{wen-projected}
\mathcal{P}_- \Big( (\gamma^0 \gamma^a p_a + k\mathbb{1})(c^{-1} A - p_\bot B)
+i \gamma^0 p_\bot(c^{-1} B + p_\bot A) \Big) \psi = 0
\end{equation}
To show that the spectrum of $\Delta$ is $\mathbb{R}$, 
we construct a generalized orthonormal basis for the eigenspace consisting of eigenvectors of eigenvalue $k{\in}\mathbb{R}$.
We observe that, 
for $\psi = e_+ + e_-$ where $e_\pm \in \ran({P}_\pm)$,
\cref{wen-projected} implies
\begin{equation}\label{wen-projected2}
(\gamma^a p_a + k\mathbb{1})(c^{-1}A - p_\bot B) e_- +\gamma^0 p_\bot(c^{-1}B + p_\bot A) e_+ = 0
\end{equation}
In order to construct a generalized orthonormal basis (normalizable to the $\delta$-function),
we can take elements of $\mathfrak{B}_+ \cup \mathfrak{B}_-$ for $\psi$ in~\cref{wen-projected}.
However, as previously mentioned, we should choose $\psi$ such that $\psi\slashed{\in}\ker(\gamma^0\gamma^j p_j + k\mathbb{1})$.
As a result, there are four possible cases to be discussed: 
\paragraph{Case 1 : $\psi \in \mathfrak{B}_-$ and $p_\bot \neq 0$} 
By \cref{wen-projected2}
\begin{equation*}
(\gamma^a p_a + k\gamma^0)(c^{-1}A - p_\bot B) \psi = 0
\end{equation*}
Now, since 
\begin{equation*}
(\gamma^a p_a + k\gamma^0)^2 = ( k^2 - p^a p_a ) \mathbb{1}= p^\bot p_\bot \mathbb{1} \neq 0
\end{equation*}
we must have 
\begin{equation*}
c^{-1} A = p_\bot B
\end{equation*}
\paragraph{Case 2 : $\psi \in \mathfrak{B}_-$ and $p_\bot = 0$}
In order to have non-trivial solution, 
we must have $A \neq 0$. 
In this case, 
\cref{wen-projected2} implies 
\begin{equation*}
(\gamma^a p_a + k \mathbb{1})\psi = 0
\end{equation*}
which is not allowed because $\psi\in\ker(\gamma^0\gamma^jp_j + k\mathbb{1})$ in this case. 
This case should thus be discarded.
\paragraph{Case 3 : $\psi \in \mathfrak{B}_+$ and $p_\bot \neq 0$}
$A$ and $B$ should verify 
\begin{equation*}
c^{-1} B + p_\bot A = 0
\end{equation*}
\paragraph{Case 4 : $\psi \in \mathfrak{B}_+$ and $p_\bot = 0$}
The terms in $B$ in the expression of $\phi$ vanish because  $p_\perp = 0$ and $A$ is determined by the normalization condition.
\\\\
%The above discussion indicates that for any $k \in \mathbb{R}$, 
Therefore, for any $k\in\mathbb{R}$, 
we can find eigenvectors of $\Delta$ with eigenvalue $k$ and a generalized orthonormal basis for the eigenspace.
\end{proof}
%\subsection{${M} = \mathbb{R}^{d-1} \times [-L, L]$ }\label{wen-subsect-saw2}
The subtlety that one should beware of in this case is the projectors $\mathcal{P}_+$.
Since $\mathcal{P}_+$ is defined by considering a set of generators of Clifford Algebra on the tangent bundle $T_{ \partial \mathcal{M}}$, 
if we would like to fixe the system of coordinates once for all,
we should also take into account the transformation of these generators. \\\\
Let $(x^0, \ldots, x^d)$ a system of coordinate of the bulk $\mathcal{M} =\mathbb{R}\times M $. 
Because of the geometric configuration of the bulk, 
this system of coordinate is global. 
The boundary is therefore $\partial \mathcal{M} = \{x^d = -L \} \cup \{ x^d = L \}$.
Furthermore, we  \\\\
We denote $(\gamma^\mu)_\mu$ the set of gamma matrices chosen on $\{x^d  = - L \}$. 
We will have then $\mathcal{P}_+\vert_{\{x^d = L\}} = \mathcal{P}_-\vert_{\{x^d = -L\}}$ and $\gamma^\mu\vert_{\{x^d = L\}}=\gamma^\mu\vert_{\{x^d = -L\}}$ for $\mu\in\llbracket 0, d-1 \rrbracket$.
When it is not specified, we refer to the gamma matrices and the projectors on $\{x^d = -L\}$. \\\\
%
As for the boundary component $\phi_|$ of a solution $\Phi= (\phi, \phi_|)$,
we can rewrite it as
\begin{equation*}
\phi_| = \mathbf{1}_{\{x^d = L \}}\phi_+ + \mathbf{1}_{\{x^d = - L \}}\phi_-
\end{equation*}
where $\phi_\pm$ represent the restriction of $\phi_|$ on $\{x^d = \pm L \}$.
Now, the boundary condition becomes
\begin{equation}\label{wen-saw2bound}
\begin{split}
\begin{cases}
\mathcal{P}_+ \phi\vert_{\{x^d = -L\}} = \phi_- \\
c^{-1}\mathcal{P}_-\phi\vert_{\{x^d = -L\}} = \mathcal{P}_-(\partial_d \phi\vert_{x^d = -L}) \\
\mathcal{P}_- \phi\vert_{\{x^d = L\}} =\phi_+ \\
-c^{-1}\mathcal{P}_+\phi\vert_{\{x^d = L\}} = \mathcal{P}_+(\partial_d \phi\vert_{x^d = L}) \\
\end{cases}
\end{split}
\end{equation}
As \cref{wen-subsect1}, the general solution $\phi$ in the bulk can be written as
\begin{equation*}
\phi 
 =\Big( (k \mathbb{1}+ \gamma^0\gamma^a p_a )(A \cos p_d x^d + B \sin p_d x^d)e^{-ip_a x^a}
+ i\gamma^0\gamma^d p_d (-A \sin p_d x^d + B \cos p_d x^d) e^{-ip_a x^a} \Big) \psi
 \end{equation*}
 where $\psi $ is a spinor and the Einstein summation over $a = 1, \ldots, d-1$ is applied. \\
 We compute
\begin{equation*}
\begin{split}
\partial_d \phi\vert_{x^d = \pm L }
= (\gamma^0\gamma^a p_a + k\mathbb{1})(\mp p_d A \sin p_d L + p_d B \cos p_d L)
+ i\gamma^0\gamma^d p_d(-p_d A \cos p_d L \mp p_d B \sin p_d L)\psi e^{-ip_a x^a}
\end{split}
\end{equation*}
Hence, the boundary condition leads to
\begin{equation*}
\begin{split}
0 = &\mathcal{P}_-\Bigg(
(\gamma^0\gamma^a p_a + k\mathbb{1})\Big((p_d A + c^{-1}B) \sin p_d L + (p_d B - c^{-1}A) \cos p_d L\Big) \\
& + i\gamma^0\gamma^d p_d \Big((-p_d A -c^{-1} B) \cos p_d L + (p_d B - c^{-1} A )\sin p_d L\Big) 
\Bigg)\psi
\end{split}
\end{equation*}
\begin{equation*}
\begin{split}
0 = &\mathcal{P}_+\Bigg(
(\gamma^0\gamma^a p_a + k\mathbb{1})\Big((- p_d A + c^{-1}B) \sin p_d L + (p_d B + c^{-1}A) \cos p_d L\Big) \\
& + i\gamma^0\gamma^d p_d \Big((-p_d A + c^{-1} B )\cos p_d L - (p_d B + c^{-1} A )\sin p_d L\Big) 
\Bigg)\psi
\end{split}
\end{equation*}
Once again, there are 4 possible cases to be discussed as in \cref{wen-subsect1}
%
\paragraph{Case 1 : $\psi \in \mathfrak{B}_-$ and $p_d \neq 0$}
The same argument as in~\cref{wen-subsect1} requires
\begin{equation}\label{wen-plates1}
\begin{cases}
(p_d A + c^{-1} B) \cos p_d L + (p_d B - c^{-1} A)\sin p_d L = 0 \\
(- p_d A + c^{-1}B)\sin p_d L - (p_d B + c^{-1} A)\cos p_d L = 0
\end{cases}
\end{equation}
In order to study~\cref{wen-plates1}, we discuss the following four sub-cases according to the coefficients of $\sin$ and $\cos$.
\subparagraph{Sub-case 1: $ p_d B - c^{-1}A = 0$}
\cref{wen-plates1} becomes
\begin{equation}\label{wen-plates1-1}
\begin{cases}
(c^{-2} + p_d ^2)\cos p_d L = 0 \\
(c^{-2} - p_d^2)\sin p_d L = 2 c^{-1}p_d \cos p_d L
\end{cases}
\end{equation}
We can verify easily that supposing $ c^{-2} + p_d^2 = 0 $ will lead to paradoxal situation. 
In effect, under this assumption, the second equation of~\cref{wen-plates1-1} will lead to $e^{-\pm c^{-1}L} = 0$. \\
Therefore, the only solutions for this sub-case are $p_d = \pm c^{-1}$ for $L$ satisfying $\cos c^{-1} L = 0$
%
\subparagraph{Sub-case 2: $ -p_d A + c^{-1}B = 0$}
By an easy calculation, we obtain the same result as in sub-case 1
%
\subparagraph{Sub-case 3: $\cos p_d L = 0 $}
Idem as in sub-case 1.
%
\subparagraph{Sub-case 4: none of the above statements holds}
\cref{wen-plates1} leads to
\begin{equation*}
\tan p_d L = \frac{p_d A + c^{-1} B }{p_d B - c^{-1} A} = 
\frac{p_d B + c^{-1}A}{p_d A- c^{-1} B}
\quad\Rightarrow\quad
(p_d ^2 + c^{-2})(A^2 - B^2 )= 0
\end{equation*}
One can easily verify that $p_d = \pm c^{-1}$ can not be solution of the equation. 
Hence, we have $A = \pm B$ and
\begin{equation}\label{wen-tan}
\tan p_d L = \frac{p_d \pm c^{-1}}{p_d \mp c^{-1}}
\end{equation}
We show that
\begin{equation*}
\tan \theta = \frac{\theta + a}{\theta - a} \quad a\in \mathbb{R} 
\end{equation*}
does not have solution $\theta \in \mathbb{C} - \mathbb{R}$.
Obviously, this equation can not have pure imaginary solutions.
If such solution $\theta = x + i y$ exists for $(x, y) \in\mathbb{R}^*\times\mathbb{R}^*$,
\begin{equation*}
\begin{split}
\tan \theta = & \frac{e^{i(x+iy)} - e^{-i(x+iy)}}{i(e^{i(x+iy)}+ e^{-i(x-iy)})} \\
= & \frac{e^{-y}\sin x + \sinh y (-i\cos x - \sin x)}{e^{-y}\sin x + \cosh y (\cos x - i\sin x)} \\
= & \frac{(e^{-y}\sin x - \sinh y \sin x )- i\cos x \sinh y}{(e^{-y} \sin x +\cosh y \cos x) - i \sin x \cosh y} \\
=& \frac{x+a+ iy}{x -a +iy}
\end{split}
\end{equation*}
which implies
\begin{equation*}
\cos x \sinh y = \sin x \cosh y
\end{equation*}
This relation shows that $\sin x \neq 0$.
We thus have
\begin{equation*}\begin{split}
\frac{e^{-y}\sin x - \sinh y \sin x}{e^{-y}\sin x + \cosh y \cos x} = &
\frac{1 - e^y \sinh y}{1+ e^y \cosh y \cot x} \\
= &\frac{1 - e^y \sinh y }{1+ e^y \frac{\cosh^2 y }{\sinh y}} \\
=& \frac{1+\frac{a}{x}}{1 - \frac{a}{x}}
\end{split}
\end{equation*}
which can not be satified because $\cosh^2 y - \sinh^2 y =1$
Thus,~\cref{wen-tan} has solutions $p_d \in \mathbb{R}$. The set of solutions of~\cref{wen-tan} is discrete.
%%
\paragraph{Case 2 : $\psi \in \mathfrak{B}_-$ and $p_d = 0$}
This case implies $A = 0$, which is forbidden because it leads to the trivial solution.
%
\paragraph{Case 3 : $\psi \in \mathfrak{B}_+$ and $p_d \neq 0$}
\cref{wen-tan} becomes
\begin{equation*}
\begin{cases}
(-p_d A - c^{-1} B)\cos p_d L + (p_d B - c^{-1}A)\sin p_d L = 0  \\
(-p_d A + c^{-1}B)\sin p_d L + (p_d B + c^{-1} A)\cos p_d L = 0 
\end{cases}
\end{equation*}
This is strictly equivalent to the case 1 of this discussion.
%
\paragraph{Case 4 : $\psi \in \mathfrak{B}_+$ and $p_d = 0$}
Leads to the trivial solution. \\\\
To conclude, we give the following proposition
\begin{proposition}
In the case where $\cos c^{-1}L \neq 0$,
the spectrum of~\cref{wen-maineq} is discrete.
For any eigenvalue $k$, there exists $p_d\in \mathbb{R}$ satisfying~\cref{wen-tan} and $-p^jp_j = k^2$.
The eigenvectors for eigenvalue $k$ can be written as 
\begin{equation*}
\phi_k = \Big(\big(i\gamma^0\gamma^a p_a+k\mathbb{1}\big)(\cos p_d x^d + \sin p_d x^d) +
i\gamma^0\gamma^d p_d(-\sin p_d x^d + \cos p_d x^d)\Big) \psi e^{-ip_a x^a}
\end{equation*}
where $\psi$ is a spinor in $\mathfrak{B}_+$ or $\mathfrak{B}_-$.
\end{proposition}









