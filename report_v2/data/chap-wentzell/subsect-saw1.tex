\subsection{Bulk $M = \mathbb{R}^{d-1} \times \mathbb{R}_+$}\label{wen-subsect1}
\begin{proposition}
Let $M = \mathbb{R}^{d-1} \times \mathbb{R}_+$. $\Delta$ is a self-adjoint operator with spectrum $\mathbb{R}$
\end{proposition}
\begin{proof}
The self-adjointness being proven, let us find the eigenvectors and eigenvalues of $\Delta$. Let $k$ an eigenvalue of $\Delta$ and $\Phi_k = (\phi_k, \phi_{| k})$. Then
\begin{equation}\label{wen-motion2}
\begin{cases}
i \gamma^0 \gamma^j \partial_j \phi_k = k \phi_k \\
-c^{-1} \gamma^0 \mathcal{P}_- \phi\vert_{\partial M} + i \gamma^0 \gamma^a \partial_a \mathcal{P}_+ \phi_{| k} = k \phi_{| k}
\end{cases}
\end{equation}
Since for all vector $\psi$,
\begin{equation*}
(i\gamma^0 \gamma^j\partial_j - k )(i\gamma^0 \gamma^j\partial_j + k )\psi = 
(- \partial^j\partial_j - k^2) \psi = 0
\end{equation*}
has plane wave solutions, 
the family
\begin{equation*}
\begin{split}
\phi_n = & (i\gamma^0\gamma^j\partial_j + k \mathbb{1}) e^{-ip_a x^a }(A\cos p_\bot x^\bot + B \sin p_\bot x^\bot) \psi_n \\
 = &\gamma^0\gamma^a p_a (A \cos p_\bot x^\bot + B \sin p_\bot x^\bot)e^{-ip_a x^a}
+ i\gamma^0\gamma^\bot p_\bot (-A \sin p_\bot x^\bot + B \cos p_\bot x^\bot) e^{-ip_a x^a} \\
& + k \mathbb{1} e^{-ip_a x^a}(A\cos p_\bot x^\bot + B \sin p_\bot x^\bot)  \\
\end{split}
\end{equation*}
with $p_j$ satisfying
\begin{equation*}
- p^j p_j = k^2
\end{equation*}
forms a basis for the solution of the first equation of \cref{wen-motion}. 
The constants $A$ and $B$ will be determined by the second equation of \cref{wen-motion} and the boundary condition. 
The only constraint on $\psi_n$ is that $\psi_n$ should not be in $\ker( \gamma^0 \gamma^j p_j + k \mathbb{1})$. 
\footnote{
As 
$(\gamma^0\gamma^j p_j + k \mathbb{1})^2 = (\gamma^j p_j + k\mathbb{1})^2 
= k^2 - (p_j)^2= 0$, 
$\gamma^0 \gamma^j p_j + k \mathbb{1}$
is nilpotent and its kernel is not reduced to $\{ 0 \}$
} 
Since we have projectors $\mathcal{P}_\pm$, 
it is natural to take the basis of the representation space as the union of an orthonormal basis $\mathfrak{B}_+$ of $\ran(\mathcal{P}_+)$ and an orthonormal basis $\mathfrak{B}_-$ of $\ran(\mathcal{P}_-)$. \\\\
Considering the second equation and the condition that an element of $\dom(\Delta)$ should satisfy, one gets
\begin{equation}\label{wen-boundary}
\begin{split}
& \mathcal{P}_+\phi\vert_{\partial M} = \mathcal{P}_+ \phi_| \\
& -c^{-1} \gamma^0 \mathcal{P}_- \phi\vert_{\partial M} + i\gamma^0\gamma^a\partial_a \mathcal{P}_+\phi\vert_{\partial M} = k \mathcal{P}_+ \phi_| 
\end{split}
\end{equation}
which implies
\begin{equation}\label{wen-boundary2}
-c^{-1} \gamma^0 \mathcal{P}_-(\phi\vert_{\partial M}) = 
\mathcal{P}_+(k\phi\vert_{\partial M} - i\gamma^0\gamma^a\partial_a\mathcal{P}_+(\phi\vert_{\partial M})) = 
i\gamma^0\mathcal{P}_-(\gamma^\bot\partial_\bot \phi\vert_{\partial M})
\end{equation}
We compute
\begin{equation*}
\begin{split}
\partial_\bot \phi \vert_{\partial M} = 
\big(\gamma^0\gamma^a p_a e^{-i_a x^a} p_\bot B - i\gamma^0\gamma^\bot p_\bot^2 A + k\mathbb{1}e^{-ip_a x^a}\big) \psi
\end{split}
\end{equation*}
Hence, \cref{wen-boundary} implies
\begin{equation}\label{wen-projected}
\mathcal{P}_- \Big( (\gamma^0 \gamma^a p_a + k\mathbb{1})(c^{-1} A - p_\bot B)
-\gamma^0 p_\bot(c^{-1} B + p_\bot A) \Big) \psi = 0
\end{equation}
Let $\psi = e_+ + e_-$ where $e_\pm \in \ran({P}_\pm)$.
\cref{wen-projected} implies
\begin{equation}\label{wen-projected2}
(\gamma^a p_a + k\mathbb{1})(c^{-1}A - p_\bot B) e_- - \gamma^0 p_\bot(c^{-1}B + p_\bot A) e_+ = 0
\end{equation}
As previously mentioned,
one can naturally consider the elements of the basis $\mathfrak{B}_+ \cup \mathfrak{B}_-$.
The four possible cases are discussed
\paragraph{Case 1 : $\psi \in \mathfrak{B}_-$ and $p_\bot \neq 0$} 
By \cref{wen-projected2}
\begin{equation*}
(\gamma^a p_a + k\gamma^0)(c^{-1}A - p_\bot B) \psi = 0
\end{equation*}
Now, since 
\begin{equation*}
(\gamma^a p_a + k\gamma^0)^2 = ( k^2 - p^a p_a ) \mathbb{1}= p^\bot p_\bot \mathbb{1} \neq 0
\end{equation*}
we must have 
\begin{equation*}
c^{-1} A = p_\bot B
\end{equation*}
\paragraph{Case 2 : $\psi \in \mathfrak{B}_-$ and $p_\bot = 0$}
In order to have non-trivial solution, 
we must have $A \neq 0$. 
In this case, 
\cref{wen-projected2} implies 
\begin{equation*}
(\gamma^a p_a + k \mathbb{1})\psi = 0
\end{equation*}
which again leads to trivial solution. 
This case should thus be discarded.
\paragraph{Case 3 : $\psi \in \mathfrak{B}_+$ and $p_\bot \neq 0$}
We get 
\begin{equation*}
c^{-1} B + p_\bot A = 0
\end{equation*}
\paragraph{Case 4 : $\psi \in \mathfrak{B}_+$ and $p_\bot = 0$}
The constants $A$ and $B$ can be chosen arbitarily in this case. \\\\
The above discussion indicates that for any $k \in \mathbb{R}$, 
we can find normalisable eigenvectors of $\Delta$ with eigenvalue $k$.
\end{proof}