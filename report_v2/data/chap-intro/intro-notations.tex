\section*{General notations and convention}
In all the discussions of this report, we work on Minkowski space-times with signature $(+, -,\ldots, -)$.
We denote $\delta^A_B$ without argument for the Kronecker delta and $\delta(x)$ with argument for the Dirac delta distribution.
$\Theta(x)$ will be use for the Heaviside step function.
$c^*$ represents the complex conjugate of $c\in\mathbb{C}$. 
Meanwhile, $T^*$ represents the adjoint of $T$ if $T$ is an operator.
As usual, $\bar{\psi}$ is the Dirac adjoint of $\psi$.
By Hamiltonian, we refer to the Hamiltonian of a dynamical system.
For a manifold $M$ on which we can define Sobolev spaces (cf \eg\cite{Hebey1996},~\cite{Eichhorn1996}), 
we denote by $W^{n,m}(M)$ the Sobolev space composed of $n$-times differentiable functions $u\in L^m(M)$ whose derivatives of order lower than $n$ are also in $L^m(M)$.
\\\\
We denote $\gamma^\mu$ for $\mu = 0, \ldots, d$ for the generators of the Clifford algebra\footnote{
For simplicity, we will sometimes call these generators gamma matrices in this report.
}, which should satisfy
\begin{equation*}
\{ \gamma^\mu, \gamma^\nu \} = \eta^{\mu\nu}
\end{equation*} 
where $\eta = \mathrm{diag}(1, -1 ,\ldots, -1)$.
Also, we choose a Hermitian representation for $\gamma^0$ and anti-Hermitian representations for the other $\gamma^i$, 
\ie
\begin{equation*}
(\gamma^0)^\dagger = \gamma^0 \quad
(\gamma^i)^\dagger = -\gamma^i
\end{equation*}
Like in many literatures, Greek letters are used for both spatial and temporal components, whereas Latin letters are only used for spatial ones. \\