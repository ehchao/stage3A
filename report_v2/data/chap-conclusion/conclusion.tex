\chapter{Conclusion and perspective}
In this report, we have shown how matching and boundary conditions can influence the vacuum polarization by effectuating computations in 1+1 dimensional case. 
The discussion in~\cref{chap-vacuum} shows how a localized  potential can have impact on charge and energy densities.
The singularity of the potential results in the discontinuity of the charge in the regions seperated by the singular point of the potential and shifts the vacuum energy density by a constant.
However, compared to the result without Kondo potential obtain in~\cite{Zahn2015}, the presence of the Kondo potential adds only constant background vacuum charge and energy density to the system. 
In~\cref{chap-wentzell}, a more generalized version of the bag boundary condition for massless fermions is proposed and we have also proven that time evolution problems of free massless fermions are well posed under this boundary condition.
Furthermore, on some manifolds, solutions of time evolution problems depend smoothly on their initial data.
The disussion in~\cref{wen-sect-ex1d} gives an idea about how vacuum energy density depends on the bulk-boundary coupling.
\\\\
For further perspectives, 
one might generalize the same type of studies in~\cref{chap-wentzell} to vector fields and gauge fields.
In both cases, it should be noted that, because of the difference between the bulk and boundary dimensions, 
there should be less components for the boundary field than for the bulk field.
The generalization of the bag boundary condition to vector fields is the extention of the perfect conductor boundary condition in electromagnetism~\cite{Stokes2015}, 
which is interpreted by vanishing Poynting vectors on the boundary.
One could also generalize the same boundary condition to massive fermions. 
As a remark, 
we should no longer use the projection spaces introduced in~\cref{chap-wentzell} when studying the well-posedness because the mass term will mix the two projection spaces.
In the massive case, one might expect that the causal propagation property studied in~\cref{wen-subsect-causal} would be a more general result because the 0-modes are not in the frequency spectrum of a free massive fermion. 




