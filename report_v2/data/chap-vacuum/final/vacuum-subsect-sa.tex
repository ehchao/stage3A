\subsection{Well-posedness of the problem~\cref{nef-Dirac}}\label{vacuum-subsect-sa}
%We would like to prove that~\cref{nef-Dirac} possesses indeed unitary time evolution solutions by functional analysis for unbounded operators~\cite{Reed1975}.
The aim of this subsection is to porve that the Hamiltonian $H$ of~\cref{nef-Dirac} has indeed a unique self-adjoint extension.
For simplicity, 
we suppose that $E = 0$.
The discussion of this subsection can be easily extended to the case of non-vanishing external electric field.\\\\
First of all, we should decide on which Hilbert space the Hamiltonian $H$ acts.
If we can show that $H$ has a self-adjoint extension for this Hilbert space, 
we know that the problem has unitary time evolution solutions which can be expressed in terms of the self-adjoint extension of $H$.
In that case, we are able to write the solution as $\phi =  e^{-iHt}\phi_0$ with $\phi_0$ being the initial data.
\\\\
%
The Hilbert space on which we are going to work is $\mathcal{H} = L^{2}(I_-, \mathbb{C}^2) \oplus L^{2}(I_+, \mathbb{C}^2)$,  where $I_- = [-\frac{L}{2}, 0)$ and $I_+ = (0, \frac{L}{2}]$.
The inner product of this Hilbert space is the sum of the usual $L^2$ inner products on $I_+$ and $I_-$, namely
\begin{equation*}
\langle \cdot, \cdot\rangle_{\mathcal{H} } = \langle \cdot, \cdot\rangle_{L^{2}(I_-, \mathbb{C}^2)} +\langle \cdot, \cdot\rangle_{L^{2}(I_+, \mathbb{C}^2)}
\end{equation*}
% with $I = [-\frac{L}{2}, \frac{L}{2}]$. 
To start, we encode the boundary condition~\cref{vacuum-bagboundcond} and the matching condition at $x^1 = 0$~\cref{nef-matching2} in the domaine of $H$ 
\begin{equation*}
\mathrm{Dom}(H) = \Big \{\phi \enskip \big\vert \enskip \phi \in W^{1,2}(I_-, \mathbb{C}^2) \oplus W^{1,2}(I_+, \mathbb{C}^2), \enskip \phi \textrm{ verifies~\cref{vacuum-bagboundcond} and~\cref{nef-matching2}} \Big \}
\end{equation*} 
Let $ \phi = \begin{pmatrix} \phi_L \\ \phi_R \end{pmatrix}, \psi = \begin{pmatrix} \psi_L \\ \psi_R \end{pmatrix} \in \mathrm{Dom}(H)$
\begin{equation}\label{sa-hamiltonian}
\begin{split}
\langle \psi, H \phi \rangle = & i \int_{I_-} ( - \psi_L^\dagger \partial \phi_L + \psi_R^\dagger \partial \phi_R )
+ i \int_{I_+} ( - \psi_L^\dagger \partial \phi_L + \psi_R^\dagger \partial \phi_R ) \\
= & i \big[-\psi_L^\dagger \phi_L + \psi_R^\dagger \phi_R \big]^{0^-}_{-\frac{L}{2}} + i \big[-\psi_L^\dagger \phi_L + \psi_R^\dagger \phi_R \big]_{0^+}^{\frac{L}{2}} \\
& - i \int_{I_-} ( - \partial \psi_L^\dagger \phi_L + \partial \psi_R^\dagger  \phi_R ) - i \int_{I_+} ( - \partial \psi_L^\dagger \phi_L + \partial \psi_R^\dagger  \phi_R ) 
\end{split}
\end{equation}
The boundary condition~\cref{vacuum-bagboundcond} implies
\begin{equation*}
i\phi_R \big(\pm \frac L 2) = \mp\phi_L\big(\pm\frac L 2\big)
\end{equation*}
Therefore
\begin{equation*}
\begin{split}
- \psi_L^\dagger(\pm \frac{L}{2}) \phi_L(\pm \frac{L}{2}) + \psi_R^\dagger(\pm \frac{L}{2}) \phi_R(\pm \frac{L}{2}) = 
0
\end{split}
\end{equation*}
The matching condition at $x^1 = 0$ gives
\begin{equation*}
\begin{split}
\big[ \psi^\dagger_L\phi_L] ^{0^+}_{0^-} & = \frac{1}{|D|^2}(A \psi_L^\dagger(0^-) + C^\dagger\psi^\dagger_R(0^-))(A \phi_L(0^-) + C\phi_R(0^-)) - \psi^\dagger_L(0^-)\phi^\dagger_L(0^-) \\
& = \frac{|C|^2}{|D|^2}\big(\psi_L(0^-)^\dagger\phi_L(0^-) + \psi_R^\dagger(0^-) \phi_R(0^-)\big) +
\frac{2A}{|D|^2}\Re{C\psi_L^\dagger \phi_R} \\
& = \big[ \psi^\dagger_R\phi_R] ^{0^+}_{0^-}
\end{split}
\end{equation*}
Hence, 
\begin{equation*}
\langle \psi, H \phi \rangle = \langle H \psi , \phi \rangle
\end{equation*}
$H$ is symmetric. 
\\\\
It would be more complicated to find the right domain of self-adjointness of $H$.
However, the exact domain of self-adjointness is not relevant for our discussion.
To prove the well-posedness of the problem, it suffices to show that $H$ is \textit{essentially self-adjoint}, \ie possesses a self-adjoint extension.
Even better, an essentially self-adjoint operator has always a unique self-adjoint extension.
A common way to prove the essential self-adjointness of a symmetric operator is by the argument of deficiency coefficients.
Let us explain the idea.
All symmetric operators are closable~\cite{Reed1981}, 
\ie
have closed extensions.
Instead of considering the operator $H$ above-defined, 
we consider its closed extension.
We denote  $\mathcal{K}_{\pm} = \ker (i \mp H^*)$ for the deficiency subspaces of $H$. 
The corollary of the Theorem X.2 of~\cite{Reed1975} states that $\dim \mathcal{K}_+ = \dim \mathcal{K}_-$ is a necessary and sufficient condition such that $H$ is essentially self-adjoint. \\\\
We start by calculating $\mathcal{K}_-$. Let $\phi \in \mathcal{K}_-$. Then $- i \phi = H^* \phi$. As $H$ is symmetric, this implies, for $\phi = \begin{pmatrix} \phi_L \\  \phi_R \end{pmatrix}$, 
\begin{equation}
i \begin{pmatrix} \phi_L \\ \phi_R \end{pmatrix} = 
i \begin{pmatrix} 1 & 0  \\ 0  &  -1 \end{pmatrix} 
\begin{pmatrix} \partial_1 \phi_L  \\  \partial_1\phi_R \end{pmatrix}
\end{equation} 
Thus, $\phi$ could be written as
\begin{equation}
\begin{pmatrix} \phi_L \\ \phi_R \end{pmatrix} = 
\Theta(-x) \begin{pmatrix} f_- e^x  \\ g_-  e^{-x} \end{pmatrix} + 
\Theta(x) \begin{pmatrix} f_+ e^x  \\ g_+  e^{-x} \end{pmatrix}
\end{equation}
%Therefore, $\dim\mathcal{K}_- = 4$.
%By the same calculation, $\dim\mathcal{K}_+ = 4$, which implies the existence of the self-adjoint extension of $H$.
%The boundary condition gives
%\begin{equation}
%\begin{cases}
%-i g_- e^{\frac{L}{2}} = - f_- e^{-\frac{L}{2}} \\
%-ig_+e^{-\frac{L}{2}} = f_+ e^{\frac{L}{2}}
%\end{cases} \quad \Leftrightarrow
%\begin{cases}
%g_- = -i f_- e^{-L} \\
%g_+ = i f_+ e^L
%\end{cases}
%\end{equation}
The matching condition~\cref{nef-matching} and the boundary condition~\cref{vacuum-bagboundcond} imply\footnote{
In effect, by integration by parts, we can show that $\psi\in\dom(H^*)$ iff 
\begin{equation*}\begin{split}
&\Big(-\psi^\dagger_L\big(\frac L 2 \big) + i\psi_R^\dagger\big(\frac L 2\big) \Big)\phi_L\big(\frac L 2\big)
-\Big(-\psi_L\big(\frac{-L}{ 2} \big) + i\psi_R^\dagger\big(\frac{ -L}{ 2}\big) \Big)\phi_L\big(\frac{ -L}{ 2}\big) \\
&+\Big(\frac{A}{|D|} \psi_L^\dagger(0^+) - \psi_L^\dagger(0^-) + \frac{C^*}{|D|}\psi_R^\dagger(0^+)\Big)\phi_L(0^-) 
+ \Big(\frac{A}{|D|} \psi_R^\dagger(0^+) - \psi_R^\dagger(0^-) + \frac{C}{|D|}\psi_L^\dagger(0^+)\Big)\phi_R(0^-) =0
\end{split}
\end{equation*}
for any $\phi\in\dom(H)$.
Therefore, an element of $\dom(H^*)$ also satisfy~\cref{nef-matching} and~\cref{vacuum-bagboundcond}.
}
\begin{equation}
f_+ \begin{pmatrix} 1 \\ ie^L \end{pmatrix}
= f_- \begin{pmatrix} \frac{A}{D} - i\frac{C}{D} e^{-L}  \\
\frac{C^*}{D} - i \frac{A}{D} e^{-L} \end{pmatrix}
\end{equation}
$f_+$ and $f_-$ are non-vanishing if and only if 
\begin{equation}
\begin{split}
& \frac A D - i\frac C D e^{-L} =  -ie^{-L}\big(\frac{ C^*}{ D} -i \frac A D e^{-L}\big) \\
\Leftrightarrow & \quad 2i A \cosh L = -C + C^*
\end{split}
\end{equation}
This equation can not hold since $A > |C|$.
As a consequence, $\dim\mathcal{K}_- = 0$. 
For $\mathcal{K}_+$, it suffices to replace $L$ by $-L$ in the above calculation and we will find $\dim\mathcal{K}_+ = 0$. 
%Therefore,
%$H$ possesses self-adjoint extension.

















