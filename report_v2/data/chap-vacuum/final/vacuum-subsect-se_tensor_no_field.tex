\subsection{Stress-energy tensor}
In an analogous way,
we can compute the stress-energy tensor in using the same renormalization prescription by point-splitting \wrt the Hadamard parametrix~\cite{Dappiaggi2009}.
In particular, in our configuration of confined space, one would expect that a correct way to define vacuum polarization will allow us to rediscover the Casimir effect~\cite{Casimir1948}.
The approach corresponds to the following computation.
\\\\
In the case of Dirac fields,
the components of the stress-energy tensor $T_{\mu\nu}$ are given by
\begin{equation}
\begin{split}
& T_{00} = \frac{i}{2} (\bar{\psi} \gamma_1 \nabla_1 \psi - \nabla_1 \bar{\psi}\gamma_1 \psi)  \\
& T_{11} = \frac{i}{2} (\bar{\psi} \gamma_0 \nabla_0 \psi - \nabla_0 \bar{\psi}\gamma_0 \psi)  \\
& T_{01} = \frac{i}{4} (\bar{\psi} \gamma_1 \nabla_0 \psi +\bar{\psi} \gamma_0 \nabla_1 \psi - \nabla_1 \bar{\psi}\gamma_0 \psi - \nabla_0 \bar{\psi}\gamma_1 \psi)  
\end{split}
\end{equation}
In terms of $\phi = \gamma^0 \psi$, 
\begin{equation*}
\bar{\psi} \gamma_1 \nabla \psi = - \phi^\dagger \gamma^1 \gamma^0 \nabla \phi
\end{equation*}
As usual, we start with the region $[-\frac{L}{2}, 0)$. By denoting
\begin{equation*}
\zeta = \gamma^1 \gamma^0 = \begin{pmatrix}
1 & 0 \\
0 & -1
\end{pmatrix}
\end{equation*}
and using the function $\xi_E$ that we have introduced in \cref{vacuum-xie}, 
the components of the renormalized two-point function of the stress-energy tensor are (up to terms of higher order and vanishing at the coinciding-point limit)
\begin{equation}\label{vacuum-stressenergy}
\begin{split}
T_{00}(x,y) = 
& \frac{i}{2}\Big(\nabla_{x^1} \big( \omega(\phi^B(x) \phi^\dagger_A(y))\zeta^A_C - H^+(x,y) \big)
- \nabla_{y^1} \big( \omega( \phi^B(x) \phi^\dagger_A(y))\zeta^A_C - H^+(x,y) \big)
\Big)\delta_B^C  \\
= & \frac{i}{2} \big( (-\xi_E'(z) - \xi_E'(w)) - \xi_E'(z) - \xi_E'(w) + \frac{i}{\pi z^2} + \frac{i}{\pi w^2} \big)   \\
T_{11}(x,y) =
& \frac{i}{2}\Big(\nabla_{x^0} \big( \omega(\phi^B(x) \phi^\dagger_A(y))\delta^A_C - H^+(x,y) \big)
- \nabla_{y^0} \big( \omega( \phi^B(x) \phi^\dagger_A(y))\delta^A_C - H^+(x,y) \big)
\Big)\delta_B^C  \\
= & - \frac{i}{2}\big( \xi_E'(z) + \xi_E'(w) + \xi_E'(z) + \xi_E'(w) - \frac{i}{\pi z^2} - \frac{i}{\pi w^2}\big) \\
T_{01}(x,y) = 
& \frac{i}{4}\Big(\nabla_{x^0} \big( \omega(\phi^B(x) \phi^\dagger_A(y))(\zeta_1)^A_C - H^+(x,y) \big) + \nabla_{x^1} \big( \omega(\phi^B(x) \phi^\dagger_A(y))\delta^A_C - H^+(x,y) \big)  \\
& - \nabla_{y^0} \big( \omega( \phi^B(x) \phi^\dagger_A(y))\zeta^A_C - H^+(x,y) \big)
- \nabla_{y^1} \big( \omega( \phi^B(x) \phi^\dagger_A(y))\delta^A_C - H^+(x,y) \big)
\Big)\delta_B^C \\
= & \frac{i}{4}\Big( \big( \xi_E'(z) - \xi_E'(w) \big) + \big(- \xi_E'(z) + \xi_E'(w) \big) - \big( - \xi_E'(z) + \xi_E'(w) \big) - \big( \xi_E'(z) - \xi_E'(w) \big) \Big) \\
= & 0
\end{split}
\end{equation}
where $z = x^0 - y^0 - x^1 + y^1$ and $w = x^0 - y^0 + x^1 - y^1$ \\
Taking the coinciding point limit, we find
\begin{equation}\label{vacuum-T-with-field}
T_{\mu\nu}(x) = 
\bigg( \frac{\pi}{2L^2}\big( -\frac{1}{3} + \frac{(\theta - \pi)^2}{\pi^2}\big) + \frac{e^2E^2(x^1)^2}{2 \pi} \bigg)
\begin{pmatrix}
1 & 0 \\ 0 & 1
\end{pmatrix}
\end{equation}
in the region $[-\frac L 2, 0)$.
In $(0, \frac L 2 ]$, 
it suffices to replace $\xi_E$ by $\chi_E$ defined by~\cref{vacuum-chie}.
It is straightforward to verify that we have exactly the same expression for the stress-energy tensor in both region.
\\\\
When the Kondo-type potential is turned off, \ie $v_3, v_+, v_- \rightarrow 0$, $\theta \rightarrow \frac 1 2 \pi$.
We obtain the same Casimir energy as calculated in~\cite{Sundberg2003}, \ie
\begin{equation*}
T_{\mu\nu} = -\frac{\pi}{24L^2}\begin{pmatrix} 1 & 0 \\ 0 & 1\end{pmatrix}
\end{equation*}
We observe that the presence of a Kondo potential in a confined space creates opposite vacuum charges in $[-\frac L 2, 0)$ and $(0, \frac L 2]$.
According to~\cref{vacuum-density_without_field}, 
this background polarization vanishes if and only if the parameter $C = 0$. 
In terms of coefficients of the Kondo potential in~\cref{nef-Dirac},
this corresponds to $v_+ = v_- = 0$.
In this case, the Kondo potential is not only Hermitian but diagonal in our system of coordinate. \\\\
%
We cite here the charge density and the stress-energy tensor obtained in~\cite{Zahn2015} for the same configuration as the ours except that there is no Kondo-type potential
\begin{equation*}
\begin{split}
& \rho = \frac 1 \pi e^2 Ex^1 \\
& T_{\mu\nu} = - \big( \frac{\pi}{24L^2} - \frac{e^2 E^2 (x^1)^2}{2\pi}\big) \begin{pmatrix} 1 & 0 \\ 0 & 1 \end{pmatrix}
\end{split}
\end{equation*}
Compared to these results,
there are only constant shifts in both charge density and stress-energy tensor in presence of the Kondo-type potential.
One can notice that the condition $C = 0$ ($\theta = \frac \pi 2$ not only leads to a vanishing vacuum charge density but also allows to obtain the same stress-energy tensor as in~\cite{Zahn2015}.













