\section{Vacuum charge and current in the spatially bounded case}
From now on, we work on the self-adjoint extension of $H$.
Let $\phi_k$ be a normalized eigenfunction for eigenvalue $k$ of the Hamiltonian $H$. 
First of all, 
let us find the expression of $\phi_k$ in the region $[-\frac{L}{2}, 0)$.
Once this has been done, 
we can find the expression of $\phi_k$ in $[-\frac L 2, 0)$ by the matching condition.
We check that in the region $[-\frac{L}{2}, 0)$, 
$\phi_k$ can be written as 
\begin{equation}\label{nef-boundCond}
\phi_k =\begin{pmatrix}
\phi_{k,L}  \\
\phi_{k,R}
\end{pmatrix} 
\quad \textrm{where $\begin{split} 
 & \phi_{k,L} = f_k e^{-\frac{i}{2}eE(x^1)^2 + ikx^1} \\
& \phi_{k,R} = g_k e^{\frac{i}{2}eE(x^1)^2 - ikx^1}
\end{split}
$}
\end{equation}
for $f, g\in \mathbb{C}$. 
The matching condition~\cref{nef-matching2} at $x^1 = 0$ requires
\begin{equation}
\begin{cases}
g = fe^{i(kL + \frac{\pi}{2}) - i\frac{eE}{8}L^2}  \\
g = \frac{A + iC^* e^{-ikL + i\frac{eE}{8}L^2 }}{A - iC e^{ikL - i\frac{eE}{8}L^2}} fe^{ikL - i\frac{eE}{8}L^2 + i\frac{\pi}{2}}
\end{cases}
\end{equation}
which implies $|f| = |g|$ and
\begin{equation}\label{nef-kn1}
kL = \textrm{Arg}(A - i|C| e^{ikL + i\eta_E}) + (n + \frac{1}{2})\pi \quad n\in \mathbb{Z}
\end{equation}
where
\begin{equation*}
\eta_E = \textrm{Arg } C - \frac{eE}{8}L^2 
\end{equation*}
The case $|C| =0$ is relatively easy to deal with. Let us focus on the cases where $|C| \neq 0$. We should consider separately \cref{nef-kn1} for $n$ odd and $n$ even because of the $2\pi$-periodicity of the exponential. \\\\
Since\footnote{
For $\alpha, \beta, \theta \in \mathbb{R}$, assuming that $\alpha + \beta \cos \theta > 0$, $\alpha + \beta e^{i \theta} = \alpha + \beta \cos \theta + i\beta \sin \theta = (\alpha^2 + \beta^2 + 2\alpha \beta \cos \theta) e^{i \delta}$ with $\delta = \arctan \frac{\beta\sin\theta}{\alpha + \beta\cos\theta}$  
} 
\begin{equation}
\begin{split}
&\textrm{Arg}(A - iC e^{ikL}) \\
= &\textrm{Arg}(A + |C| e^{i(kL - \frac{\pi}{2} + \eta_E)}) \\
= & \arctan \bigg( \frac{|C| \sin(kL - \frac{\pi}{2} + \eta_E)}{A + | C| \cos(kL - \frac{\pi}{2} + \eta_E) }\bigg)
\end{split}
\end{equation}
Therefore, by \cref{nef-kn1}, $k$ must satisfy
\begin{equation}\label{nef-arctan}
\begin{split}
& \frac{|C| \sin(kL - \frac{\pi}{2} + \eta_E)}{A + | C| \cos(kL - \frac{\pi}{2} + \eta_E) } =  - \cot kL  \\
\Leftrightarrow \quad & A \cot kL = |C| \cos(kL + \eta_E) - |C| \cot kL \sin(kL + \eta_E)  \\
%
\Leftrightarrow\quad &
A \cos kL + |C| \sin\eta_E= 0
\end{split}
\end{equation}
We want to find $k$ such that $kL \in [0, \pi]$ in order to coincide it with the allowed values of $\arctan$.
In this case, the only allowed value for $kL$ is given by
\begin{equation*}
kL = \arccos \big(-\frac{|C|\sin\eta_E}{A}\big)
\end{equation*}
For all even $n$, the corresponding mode $k_n$ is equal to this value modulo $2 \pi$. \\
For odd $n$, the calculation is similar.
We try to find $k$ such that $kL - \pi \in [0, \pi]$, which gives 
\begin{equation*}
kL = 2\pi - \arccos \big(-\frac{|C|}{A}\big)
\end{equation*} 
And thus for all odd $n$, the corresponding mode $k_n$ is equal to this value modulo $2 \pi$.\\
To sum up, the possible values of $k$ are given by
\begin{equation*}
k_{n} = \frac{(-1)^n}{L}\theta  + \frac{\pi}{L}n 
\end{equation*}
where
\begin{equation*}
\theta = \arccos\bigg( \frac{-|C| \sin \eta_E}{A} \bigg)
\end{equation*}
The coefficients $f_{n}$ and $g_{n}$ for the mode $k_{ n}$ can be determined in using the normalization condition  $\int_{[-\frac{L}{2}, \frac{L}{2}]}\phi^\dagger \phi = 1$. 
In the region $[-\frac{L}{2}, 0)$ , $\phi^\dagger \phi = | f |^2 + | g |^2$. Whereas in the region $(0, \frac{L}{2}]$, 
\begin{equation}\label{nef-norm1}
\begin{split}
\phi^\dagger \phi & = \begin{pmatrix}
\frac{1}{D^*}(Af^* +  C^*g^*)e^{-ikx^1}  & \frac{1}{D^*}(C f^* + Ag^*)e^{ikx^1} 
\end{pmatrix}\begin{pmatrix}
\frac{1}{D}(Af +  Cg)e^{ikx^1}  \\
 \frac{1}{D}(C^* f + Ag)e^{-ikx^1} 
\end{pmatrix}  \\
 & =
\frac{A^2 + | C|^2}{| D |^2}(|f|^2 + |g|^2) + 4\frac{A}{|D|^2}\Re \{C f^* g\}
\end{split}
\end{equation}
By the first equation of \cref{nef-boundCond}, the last term of~\cref{nef-norm1} is 
\begin{equation*}
4\frac{A |C|}{|D|^2}| f|^2\Re\{e ^{-i(kL + \frac{\pi}{2} - \eta_E)}\} = 
- 4\frac{A |C|}{|D|^2}| f|^2\sin( kL - \eta_E) 
\end{equation*}
Hence
\begin{equation*}
 | f_{n} | =  \sqrt{\frac{1}{L(\alpha - \beta \sin (k_{n} L - \eta_E))}}  
\end{equation*}
where 
\begin{equation*}
\alpha = 1+\frac{A^2 + |C|^2}{|D|^2} \quad,\quad
\beta = \frac{2 A |C|}{|D|^2}
\end{equation*}
We have thus found an eigenfunction for eigenvalue $k_n$
\begin{equation*} 
\begin{split}
\phi_{k_{n}} = 
& \sqrt{\frac{1}{L(\alpha - \beta \sin (k_{n}L - \eta_E))}} \Bigg( 
\begin{pmatrix}
1 & 0 \\
0  & e^{-i(kL + \frac{\pi}{2}) + i\frac{eE}{8}L^2}
\end{pmatrix}
\Theta(-x^1) + \\
& \begin{pmatrix}
\frac{A}{D}  +  \frac{C}{D} e^{-i(kL + \frac{\pi}{2})+ i\frac{eE}{8}L^2} & 0 \\
0  & \frac{C^*}{D}  + \frac{A}{D}e^{-i(kL + \frac{\pi}{2} ) + i\frac{eE}{8}L^2}
\end{pmatrix}\Theta(x^1)\Bigg)
\begin{pmatrix}
e^{ik_{n} x^1 -  i\frac{eE}{2}(x^1)^2} \\
e^{- ik_{n} x^1 + i\frac{eE}{2}(x^1)^2}
\end{pmatrix}
\end{split}
\end{equation*}
Let us compute now the vacuum 1+1 current~\cref{vacuum-currentexpression}.
As we will trace the two-point function defined in~\cref{vacuum-hadamardstate} with $\gamma^\mu$, 
only the off-diagonal terms of the two-point function should be considered.
In terms of $\phi$, 
the two-point function~\cref{vacuum-hadamardstate} becomes
\begin{equation}\label{vacuum-calculhadamard}
\omega(\psi^B(x) \bar{\psi_A}(y)) = 
(\gamma^0)^B_C \omega(\phi^C(x) \phi^\dagger_A(y)) = (\gamma^0)^B_C
\int_{E_k > 0} \phi^C_k(x) \phi^\dagger_{k,A}(y) e^{-i(x^0 - y^0) E_k} \dd k
%\quad \textrm{for $A = 1,2$}
\end{equation}
Let us calculate explicitly the two-point function~\cref{vacuum-calculhadamard} in the region $[-\frac L 2,0)$. 
For $A =1, B= 2$, 
with 
\begin{equation*}
z =x^0 - y^0 - x^1 +y^1
\end{equation*}
\cref{vacuum-calculhadamard} becomes
\begin{equation*}
\begin{split}
& \sum_{2p \geq 0} \frac{e^{-i(\theta + 2p\pi)\frac{z}{L}+ \frac{ieE}{2}(x^1 - y^1)(x^1+y^1)} }{L(\alpha - \beta \sin (\theta - \eta_E))} 
+ \sum_{2p+1 \geq 0} \frac{e^{-i(- \theta + (2p+2)\pi)\frac{z}{L}+ \frac{ieE}{2}(x^1 - y^1)(x^1+y^1)}}{L(\alpha + \beta \sin (\theta + \eta_E))}\\
%
=& 
\frac{1}{2i L\sin\frac{\pi}{L}z} \bigg( \frac{e^{i(-\theta + \pi)\frac{z}{L}+ \frac{ieE}{2}(x^1 - y^1)(x^1+y^1)}}{\alpha - \beta \sin (\theta - \eta_E)}
+ \frac{e^{i(\theta - \pi) \frac{z}{L}+ \frac{ieE}{2}(x^1 - y^1)(x^1+y^1)}}{\alpha + \beta \sin (\theta + \eta_E)}
\bigg)
\end{split}
\end{equation*}
Using\footnote{
As
\begin{equation*}
\begin{split}
& \alpha^2 - \beta ^ 2 = 2 \alpha \\
&  \beta^2 \big(1 + \frac{|C|^2}{A^2} \big) - 2 \alpha \beta \frac{|C|}{A} \\
= & \big(2\frac{A |C|}{|D|^2} \big)^2 \big( 1+ \frac{|C|^2}{A^2} \big) - 4\big( 1+ \frac{|C|^2}{D^2}))\big(2\frac{A |C|}{|D|^2} \big) \frac{|C|}{A}  \\
= & 4 \frac{A^2 |C|^2}{|D|^4} + 4\frac{|C|^4}{|D|^4} - 8\frac{|C|^2}{|D|^2} - 8\frac{|C|^4}{|D|^4} \\
= & -2 \beta \frac{|C|}{A}
\end{split}
\end{equation*}
we have
\begin{equation*}
\begin{split}
& \frac{1}{\alpha - \beta \sin (\theta - \eta_E)}
   + \frac{1}{\alpha + \beta \sin (\theta + \eta_E)} \\
= & \frac{2(\alpha + \beta \sin \eta_E \cos \theta)}{(\alpha + \beta \sin \eta_E \cos \theta)^2 - \beta^2 \sin^2 \theta \cos^2 \eta_E} \\
= & \frac{2(\alpha - \beta \frac{|C|}{A} \sin^2 \eta_E)}{\alpha^2 - \beta^2 + \beta^2 \sin^2 \eta_E (1 + \frac{|C|^2}{A^2}) - 2 \alpha \beta \frac{|C|}{A} \sin^2 \eta_E} \\
=& 1
\end{split}
\end{equation*}
}
\begin{equation}\label{nef-lourdeur}
\frac{1}{\alpha - \beta \sin (\theta - \eta_E)}
   + \frac{1}{\alpha + \beta \sin (\theta + \eta_E)} 
= 1
\end{equation}
and
\begin{equation*}
\frac{1}{2i \sin \frac{\pi}{L}z } = \frac{-iL}{2 \pi z} - \frac{i \pi z}{12L} + \mathcal{O}(z^3) 
\end{equation*}
In the region $[-\frac{L}{2}, 0)$, we have, up to terms of higher order,
\begin{equation*}
\begin{split}
\omega(\psi^2(x) \bar{\psi_1}(y)) = 
\omega(\phi^1(x) \phi^\dagger_1(y)) = \xi_E(x^0 - y^0 - x^1 + y^1, x^1, y^1) 
\end{split}
\end{equation*}
where $\xi_E$ is defined by
\begin{equation}\label{vacuum-xie}
\begin{split}
\xi_E(z, x^1, y^1)
= & \frac{-i}{2 \pi z} + \frac{1}{2\pi L}\Big( \frac{\beta \sin \theta \cos \eta_E}{\alpha + \beta \sin \eta_E \cos \theta}\Big) (-\theta + \pi) 
\Big(1+ \frac{ieE}{2}(x^1 - y^1)(x^1 + y^1) \Big) \\
& + \frac{eE}{4 \pi}\frac{(x^1 - y^1)(x^1 + y^1)}{z}     
+ \frac{i\pi}{4 L^2}\big( -\frac{1}{3} + \frac{(\theta - \pi)^2}{\pi^2}\big)z
+ \frac{i e^2 E^2}{16 \pi z} (x^1 - y^1)^2 (x^1 + y^1)^2
\end{split}
\end{equation}
Analogously, for $A=2,B=1$, we have, up to terms of higher order,
\begin{equation*}
 \omega(\psi^1(x) \bar{\psi_2}(y)) = 
\omega(\phi^2(x) \phi^\dagger_2(y)) = \xi_E(x^0 - y^0 + x^1 - y^1, x^1, y^1)
\end{equation*}
Similarly, we have the expressions of the components of the two-point function up to terms of higher order in the region $(0, \frac{L}{2}]$
\begin{equation*}
\begin{split}
\omega(\psi^2(x) \bar{\psi_1}(y)) = 
\omega(\phi^1(x) \phi^\dagger_1(y)) = \chi_E(x^0 - y^0 - x^1 + y^1, x^1, y^1) \\
 \omega(\psi^1(x) \bar{\psi_2}(y)) = 
\omega(\phi^2(x) \phi^\dagger_2(y)) = \chi_E(x^0 - y^0 + x^1 - y^1, x^1, y^1)
\end{split}
\end{equation*}
where $\chi_E$ is defined by
\begin{equation}\label{vacuum-chie}
\begin{split}
\chi_E(z, x^1, y^1) = 
& \frac{-i}{2 \pi z} + \frac{1}{2\pi L}\Big( \frac{\beta \sin \theta \cos \eta_E}{\alpha + \beta \sin \eta_E \cos \theta}\Big) (\theta - \pi)\Big(1+ \frac{ieE}{2}(x^1 - y^1)(x^1 + y^1) \Big) \\
&+ \frac{eE}{4 \pi}\frac{(x^1 - y^1)(x^1 + y^1)}{z}  
+ \frac{i\pi}{4 L^2}\big( -\frac{1}{3} + \frac{(\theta - \pi)^2}{\pi^2}\big)z
+ \frac{i e^2 E^2}{16 \pi z} (x^1 - y^1)^2 (x^1 + y^1)^2
+ \mathcal{O}(z^2)
\end{split}
\end{equation}
To sum up,
subtracting the Hadamard parametrix~\cref{vacuum-hadamardparametrix} and tracing with the gamma matrices, 
the expression of the vacuum charge density defined by~\cref{vacuum-currentexpression} is
\begin{equation}\label{vacuum-rho-with-field}
\rho(x) = \frac{e}{\pi L}\Big( \frac{\beta \sin \theta \cos \eta_E}{\alpha + \beta \sin \eta_E \cos \theta}\Big) (-\theta + \pi)
\Big(\Theta(- x^1) - \Theta(x^1))\Big) + \frac{e^2 E}{\pi} x^1
\end{equation}
and the current density is zero. \\\\
%
We cite here the vacuum charge density found in~\cite{Zahn2015} for the same configuration without Kondo-type potential
\begin{equation*}
\rho(x) = \frac{e^2 E}{\pi}x^1
\end{equation*}
and the vacuum current density is also 0.
Compared to~\cite{Zahn2015}, 
we see that the Kondo-type potential creates an offset on the charge density between the two regions seperated by the singular point.
When $E\rightarrow 0$, the first term of the \lhs of~\cref{vacuum-rho-with-field} does not vanish if $|C| \neq 0$.
Concretely, this explains that the interaction between spinors of different chiralities due to the Kondo-type potential (the off-diagonal terms) leads to a background polarization before the introduction of the electric field.
%%%%%%%%%%%%%%%%%%%%%%%%%%%%%%%%
\begin{remark}
The case of a non-confined space-time is partially treated in~\cref{appendix-vac}. 
We show how this kind of calculation could be done for non-discrete energy levels. 
We yield indeed the same result for the vacuum polarization as taking the $L\rightarrow+\infty$ limit in~\cref{vacuum-rho-with-field}. 
However, the self-adjointness of the Hamiltonian of the problem is assumed but not proven for that case.
\end{remark}
