\section{Discussion}
The presence of a Kondo potential in a confined space creates opposite vacuum charges in $[-\frac L 2, 0)$ and $(0, \frac L 2]$.
According to~\cref{vacuum-density_without_field}, 
this background polarization vanishes if and only if the parameter $C = 0$. 
In terms of coefficients in Kondo potential in~\cref{nef-Dirac},
this corresponds to $v_+ = v_- = 0$.
In this case, the Kondo potential is not only Hermitian but diagonal in our system of coordinate. \\\\
%
We cite here the charge density and the stress-energy tensor obtained in~\cite{Zahn2015}
\begin{equation*}
\begin{split}
& \rho = \frac 1 \pi e^2 Ex^1 \\
& T_{\mu\nu} = - \big( \frac{\pi}{24L^2} - \frac{e^2 E^2 (x^1)^2}{2\pi}\big) \begin{pmatrix} 1 & 0 \\ 0 & 1 \end{pmatrix}
\end{split}
\end{equation*}
Compared to these results, 
\cref{vacuum-sect-field} shows that there are only constant shifts in both charge density and stress-energy tensor due to the polarized background.
One can notice that the condition $C = 0$ not only leads to a vanishing vacuum charge density but also allows to obtain the same stress-energy tensor as in~\cite{Zahn2015}.
\\\\
On the other hand, when we enlarge the confined space by doing $L \rightarrow \infty$, 
the results of~\cref{sect-nef} and~\cref{vacuum-sect-field} coincide with the results of~\cref{vacuum-without_bdy}.
Mean while, the charge density in the case with external electric field~\cref{vacuum-charge_nbdy} implies a diverged total charge.
In effect, this situation could be discarded since a constant electric field in an infinite space is not physically relevant.































