\subsection{Non-linear equation}
We put the gauge field part into the action~\cref{wen-action} and use covariant derivative instead. 
The total action is now
\begin{equation*}
\begin{split}
\mathcal{S} = & \frac{1}{2}i\int_M \bar{\psi} \gamma^\mu (\partial_\mu+i A_\mu) \psi - (\partial_\mu - iA_\mu) \bar{\psi} \gamma^\mu \psi 
+ \frac{1}{2}\int_{\partial M} ic \bar{\psi} \gamma^\alpha (\partial_\alpha+iA_\alpha) (1 - i \gamma^\bot) \psi
+ \bar{\psi} \psi \\ 
%
& - \frac 1 4 \int_M F^{\mu\nu}F_{\mu\nu} - \frac 1 4 \int_{\partial M} F^{\alpha\beta}F_{\alpha\beta}
\end{split}
\end{equation*}
From this action, we can derive a system of equations containing the equations of motion of the Dirac fields as~\cref{wen-motion} and the equations coupling the matter field to the gauge field.
The coupling equations of motion are
\begin{equation}\label{ym-coupling}
\begin{cases}
\nabla^A_\mu F^{\mu\nu} = \psi^\dagger\gamma^0\gamma^\nu\psi \\
%
\nabla^{\bar{A}}_\alpha \bar{F}^{\alpha\beta} = \psi_|^\dagger\gamma^0\gamma^\beta\psi_| -c^{-1}\bar{F}_{\nu\beta}n^\nu 
\end{cases}
\end{equation}
It could be shown that the tangential component of the current term in the bulk is equal to the current term on the boundary as long as we have $\mathcal{P}_+\phi\vert_{\partial M} = \phi_|$. 
In effect, this condition implies
\begin{equation}\label{ym-current}
\phi = 2 \phi_| - i\gamma^\bot\phi
\end{equation}
Multiplying~\cref{ym-current} by $(\gamma^j\phi)^\dagger\gamma^0$ on the left, we have
\begin{equation}\label{ym-current1}
(\gamma^j\phi)^\dagger\gamma^0\phi = 2(\gamma^j\phi)^\dagger\gamma^0\phi_| - i(\gamma^j\phi)^\dagger\gamma^0\gamma^\bot\phi
\end{equation}
Taking the adjoint of~\cref{ym-current} multiplied by $\gamma^0$ and multiplying by $\gamma^j\phi$ on the right, we get
\begin{equation}\label{ym-current2}
(\gamma^0\phi)^\dagger\gamma^j\phi = 2\phi^\dagger_|\gamma^0\gamma^j\phi - i\phi^\dagger\gamma^\bot\gamma^0\gamma^j\phi
\end{equation}
%perhaps put earlier?
\begin{remark}
Choosing $j = \bot$, one gets from~\cref{ym-current1} and~\cref{ym-current2}
\begin{equation}\label{ym-current3}
\phi^\dagger\gamma^0\gamma^\bot\phi = i\phi^\dagger(\mathcal{P}_+\gamma^0 - \gamma^0\mathcal{P}_+)\phi  
\end{equation}
In terms of $\psi = \gamma^0\phi$, we can write the \textbf{in-coming} current normal to the boundary as
\begin{equation}\label{ym-current3}
J^\bot = i\psi^\dagger(\gamma^0\mathcal{P}_- - \mathcal{P}_-\gamma^0 )\psi
\end{equation}
We can also check that the current on the boundary is not conserved but is consistent with the current coming from the bulk. 
The boundary current density can be expressed as
\begin{equation*}
J_|^\alpha = i c \psi^\dagger_|\gamma^0\gamma^\alpha\psi_|
\end{equation*}
We recall the equation of motion of the Dirac field on the boundary in terms of $\psi$
\begin{equation*}
i\gamma^\alpha\partial_\alpha\psi_| = -c^{-1}\mathcal{P}_+\psi
\end{equation*}
Hence, the on-shell divergence of the current density is
\begin{equation*}
\begin{split}
\partial_\alpha J_|^\alpha  = & c\big((\partial_\alpha\psi_|^\dagger)\gamma^0\gamma^\alpha\psi_| + \psi_|^\dagger\gamma^0\gamma^\alpha(\partial_\alpha\psi) \big)\\
= &
-i\psi^\dagger\vert_{\partial M}\mathcal{P}_+ \gamma^0\psi_| + i\psi_|^\dagger\gamma^0 \mathcal{P}_+\psi\vert_{\partial M} \\
= & -J^\bot
\end{split}
\end{equation*}
\end{remark}
%check the stress tensor energy for the elm field
\begin{remark}
The same verification of consistency can be done for the vector field here.  
The stress-energy tensor of the vector field on the boundary is given by
\begin{equation*}
\bar{T}^{\alpha\beta} = c\big( \bar{F}^{\alpha\gamma}\bar{F}^{\beta}_{\enskip\gamma}
- \frac 1 4 \eta^{\alpha\beta}\bar{F}^{\gamma\delta}\bar{F}_{\gamma\delta} \big)
\end{equation*}
Its divergence is 
\begin{equation*}
\begin{split}
\partial_\alpha \bar{T}^{\alpha\beta} = & c\big(
(\partial_\alpha \bar{F}^{\alpha\gamma})\bar{F}^{\beta}_{\enskip\gamma} +
\bar{F}^{\alpha\gamma}(\partial_\alpha\bar{F}^{\beta}_{\enskip\gamma})
- \frac 1 4 \eta^{\alpha\beta}\partial_\alpha(\bar{F}^{\gamma\delta}\bar{F}_{\gamma\delta}) \big)\\
\underset{\textrm{on-shell}}= & 
- F^{\bot\alpha}\bar{F}^\beta_{\enskip\alpha}
\end{split}
\end{equation*}
The last two terms on the right hand side of the first line are cancelled out by symmetry and the first Bianchi identity that $\bar{F}$ satisfies.
It is then obvious that
\begin{equation*}
\partial_\alpha \bar{T}^{\alpha\beta} = -T^{\bot\beta}\vert_{\partial M}
\end{equation*}
\end{remark}
%Some little discovery
\begin{remark}
%As one might notice,~\cref{ym-coupling} also imposes some condition on the current density.
%As an example, take $M = \mathbb{R}_+\times \mathbb{R}_+\times\mathbb{R}^{d-1}$.
Suppose that we have $\psi = \gamma^0\phi$ such that $\phi\in \cap_{k\in\mathbb{N}^*}\Delta^k$ where $\Delta$ is the Hamiltonian~\cref{wen-hamiltonian}.
We can show that the~\cref{ym-coupling} is consistent with this solution for $\psi$.
Since we work with flat space-time, 
the difference between the two equations of~\cref{ym-coupling} reads
\begin{equation*}
\begin{split}
\partial_\bot F^{\bot \mu}\vert_{\partial M} = & \psi^\dagger\gamma^0\gamma^\mu\psi - \psi^\dagger\gamma^0\gamma^\mu\psi_| + c^{-1} F^{\bot \mu} \\
= & \psi^\dagger\mathcal{P}_+\gamma^0\gamma^\mu\psi + c^{-1}F^{\bot \mu }\vert_{\partial M}
\end{split}
\end{equation*}
Its divergence is
\begin{equation}\label{ym-consist1}
\begin{split}
\partial_\mu(\partial_\bot F^{\bot \mu})\vert_{\partial M} = & \partial_\mu(\psi^\dagger\mathcal{P}_+\gamma^0\gamma^\mu\psi) + c^{-1}\partial_\mu F^{\bot \mu}\vert_{\partial M} \\
\underset{\textrm{on-shell}}{=} & 
i(\partial_\bot\psi^\dagger\gamma^0\mathcal{P}_+\psi)\vert_{\partial M} - 
i(\psi^\dagger\mathcal{P}_+\gamma^0\partial_\bot\psi)\vert_{\partial M}
- c^{-1} \psi^\dagger\gamma^0\gamma^\bot\psi \\
%= & \Re (i\partial_\bot \psi^\dagger\gamma^0\mathcal{P}_+\psi)\vert_{\partial M} - c^{-1} J^\bot \\
\end{split}
\end{equation}
On the other hand
\begin{equation}\label{ym-consist2}
\begin{split}
(\partial_\bot\partial_\mu F^{\bot \mu})\vert_{\partial M} = & -\partial_\bot J^\bot\vert_{\partial M} \\
= & -\partial_\bot \psi^\dagger\gamma^0\gamma^\bot\psi 
-\psi^\dagger\gamma^0\gamma^\bot\partial_\bot\psi
\end{split}
\end{equation}
Equaling~\cref{ym-consist1} and~\cref{ym-consist2} and using
\begin{equation*}
\phi\in\dom(\Delta^2) \quad\Rightarrow\quad 
\mathcal{P}_+(\partial_\bot\psi\vert_{\partial M} - c^{-1} \psi\vert_{\partial M})
\end{equation*}
we find
\begin{equation*}
\begin{split}
0 = &
i \big( \partial_\bot\psi^\dagger\gamma^0\mathcal{P}_-\psi 
- \psi^\dagger\mathcal{P}_-\gamma^0\partial_\bot\psi \big)\vert_{\partial M}  
-c^{-1}J^\bot \\
= &
i c^{-1}\big( \psi^\dagger\mathcal{P}_+\gamma^0\psi 
- \psi^\dagger\gamma^0\mathcal{P}_+\psi \big)\vert_{\partial M} 
-c^{-1}J^\bot
\end{split}
\end{equation*}
We find again the expression of $J_\bot$~\cref{ym-current3}.
\end{remark}










