\section{Equation of motion}
Given $M$ a $(d +1)$-dimensional smooth complete Lorentzian spin manifold with boundary. 
We suppose furthermore that the torsion of $\{M,d\}$ is reduced to $\{0\}$ and that we can define Sobolev spaces corresponding to all order of derivatives on $M$. 
Let us now derive the equations of motion from the action~\cref{ym-action}.
A variation $A \rightarrow A + \delta A$ leads to $\mathcal{S}_{YM} \rightarrow \mathcal{S}_{YM} + \delta\mathcal{S}_{YM}$ where
%
\begin{equation*}
\begin{split}
\delta \mathcal S_{YM} = & -\frac 1 2 \Big(
 \langle \dd \delta A, \dd A +A\wedge A  \rangle_{L^2(M)} +  
 \langle \delta A \wedge A, \dd A + A\wedge A \rangle_{L^2(M)} + 
 \langle A \wedge \delta A , \dd A + A\wedge A \rangle_{L^2(M)}  \\ & + 
 c (\langle \dd \delta A, \dd A +A\wedge A\rangle_{L^2(\partial M)}+
 \langle \delta A \wedge A, \dd A + A\wedge A  \rangle_{L^2(\partial M)} + 
 \langle A \wedge \delta A , \dd A + A\wedge A \rangle_{L^2(\partial M)} 
 )
 \Big) \\
 %
 \underset{\textrm{\cref{ym-stokes}}}{=} & - \frac{1}{2} \Big(
 \langle \delta A_t, (\dd A + A\wedge A)_n \rangle_{L^2(\partial M)} +
 \langle \delta A, \dd^c (\dd A + A\wedge A) \rangle_{L^2(M)} + 
 \langle \delta A \wedge A,  \dd A + A \wedge A \rangle_{L^2(M)} \\ &+
 \langle A \wedge \delta A,  \dd A + A \wedge A \rangle_{L^2(M)}  +
 \langle \delta A, \dd^c (\dd A + A\wedge A) \rangle_{L^2(\partial M)} + 
 \langle \delta A \wedge A, \dd A + A \wedge A \rangle_{L^2(\partial M)} \\&+
 \langle A \wedge \delta A, \dd A + A \wedge A \rangle_{L^2(\partial M)}
 \Big) \\
 %
\underset{\textrm{\cref{ym-wedge}}}{=} & - \frac 1 2 \Big(
\langle \delta A, (\dd^c + [A, \cdot ])(\dd A+ A \wedge A) \rangle_{L^2(M)} \\ &+
c \big( c^{-1}\langle \delta A_t, (\dd A + A\wedge A)_n \rangle_{L^2(\partial M)} +
\langle \delta A, (\dd^c + [A, \cdot ])(\dd A+ A \wedge A) \rangle_{L^2(\partial M)} \big)
\Big)
\end{split}
\end{equation*}
To compute the codifferential of 2-forms, the dimension of the space should be taken into account. In the bulk, we have
\begin{equation*}
\begin{split}
\dd^c F_{\mu\nu} \dd x^\mu \wedge \dd x^\nu = & (-1)^{d+3} \star \dd \star (F_{\mu\nu} \dd x^\mu \wedge \dd x^\nu ) \\
%
= & \frac{(-1)^{d+3}}{2 (d-1)!} \star \dd \Big( F_{\mu\nu} \varepsilon^{\mu\nu}_{\enskip \lambda_1\cdots \lambda_{d-1}} \bigwedge_{i} \dd x^{\lambda_i} \Big) \\
%
= & \frac{(-1)^{d+1}}{2 (d-1)!} \star \Big( \partial_\sigma F_{\mu\nu} \varepsilon^{\mu\nu}_{\enskip \lambda_1\cdots \lambda_{d-1}} \enskip \dd x^\sigma\wedge \big(\bigwedge_{i} \dd x^{\lambda_i} \big)  \Big) \\
%
= & \frac{1}{2(d-1)!} \star \Big( \partial_\sigma F_{\mu\nu} \varepsilon^{\mu\nu}_{\enskip \lambda_1\cdots \lambda_{d-1}} \enskip \big(\bigwedge_{i} \dd x^{\lambda_i} \big) \wedge \dd x^\sigma \Big) \\
%
= & \frac{1}{2(d-1)!d !} \enskip  \partial^\sigma F_{\mu\nu} \varepsilon^{\mu\nu}_{\enskip \lambda_1\cdots \lambda_{d-1}} \varepsilon^{\lambda_1 \cdots\lambda_d}_{\enskip\enskip\enskip\enskip \sigma\eta} \enskip \dd x^\eta \\
%
= & - \partial^\mu F_{\mu\nu} \dd x^\nu
\end{split}
\end{equation*}
Motivated by this computation, we propose to study the following system of equations of motion
\begin{equation}\label{ym-motion1}
\begin{cases}
\partial^\mu F_{\mu\nu} +  [A^\mu, F_{\mu\nu}] = 0 \quad\textrm{ in $M$}\\
%
\partial^\alpha \bar F_{\alpha\beta} +  [\bar A^\alpha, \bar F_{\alpha\beta}] = 
 - c^{-1} F_{\nu\beta}n^\nu
\quad\textrm{ on $\partial M$}
\end{cases}
\end{equation}
where the quantities with bar live on the boundary and $n$ is the unit \textbf{in-coming} vector normal to $\partial M$\\
Compared to the original work of C.N. Yang and R. L. Mills~\cite{Yang1954}, there is a supplementary source term in the equation of motion on the boundary due to the field of the bulk.\\\\
Denote $E_i = F_{0i}$ and $\bar E_a = \bar F_{0a} $. 
We choose the temporal gauge for the following calculation, \ie $A_0 = 0$
We can split~\cref{ym-motion1} into the dynamical system 
\begin{equation}\label{ym-dyn}
\begin{cases}
\partial_0 A_i = E_i \\
\partial_0 E_i =  \partial_j F_{ji} + [A_j, F_{ji}] \\
\partial_0 \bar A_a = \bar E_a \\
\partial_0 \bar E_a =  \partial_b \bar F_{ba} +  [\bar A_b, \bar F_{ba}] + c^{-1} F_{\nu a} n^\nu \\
\end{cases}
\end{equation}
and the constraint equation
\begin{equation}\label{ym-const}
\begin{cases}
\partial_i E_i  = - [A_i, E_i] \\
\partial_a \bar E_a = - [\bar A_a , \bar E_a]  + c^{-1} E_{b}n^b
\end{cases}
\end{equation}
where repeated latin indices are summed (since the Minkowski metric is $g_{ij} = \delta_{ij}$ under our convention). \\\\
Before defining the Hamiltonian of the system, we have to ensure that if a data $(A, E, \bar A, \bar E)$ satisfies~\cref{ym-const} at a given moment and is a solution of~\cref{ym-dyn}, then its evolution will always satisfy~\cref{ym-const}.
This consequence is general enough~\cite{Arnowitt1962}. \\\\
%
We define the covariant derivative corresponding to the field $A$ by 
\begin{equation*}
\nabla^A_\mu \equiv \partial_\mu + [A_\mu, \cdot]
\end{equation*}
With this notation, the constraint equations~\cref{ym-const} become
\begin{equation*}
\begin{cases}
\nabla^A_i E_i = 0 \\
%
\nabla^{\bar{A}}_i \bar E_i = - c^{-1} E_b n^b 
\end{cases}
\end{equation*}
In the bulk, we compute
\begin{equation*}
\begin{split}
\frac{d}{dt}\nabla^A_j E^j = & \partial_0 \nabla^A_j E^j  \\
= & \partial_0(\partial_j E^j + [A_j , E^j])  \\
= & \partial_j (\partial^i F_{ij} + [A^i, F_{ij}]) + [E_j, E^j] + [A_j, \partial^i F_{ij} + [A^i, F_{ij}]] \\
= & \partial^j\partial^i F_{ij}  + [A^i, \partial^j F_{ij}] + [A^j, \partial^i F_{ij}] + [\partial^j A^i, F_{ij}]+ [A^j, [A^i, F_{ij}]]
\end{split}
\end{equation*}
Since $F$ is anti-symmetric, the first three terms vanish. 
And we have 
\begin{equation}
[\partial^j A^i, F_{ij}] = \frac 1 2 \big( [F^{ji}, F_{ij}] - [[A^j, A^i], F_{ij}] \big)
\end{equation}
\begin{equation}
\begin{split}
&[A^j, [A^i, F_{ij}]] = - [A^i, [A^j, F_{ij}]]
= [A^j, [F_{ij}, A^i]] + [F_{ij}, [A^i, A^j]] \\
 \quad\Leftrightarrow\quad &
[A^j, [A^i, F_{ij}]] = \frac 1 2 [F_{ij}, [A^i, A^j]]
\end{split}
\end{equation}
The sum of the last two terms vanishes. \\\\
On the boundary, on the contrary, the preservation of the constraint equation is not automatically guaranteed. As we can see,
\begin{equation*}
\begin{split}
\frac{d}{dt}\nabla^{\bar A}_a \bar{E}^a = & 
 c^{-1}\big(\partial_a F_{ja}+ [\bar{A}_a, F_{ja}]\big) n^j
\end{split}
\end{equation*}
However, with the domain of the Hamiltonian of the dynamical system~\cref{ym-dyn} that we will write in the next subsection,~\cref{ym-const} can be preserved by choosing Cauchy data such that $A_t = \bar A$.
%%%%%%%%%%
\subsection{Linearized system}
Let us begin by studying~\cref{ym-dyn} up to first order in $A$ and $\bar{A}$. 
We define the operator $G_L$ with
\begin{equation}
G_L (A, E, \bar{A}, \bar{E}) = (E, -\dd^c \dd A, \bar{E}, -\dd^c \dd \bar{A} + c^{-1} A_\bot)
\end{equation}
where $A_\bot$ is the component of $A$ which is perpendicular to the boundary (inward directed). \\\\

\subsection{Non-linear equation}
We put the gauge field part into the action~\cref{wen-action} and use covariant derivative instead. 
The total action is now
\begin{equation*}
\begin{split}
\mathcal{S} = & \frac{1}{2}i\int_M \bar{\psi} \gamma^\mu (\partial_\mu+i A_\mu) \psi - (\partial_\mu - iA_\mu) \bar{\psi} \gamma^\mu \psi 
+ \frac{1}{2}\int_{\partial M} ic \bar{\psi} \gamma^\alpha (\partial_\alpha+iA_\alpha) (1 - i \gamma^\bot) \psi
+ \bar{\psi} \psi \\ 
%
& - \frac 1 4 \int_M F^{\mu\nu}F_{\mu\nu} - \frac 1 4 \int_{\partial M} F^{\alpha\beta}F_{\alpha\beta}
\end{split}
\end{equation*}
From this action, we can derive a system of equations containing the equations of motion of the Dirac fields as~\cref{wen-motion} and the equations coupling the matter field to the gauge field.
The coupling equations of motion are
\begin{equation*}
\begin{cases}
\nabla^A_\mu F^{\mu\nu} = i\phi^\dagger\gamma^\nu\gamma^0\phi \\
%
\nabla^{\bar{A}}_\alpha \bar{F}^{\alpha\beta} = i\phi_|^\dagger\gamma^\beta\gamma^0\phi_| -c^{-1}\bar{F}_{\nu\beta}n^\nu 
\end{cases}
\end{equation*}
It could be shown that the tangential component of the current term in the bulk is equal to the current term on the boundary as long as we have $\mathcal{P}_+\phi\vert_{\partial M} = \phi_|$. 
In effect, this condition implies
\begin{equation}\label{ym-current}
\phi = 2 \phi_| - i\gamma^\bot\phi
\end{equation}
Multiplying~\cref{ym-current} by $(\gamma^j\phi)^\dagger\gamma^0$ on the left, we have
\begin{equation}\label{ym-current1}
(\gamma^j\phi)^\dagger\gamma^0\phi = 2(\gamma^j\phi)^\dagger\gamma^0\phi_| - i(\gamma^j\phi)^\dagger\gamma^0\gamma^\bot\phi
\end{equation}
Taking the adjoint of~\cref{ym-current} multiplied by $\gamma^0$ and multiplying by $\gamma^j\phi$ on the right, we get
\begin{equation}\label{ym-current2}
(\gamma^0\phi)^\dagger\gamma^j\phi = 2\phi^\dagger_|\gamma^0\gamma^j\phi - i\phi^\dagger\gamma^\bot\gamma^0\gamma^j\phi
\end{equation}
%perhaps put earlier?
Choosing $j = \bot$, one gets from~\cref{ym-current1} and~\cref{ym-current2}
\begin{equation}\label{ym-current3}
\phi^\dagger\gamma^0\gamma^\bot\phi = i\phi^\dagger(\mathcal{P}_+\gamma^0 - \gamma^0\mathcal{P}_+)\phi  
\end{equation}
By comparing~\cref{ym-current3} to its adjoint, we find
\begin{equation*}
\phi^\dagger\gamma^0\gamma^\bot\phi\vert_{\partial M} = 0
\end{equation*}
As in the normal bag boundary condition, there is no current in the direction perpendicular to the boundary. \\\\
We can also check that the current on the boundary is conserved. 
The boundary current density can be expressed as
\begin{equation*}
J_|^\alpha = i \phi_|\gamma^\alpha\gamma^0\phi_|
\end{equation*}
Hence, the on-shell divergence of the current density is
\begin{equation*}
\begin{split}
\nabla_\alpha J_|^\alpha  = & (\nabla_\alpha\phi_|^\dagger)\gamma^\alpha\gamma^0\phi_| + \phi_|^\dagger\gamma^\alpha\gamma^0(\nabla_\alpha\phi) \\
= &
-ic^{-1}\mathcal{P}_+\phi^\dagger\vert_{\partial M} \gamma^\alpha\gamma^0\phi_| + i\phi_|^\dagger\gamma^\alpha\gamma^0 c^{-1}\mathcal{P}_+\phi\vert_{\partial M} \\
= & 0
\end{split}
\end{equation*}
This result is expected as there is no current coming from the bulk.

































