%\paragraph{Convention} In this section, we will work on Minkowski space-time with signature $(-, +,\ldots,+)$.
\section{Preliminary}
We recall briefly some mathematical notions which will be used in this section (see~\cite{Zeidler1995}~\cite{Ivancevic2011}~\cite{Frankel1997} for more details).
Let $M$ be a $n$-dimensional oriented manifold with boundary $\partial M$.
We denote $\Omega^p(U)$ the set of $p$-forms in $U\subseteq M$. 
\begin{definition}
The \textbf{Hodge star operator} $\star : \Omega^p(U) \mapsto \Omega^{n- p}(U)$ for all $p\in\mathbb{N}$ is defined by the following mapping
\begin{equation*}
\star ( \dd x^{\mu_1}\wedge\ldots\wedge\dd x^{\mu_p}) = 
\frac{1}{(n-p)!}\varepsilon^{\mu_1\ldots\mu_p}_{\quad\mu_{p+1}\ldots\mu_n}(\dd x^{\mu_{p+1}}\wedge\ldots\wedge\dd x^{\mu_n})
\end{equation*}
where $\varepsilon$ is the Levi-Civita tensor where $\varepsilon_{0\ldots p} = 1$ \\\\
For a $p$-form $F_{(p)}$, the mapping of $\star$ is given by
\begin{equation*}
\star F_{(p)} = \frac 1 {p !} F_{\mu_1\ldots\mu_p} \star(\dd x^{\mu_1}\wedge\ldots\wedge \dd x^{\mu_p})
\end{equation*}
\end{definition}
%
\begin{definition}
Let $d$ be the exterior differential operator on $\Omega^p(M)$. We define the \textbf{codifferential operator} $\dd^c: \Omega^p(M) \mapsto \Omega^{p-1}(M)$ by the mapping
\begin{equation*}
\dd^c \omega = (-1)^{n(p+1)+1 + s}\star d \star \omega
\end{equation*}
for all $\omega\in\Omega^p(M)$ where $1 \leq p \leq n$ and $s $ is the signature of the metric. 
\end{definition}
%
One can get the mapping $\star^2$ by simple calculation
\begin{proposition}
Let $F_{(p)}$ be a $p$-form on $\mathbb{R}^{n-m, m}$. Then
\begin{equation*}
\star^2 F_{(p)} = (-1)^{p(n-p)+m}F_{(p)}
\end{equation*}
\end{proposition}
%
We can define an inner product $\langle \cdot, \cdot \rangle_{L^2(M)}$ on $\Omega^p(M)\cap L^2(M)$ by
\begin{equation*}
\langle w, v \rangle_{L^2(M)} = \int_M w \wedge \star v
\end{equation*}
We introduce the notion of tangential and normal component of a $k$-form of $M$ on the boundary $\partial M$ (see~\cite{Schwarz2006}).
\begin{definition}
Let $\omega$ be a $k$-form of $M$. 
Each vector field $X$ on the tangent bundle $TM$ can be decomposed into its tangential part $X^\parallel$ and its normal part $X^\bot$ \wrt $\partial M$.
We define the \textbf{tangential component} $\omega_t$ and the $\textbf{normal component}$ $\omega_n$ as following
\begin{equation*}
\begin{split}
\omega_t(X_1, \ldots, X_k) = & \omega(X_1^\parallel ,\ldots, X_k^\parallel) \\
\omega_n(X_1, \ldots, X_k) =& \omega\vert_{\partial M} -  \omega_t(X_1, \ldots, X_k) 
\end{split}
\end{equation*}
where $X_1, \ldots, X_k$  are $k$ vector fields of the tangent bundle $TM\vert_{\partial M}$.
\end{definition}
The following version of the Stokes' theorem will be useful.
\begin{proposition}
Let $w$ and $v$ be two $1$-forms. Then
\begin{equation}\label{ym-stokes}
\langle \dd w,  \dd v \rangle_{L^2(M)} =
\int_{\partial M} w_t\wedge \star \dd v_n  +
\langle w, \dd^c \dd v \rangle_{L^2(M)}
\end{equation}
where the indices $t$ and $n$ are tangential and normal outward component to $\partial M$ and $s$ is the signature of the metric ($s=0$ for Riemannian and $s = 1$ for Lorentzian).
\end{proposition}
%
We introduce the following relation which will be useful when deriving the equation of motion
\begin{proposition}\label{ym-wedge}
Let $A$ be a 2-form and $B$ and $C$ be two 1-forms. Then
\begin{equation}
\langle A, B\wedge C \rangle_{L^2(M)} + \langle A, C\wedge B \rangle_{L^2(M)} =
- 2 \langle B, [C, A] \rangle_{L^2(M)}
\end{equation}
where $[C, A]$ is the 1-form defined by
\begin{equation*}
[C, A]_{i} \equiv C^j A_{ji} - A_{ji}C^j
\end{equation*}
where the summation over $\nu$ is understood.
\end{proposition}
\begin{proof}
In terms of components, we have (repeated indices are summed)
\begin{equation*}
\begin{split}
\langle A, B\wedge C \rangle_{L^2(M)} + \langle A, C\wedge B \rangle_{L^2(M)} = & 
A_{ij}(B^i C^j - C^j B^i) + A_{ij}(C^i B^j - B^j C^i) \\
\underset{A_{ij} = -A_{ji}}{=} & 2 A_{ij}(B^i C^j - C^j B^i)\\
\underset{\textrm{symmetry of }\langle \cdot, \cdot \rangle}{ = } &2 (B^i C^j - C^j B^i)A_{ij}
\end{split}
\end{equation*}
Consider the 1-forms $D$ and $E$ where $D_i = A_{ij}C^j$ and $E_i = B^j A_{ji}$, we can show that 
\begin{equation*}
\langle B, D\rangle_{L^2{(M)}} = \langle E, C \rangle_{L^2(M)}
\end{equation*}
which allows us to conclude.
\end{proof}
We are interested in certain types of mainfold.
Especially, we want them to have some nice cohomological properties. 
Some important results of~\cite{Goldshtein2006} will be used to define the framework.
For this purpose, we introduce them here.
\begin{definition}
Given a Banach complex $\{F^k, d\}$ we define the \textbf{torsion} of the complex $F^*$ by 
\begin{equation*}
T^k(F^*) \equiv H^k / \bar{H^k}
\end{equation*}
where $H^k(F^*) = \ker(d: F^k \rightarrow F^{k+1})/ \im(d: F^{k-1} \rightarrow F^k)$ is the cohomology of the complex $F^*$
\end{definition}
The following theorem is a generalized version of the Hodge-Kodaira decomposition.
\begin{theorem}\label{ym-kodaira}
For any complete Riemannian manifold $(M,g)$, the following conditions are equivalent :
\begin{enumerate}
%
\item[(i)]
\begin{equation*}
L^2(M, \Lambda^k) = \im \dd \oplus \im\dd^c \oplus 
\big(L^2(M, \Lambda^k)\cap\ker(\dd \dd^c + \dd^c \dd)\big)
\end{equation*}
where the decomposition is orthogonal
%
\item[(ii)]
\begin{equation*}
T_2^k(M) = T_2^{n-k}(M) = \{0 \}
\end{equation*}
\end{enumerate}
where $\Lambda^k$ is the $k$th exterior power.
\end{theorem}
Finally, we will use the generalized version of the Poincar�-Wirtinger inequalities
\begin{theorem}\label{ym-pw}
Denote $\Omega^k_{q,p}(M) \equiv \{ \omega \in L^q(M, \Lambda^k) \vert \dd \omega \in L^p \}$ and $H^k_{q,p}$ the corresponding cohomology. Then
\begin{enumerate}
\item $H^k_{q,p}(M,g) = 0$ if and only if there exists a constant $C<+\infty$ such that for any closed $p$-integrable differential form $\omega$ of degree $k$, there exists a differential form $\theta$ of degree $k-1$ such that $\dd \theta = \omega$ and
\begin{equation}
\| \theta \|_{L^q} \leq C \| \omega \|_{L^p}
\end{equation}
%
\item
If $T^k_{q,p}(M)=\{0\}$, there exists a constant $C'$ such that for any differential form $\theta \in \Omega^{k-1}_{q,p}(M)$ of degree $k-1$, there exists a closed form $\zeta\in\ker(d: F^{k-1} \rightarrow F^{k})$ such that 
\begin{equation}
\| \theta - \zeta \|_{L^q} \leq C' \| \dd \theta \|_{L^p}
\end{equation}
\end{enumerate}
\end{theorem}



























