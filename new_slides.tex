\documentclass[english]{beamer}
\usetheme{default}
\usepackage{lmodern}
\usepackage[utf8]{inputenc}
\usepackage[T1]{fontenc}
\usepackage{babel}
\usepackage{color}
\usepackage{listings}
\usepackage{booktabs}
\usepackage{amsmath,amsthm,amssymb,graphicx,mathrsfs}
\usepackage{stmaryrd}
\usepackage{slashed}
%\usepackage[hmargin={3.5cm,3.5cm},top=4cm,bottom=4cm]{geometry} %problem...

\graphicspath{{images/}}

\newcommand{\vb}[1]{\mathbf{#1}}
\newcommand{\V}[1]{\textnormal{#1}}
\newcommand{\dd}[0]{\textrm{d}}
\newcommand{\bra}[1]{\langle#1 |}
\newcommand{\ket}[1]{| #1 \rangle}
\newcommand{\vdot}[0]{\boldsymbol\cdot}
\newcommand{\dom}{\mathrm{Dom}}

\newtheorem{remark}{Remark}
\newtheorem{proposition}{Proposition}

\setbeamertemplate{footline}[frame number]{}
\setbeamertemplate{navigation symbols}{}
\setbeamertemplate{caption}[numbered]


%%%%%%%%%%%%%%%%%%%%%%%%%%%%
\lstset{%
  backgroundcolor=\color{white},   % choose the background color; you must add \usepackage{color} or \usepackage{xcolor}
  basicstyle=\small,        % the size of the fonts that are used for the code
  commentstyle=\color{blue},    % comment style
  extendedchars=true,              % lets you use non-ASCII characters; for 8-bits encodings only, does not work with UTF-8
  keepspaces=true,                 % keeps spaces in text, useful for keeping indentation of code (possibly needs columns=flexible)
  keywordstyle=\color{red},       % keyword style
  language=Fortran,                 % the language of the code
  rulecolor=\color{black},         % if not set, the frame-color may be changed on line-breaks within not-black text (e.g. comments (green here))
  showspaces=false,                % show spaces everywhere adding particular underscores; it overrides 'showstringspaces'
  showstringspaces=false,          % underline spaces within strings only
  showtabs=false,                  % show tabs within strings adding particular underscores
  stringstyle=\color{green},     % string literal style
  tabsize=2,                       % sets default tabsize to 2 spaces
  frame=single
}



%%%%%%%%%%%%%%%%%%%%%%%%%%%%%%%%%%%%%%%%%%%%%%%%%%%%%%%%%%%%%%%%%%%%%%%%
\title{Boundary and matching conditions for quantized Dirac fields}
\author{En-Hung CHAO}
\institute{{\'E}cole Polytechnique \& Universit{\"a}t Leipzig}
\date{\today}
%%%%%%%%%%%%%%%%%%%%%%%%%%%%%%%%%%%%%%%%%%%%%%%%%%%%%%%%%%%%%%%%%%%%%%%%
%%%%%%%%%%%%%%%%%%%%%%%%%%%%%%%%%%%%%%%%%%%%%%%%%%%%%%%%%%%%%%%%%%%%%%%%
\begin{document}
%\selectlanguage{english}
%\graphicspath{{../images/}}

\begin{frame}
\titlepage%
\centerline{supervised by Jochen Zahn (Unisit{\"a} Leipzig)}
\end{frame}
%%%%%%%%%%%%%%%%%%%%%%%%%%%%%%%%%%%%%%%
%%%%%%%%%%%%%%%%%%%%%%%%%%%%%%%%%
\begin{frame}
\frametitle{Outline}
\framesubtitle{}

\begin{enumerate}

\item Introduction
\item Vacuum polarization in presence of a Kondo-type delta potential
\item Generalization of Wentzell boundary condition for massless fermions
\item Conclusion

\end{enumerate}

\end{frame}
%%%%%%%%%%%%%%%%%%%
\begin{frame}
\frametitle{Introduction}
\framesubtitle{}

\begin{itemize}
\item<1-> Well-posedness of the system under the matching and boundary conditions: unitary time evolution and causal propagation
\item<2-> Vacuum polarization 
	\begin{itemize}
		\item Define the expectation of vacuum current by point-splitting prescription as a Hadamard state
		\item Renormalization by subtraction the global singularity of Hadamard states
	\end{itemize}
\end{itemize}

\end{frame}
%%%%%%%%%%%%%%%%%%
\begin{frame}[shrink=30]
\frametitle{\small{Vacuum polarization in presence of a Kondo-type delta potential}}
\framesubtitle{}
\begin{itemize}
%
\item<1-> 
The equation of motion for a massless spin-$\frac 1 2$ fermion in presence of a Kondo potential \\
\tiny\color{blue}[J. Erdmenger, C. Hoyos, A. O?Bannon and J. Wu, JHEP
, 12:086, 2013]\color{black}\normalsize

\begin{equation*}
i \partial_0 \phi = 
\underbrace{\begin{pmatrix} 
-1 & 0 \\
0 & 1 
\end{pmatrix} i \partial_1 \phi }_{H\phi}+
\begin{pmatrix}
v_3 & v_- \\
v_+ & -v_3
\end{pmatrix} \delta(x^1) \phi
\end{equation*}
where $\phi = \gamma^0\psi$ for some Dirac field $\psi$,  $v_3, v_+ + v_- \in \mathbb{R}$ and $ v_+ - v_-\in i \mathbb{R}$
%
\item<2-> Matching condition (m.c.) at $x^1 = 0$ 
%
\item<3-> Bag boundary condition (b.d.c.) for confined cases 
\begin{equation*}
- i\gamma^1 \phi \Big\vert_{\pm \frac{L}{2}} = \pm \phi \Big\vert_{\pm \frac{L}{2}}
\end{equation*}

%
\item<4-> In the confined case, the problem $i\partial_0\phi = H\phi$ is well-posed since $H$ is an essentially self-adjoint operator of the Hilbert space
\begin{equation*}
\mathcal{H} = L^{2}\Big(\big[-\frac L 2, 0\big), \mathbb{C}^2\Big) \oplus L^{2}\Big(\big(0,\frac L 2\big], \mathbb{C}^2\Big) 
\quad,\quad
\langle \cdot, \cdot\rangle_{\mathcal{H} } = \langle \cdot, \cdot\rangle_{L^{2}\big(\big[-\frac L 2, 0\big), \mathbb{C}^2\big)} +\langle \cdot, \cdot\rangle_{L^{2}\big(\big(0,\frac L 2\big], \mathbb{C}^2\big)}
\end{equation*}

 on the domain
\begin{equation*}
\mathrm{Dom}(H) = \Big \{\phi \enskip \big\vert \enskip \phi \in W^{1,2}(I_-, \mathbb{C}^2) \oplus W^{1,2}(I_+, \mathbb{C}^2), \enskip \phi \textrm{ b.d.c + m.c.} \Big \}
\end{equation*} 


\end{itemize}
\end{frame}
%%%%%%%%%%%%%%%%%%
\begin{frame}[shrink=30]
\frametitle{\small{Vacuum polarization in presence of a Kondo-type delta potential}}
\framesubtitle{Definition of vacuum current and stress-energy tensor}

\begin{itemize}
\item<1-> Define the Hadamard state
\\\tiny\color{blue}[J. Schlemmer and J. Zahn, Ann. of Phys. 2015]\color{black}\normalsize
\begin{equation*}
\begin{split}
\omega(\psi^B(x)\bar{\psi}_A(y)) = & \int_{E_k >0} \psi_k^B(x)\bar{\psi}_{A,k}(y)e^{-i(x^0-y^0)E_k} \dd k \\
\omega(\bar{\psi}_A(y)\psi^B(x)) = & \int_{E_k <0} \psi_k^B(x)\bar{\psi}_{A,k}(y)e^{-i(x^0-y^0)E_k} \dd k 
\end{split}
\end{equation*}
%
\item<2-> Renormalized vacuum current expectation
\begin{equation*}
j^\mu(x) = \lim_{y \rightarrow x} \gamma^A_B \big(
\omega(\psi^B(x)\bar{\psi}_A(y)) - H^B_A (x, y)\big)
\end{equation*}

%
\item<3-> Renormalized vacuum stress-energy tensor expectation

\begin{equation*}
\begin{split}
& T_{00} = \frac{i}{2} (\bar{\psi} \gamma_1 \nabla_1 \psi - \nabla_1 \bar{\psi}\gamma_1 \psi) , \quad
 T_{11} = \frac{i}{2} (\bar{\psi} \gamma_0 \nabla_0 \psi - \nabla_0 \bar{\psi}\gamma_0 \psi)  \\
& T_{01} = \frac{i}{4} (\bar{\psi} \gamma_1 \nabla_0 \psi +\bar{\psi} \gamma_0 \nabla_1 \psi - \nabla_1 \bar{\psi}\gamma_0 \psi - \nabla_0 \bar{\psi}\gamma_1 \psi)  
\end{split}
\end{equation*}

\begin{equation*}
\begin{split}
T_{00}(x,y) = 
\frac{i}{2}\Big( & 
\nabla_{x^1}\big(\omega(  \psi^B(x) \bar{\psi}_A(y))-H^B_A(x,y)\big)(\gamma_1)^A_B \\
& - \nabla_{y^1}\Big(\omega( \psi^B(x)  \bar{\psi}_A(y)) - H^B_A(x,y)\big)(\gamma_1)^A_B \Big)   
\end{split}
\end{equation*}

\end{itemize}
\end{frame}
%%%%%%%%%%%%%%%%%%%%%%%%%%%%%%%%%%%%%%

\begin{frame}[shrink=30]
\frametitle{\small{Vacuum polarization in presence of a Kondo-type delta potential}}
\framesubtitle{}

\begin{itemize}

\item<1-> Confined and with constant external electric field $E$ with static gauge for the vector potential $A^\mu(x) = (Ex^0, 0)$

\begin{itemize}
\item<2-> Eigenvalues of $H_E = H + eEx^1 $ (modes)
\begin{equation*}
k_{n} = \frac{(-1)^n}{L}  \theta + \frac{2\pi}{L}n 
\end{equation*}
where $\theta = \theta(E, v_3, v_+, v_-)$
%
\item<3-> Vacuum polarization and stress-energy tensor
\begin{equation*}
\rho(x) = \frac{e}{\pi L}\Big( \frac{\beta \sin \theta \cos \eta}{\alpha + \beta \sin \eta \cos \theta}\Big) (-\theta + \pi)
\Big(\Theta(- x^1) - \Theta(x^1))\Big) + \frac{e^2 E}{\pi} x^1
\end{equation*}
%
\begin{equation*}
T_{\mu\nu}(x) = 
\bigg( \frac{\pi}{2L^2}\big( -\frac{1}{3} + \frac{(\theta - \pi)^2}{\pi^2}\big) + \frac{e^2E^2(x^1)^2}{2 \pi} \bigg)
\begin{pmatrix}
1 & 0 \\ 0 & 1
\end{pmatrix}
\end{equation*}
%
\item<3-> When $v_3 = v_\pm = E = 0$, we obtain the same result for the stress-energy tensor for Casimir effect \\
\tiny\color{blue}[P. Sundberg and R. L. Jaffe, Annals Phys., 309:442?458, 2004]\normalsize\color{black}
\end{itemize}
%
\item Non-confined case
\begin{itemize}
\item<4-> Wave functions are normalized to $\delta$-function.
\item<5-> Same results at $L\rightarrow +\infty $ limit.
\end{itemize}

\end{itemize}

\end{frame}
%%%%%%%%%%%%%%%%%%
\begin{frame}[shrink=30]
\frametitle{\small{Generalization of Wentzell boundary condition for massless fermions}}
\framesubtitle{}

$\mathcal{M}$ a d+1 dimensional spin manifold (Minkowski) with time-like static boundary $\partial\mathcal{M}$

\begin{itemize}
\item<2-> We consider the action

\begin{equation*}
\mathcal{S} = \frac{1}{2}i\int_{\mathcal{M}} \bar{\psi} \gamma^\mu \partial_\mu \psi - \partial_\mu \bar{\psi} \gamma^\mu \psi 
+ \frac{1}{2}\int_{\partial \mathcal{M}} ic \bar{\psi} \gamma^\alpha \partial_\alpha (1 - i \gamma^\bot) \psi
+ \bar{\psi} \psi
\end{equation*}
%motivated by holographic renormalization \\
where $\gamma^\bot = n_j\gamma^j$ with $n$ unit inward vector normal to the boundary.\\
\tiny\color{blue}[M. Henningson and K. Sfetsos,
Phys. Lett. B, 431(1):63-68, 1998
] \color{black}\normalsize
\tiny\color{blue} [R. Contino and A. Pomarol JHEP, 11:058, 2004]
\color{black}\normalsize
%
%
\item<3-> EOM in terms of $\phi = \gamma^0 \psi$
\begin{equation*}
\begin{cases}
i \partial_0 \phi = i \gamma^0 \gamma^j \partial_j \phi   \quad \textrm{in $\mathcal{M}$}\\
i \partial_0(1 + i\gamma^\bot) \phi = i\gamma^0 \gamma^a \partial_a (1+ i\gamma^\bot)\phi - c^{-1} \gamma^0(1 - i \gamma^{\bot})\phi \quad \textrm{on $\partial \mathcal{M}$}
\end{cases}
\end{equation*}
%
\item<4-> Bag boundary condition when $c\rightarrow0$ 
%
\item<5-> On the Hilbert space 
\begin{equation*}
\mathcal{H} = L^{2}(M,E)\oplus L^{2}(\partial M, \mathcal{P}_+ F) \quad,\quad
\langle \cdot, \cdot \rangle _\mathcal{H} = \langle \cdot, \cdot \rangle _{L^2(M)} + c \langle \cdot, \cdot \rangle _{L^2(\partial M)}
\end{equation*}
 where $M$ is an equal-time hypersurface of $\mathcal{M}$ and $\partial M = M\cap \partial \mathcal{M}$, we define


\begin{equation*}
\Delta = \begin{pmatrix}
i \gamma^0 \slashed{\partial}  & 0 \\
-c^{-1} \gamma^0 \mathcal{P}_- \cdot \vert_{\partial M}&  i\gamma^0 \slashed{\partial}_| \mathcal{P}_+
\end{pmatrix}
\end{equation*}
where $\slashed{\partial} = \gamma^j\partial_j$ for
$j \in \llbracket 1 , d \rrbracket$, $\slashed{\partial}_| = h^{ab} \gamma_{a} \partial_{a}$ where $h$ is the induced metric on the boundary and $
\mathcal{P}_\pm = \frac{1}{2}(1 \pm i n_j\gamma^j) $
is a Hermitian projector.

\end{itemize}

\end{frame}
%%%%%%%%%%%%%%%%%%
\begin{frame}[shrink=30]
\frametitle{\small{Generalization of Wentzell boundary condition for massless fermions}}
\framesubtitle{}

\begin{itemize}

\item<1-> $\Delta$ is self-adjoint on the domain

\begin{equation*}
\dom( \Delta) = \{ \Phi = (\phi, \phi_|) \in W^{1,2}(M)\times W^{1,2}(
\partial M) \enskip | \enskip \mathcal{P}_+\phi \vert_{\partial M} - \phi_| = 0 \} 
\end{equation*}  

$\Rightarrow$ unitary time evolution.
\newline

\item<2-> \textbf{Causal propagation}\\
We denote
\begin{equation*}
\mathcal{H}^k(M) = \cap_{s=0}^{k} \dom(\Delta), 
\quad
\| \cdot \|_{\mathcal{H}^k(M)} = \| \Delta^k \cdot \|_{\mathcal{H}(M)}
\end{equation*}

\begin{proposition}
Let $\Sigma_0$ and $\Sigma_1$ two equal-time surfaces.
Let $k \in \mathbb{N}^*$.
Suppose that the initial data $\Phi_0 = (\phi, \phi_|)\enskip\slashed{\in} \ker \Delta$.
%With the same notations and the same conditions as in \cref{wen-propcau}, 
The smooth solution $\Phi$ depends continuously on the initial data $(\phi, \phi_|)$ on a Cauchy surface $\mathcal{S}_0 \subset \Sigma_0$ in the sense that
\begin{equation*}
\big\| \frac{\partial^k}{\partial t^k} \Phi\big\|_{\mathcal{H}(\mathcal{S}_1)}
\leq
\big\| \Phi\big\|_{\mathcal{H}^{k}(\mathcal{S}_0)}
\end{equation*}
for any $\mathcal{S}_1 \in D^+(\mathcal{S}_0)\cap\Sigma_1$
where $D^+(\mathcal{S}_0)$ is the future domain of dependence of $\mathcal{S}_0$.
\end{proposition}

\item<3-> Two examples 

	\begin{itemize}
		\item<4-> $M = \mathbb{R}^{d-1}\times \mathbb{R}_+$ the spectrum is $\mathbb{R}$
		\item<5-> $M = \mathbb{R}^{d-1}\times [-L, L]$ the spectrum is
		\begin{equation*}
		\{k\in\mathbb{R} \enskip\vert\enskip k^2 = p^2 + q^2 \enskip\mathrm{ where }\enskip p\in\mathbb{R}^{d-1}, \tan q L = \frac{\mp q + c^{-1}}{q \pm c^{-1}}\}
		\end{equation*}
		$\Rightarrow$ Causal propagation in $d=1$
	\end{itemize}
\end{itemize}

\end{frame}
%%%%%%%%%%%%%%%%%%
\begin{frame}[shrink=30]
\frametitle{Conclusion}
\framesubtitle{}

\begin{itemize}
\item To summarize
\end{itemize}
\end{frame}
\end{document}

















