%\PassOptionToPackage{english}{babel}
\documentclass[french]{beamer}
\usetheme{Warsaw}
\usepackage{lmodern}
\usepackage[utf8]{inputenc}
\usepackage[T1]{fontenc}
\usepackage{babel}
\usepackage{color}
\usepackage{listings}
\usepackage{booktabs}
\usepackage{amsmath,amsthm,amssymb,graphicx,mathrsfs}
%\usepackage[hmargin={3.5cm,3.5cm},top=4cm,bottom=4cm]{geometry} %problem...

\newcommand{\vb}[1]{\mathbf{#1}}
\newcommand{\V}[1]{\textnormal{#1}}
\newcommand{\dd}[0]{\textrm{d}}
\newcommand{\bra}[1]{\langle#1 |}
\newcommand{\ket}[1]{| #1 \rangle}
\newcommand{\vdot}[0]{\boldsymbol\cdot}

\setbeamertemplate{footline}[frame number]{}
%\setbeamertemplate{navigation symbols}{}
\setbeamertemplate{caption}[numbered]


%%%%%%%%%%%%%%%%%%%%%%%%%%%%
\lstset{%
  backgroundcolor=\color{white},   % choose the background color; you must add \usepackage{color} or \usepackage{xcolor}
  basicstyle=\small,        % the size of the fonts that are used for the code
  commentstyle=\color{blue},    % comment style
  extendedchars=true,              % lets you use non-ASCII characters; for 8-bits encodings only, does not work with UTF-8
  keepspaces=true,                 % keeps spaces in text, useful for keeping indentation of code (possibly needs columns=flexible)
  keywordstyle=\color{red},       % keyword style
  language=Fortran,                 % the language of the code
  rulecolor=\color{black},         % if not set, the frame-color may be changed on line-breaks within not-black text (e.g. comments (green here))
  showspaces=false,                % show spaces everywhere adding particular underscores; it overrides 'showstringspaces'
  showstringspaces=false,          % underline spaces within strings only
  showtabs=false,                  % show tabs within strings adding particular underscores
  stringstyle=\color{green},     % string literal style
  tabsize=2,                       % sets default tabsize to 2 spaces
  frame=single
}



%%%%%%%%%%%%%%%%%%%%%%%%%%%%%%%%%%%%%%%%%%%%%%%%%%%%%%%%%%%%%%%%%%%%%%%%
\title{Quantum field theories in external potentials}
\author{En-Hung CHAO}
\institute{Ecole Polytechnique}
\date{\today}
%%%%%%%%%%%%%%%%%%%%%%%%%%%%%%%%%%%%%%%%%%%%%%%%%%%%%%%%%%%%%%%%%%%%%%%%
%%%%%%%%%%%%%%%%%%%%%%%%%%%%%%%%%%%%%%%%%%%%%%%%%%%%%%%%%%%%%%%%%%%%%%%%
\begin{document}
%\selectlanguage{english}
%\graphicspath{{../images/}}

\begin{frame}
\titlepage%
\end{frame}
%%%%%%%%%%%%%%%%%%%%%%%%%%%%%%%%%%%%%%%
%%%%%%%%%%%%%%%%%%%%%%%%%%%%%%%%%
\begin{frame}
\frametitle{Outline}
\framesubtitle{}
\begin{enumerate}
 \item Introduction 
  \begin{enumerate}
  \item Elements of Quantum Field Theory in Curved Space-Time (QFTCST)
  \item Kondo effect
  \item Brief review of AdS/CFT correspondence and holographic renormalization
  \end{enumerate}
 \item Vacuum polarization in 1+1 dimension in presence of a Kondo type potential
  \begin{enumerate}
  \item Hadamard parametrix and vacuum polarization
  \item Results
  \end{enumerate}
 \item Wentzell boundary condition for Dirac fields
  \begin{enumerate}
  \item Well-posedness of an action
  \item 3 examples
  \item Vacuum polarization and stress-energy tensors in 1+1 dimensional
  \end{enumerate}
 \item Conclusion and prospective
\end{enumerate}
\end{frame}
%%%%%%%%%%%%%%%%%%%%%%%%%%%%%%%%%%%%%%%
\begin{frame}
\frametitle{Introduction}
\framesubtitle{Elements of QFTCST}

\begin{itemize}
\item Difficulty in defining the ground state in curved space-time for usual construction of Hilbert space
\item Algebraic approach formulated with an algebra of quantum observables $\mathscr{A}(M,g)$
\item Distributional nature of fields
\item Postulates for scalar field theory
 \begin{enumerate}
  \item \textbf{Linearity} $\phi(c_1 f_1 + c_2 f_2) = c_1 \phi(f_1) + c_2 \phi(f_2)$ for $c_1, c_2 \in \mathbb{C}$
%
\item \textbf{Klein-Gordon equation} $\phi\big( (\Box_g - m^2)f \big) = 0$
%
\item \textbf{Hermitian field} $\phi(f)^* = \phi(\bar{f})$
%
\item \textbf{Canonical commutation relation (CCR)} $[\phi(f_1), \phi(f_2)] = iE(f_1, f_2) \mathbf{1}$
 \end{enumerate}
\end{itemize}

\end{frame}
%%%%%%%%%%%%%%%%%%%%%%%%%%%%%%%%%%%%%%
\begin{frame}
\frametitle{Introduction}
\framesubtitle{Elements of QFTCST}
\begin{itemize}

\item \textbf{Physical state} $\omega: \mathscr{A}(M,g) \rightarrow \mathbb{C}$ with $\omega(\mathbf{1}) = 1$ and the positivity $\omega(a^*a) \geq 0$

\item \textbf{GNS construction} corresponding to the state 

\item Equivalent theory for Dirac field : Dirac equation and \textit{anti}-commutation relation
\begin{equation*}
\{\psi(f_1), \psi(f_2)\} = i S(f_1, f_2) \mathbf{1}
\end{equation*}


\end{itemize}

\end{frame}
%%%%%%%%%%%%%%%%%%%%%%%%%%%%%%%%%%%%%%
\begin{frame}
\frametitle{Introduction}
\framesubtitle{Elements of QFTCST}

\begin{itemize}

\item \textbf{Hadamard states} and characterization
 \begin{equation*}
\begin{split}
\omega(\psi^B(x)\bar{\psi}_A(x')) + \omega(\bar{\psi}_A(x')\psi^B(x)) = &
iS^B_A(x,x') \\
\overline{\omega(\bar{\psi}(u)\psi(\bar{v}))} = & \omega(\bar{\psi}(v)\psi(\bar{u}))
\end{split}
\end{equation*}

\item Dirac's method and vacuum current as a Hadamard state

\begin{equation*}
\begin{split}
\omega(\psi^B(x)\bar{\psi}_A(y)) = & \int_{E_k >0} \psi_k^B(x)\bar{\psi}_{A,k}(y)e^{-i(x^0-y^0)E_k} \dd k \\
\omega(\bar{\psi}_A(y)\psi^B(x)) = & \int_{E_k <0} \psi_k^B(x)\bar{\psi}_{A,k}(y)e^{-i(x^0-y^0)E_k} \dd k 
\end{split}
\end{equation*}

\begin{equation*}
j^\mu(x) = \lim_{y \rightarrow x} \gamma^A_B \big(
\omega(\psi^B(x)\bar{\psi}_A(y)) - H^B_A (x, y)\big)
\end{equation*}


\end{itemize}

\end{frame}
%%%%%%%%%%%%%%%%%%%%%%%%%%%%%%%%%%%%%%
\begin{frame}
\frametitle{Kondo effect}
\framesubtitle{}
\begin{itemize}
\item Anomalies in electrical resistance when temperature decreases

\end{itemize}

\end{frame}
%%%%%%%%%%%%%%%%%%%%%%%%%%%%%%%%%%%%%%
\begin{frame}
\frametitle{Kondo effect}
\framesubtitle{}
\begin{itemize}
\item Punctual spin interaction between electrons and impurities
\item Hamiltonian
\begin{equation*}
H_K = \psi_\alpha^\dagger \frac{-\nabla^2}{2m}\psi_\alpha +
\frac 1 2\lambda_K \delta(\vec{x})\vec{S}\cdot \psi_{\alpha'}^\dagger  \vec{\sigma}_{\alpha' \alpha} \psi_\alpha
\end{equation*}

\item How could such a punctual interaction change the vacuum polarization ?
\end{itemize}

\end{frame}
%%%%%%%%%%%%%%%%%%%%%%%%%%%%%%%%%%%%%%
\begin{frame}
\frametitle{Introduction}
\framesubtitle{Holographic renormalization}

\begin{itemize}
\item Strong/weak coupling duality and IR-UV connection in anti-deSitter/Conformal Field Theory (AdS/CFT) correspondence

\item Example for scalar field

\begin{equation*}
\begin{split}
\mathcal{S}[\phi] = & -\int_{B_{d+1}} \sqrt{g} \phi D_i D^i \phi + 
\lim_{\epsilon\rightarrow 0}\int_{T_\epsilon}  \sqrt{h} \phi (\vec{n}\cdot\vec{\nabla})\phi \\
%
& \sim \int \dd \mathbf{x} \dd \mathbf{x}' 
\frac{\phi_0(\mathbf{x})\phi_0(\mathbf{x}')}{|\mathbf{x} - \mathbf{x}'|^{2d}}
\end{split}
\end{equation*}

$\Rightarrow$ the same divergence as in CFT

\end{itemize}

\end{frame}
%%%%%%%%%%%%%%%%%%%%%%%%%%%%%%%%%%%%%%
\begin{frame}
\frametitle{Introduction}
\framesubtitle{Holographic renormalization}

\begin{itemize}

\item Boundary field and holographic renormalization

\item Type of action appearing in holographic renormalization for scalar case
\begin{equation*}
\begin{split}
\mathcal{S} = & \mathcal{S}_{\mathrm{bulk}} + \mathcal{S}_{\mathrm{bdy}}
\\ = &
-\frac 1 2 \int_M g^{\mu\nu} \partial_\mu \phi \partial_{\nu} + 
\mu^2\phi^2 - \frac c 2 \int_{\partial M}h^{\mu\nu}\partial_\mu\phi\partial_\nu\phi + \mu^2\phi^2
\end{split}
\end{equation*}

\item The case of Dirac fields and the vacuum polarization under the new boundary condition?

\end{itemize}

\end{frame}
%%%%%%%%%%%%%%%%%%%%%%%%%%%%%%%%%%%%%%
\begin{frame}[shrink=20]
\frametitle{Vacuum polarization in 1+1 dimension in presence of a Kondo type potential}
\framesubtitle{Hadamard parametrix and vacuum polarization}

\begin{itemize}
\item Representation of the Dirac gamma matrices
\begin{equation*}
\gamma^0 = \begin{pmatrix}
0 & 1 \\
1 & 0 \end{pmatrix}  \quad  \gamma^1 = \begin{pmatrix}
0  & 1 \\
-1 & 0
\end{pmatrix}
\end{equation*}

\item Hadamard parametrix for a spin-$\frac 1 2$ massless particle with charge $+e$ with external potential $A^\mu(x) = (Ex^1, 0 )$
\begin{equation*}
\begin{split}
& H^\pm (x, y)^1_2 = \frac{-i}{2\pi}\frac{1-\frac i 2 e E(x^1 + y^1)(x^0-y^0) 
- \frac 1 8 (eE)^2(x^1 + y^1)^2(x^0 - y^0)^2}{x^0 - y^0 + x^1 - y^1 \mp i \epsilon}  \\
& H^\pm (x, y)^2_1 = \frac{-i}{2\pi}\frac{1-\frac i 2 e E(x^1 + y^1)(x^0-y^0) 
- \frac 1 8 (eE)^2(x^1 + y^1)^2(x^0 - y^0)^2}{x^0 - y^0 - x^1 + y^1 \mp i \epsilon}
\end{split}
\end{equation*}

\end{itemize}

\end{frame}
%%%%%%%%%%%%%%%%%%%%%%%%%%%%%%%%%%%%%%
\begin{frame}[shrink=10]
\frametitle{Vacuum polarization in 1+1 dimension in presence of a Kondo type potential}
\framesubtitle{Hadamard parametrix and vacuum polarization}

We start with $E = 0$ and a confined space $x^1 \in [\-\frac L 2 , \frac L 2]$
\begin{itemize}
\item We denote $\phi = \gamma^0\psi$ where $\psi$ is a massless Dirac field satisfying 
\begin{equation*}
i\gamma^\mu\partial_\mu \psi = 0
\end{equation*}
outside the singularity
\item Imposing a Kondo type potential due to the impurity at $x^1 = 0$,
we consider $\phi$ satisfying

\begin{equation*}
i \partial_0 \phi = 
\begin{pmatrix} 
-1 & 0 \\
0 & 1 
\end{pmatrix} i \partial_1 \phi +
\begin{pmatrix}
v_3 & v_- \\
v_+ & -v_3
\end{pmatrix} \delta(x_1) \phi
\end{equation*}
where $v_3, v_+ + v_- \in \mathbb{R}$ and $ v_+ - v_-\in i \mathbb{R}$
\end{itemize}

\end{frame}
%%%%%%%%%%%%%%%%%%%%%%%%%%%%%%%%%%%%%%
\begin{frame}
\frametitle{Vacuum polarization in 1+1 dimension in presence of a Kondo type potential}
\framesubtitle{Hadamard parametrix and vacuum polarization}

\begin{itemize}
\item Matching condition at $x^1 = 0$ for $\phi =
\begin{pmatrix}
\phi_L \\
\phi_R
\end{pmatrix}$

\begin{equation*}
\begin{pmatrix}
\phi_L(0^+) \\
\phi_R(0^+)
\end{pmatrix} = \begin{pmatrix}
\frac{A}{D}  & \frac{C}{D} \\
\frac{C^*}{D} & \frac{A}{D}
\end{pmatrix}\begin{pmatrix}
\phi_L(0^-) \\
\phi_R(0^-)
\end{pmatrix}
\end{equation*}
where  $\Sigma = v_+ ^ 2 + v_- ^ 2 + v_3 ^ 2$, $A = 1+ \frac{1}{4}\Sigma$, $C = -iv_-$, $D = 1-\frac{1}{4}\Sigma + iv_3$.

\item Bag boundary condition (in terms of $\psi = \gamma^0\phi$)
\begin{equation*}
i\gamma^1 \psi \Big\vert_{\pm \frac{L}{2}} = \pm \psi \Big\vert_{\pm \frac{L}{2}}
\end{equation*}

\item The problem is well posed because the Hamiltonian of the dynamical system is self-adjoint under these conditions

\end{itemize}

\end{frame}
%%%%%%%%%%%%%%%%%%%%%%%%%%%%%%%%%%%%%%
\begin{frame}
\frametitle{Vacuum polarization in 1+1 dimension in presence of a Kondo type potential}
\framesubtitle{Hadamard parametrix and vacuum polarization}


\end{frame}
%%%%%%%%%%%%%%%%%%%%%%%%%%%%%%%%%%%%%%
\begin{frame}
\frametitle{}
\framesubtitle{}


\end{frame}
%%%%%%%%%%%%%%%%%%%%%%%%%%%%%%%%%%%%%%
\begin{frame}
\frametitle{}
\framesubtitle{}


\end{frame}
%%%%%%%%%%%%%%%%%%%%%%%%%%%%%%%%%%%%%%
\begin{frame}
\frametitle{}
\framesubtitle{}


\end{frame}
%%%%%%%%%%%%%%%%%%%%%%%%%%%%%%%%%%%%%%
\begin{frame}
\frametitle{}
\framesubtitle{}


\end{frame}
%%%%%%%%%%%%%%%%%%%%%%%%%%%%%%%%%%%%%%d
\end{document}
















