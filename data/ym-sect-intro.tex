\section{Motivation}
It is curious to know if the system studied in~\cref{chap-wentzell} can be coupled with a gauge field by Yang-Mills theory. 
We start by investigate the equations of motion (field part) derived from the following action
\begin{equation*}
\mathcal{S}_{YM} = -\frac 1 4 \int_M F^{\mu\nu} F_{\mu\nu} 
-\int_{\partial M} \frac 1 4 c\ \bar{F}^{\alpha\beta} \bar{F}_{\alpha\beta} - \bar{A}_\alpha \grad^\alpha A^{\bot} 
\end{equation*}
As suggested in the litterature, 
%TODO +ref
we choose to work with the temporal gauge ($A_0 = 0$). 
We get the following system of linearized equations by applying variational method to the action
\begin{equation}
\begin{split}
& \partial_0 A = E, \quad \partial_0 E = \delta A_{df} \\
& \partial_0 \bar{A} = \bar{E} , \quad \partial \bar{E} = \delta \bar{A}_{df} - c^{-1}\partial_\bot A\vert_{\partial M}
\end{split}
\end{equation}
where the Helmholtz-Hodge decomposition is used and $df$ means the divergence-free part.
%TODO add reference and more precisions.
