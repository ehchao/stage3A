\section{Introduction}
The AdS (anti-deSitter)/CFT (conformal field theory) correspondance has been a very active field of research in theoretical physics these years. 
This principle is also called holography.
%add ref??
%Maldacena's paper?
The main idea of the theory is to study the duality between the string theory in an AdS space-time bulk and the quantum field theory living on the boundary of the AdS space-time.
Another interesting application of the AdS/CFT correspondance is to study the strong/weak coupling duality, \ie
a strong coupling of the quantum field theory on the boundary corresponds to a weak coupling of the string theory in the bulk. 
For instance,~\cite{Skenderis2002} gives examples to calculate renormalized correlation functions\footnote{
See~\eg\cite{Peskin1995} for the definition.
} of the quantum field theory by performing computations on the gravity side.
The strong couplings in QCD might be better understood in using the holography principal. 
Based on the observation of the IR-UV connection~\cite{Susskind1998}, 
\ie the fact that the ultraviolet divergence of the boundary theory corresponds to the infrared divergence of the bulk theory, 
we can renormalize a theory by undergoing the holographic renormalization procedure~\cite{Skenderis2002}. \\\\
%
However, one can wonder how the boundary field should be built in order to ensure that the two-point function constructed on the boundary has the behaviors that correspond to the bulk field. 
\cite{Zahn2016} proposes thus an additional boundary action for a scalar field, namely, by considering the following action
\begin{equation*}
\mathcal{S} = \mathcal{S}_{\mathrm{bulk}} + \mathcal{S}_{\mathrm{bdy}} = 
-\frac 1 2 \int_M g^{\mu\nu} \partial_\mu \phi \partial_{\nu} + 
\mu^2\phi^2 - \frac c 2 \int_{\partial M}h^{\mu\nu}\partial_\mu\phi\partial_\nu\phi + \mu^2\phi^2
\end{equation*}
and studies the time evolution and the quantization of the field.
Indeed, the boundary term appears as a counter-term in the holographic renormalization theory~\cite{Skenderis2002}.
In this report, we will try to extend this kind of studies to the Dirac field case by proposing an action and studying the dynamics of the Dirac field under the induced boundary condition.
Some methods of functional analysis are used and we can refer to~\cite{Reed1981} and~\cite{Reed1975} for these tools. \\\\
%
Before showing the main results, 
we give a brief description of the AdS/CFT correspondance by using the example of a massless scalar field from~\cite{Witten1998} and~\cite{Skenderis2002}. 
%
\paragraph{AdS/CFT correspondance in scalar field case}
The metric of the $(d+1)$-AdS space-time can be identified as the open unit ball $B_{d+1}$ in a Euclidean space $\mathbb{R}^{d+1}$ with coordinates $y_0, \ldots, y_d$ such that $\sum_{i=0}^d y_i^2 <1$ with the metric
\begin{equation*}
ds^2 = \frac{4\sum_{i=0}^d dy_i^2}{(1 - |y|^2)^2}
\end{equation*}
We can include the infinity boundary by taking the closure of the unit ball $B_{d+1}$. 
As one can notice, the singularity of the metric at infinity ($|y|^2 = 1$) can be resolved by studying the conformal transformation of the metric, 
\ie by replacing $ds^2$ by $d\tilde{s}^2$ which is defined by
\begin{equation*}
d\tilde{s}^2 = (1 - |y|^2) ds^2
\end{equation*}
We consider a field theory with the action of a massless scalar field $\phi$
\begin{equation}\label{wen-adscft1}
\mathcal{S}[\phi] = \frac 1 2 \int_{B_{d+1}} \dd^{d+1} y \sqrt{g} |\dd \phi|^2
\end{equation}
where $g$ is the absolute value of the determinant of the metric tensor.
By doing integration by part to~\cref{wen-adscft1}, we find
\begin{equation*}
\mathcal{S}[\phi] = -\int_{B_{d+1}} \sqrt{g} \phi D_i D^i \phi + 
\lim_{\epsilon\rightarrow 0}\int_{T_\epsilon}  \sqrt{h} \phi (\vec{n}\cdot\vec{\nabla})\phi
\end{equation*}
where $T_\epsilon$ is a family of surfaces parametrized by $\epsilon$ and converges to the boundary of $B_{d+1}$ when $\epsilon\rightarrow 0$ and $h$ is the absolute value of the determinant of the induced metric on the boundary.
The first term of the \rhs vanishes on-shell and the action can be written in terms of the boundary field $\phi_0$
\begin{equation*}
\mathcal{S}[\phi] \sim \int \dd \mathbf{x} \dd \mathbf{x}' 
\frac{\phi_0(\mathbf{x})\phi_0(\mathbf{x}')}{|\mathbf{x} - \mathbf{x}'|^{2d}}
\end{equation*}
Consider a field $\mathcal{O}$ whose source is the boundary field $\phi$. 
It is defined by~\cite{Gubser1998}
\begin{equation*}
\mathcal{S}[\phi] = \int \dd \mathbf{x} \phi_0(\mathbf{x})\mathcal{O}(\mathbf{x})
\end{equation*}
It could then be shown by the prescription of the section 2 of~\cite{Skenderis2002} that the two-point function of the operator $\mathcal{O}$ is a multiple of $|\mathbf{x} - \mathbf{x}'|^{-2d}$, which agrees with the two-point function of the $d$-dimensional conformal field theory~\cite{Qualls2015}. \\\\
%
Among recent works on the AdS/CFT correspondance of Dirac fields, we can find different propositions for boundary action.
For instance, 
~\cite{Henningson1998} proposes to put a boundary action which involves the product of the bulk field $\psi$ and its Dirac adjoint $\bar{\psi}$. 
This term in $\bar{\psi}\psi$ looks like a mass term living on the boundary.
On the other hand,~\cite{Contino2005} suggests to construct a boundary action such that a chiral component of the bulk field vanishes on the boundary.
This boundary condition is also known as bag boundary condition, proposed in~\cite{Chodos1974}\footnote{
A direct consequence of the bag boundary condition is that the component normal to the boundary of the density of current in the bulk will vanish. 
This kind of boundary condition is used to study the confinement of quarks~\cite{Hasenfratz1978}.
}.
Under the setting of~\cite{Contino2005}, one can introduce terms that only contain the chiral component which is not imposed to vanish on the boundary.
Taking into account these two possibilities of boundary action construction, 
one can wonder if a more general action could be made. 
%
%
%
%%%%%%%%%%%%%%%%%%
%%%%%%%%%%%%%%%%
\section{The action of the problem}
Combining the boundary actions of~\cite{Henningson1998} and~\cite{Contino2005}, we propose to study the following action
\begin{equation}\label{wen-action}
\mathcal{S} = \frac{1}{2}i\int_{\mathcal{M}} \bar{\psi} \gamma^\mu \partial_\mu \psi - \partial_\mu \bar{\psi} \gamma^\mu \psi 
+ \frac{1}{2}\int_{\partial \mathcal{M}} ic \bar{\psi} \gamma^\alpha \partial_\alpha (1 - i \gamma^\bot) \psi
+ \bar{\psi} \psi
\end{equation}
for a certain constant $c >0$. 
$\gamma^\bot$ here represents the component perpendicular to $\partial \mathcal{M}$ in the incoming direction. 
As we will see later, the supplementary $\frac 1 2 (1-i\gamma^\bot)$ acts as a projector on the boundary field.
\\\\
When $c \rightarrow 0$, the boundary condition becomes the bag condition. 
Furthermore, we suppose $\partial(\partial \mathcal{M}) = \emptyset$.
By variational method, we can derive the equation of motion in the bulk and on the boundary
\begin{equation}\label{wen-motion}
\begin{cases}
i \gamma^\mu \partial_\mu \psi = 0  \quad \textrm{in $\mathcal{M}$}\\
i \gamma^\alpha \partial_\alpha (1 - i\gamma^\bot) \psi = - c^{-1}(1 + i\gamma^{\bot}) \psi \quad \textrm{on $\partial \mathcal{M}$}
\end{cases}
\end{equation}
We define $\phi = \gamma^0 \psi$. 
\cref{wen-motion} can be written as 
\begin{equation}\label{wen-maineq}
\begin{cases}
i \partial_0 \phi = i \gamma^0 \gamma^j \partial_j \phi   \quad \textrm{in $\mathcal{M}$}\\
i \partial_0(1 + i\gamma^\bot) \phi = i\gamma^0 \gamma^a \partial_a (1+ i\gamma^\bot)\phi - c^{-1} \gamma^0(1 - i \gamma^{\bot})\phi \quad \textrm{on $\partial \mathcal{M}$}
\end{cases}
\end{equation}
One can notice that the boundary condition implies constraints on only certain components of $\phi$. 
For instance, for $\dim \mathcal{M} = 3$, we can construct the following gamma matrices as suggested in~\cite{Polchinski1998}
\begin{equation*}
\gamma^0 = i\begin{pmatrix} 0 & 1 \\ -1 & 0 \end{pmatrix}  \quad
\gamma^1 = i\begin{pmatrix} 0 & 1 \\ 1 & 0 \end{pmatrix}  \quad
\gamma^2 = i\begin{pmatrix} 1 & 0 \\ 0 & -1 \end{pmatrix}  
\end{equation*}
Suppose that the inward normal vector of $\partial \mathcal{M}$ is $e_2$ at all point.
We have
\begin{equation*}
1 - i\gamma^\bot = 
\begin{pmatrix} 2 & 0 \\ 0 & 0\end{pmatrix} = 2 \mathcal{P}
\end{equation*}
where $\mathcal{P}$ is one of the chiral projectors on the boundary. 
More generally, for any space-like unit vector $n_j$,
\begin{equation*}
\mathcal{P}_\pm = \frac{1}{2}(1 \pm i n_j\gamma^j) 
\end{equation*}
are Hermitian projectors since 
\begin{equation*}
\mathcal{P}_\pm^\dagger = 
\frac{1}{2}(1 \mp i (n_j \gamma^j)^\dagger)=
\frac{1}{2}(1 \pm i n_j \gamma^j)
\end{equation*}
and
\begin{equation*}
(\mathcal{P}_\pm)^{2} = \frac{1}{4}(2\pm 2i \gamma^\bot) = \mathcal{P}_\pm
\end{equation*}
Furthermore, they have the same rank (equal to the half of the $n$-dimension of the representation space) since $\gamma^0$ is of maximal rank and
\begin{equation*}
\gamma^0\mathcal{P}_\pm = \mathcal{P}_\mp\gamma^0
\end{equation*}
In a more general way, 
we can rewrite~\cref{wen-maineq} as a time evolution problem with Cauchy data
$\Phi = \begin{pmatrix}
\phi, \phi_|
\end{pmatrix}$
\begin{equation*}
i\partial_0 \Phi = \Delta \Phi
\end{equation*}
where $\phi$ is the bulk field, $\phi_|$ is the boundary field and the operator $\Delta$ is defined as
\begin{equation}\label{wen-hamiltonian}
\Delta = \begin{pmatrix}
i \gamma^0 \slashed{\partial}  & 0 \\
-c^{-1} \gamma^0 \mathcal{P}_- \cdot \vert_{\partial M}&  i\gamma^0 \slashed{\partial}_|
\end{pmatrix}
\end{equation}
where $\slashed{\partial} = \gamma^j\partial_j$ for
$j \in \llbracket 1 , d \rrbracket$ and $\slashed{\partial}_| = h^{ab} \gamma_{a} \partial_{b}$ where $h$ is the induced metric on the boundary.
For simplicity, we will work on flat boundaries.
We choose a coordinate system where $n^\perp = n_d$.
Under this choice,
$\partial_|$ can be written in a more concise way
\begin{equation*}
\slashed \partial_| = \gamma^a\partial_a \quad\mathrm{for}\quad 
a \in \llbracket 1, \ldots, d-1 \rrbracket
\end{equation*}
We will take the above choice for the rest of the chapter. 

















