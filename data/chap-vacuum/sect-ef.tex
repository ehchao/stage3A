%%%%%%%%%%%%%%%%%%%%%%%%
%%%%%%%%%%%%%%%%%%%%%%%%%%%
\section{With static constant electric field}
We introduce now a static constant electric field whose electromagnetic potential components are 
\begin{equation*}
(A_0, A_1) = (E x^1, 0)
\end{equation*}
\cref{nef-Dirac} turns out to be
\begin{equation} 
i \partial \phi = 
\begin{pmatrix} 
-1 & 0 \\
0 & 1 
\end{pmatrix} i \partial \phi + eEx^1 \phi +
\begin{pmatrix}
v_3 & v_- \\
v_+ & -v_3
\end{pmatrix} \delta(x_1) \phi
\end{equation}
For eigenvalue $k$ of the new Hamiltonian, the eigenvector in the region $[-\frac{L}{2}, 0)$ can be written as 
\begin{equation}
\phi_k =\begin{pmatrix}
\phi_{k,L}  \\
\phi_{k,R}
\end{pmatrix}
\quad \textrm{where $\begin{split} 
 & \phi_{k,L} = f_k e^{-\frac{i}{2}eE(x^1)^2 + ikx^1} \\
& \phi_{k,R} = g_k e^{\frac{i}{2}eE(x^1)^2 - ikx^1}
\end{split}
$}
\end{equation}
for $f, g\in \mathbb{C}$. The matching condition at $x^1 = 0$ is the same as \cref{nef-matching2}. With the same boundary conditions, we can derive a relation similar to \cref{nef-boundCond}
\begin{equation}
\begin{cases}
g = fe^{i(kL + \frac{\pi}{2}) - i\frac{eE}{8}L^2}  \\
g = \frac{A + iC^* e^{-ikL + i\frac{eE}{8}L^2 }}{A - iC e^{ikL - i\frac{eE}{8}L^2}} fe^{ikL - i\frac{eE}{8}L^2 + i\frac{\pi}{2}}
\end{cases}
\end{equation}�
which implies $|f| = |g|$ and
\begin{equation}
kL = \textrm{Arg}(A - iC e^{ikL -i\frac{eE}{8}L^2}) + (n + \frac{1}{2}) \quad n\in \mathbb{Z}
\end{equation}
This equation can be solved in the same way which has been done for \cref{nef-boundCond1} by replacing $C$ by $Ce^{-i\frac{eE}{4}L^2}$. \\
Thus, by denoting $\eta = \textrm{Arg } C - \frac{eE}{8}L^2 $, an eigenvalue $k$ of the Hamiltonian can take the following values
\begin{equation}
k_{n} = \frac{(-1)^n}{L} \theta+ \frac{2\pi}{L}n 
\quad \textrm{where $\theta = \arccos\bigg( \frac{-|C| \sin \eta}{A} \bigg)$}
\end{equation}
The solution space is therefore spanned by
\begin{equation} 
\begin{split}
\phi_{k_{n}} = 
& \sqrt{\frac{1}{L(\alpha - \beta \sin (k_{n}L - \eta))}} \Bigg( 
\begin{pmatrix}
1 & 0 \\
0  & e^{-i(kL + \frac{\pi}{2}) + i\frac{eE}{8}L^2}
\end{pmatrix}
\Theta(-x^1) + \\
& \begin{pmatrix}
\frac{A}{D}  +  \frac{C}{D} e^{-i(kL + \frac{\pi}{2})+ i\frac{eE}{8}L^2} & 0 \\
0  & \frac{C^*}{D}  + \frac{A}{D}e^{-i(kL + \frac{\pi}{2} ) + i\frac{eE}{8}L^2}
\end{pmatrix}\Theta(x^1)\Bigg)
\begin{pmatrix}
e^{ik_{n} x^1 -  i\frac{eE}{2}(x^1)^2} \\
e^{- ik_{n} x^1 + i\frac{eE}{2}(x^1)^2}
\end{pmatrix}
\end{split}
\end{equation}
With $z = x^0 - y^0 - x^1 + y^1$, we define
\begin{equation}
\begin{split}
\xi_E(z, x^1, y^1) = & \Big( \frac{-i}{2 \pi z} - \frac{i \pi z}{12L^2} + \mathcal{O}(z^3) \Big)
\Big( \frac{1}{\alpha - \beta\sin(\theta - \eta)}e^{i\frac{(-\theta + \pi)z}{L} + \frac{ieE}{2}((x^1)^2 - (y^1)^2)} + \\
& \frac{1}{\alpha + \beta\sin(\theta + \eta)}e^{i\frac{(\theta - \pi) z}{L} + \frac{ieE}{2}((x^1)^2 - (y^1)^2)} \Big)  \\
= & \Big( \frac{-i}{2 \pi z} - \frac{i \pi z}{12L^2} + \mathcal{O}(z^3) \Big)
\Big( 1 + \frac{i(-\theta + \pi)\frac{z}{L}}{\alpha - \beta\sin(\theta - \eta)} + \frac{i(\theta - \pi)\frac{z}{L}}{\alpha + \beta\sin(\theta + \eta)}  \\
& + \frac{1}{2}\Big(\frac{(-\theta + \pi)^2}{\alpha - \beta \sin (\theta - \eta)}  
+ \frac{(\theta - \pi)^2}{\alpha + \beta \sin (\theta + \eta)} \Big)\frac{z^2}{L^2} \\
& + \frac{ieE}{2}(x^1 - y^1)(x^1 + y^1) - \frac{e^2 E^2}{8}(x^1 - y^1)^2 (x^1 + y^1)^2
+ \mathcal{O}(z^3)  \Big)  \\
= & \frac{-i}{2 \pi z} + \frac{1}{2\pi L}\Big( \frac{\beta \sin \theta \cos \eta}{\alpha + \beta \sin \eta \cos \theta}\Big) (-\theta + \pi) + \frac{eE}{4 \pi}\frac{(x^1 - y^1)(x^1 + y^1)}{z}     \\
& + \frac{i\pi}{4 L^2}\big( -\frac{1}{3} + \frac{(\theta - \pi)^2}{\pi^2}\big)z
+ \frac{i e^2 E^2}{16 \pi z} (x^1 - y^1)^2 (x^1 + y^1)^2
+ \mathcal{O}(z^2)
\end{split}
\end{equation}
In the region $[-\frac{L}{2}, 0)$, the two-point functions defined in the last section become
\begin{equation*}
\begin{split}
\omega(\psi^2(x) \bar{\psi_1}(y)) = 
\omega(\phi^1(x) \phi^\dagger_1(y)) = \xi_E(x^0 - y^0 - x^1 + y^1, x^1, y^1) \\
 \omega(\psi^1(x) \bar{\psi_2}(y)) = 
\omega(\phi^2(x) \phi^\dagger_2(y)) = \xi_E(x^0 - y^0 + x^1 - y^1, x^1, y^1)
\end{split}
\end{equation*}
Similarly, by defining 
\begin{equation}
\begin{split}
\chi_E(z, x^1, y^1) = 
& \frac{-i}{2 \pi z} + \frac{1}{2\pi L}\Big( \frac{\beta \sin \theta \cos \eta}{\alpha + \beta \sin \eta \cos \theta}\Big) (\theta - \pi) + \frac{eE}{4 \pi}\frac{(x^1 - y^1)(x^1 + y^1)}{z}     \\
& + \frac{i\pi}{4 L^2}\big( -\frac{1}{3} + \frac{(\theta - \pi)^2}{\pi^2}\big)z
+ \frac{i e^2 E^2}{16 \pi z} (x^1 - y^1)^2 (x^1 + y^1)^2
+ \mathcal{O}(z^2)
\end{split}
\end{equation}
we have the expressions for the states in the region $(0, \frac{L}{2}]$
\begin{equation*}
\begin{split}
\omega(\psi^2(x) \bar{\psi_1}(y)) = 
\omega(\phi^1(x) \phi^\dagger_1(y)) = \chi_E(x^0 - y^0 - x^1 + y^1, x^1, y^1) \\
 \omega(\psi^1(x) \bar{\psi_2}(y)) = 
\omega(\phi^2(x) \phi^\dagger_2(y)) = \chi_E(x^0 - y^0 + x^1 - y^1, x^1, y^1)
\end{split}
\end{equation*}
The coinciding point limit gives the charge density of the vacuum polarisation and the stress-tensor energy in the whole region (compared to the results of~\cite{Zahn2015})
\begin{equation}
\rho(x) = \frac{e}{\pi L}\Big( \frac{\beta \sin \theta \cos \eta}{\alpha + \beta \sin \eta \cos \theta}\Big) (-\theta + \pi)
\Big(\Theta(- x^1) - \Theta(x^1))\Big) + \frac{e^2 E}{\pi} x^1
\end{equation}
\begin{equation}
T_{ab}(x) = 
\bigg( \frac{\pi}{2L^2}\big( -\frac{1}{3} + \frac{(\theta - \pi)^2}{\pi^2}\big) + \frac{e^2E^2(x^1)^2}{2 \pi} \bigg)
\begin{pmatrix}
1 & 0 \\ 0 & 1
\end{pmatrix}
\end{equation}
