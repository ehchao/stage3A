%\section*{Notations and convention}
%The signature of our space-time is $(+, -)$.
We choose the following representations for the Dirac gamma matrices for this chapter
\begin{equation*}
\gamma^0 = \begin{pmatrix}
0 & 1 \\
1 & 0 \end{pmatrix}  \quad  \gamma^1 = \begin{pmatrix}
0  & 1 \\
-1 & 0
\end{pmatrix}
\end{equation*}
%
\section{Vacuum polarization in a confined space without external electric field}\label{sect-nef}
We consider a massless Dirac field in $1+1$-dimensional space-time spatially confined in the region $[-\frac L 2 , \frac L 2 ]$.
Denoting $\phi = \gamma^0\psi$, the Dirac equation of the massless field
\begin{equation*}
i\gamma^\mu\partial_\mu \psi = 0
\end{equation*}
can be written as
\begin{equation*}
i\partial_0\phi = i \begin{pmatrix} -1 & 0 \\ 0 & 1 \end{pmatrix}\partial_1\phi
\end{equation*}
%
As for the boundary condition on $\pm \frac L 2$, we impose the bag boundary condition~\cite{Chodos1974}, \ie 
\begin{equation*}
i\gamma^1 \psi \eval{\pm \frac{L}{2}} = \pm \psi \eval{\pm \frac{L}{2}}
\end{equation*}
In terms of $\phi = \gamma^0\psi = \begin{pmatrix} \phi_L \\ \phi_R \end{pmatrix}$.
In terms of $\phi = \gamma^0 \psi$, the bag boundary condition becomes
\begin{equation}\label{vacuum-bagboundcond}
\begin{pmatrix}
-i \phi_R \\
i \phi_L
\end{pmatrix} \eval{\pm \frac{L}{2}} = \pm \begin{pmatrix}
\phi_L \\
\phi_R
\end{pmatrix} \eval{\pm \frac{L}{2}}
\end{equation}
%
As shown in~\cite{Zahn2015}, when there is no external electric field, no polarization occurs. 
However, one can wonder if this result is always valid when there is a punctual interaction in the confined space.
Inspired by~\cref{vacuum-kondohamiltonian}, we introduce a delta potential term at $x^1 = 0$.
We can write down the equations of motion of the problem as following
\begin{equation}\label{nef-Dirac}
i \partial_0 \phi = 
\begin{pmatrix} 
-1 & 0 \\
0 & 1 
\end{pmatrix} i \partial_1 \phi +
\begin{pmatrix}
v_3 & v_- \\
v_+ & -v_3
\end{pmatrix} \delta(x_1) \phi
\end{equation}
where $v_3, v_+ + v_- \in \mathbb{R}$ and $ v_+ - v_-\in i \mathbb{R}$.\footnote{
In effect, the Kondo potential (the second term of the \rhs of~\cref{vacuum-kondohamiltonian}) is expressed as linear combination of the Pauli matrices
\begin{equation*}
\sigma^1 = \begin{pmatrix} 0 & 1 \\ 1 & 0 \end{pmatrix} \quad
\sigma^2 = \begin{pmatrix} 0 & -i \\ i & 0 \end{pmatrix} \quad
\sigma^3 = \begin{pmatrix} 1 & 0 \\ 0 & -1 \end{pmatrix} 
\end{equation*}
with real coefficients.
}
By denoting $\phi =
\begin{pmatrix}
\phi_L \\
\phi_R
\end{pmatrix}$
, \cref{nef-Dirac} leads to
\begin{equation}
\begin{cases}
i \partial_0 \phi_L = -i\partial_1 \phi_L + (v_3 \phi_L + v_- \phi_R) \delta(x_1) \\
i \partial_0 \phi_R = i\partial_1 \phi_R + (v_+ \phi_L - v_3 \phi_R) \delta(x_1)
\end{cases}
\end{equation}
The presence of a punctual potential might impose some matching condition at the singular point (See \eg Appendix J of~\cite{albeverio1988solvable}).
Let us consider now the matching conditions at $x^1=0$.
A solution of~\cref{nef-Dirac} is not continuous at $x^1 = 0$. 
By integrating~\cref{nef-Dirac} over $[-\epsilon, \epsilon]$ and taking the limite for $\epsilon \rightarrow 0$,
we have the matching condition
\begin{equation}\label{nef-matching}
\begin{cases}
-i(\phi_L(0^+) - \phi_L(0^-)) + \frac{1}{2}(v_3 (\phi_L(0^+) + \phi_L(0^-))+ v_- (\phi_R(0+) + \phi_R(0^-))) = 0 \\
i(\phi_R(0^+) - \phi_R(0^-)) + \frac{1}{2}(v_+ (\phi_L(0^+) + \phi_L(0^-)) - v_3 (\phi_R(0^+) + \phi_R(0^-))) = 0
\end{cases}
\end{equation}
Denote 
\begin{equation*}
\Sigma = v_+ ^ 2 + v_- ^ 2 + v_3 ^ 2
\end{equation*}
One can notice that this system does not have unique solution when $1 - \frac{1}{4}\Sigma + iv_3 = 0$. This case will be excluded in the following.
Therefore, the matching condition \cref{nef-matching} implies 
\begin{equation}\label{nef-matching2}
\begin{pmatrix}
\phi_L(0^+) \\
\phi_R(0^+)
\end{pmatrix} = \begin{pmatrix}
\frac{A}{D}  & \frac{C}{D} \\
\frac{C^*}{D} & \frac{A}{D}
\end{pmatrix}\begin{pmatrix}
\phi_L(0^-) \\
\phi_R(0^-)
\end{pmatrix}
\end{equation}
where 
\begin{equation*}
A = 1+ \frac{1}{4}\Sigma \quad, \quad
C = -iv_-  \quad, \quad
D = 1-\frac{1}{4}\Sigma + iv_3
\end{equation*}
%\subsection{Verification of the consistency of the solution}
Let us justify the consistency of \cref{nef-matching2}. From \cref{nef-basisSol}, we can consider $\phi_{L,k}$ ($\phi_{R,k}$) in the region $x^1 < 0$ ($x^1 > 0 $) as an "in-coming" wave \wrt the origin and $\phi_{R,k}$ ($\phi_{L,k}$) in the region $x^1 > 0$ ($x^1 < 0 $) as an "out-coming" wave \wrt the origin. In the stationary regime, the total density of probability of the in-coming waves should be equal to the total density of probability of the out-coming waves. 
One can re-write \cref{nef-matching2} in the following way
\begin{equation}\label{nef-consistency}
\begin{pmatrix}
\phi_L(0^+) \\
\phi_R(0^-)
\end{pmatrix}=\begin{pmatrix}
\frac{A}{D} - \frac{| C |^2}{DA} & \frac{C}{A}  \\
-\frac{C^*}{A}  &  \frac{D}{A}
\end{pmatrix}\begin{pmatrix}
\phi_L(0^-) \\
\phi_R(0^+)
\end{pmatrix}
\end{equation} 
Since $|A|^2 = |C|^2 + |D|^2$, the matrix on the \rhs is an unitary matrix. It follows that the norm of the vector on the \lhs is equal to the norm of the vector on the \rhs, which is indeed what we try to prove.
%
However, because of the $\delta$-functions in \cref{nef-Dirac}, 
the well-posedness of the problem is not guaranteed,
\ie it is uncertain if the system has a unitary time evolution solution~\cite{Reed1981}.
We should thus check the self-adjointness of the Hamiltonian of the dynamical system~\cref{nef-Dirac}
This aspect will be tackled in the next subsection.
%
\subsection{Well-posedness of the problem~\cref{nef-Dirac}}\label{vacuum-subsect-sa}
We would like to prove that~\cref{nef-Dirac} possesses indeed unitary time evolution solutions by functional analysis for unbounded operators~\cite{Reed1975}.
The Hamiltonian $H$ of the dynamical system~\cref{nef-Dirac} as an operator is defined as
\begin{equation}
H \phi = i \begin{pmatrix}
-1  &  0 \\
0  &  1  \end{pmatrix} \partial_1 \phi 
\end{equation}
First of all, we should decide on which Hilbert space the Hamiltonian $H$ acts.
If we can show that $H$ has a self-adjoint extension for this Hilbert space, 
we know that the problem has unitary time evolution solutions which can be expressed in terms of the self-adjoint extension of $H$.\\\\
%
The Hilbert space on which we are going to work is $\mathcal{H} = L^{2}(I_-, \mathbb{C}^2) \oplus L^{2}(I_+, \mathbb{C}^2)$,  where $I_- = [-\frac{L}{2}, 0)$ and $I_+ = (0, \frac{L}{2}]$.
The inner product of this Hilbert space is the sum of the usual $L^2$ inner products on $I_+$ and $I_-$, namely
\begin{equation*}
\langle \cdot, \cdot\rangle_{\mathcal{H} } = \langle \cdot, \cdot\rangle_{L^{2}(I_-, \mathbb{C}^2)} +\langle \cdot, \cdot\rangle_{L^{2}(I_+, \mathbb{C}^2)}
\end{equation*}
% with $I = [-\frac{L}{2}, \frac{L}{2}]$. 
To start, we encode the boundary condition~\cref{vacuum-bagboundcond} and the matching condition at $x^1 = 0$~\cref{nef-matching2} in the domaine of $H$ 
\begin{equation*}
\mathrm{Dom}(H) = \Big \{\phi \enskip \big\vert \enskip \phi \in W^{1,2}(I_-, \mathbb{C}^2) \oplus W^{1,2}(I_+, \mathbb{C}^2), \enskip \phi \textrm{ verifies~\cref{vacuum-bagboundcond} and~\cref{nef-matching2}} \Big \}
\end{equation*} 
where the matching condition at 0 is given in \cref{nef-matching2}\\\\
Let $ \phi = \begin{pmatrix} \phi_L \\ \phi_R \end{pmatrix}, \psi = \begin{pmatrix} \psi_L \\ \psi_R \end{pmatrix} \in \mathrm{Dom}(H)$
\begin{equation}\label{sa-hamiltonian}
\begin{split}
\langle \psi, H \phi \rangle = & i \int_{I_-} ( - \psi_L^\dagger \partial \phi_L + \psi_R^\dagger \partial \phi_R )
+ i \int_{I_+} ( - \psi_L^\dagger \partial \phi_L + \psi_R^\dagger \partial \phi_R ) \\
= & i \big[-\psi_L^\dagger \phi_L + \psi_R^\dagger \phi_R \big]^{0^-}_{-\frac{L}{2}} + i \big[-\psi_L^\dagger \phi_L + \psi_R^\dagger \phi_R \big]_{0^+}^{\frac{L}{2}} \\
& - i \int_{I_-} ( - \partial \psi_L^\dagger \phi_L + \partial \psi_R^\dagger  \phi_R ) - i \int_{I_+} ( - \partial \psi_L^\dagger \phi_L + \partial \psi_R^\dagger  \phi_R ) 
\end{split}
\end{equation}
The boundary condition~\cref{vacuum-bagboundcond} implies
\begin{equation*}
i\phi_R \big(\pm \frac L 2) = \mp\phi_L\big(\pm\frac L 2\big)
\end{equation*}
Therefore
\begin{equation*}
\begin{split}
- \psi_L^\dagger(\pm \frac{L}{2}) \phi_L(\pm \frac{L}{2}) + \psi_R^\dagger(\pm \frac{L}{2}) \phi_R(\pm \frac{L}{2}) = 
0
\end{split}
\end{equation*}
The matching condition at $x^1 = 0$ gives
\begin{equation*}
\begin{split}
\big[ \psi^\dagger_L\phi_L] ^{0^+}_{0^-} & = \frac{1}{|D|^2}(A \psi^\dagger(0^-) + C^\dagger\psi^\dagger_R(0^-))(A \phi_L(0^-) + C\phi_R(0^-)) - \psi^\dagger_L(0^-)\phi^\dagger_L(0^-) \\
& = \frac{|C|^2}{|D|^2}\big(\psi_L(0^-)^\dagger\phi_L(0^-) + \psi_R^\dagger(0^-) \phi_R(0^-)\big) +
\frac{2A}{|D|^2}\Re{C\psi_L^\dagger \phi_R} \\
& = \big[ \psi^\dagger_R\phi_R] ^{0^+}_{0^-}
\end{split}
\end{equation*}
Hence, 
\begin{equation*}
\langle \psi, H \phi \rangle = \langle H \psi , \phi \rangle
\end{equation*}
$H$ is symmetric. 
\\\\
It would be more complicated to find the right domain of self-adjointness of $H$.
However, the exact domain of self-adjointness is not what interests us the most. 
To prove the well-posedness of the problem, it suffices to show that $H$ is \textit{essentially self-adjointness}, \ie possesses a self-adjoint extension.
A basic criterion of essential self-adjointness is given by~\cite{Reed1981}
\begin{theorem}
Let $T$ be a symmetric operator on a Hilbert space $ \mathcal{H}$. The 3 following statement are equivalent 
\begin{enumerate}
\item $T$ is essentially self-adjoint
\item $\ker(T^* \pm i) = \{0\}$
\item $\ran(T \pm i )$ are dense
\end{enumerate} 
\end{theorem}
We denote  $\mathcal{K}_{\pm} = \ker (i \mp H^*)$ for the deficiency subspaces of $H$. The corollary of the Theorem X.2 of~\cite{Reed1975} states that $\dim \mathcal{K}_+ = \dim \mathcal{K}_-$ is a necessary and sufficient condition such that $H$ possesses an self-adjoint extension (all closed extension of $H$ is self-adjoint if this two numbers are equal to zero). \\\\
We start by calculate $\mathcal{K}_-$. Let $\phi \in \mathcal{K}_-$. Then $- i \phi = H^* \phi$. As $H$ is symmetric, this implies, for $\phi = \begin{pmatrix} \phi_L \\  \phi_R \end{pmatrix}$, 
\begin{equation}
i \begin{pmatrix} \phi_L \\ \phi_R \end{pmatrix} = 
i \begin{pmatrix} 1 & 0  \\ 0  &  -1 \end{pmatrix} 
\begin{pmatrix} \partial_1 \phi_L  \\  \partial_1\phi_R \end{pmatrix}
\end{equation} 
Thus, $\phi$ could be written as
\begin{equation}
\begin{pmatrix} \phi_L \\ \phi_R \end{pmatrix} = 
\Theta(-x) \begin{pmatrix} f_- e^x  \\ g_-  e^{-x} \end{pmatrix} + 
\Theta(x) \begin{pmatrix} f_+ e^x  \\ g_+  e^{-x} \end{pmatrix}
\end{equation}
The boundary condition gives
\begin{equation}
\begin{cases}
-i g_- e^{\frac{L}{2}} = - f_- e^{-\frac{L}{2}} \\
-ig_+e^{-\frac{L}{2}} = f_+ e^{\frac{L}{2}}
\end{cases} \quad \Leftrightarrow
\begin{cases}
g_- = -i f_- e^{-L} \\
g_+ = i f_+ e^L
\end{cases}
\end{equation}
We have found that the matching condition at $x=0$ gives a linear transformation, namely, $\phi(0^+) = T\phi(0^-)$ with $T = \frac{1}{D}\begin{pmatrix} A & C \\ C^* & A \end{pmatrix}$. 
With the boundary conditions, this implies
\begin{equation}
f_+ \begin{pmatrix} 1 \\ ie^L \end{pmatrix}
= f_- T \begin{pmatrix} 1 \\ -ie^{-L} \end{pmatrix}
= f_- \begin{pmatrix} \frac{A}{D} - i\frac{C}{D} e^{-L}  \\
\frac{C^*}{D} - i \frac{A}{D} e^{-L} \end{pmatrix}
\end{equation}
$f_+$ and $f_-$ are non-vanishing if and only if 
\begin{equation}
\begin{split}
& \frac A D - i\frac C D e^{-L} =  -ie^{-L}\big(\frac{ C^*}{ D} -i \frac A D e^{-L}\big) \\
\Leftrightarrow & \quad 2i A \cosh L = -C + C^*
\end{split}
\end{equation}
This equation can not hold since $A > |C|$.
As a consequence, $\dim\mathcal{K}_- = 0$. 
For $\mathcal{K}_+$, it suffices to replace $L$ by $-L$ in the above calculation and we will find $\dim\mathcal{K}_+ = 0$. 
Therefore,
$H$ possesses self-adjoint extension.


















\subsection{Vacuum charge and current in the spatially bounded case}
From now on, we work on the self-adjoint extension of the Hamiltonian of the problem~\cref{nef-Dirac}.
The space of eigenvectors for eigenvalue $k$ is spanned by
\begin{equation}\label{nef-basisSol}
\phi_{L,k} = 
\begin{pmatrix}
1 \\
0
\end{pmatrix} e^{ikx^1} \quad \textrm{and} \quad
\phi_{R,k} = 
\begin{pmatrix}
0 \\
1
\end{pmatrix} e^{-ikx^1}
\end{equation}
%and $\mu(k)$ is the spectral measure for the eigenvalue $k$ (see \eg Chap. 8 of~\cite{Reed1981} for definition).  \\\\
%
Let us look for solutions for the eigenvalue $k$.
Suppose that for the region $x^1<0$, 
 $\phi_L = f e^{ik x^1}$ and $\phi_R = g e^{-ikx^1}$, where $f$ and $g$ are complex numbers that we have to determine. 
According to \cref{nef-matching2}, the components of the solution in the region $x^1 > 0$ should be $\phi_L = \frac{1}{D} (Af+Cg) e^{-ik(x^0 - x^1)}$ and $\phi_R = \frac{1}{D}(C^* f + Ag ) e^{-ik(x^0 + x^1)}$. 
Note that the solution on the whole space $x^1 \in [-\frac{L}{2}, \frac{L}{2}] - \{0\}$ is totally determined by $f$ and $g$ due to the matching condition \cref{nef-matching}. \\\\
The boundary condition implies
\begin{equation}
\begin{cases}
-i e^{ik \frac{L}{2}} g = -f e^{-ik \frac{L}{2}}  \quad \textrm{, at $ x^1 = -\frac{L}{2}$}  \\
\frac{A}{D} f e^{ik \frac{L}{2}} + \frac{C}{D} g e^{ik \frac{L}{2}} = -i (\frac{C^*}{D} f e^{-ik \frac{L}{2}} + \frac{A}{D} g e^{-ik \frac{L}{2}})   \quad \textrm{, at $x^1 = \frac{L}{2}$}
\end{cases}
\end{equation}
which can be re-arranged as
\footnote{We can verify that, as $|A|^2 - |C|^2 = |D|^2 > 0$ by assumption, $iA + C$ is always non-vanishing.} 
\begin{equation}\label{nef-boundCond}
\begin{cases}
g = f e^{-i(kL+ \frac{\pi}{2})}  \\
g = \frac{A + iC^* e^{-ikL}}{- C e^{ikL} - iA} f e^{ikL}
\end{cases}
\end{equation}
For a non-vanishing solution, this implies
\begin{equation}\label{nef-boundCond1}
e^{-i(kL + \frac{\pi}{2})} = \frac{A + iC^* e^{-ikL}}{(A + iC^* e^{-ikL})^*} e^{i(kL + \frac{\pi}{2})}
\end{equation}
and 
\begin{equation}\label{nef-boundCond2}
| f | = | g |
\end{equation}
Thus, according to \cref{nef-boundCond1},
\begin{equation}\label{nef-kn1}
kL =  \textrm{Arg}(A - iC e^{ikL}) + \big(n+\frac{1}{2} \big)\pi   \quad \textrm{for n $\in \mathbb{Z}$}
\end{equation}
The case $|C| =0$ is relatively easy to deal with. Let us focus on the cases where $|C| \neq 0$. We should consider separately \cref{nef-kn1} for $n$ odd and $n$ even because of the $2\pi$-periodicity of the exponential term. \\\\
Let us start with the case where $n$ is even. 
For 
\begin{equation*}
C = |C| e^{i\eta} \neq 0 
\end{equation*}
it follows\footnote{
For $\alpha, \beta, \theta \in \mathbb{R}$, assuming that $\alpha + \beta \cos \theta > 0$, $\alpha + \beta e^{i \theta} = \alpha + \beta \cos \theta + i\beta \sin \theta = (\alpha^2 + \beta^2 + 2\alpha \beta \cos \theta) e^{i \delta}$ with $\delta = \arctan \frac{\beta\sin\theta}{\alpha + \beta\cos\theta}$  
} 
\begin{equation}
\begin{split}
&\textrm{Arg}(A - iC e^{ikL}) \\
= &\textrm{Arg}(A + |C| e^{i(kL - \frac{\pi}{2} + \eta)}) \\
= & \arctan \bigg( \frac{|C| \sin(kL - \frac{\pi}{2} + \eta)}{A + | C| \cos(kL - \frac{\pi}{2} + \eta) }\bigg)
\end{split}
\end{equation}
We want to find $k$ such that $kL \in [0, \pi]$ in order to coincide it with the allowed values of $\arctan$.
Therefore, by \cref{nef-kn1}, $k$ must satisfy
\begin{equation}\label{nef-arctan}
\begin{split}
& \frac{|C| \sin(kL - \frac{\pi}{2} + \eta)}{A + | C| \cos(kL - \frac{\pi}{2} + \eta) } =  - \cot kL  \\
\Leftrightarrow \quad & A \cot kL = |C| \cos(kL + \eta) - |C| \cot kL \sin(kL + \eta)  \\
%
\Leftrightarrow\quad
A \cos kL + |C| \sin\eta= 0
\end{split}
\end{equation}
Hence,
\begin{equation*}
kL = \arccos \big(-\frac{|C|\sin\eta}{A}\big)
\end{equation*}
For all even $n$, the corresponding mode $k_n$ is equal to this value modulo $2 \pi$. \\
For odd $n$, the calculation is similar.
We try to find $k$ such that $kL - \pi \in [0, \pi]$, which gives $kL = 2\pi - \arccos \big(-\frac{|C|}{A}\big)$. 
And thus for all odd $n$, the corresponding mode $k_n$ is equal to this value modulo $2 \pi$.\\
To sum up, the possible values of $k$ are given by
\begin{equation*}
k_{n} = \frac{(-1)^n}{L}\theta  + \frac{\pi}{L}n 
\end{equation*}
where
\begin{equation*}
\theta = \arccos\bigg( \frac{-|C| \sin \eta}{A} \bigg)
\end{equation*}
The coefficients $f_{n}$ and $g_{n}$ for the mode $k_{ n}$ can be determined by using the normalization condition  $\int_{[-\frac{L}{2}, \frac{L}{2}]}\phi^\dagger \phi = 1$. 
In the region $[-\frac{L}{2}, 0)$ , $\phi^\dagger \phi = | f |^2 + | g |^2$. Whereas in the region $(0, \frac{L}{2}]$, 
\begin{equation}\label{nef-norm1}
\begin{split}
\phi^\dagger \phi & = \begin{pmatrix}
\frac{1}{D^*}(Af^* +  C^*g^*)e^{-ikx^1}  & \frac{1}{D^*}(C f^* + Ag^*)e^{ikx^1} 
\end{pmatrix}\begin{pmatrix}
\frac{1}{D}(Af +  Cg)e^{ikx^1}  \\
 \frac{1}{D}(C^* f + Ag)e^{-ikx^1} 
\end{pmatrix}  \\
 & =
\frac{A^2 + | C|^2}{| D |^2}(|f|^2 + |g|^2) + 4\frac{A}{|D|^2}\Re \{C f^* g\}
\end{split}
\end{equation}
By the first equation of \cref{nef-boundCond}, the last term of~\cref{nef-norm1} is 
\begin{equation*}
4\frac{A |C|}{|D|^2}| f|^2\Re\{e ^{-i(kL + \frac{\pi}{2} - \eta)}\} = 
- 4\frac{A |C|}{|D|^2}| f|^2\sin( kL - \eta) 
\end{equation*}
Hence, the normalization condition and \cref{nef-boundCond2} imply
\begin{equation*}
 | f_{n} | =  \sqrt{\frac{1}{L(\alpha - \beta \sin (k_{n} L - \eta))}}  
\end{equation*}
where 
\begin{equation*}
\alpha = 1+\frac{A^2 + |C|^2}{|D|^2} \quad,\quad
\beta = \frac{2 A |C|}{|D|^2}
\end{equation*}
Therefore, we have found an eigenvector for the eigenvalue $k_n > 0$ 
\begin{equation}
\begin{split}
\phi_{k_{n}} = 
& \sqrt{\frac{1}{L(\alpha - \beta \sin (k_{n}L - \eta))}} \Bigg( 
\begin{pmatrix}
1 & 0 \\
0  & e^{-i(kL + \frac{\pi}{2})}
\end{pmatrix}
\Theta(-x^1) + \\
& \begin{pmatrix}
\frac{A}{D}  +  \frac{C}{D} e^{-i(kL + \frac{\pi}{2})} & 0 \\
0  & \frac{C^*}{D}  + \frac{A}{D}e^{-i(kL + \frac{\pi}{2})}
\end{pmatrix}\Theta(x^1)\Bigg)
\begin{pmatrix}
e^{ik_{n} x^1} \\
e^{- ik_{n} x^1}
\end{pmatrix}
\end{split}
\end{equation}
where $\Theta$ is the Heaviside step function.\\\\
Let us compute now the vacuum 1+1 current~\cref{vacuum-currentexpression}.
As we will multiply the two-point function defined in~\cref{vacuum-hadamardstate} by $\gamma^i$, 
only the off-diagonal terms of the two-point function should be considered.
In terms of $\phi$, these terms are expressed as
\begin{equation}\label{vacuum-calculhadamard}
\omega(\psi^B(x) \bar{\psi_A}(y)) = 
(\gamma^0)^B_C \omega(\phi^C(x) \phi^\dagger_A(y)) = (\gamma^0)^B_C
\int_{E_k > 0} \phi^C(x) \phi^\dagger_A(y) e^{-i(x^0 - y^0) E_k} \dd k
%\quad \textrm{for $A = 1,2$}
\end{equation}
Let us begin by considering the region $x^1, y^1 < 0$. 
For $A =1, B= 2$, with $z =x^0 - y^0 - x^1 +y^1$,~\cref{vacuum-calculhadamard} becomes
\begin{equation*}
\begin{split}
& \sum_{2p \geq 0} \frac{e^{-i(\theta + 2p\pi)\frac{z}{L}}}{L(\alpha - \beta \sin (\theta - \eta))} 
+ \sum_{2p+1 \geq 0} \frac{e^{-i(- \theta + (2p+2)\pi)\frac{z}{L}}}{L(\alpha + \beta \sin (\theta + \eta))}\\
%
=& 
\frac{1}{2i L\sin\frac{\pi}{L}z} \bigg( \frac{e^{i(-\theta + \pi)\frac{z}{L}}}{\alpha - \beta \sin (\theta - \eta)}
+ \frac{e^{i(\theta - \pi) \frac{z}{L}}}{\alpha + \beta \sin (\theta + \eta)}
\bigg)
\end{split}
\end{equation*}
Developping the term in the parenthesis up to $\mathcal{O}(z^0)$, we get
\begin{equation}
\begin{split}
& \frac{1}{\alpha - \beta \sin (\theta - \eta)}
   + \frac{1}{\alpha + \beta \sin (\theta + \eta)} \\
= & \frac{2(\alpha + \beta \sin \eta \cos \theta)}{(\alpha + \beta \sin \eta \cos \theta)^2 - \beta^2 \sin^2 \theta \cos^2 \eta} \\
= & \frac{2(\alpha - \beta \frac{|C|}{A} \sin^2 \eta)}{\alpha^2 - \beta^2 + \beta^2 \sin^2 \eta (1 + \frac{|C|^2}{A^2}) - 2 \alpha \beta \frac{|C|}{A} \sin^2 \eta} \\
\end{split}
\end{equation}
As
\begin{equation*}
\begin{split}
& \alpha^2 - \beta ^ 2 = 2 \alpha \\
&  \beta^2 \big(1 + \frac{|C|^2}{A^2} \big) - 2 \alpha \beta \frac{|C|}{A} \\
= & \big(2\frac{A |C|}{|D|^2} \big)^2 \big( 1+ \frac{|C|^2}{A^2} \big) - 4\big( 1+ \frac{|C|^2}{D^2}))\big(2\frac{A |C|}{|D|^2} \big) \frac{|C|}{A}  \\
= & 4 \frac{A^2 |C|^2}{|D|^4} + 4\frac{|C|^4}{|D|^4} - 8\frac{|C|^2}{|D|^2} - 8\frac{|C|^4}{|D|^4} \\
= & -2 \beta \frac{|C|}{A}
\end{split}
\end{equation*}
we have
\begin{equation}\label{nef-lourdeur}
\frac{1}{\alpha - \beta \sin (\theta - \eta)}
   + \frac{1}{\alpha + \beta \sin (\theta + \eta)} 
= 1
\end{equation}
%Therefore, the singularity of $\mathcal{O}(z^{-1})$ is the same as for the Hadamard parametrix of the vacuum case.\\\\
We calculate now the vacuum polarization in the region $[-\frac{L}{2}, 0)$. Since
\begin{equation*}
\frac{1}{2i \sin \frac{\pi}{L}z } = \frac{-iL}{2 \pi z} - \frac{i \pi z}{12L} + \mathcal{O}(z^3) 
\end{equation*}
using \cref{nef-lourdeur} and denoting
\begin{equation}\label{nef-xi}
\begin{split}
\xi(z) = & \Big( \frac{-i}{2 \pi z} - \frac{i \pi z}{12L^2} + \mathcal{O}(z^3) \Big)
\Big( 1 + \frac{i(-\theta + \pi)\frac{z}{L}}{\alpha - \beta\sin(\theta - \eta)} + \frac{i(\theta - \pi)\frac{z}{L}}{\alpha + \beta\sin(\theta + \eta)}  \\
& - \frac{1}{2}\Big(\frac{(-\theta + \pi)^2}{\alpha - \beta \sin (\theta - \eta)}  
+ \frac{(\theta - \pi)^2}{\alpha + \beta \sin (\theta + \eta)} \Big)\frac{z^2}{L^2}
+  \mathcal{O}(z^3) \Big)  \\
= & \frac{-i}{2 \pi z} + \frac{1}{2\pi L}\Big( \frac{-\theta + \pi}{\alpha - \beta\sin(\theta - \eta)} + \frac{\theta - \pi}{\alpha + \beta\sin(\theta + \eta)} \Big)  
 + \frac{i\pi}{4 L^2}z \big( -\frac{1}{3} + \frac{(\theta - \pi)^2}{\pi^2}\big) + \mathcal{O}(z^2) \\
= &  \frac{-i}{2 \pi z} 
+ \frac{1}{2\pi L}\Big( \frac{\beta \sin \theta \cos \eta}{\alpha + \beta \sin \eta \cos \theta}\Big) (-\theta + \pi) 
+ \frac{i\pi}{4 L^2}\big( -\frac{1}{3} + \frac{(\theta - \pi)^2}{\pi^2}\big)z+ \mathcal{O}(z^2)
\end{split}
\end{equation}
we thus have
\begin{equation*}
\omega(\psi^2(x) \bar{\psi_1}(y)) = \omega(\phi^1(x) \phi^\dagger_1(y)) 
= \xi( x^0 - y^0 - x^1 +y^1)
\end{equation*}
\begin{equation*}
\omega(\psi^1(x) \bar{\psi_2}(y)) =  \omega(\phi^2(x) \phi^\dagger_2(y)) 
= \xi(x^0 - y^0 + x^1 -y^1)
\end{equation*}
We subtract then from the above the Hadamard parametrix~\cref{vacuum-hadamardparametrix} and trace them with the gamma matrices in order to get the vacuum charge and current densities.
In the region $[-\frac{L}{2}, 0)$, the charge density is
\begin{equation}
\rho(x) = \frac{e}{\pi L}\Big( \frac{\beta \sin \theta \cos \eta}{\alpha + \beta \sin \eta \cos \theta}\Big) (-\theta + \pi)
\end{equation}
The same calculation allows us to get the two point function in the region $(0, \frac{L}{2}]$. By denoting
\begin{equation*}
\begin{split} 
\chi(z) = & \omega(\phi^1(x) \phi^\dagger_1(y)) \\
= & \Big(  \frac{-i}{2 \pi z} - \frac{i \pi z}{12L^2} + \mathcal{O}(z^3) \Big)  \bigg( 1 + \frac{i(-\theta + \pi)\frac{z}{L}}{\alpha + \beta\sin(\theta + \eta)}  
+ \frac{ i (\theta - \pi) \frac{z}{L}}{\alpha - \beta\sin(\theta - \eta)}   \\
& - \frac{1}{2}\Big(\frac{(-\theta + \pi)^2}{\alpha + \beta \sin (\theta + \eta)}  
+ \frac{(\theta - \pi)^2}{\alpha - \beta \sin (\theta - \eta)} \Big)\frac{z^2}{L^2}
+ \mathcal{O}(z^3) \bigg)  \\
= & \frac{-i}{2 \pi z} - \frac{1}{2\pi L} \Big( \frac{\beta \sin \theta \cos \eta}{\alpha + \beta \sin \eta \cos \theta}\Big) (-\theta + \pi) 
+ \frac{i\pi}{4 L^2}\big( -\frac{1}{3} + \frac{(\theta - \pi)^2}{\pi^2}\big)z
+ \mathcal{O}(z^2)
\end{split}
\end{equation*}
we find
\begin{equation*}
\omega(\psi^2(x) \bar{\psi_1}(y)) = \chi(x^0 - y^0 - x^1 + y^1)
\end{equation*}
\begin{equation*}
\omega(\psi^1(x) \bar{\psi_2}(y)) = \chi(x^0 - y^0 + x^1 - y^1)
\end{equation*}
Hence, the charge density in the whole space $[-\frac{L}{2}, \frac{L}{2}] - \{0\}$ is
\begin{equation}\label{vacuum-density_without_field}
\begin{split}
\rho(x) = \frac{e}{\pi L}\Big( \frac{\beta \sin \theta \cos \eta}{\alpha + \beta \sin \eta \cos \theta}\Big) (-\theta + \pi) \Big( \Theta(-x^1) - \Theta(x^1)\Big)
\end{split}
\end{equation}
and the current density is zero. 
%%%%%%%%%%%%%%%%%%%%%%%%%%%%%%%%

\subsection{Stress-energy tensor}
It is also worthy to compute the stress-energy tensor of the system in order to see how it is related to the vacuum polarization that we have calculated.
With its classical form, one can write down the components of stress-energy tensor as
\begin{equation}
\begin{split}
& T_{00} = \frac{i}{2} (\bar{\psi} \gamma_1 \nabla_1 \psi - \nabla_1 \bar{\psi}\gamma_1 \psi)  \\
& T_{11} = \frac{i}{2} (\bar{\psi} \gamma_0 \nabla_0 \psi - \nabla_0 \bar{\psi}\gamma_0 \psi)  \\
& T_{01} = \frac{i}{4} (\bar{\psi} \gamma_1 \nabla_0 \psi +\bar{\psi} \gamma_0 \nabla_1 \psi - \nabla_1 \bar{\psi}\gamma_0 \psi - \nabla_0 \bar{\psi}\gamma_1 \psi)  
\end{split}
\end{equation}
In the two-point function formulation, the expectation value of $T_{ab}$ corresponds to the regular part of the state $\omega$ previously defined. For example\footnote{Normal ordering is applied here when calculating the two-point function. The anti-commutation relation of the state is involved.},
\begin{equation}
\begin{split}
T_{00}(x,y) = &
- \frac{i}{2}\big(\omega( \nabla_1 \psi^B(x) \bar{\psi}_A(y))(\gamma_1)^A_B - \omega( \psi^B(x) \nabla_1 \bar{\psi}_A(y))(\gamma_1)^A_B \big) - \textrm{singular part}  \\
\end{split}
\end{equation}
Practically, the singular part corresponds to the covariant derivatives of the Hadamard parametrix. In terms of $\phi = \gamma^0 \psi$, 
\begin{equation*}
\bar{\psi} \gamma_1 \nabla \psi = - \phi^\dagger \gamma^1 \gamma^0 \nabla \phi
\end{equation*}
As usual, we start with the region $[-\frac{L}{2}, 0)$. By denoting
\begin{equation*}
\zeta = \gamma^1 \gamma^0 = \begin{pmatrix}
1 & 0 \\
0 & -1
\end{pmatrix}
\end{equation*}
using the function $\xi$ that we have introduced in \cref{nef-xi}, the states evaluated at the components of the stress-energy tensor can be written as
\begin{equation}
\begin{split}
T_{00}(x,y) = 
& \frac{i}{2}\Big(\nabla_{x^1} \big( \omega(\phi^B(x) \phi^\dagger_A(y))\zeta^A_C - H^+(x,y) \big)
- \nabla_{y^1} \big( \omega( \phi^B(x) \bar{\phi}_A(y))\zeta^A_C - H^+(x,y) \big)
\Big)\delta_B^C  \\
= & \frac{i}{2} \big( (-\xi'(z) - \xi'(w)) - \xi'(z) - \xi'(w) + \frac{i}{\pi z} + \frac{i}{\pi w} \big)   \\
T_{11}(x,y) =
& - \frac{i}{2}\big( \xi'(z) + \xi'(w) + \xi'(z) + \xi'(w) - \frac{i}{\pi z} - \frac{i}{\pi w}\big) \\
T_{01}(x,y) = 
& \frac{i}{4}\Big(\nabla_{x^0} \big( \omega(\phi^B(x) \phi^\dagger_A(y))(\zeta_1)^A_C - H^+(x,y) \big) + \nabla_{x^1} \big( \omega(\phi^B(x) \phi^\dagger_A(y))\delta^A_C - H^+(x,y) \big)  \\
& - \nabla_{y^0} \big( \omega( \phi^B(x) \bar{\phi}_A(y))\zeta^A_C - H^+(x,y) \big)
- \nabla_{y^1} \big( \omega( \phi^B(x) \bar{\phi}_A(y))\delta^A_C - H^+(x,y) \big)
\Big)\delta_B^C \\
= & \frac{i}{4}\Big( \big( \xi'(z) - \xi'(w) \big) + \big(- \xi'(z) + \xi'(w) \big) - \big( - \xi'(z) + \xi'(w) \big) - \big( \xi'(z) - \xi'(w) \big) \Big) \\
= & 0
\end{split}
\end{equation}
where $z = x^0 - y^0 - x^1 + y^1$ and $w = x^0 - y^0 + x^1 - y^1$ \\
Taking the coinciding point limit, we find
\begin{equation}
T_{ab} = \frac{ \pi}{2 L^2} \big( -\frac{1}{3} + \frac{(\theta - \pi)^2}{\pi^2}\big)\begin{pmatrix}
1  & 0 \\ 0  &  1
\end{pmatrix}
\end{equation}


%
%
%
%


















