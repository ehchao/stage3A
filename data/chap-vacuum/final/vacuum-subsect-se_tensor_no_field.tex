\subsection{Stress-energy tensor}
It is also worth it to compute the stress-energy tensor of the system in order to see how it is related to the vacuum polarization that we have calculated.
In particular, in our configuration where the space is confined, one would expect that a correct way to define vacuum polarization will allow us to rediscover the Casimir effect~\cite{Casimir1948}.\\\\
%
The components of the usual stress-energy tensor $T_{\mu\nu}$ for Dirac fields are given by
\begin{equation}
\begin{split}
& T_{00} = \frac{i}{2} (\bar{\psi} \gamma_1 \nabla_1 \psi - \nabla_1 \bar{\psi}\gamma_1 \psi)  \\
& T_{11} = \frac{i}{2} (\bar{\psi} \gamma_0 \nabla_0 \psi - \nabla_0 \bar{\psi}\gamma_0 \psi)  \\
& T_{01} = \frac{i}{4} (\bar{\psi} \gamma_1 \nabla_0 \psi +\bar{\psi} \gamma_0 \nabla_1 \psi - \nabla_1 \bar{\psi}\gamma_0 \psi - \nabla_0 \bar{\psi}\gamma_1 \psi)  
\end{split}
\end{equation}
Using again the point-splitting formalism~\cite{Dappiaggi2009} gives a way to obtain renormalized stress-energy tensor using Hadamard parametrix.
The approach corresponds to the following computation.
In terms of $\phi = \gamma^0 \psi$, 
\begin{equation*}
\bar{\psi} \gamma_1 \nabla \psi = - \phi^\dagger \gamma^1 \gamma^0 \nabla \phi
\end{equation*}
As usual, we start with the region $[-\frac{L}{2}, 0)$. By denoting
\begin{equation*}
\zeta = \gamma^1 \gamma^0 = \begin{pmatrix}
1 & 0 \\
0 & -1
\end{pmatrix}
\end{equation*}
using the function $\xi$ that we have introduced in \cref{nef-xi}, 
the renormalized components of the stress-energy tensor are
\begin{equation}\label{vacuum-stressenergy}
\begin{split}
T_{00}(x,y) = 
& \frac{i}{2}\Big(\nabla_{x^1} \big( \omega(\phi^B(x) \phi^\dagger_A(y))\zeta^A_C - H^+(x,y) \big)
- \nabla_{y^1} \big( \omega( \phi^B(x) \phi^\dagger_A(y))\zeta^A_C - H^+(x,y) \big)
\Big)\delta_B^C  \\
= & \frac{i}{2} \big( (-\xi'(z) - \xi'(w)) - \xi'(z) - \xi'(w) + \frac{i}{\pi z^2} + \frac{i}{\pi w^2} \big)   \\
T_{11}(x,y) =
& \frac{i}{2}\Big(\nabla_{x^0} \big( \omega(\phi^B(x) \phi^\dagger_A(y))\delta^A_C - H^+(x,y) \big)
- \nabla_{y^0} \big( \omega( \phi^B(x) \phi^\dagger_A(y))\delta^A_C - H^+(x,y) \big)
\Big)\delta_B^C  \\
= & - \frac{i}{2}\big( \xi'(z) + \xi'(w) + \xi'(z) + \xi'(w) - \frac{i}{\pi z^2} - \frac{i}{\pi w^2}\big) \\
T_{01}(x,y) = 
& \frac{i}{4}\Big(\nabla_{x^0} \big( \omega(\phi^B(x) \phi^\dagger_A(y))(\zeta_1)^A_C - H^+(x,y) \big) + \nabla_{x^1} \big( \omega(\phi^B(x) \phi^\dagger_A(y))\delta^A_C - H^+(x,y) \big)  \\
& - \nabla_{y^0} \big( \omega( \phi^B(x) \phi^\dagger_A(y))\zeta^A_C - H^+(x,y) \big)
- \nabla_{y^1} \big( \omega( \phi^B(x) \phi^\dagger_A(y))\delta^A_C - H^+(x,y) \big)
\Big)\delta_B^C \\
= & \frac{i}{4}\Big( \big( \xi'(z) - \xi'(w) \big) + \big(- \xi'(z) + \xi'(w) \big) - \big( - \xi'(z) + \xi'(w) \big) - \big( \xi'(z) - \xi'(w) \big) \Big) \\
= & 0
\end{split}
\end{equation}
where $z = x^0 - y^0 - x^1 + y^1$ and $w = x^0 - y^0 + x^1 - y^1$ \\
Taking the coinciding point limit, we find
\begin{equation*}
T_{\mu\nu} = \frac{ \pi}{2 L^2} \big( -\frac{1}{3} + \frac{(\theta - \pi)^2}{\pi^2}\big)\begin{pmatrix}
1  & 0 \\ 0  &  1
\end{pmatrix}
\end{equation*}
This expression is valid for $x\in[-\frac L 2 , \frac L 2] - \{0\}$.\\\\
%
We can verify that when the delta potential is turned off, \ie $v_3, v_+, v_- \rightarrow 0$, $\theta \rightarrow \frac 1 2 \pi$.
We obtain the same Casimir energy calculated in~\cite{Sundberg2003}, \ie
\begin{equation*}
T_{\mu\nu} = -\frac{\pi}{24L^2}\begin{pmatrix} 1 & 0 \\ 0 & 1\end{pmatrix}
\end{equation*}












