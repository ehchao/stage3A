\subsection{Stress-energy tensor}
It is also worth it to compute the stress-energy tensor of the system in order to see how it is related to the vacuum polarization that we have calculated.
With its classical form, one can write down the components of stress-energy tensor as
\begin{equation}
\begin{split}
& T_{00} = \frac{i}{2} (\bar{\psi} \gamma_1 \nabla_1 \psi - \nabla_1 \bar{\psi}\gamma_1 \psi)  \\
& T_{11} = \frac{i}{2} (\bar{\psi} \gamma_0 \nabla_0 \psi - \nabla_0 \bar{\psi}\gamma_0 \psi)  \\
& T_{01} = \frac{i}{4} (\bar{\psi} \gamma_1 \nabla_0 \psi +\bar{\psi} \gamma_0 \nabla_1 \psi - \nabla_1 \bar{\psi}\gamma_0 \psi - \nabla_0 \bar{\psi}\gamma_1 \psi)  
\end{split}
\end{equation}
In the two-point function formulation, the expectation value of $T_{ab}$ corresponds to the regular part of the state $\omega$ previously defined. For example
\begin{equation}
\begin{split}
T_{00}(x,y) = &
\frac{i}{2}\big(\omega( \nabla_1 \psi^B(x) \bar{\psi}_A(y))(\gamma_1)^A_B - \omega( \psi^B(x) \nabla_1 \bar{\psi}_A(y))(\gamma_1)^A_B \big) - \textrm{singular part}  \\
\end{split}
\end{equation}
The singular part corresponds to the covariant derivatives of the Hadamard parametrix.
%add ref??? 
In terms of $\phi = \gamma^0 \psi$, 
\begin{equation*}
\bar{\psi} \gamma_1 \nabla \psi = - \phi^\dagger \gamma^1 \gamma^0 \nabla \phi
\end{equation*}
As usual, we start with the region $[-\frac{L}{2}, 0)$. By denoting
\begin{equation*}
\zeta = \gamma^1 \gamma^0 = \begin{pmatrix}
1 & 0 \\
0 & -1
\end{pmatrix}
\end{equation*}
using the function $\xi$ that we have introduced in \cref{nef-xi}, 
the renormalized components of the stress-energy tensor are
%the states evaluated at the components of the stress-energy tensor can be written as
\begin{equation}\label{vacuum-stressenergy}
\begin{split}
T_{00}(x,y) = 
& \frac{i}{2}\Big(\nabla_{x^1} \big( \omega(\phi^B(x) \phi^\dagger_A(y))\zeta^A_C - H^+(x,y) \big)
- \nabla_{y^1} \big( \omega( \phi^B(x) \bar{\phi}_A(y))\zeta^A_C - H^+(x,y) \big)
\Big)\delta_B^C  \\
= & \frac{i}{2} \big( (-\xi'(z) - \xi'(w)) - \xi'(z) - \xi'(w) + \frac{i}{\pi z^2} + \frac{i}{\pi w^2} \big)   \\
T_{11}(x,y) =
& - \frac{i}{2}\big( \xi'(z) + \xi'(w) + \xi'(z) + \xi'(w) - \frac{i}{\pi z^2} - \frac{i}{\pi w^2}\big) \\
T_{01}(x,y) = 
& \frac{i}{4}\Big(\nabla_{x^0} \big( \omega(\phi^B(x) \phi^\dagger_A(y))(\zeta_1)^A_C - H^+(x,y) \big) + \nabla_{x^1} \big( \omega(\phi^B(x) \phi^\dagger_A(y))\delta^A_C - H^+(x,y) \big)  \\
& - \nabla_{y^0} \big( \omega( \phi^B(x) \bar{\phi}_A(y))\zeta^A_C - H^+(x,y) \big)
- \nabla_{y^1} \big( \omega( \phi^B(x) \bar{\phi}_A(y))\delta^A_C - H^+(x,y) \big)
\Big)\delta_B^C \\
= & \frac{i}{4}\Big( \big( \xi'(z) - \xi'(w) \big) + \big(- \xi'(z) + \xi'(w) \big) - \big( - \xi'(z) + \xi'(w) \big) - \big( \xi'(z) - \xi'(w) \big) \Big) \\
= & 0
\end{split}
\end{equation}
where $z = x^0 - y^0 - x^1 + y^1$ and $w = x^0 - y^0 + x^1 - y^1$ \\
Taking the coinciding point limit, we find
\begin{equation}
T_{\mu\nu} = \frac{ \pi}{2 L^2} \big( -\frac{1}{3} + \frac{(\theta - \pi)^2}{\pi^2}\big)\begin{pmatrix}
1  & 0 \\ 0  &  1
\end{pmatrix}
\end{equation}
This expression is valid for $x\in[-\frac L 2 , \frac L 2] - \{0\}$.
