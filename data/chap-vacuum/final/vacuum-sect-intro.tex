\section{Introduction}
Quantum field theories in external potential have gained importance these years due to the progress in experimental technology on intensive lasers and measurement of higher precision.
Especially, the quantum electrodynamics in such circumstance becomes a great interest~\cite{Mohr1998}. 
%add more blabla stuffs?
%Quantum electrodynamics has been well tested in weak external fields.
Its present form is basically due to Schwinger. 
In his paper~\cite{Schwinger1951}, the point-splitting formalism is used. 
However, it seems that his approach in~\cite{Schwinger1951} leads to some results which are not gauge invariant. 
~\cite{Zahn2015} highlights this point and proposes another approach using Dirac's method~\cite{Dirac1934} and some notions developped in \textit{quantum field theory in curved space-time} (QFT in curved space-time). 
Dirac's method consists of considering the vacuum as a space occupied by negative-energy positrons and that nearly all the positive-energy states are unoccupied.
The positive-energy that occupied by positrons 
%~\cite{Schwinger1951}
%
%Hadamard parametrix
A review on the necessary notions for this report will be given in the next paragraph.
For a more complete review of this subject, we can refer to~\cite{Hollands2014}. \\\\
%
QFT in curved space-time is a semi-classical theory aiming to taking into account the curvature of space-time in quantum field theory.
For a given curved Lorentzian space-time $(M,g)$,
the construction of the theory is based on algebraic approach.
Instead of considering a given Hilbert space, 
the theory is formulated with an algebra of quantum observables $\mathscr{A}(M,g)$.
As done in the section 2 of~\cite{Hollands2014}, we give here a brief review of QFT in curved space-time for quantum algebra of observables generated by a Klein-Gordon field $\phi$. 
However, the field $\phi$ should be treated as general function, \ie  distribution.
As a quantized field, the contribution of high frequency modes makes it difficult to define $\phi$ at a precise point $x$.
Hence, $\phi$ should be smeared with some test function $f\in C^\infty_0(M)$ when we want to study related relations.
The smearing $\phi(f)$ is given by
\begin{equation*}
\phi(f) = \int_M \phi f
\end{equation*}
We can then construct $\mathscr{A}(M,g)$ by starting with the free *-algebra generated by unit element $\mathbf{1}$ and elements $\phi(f)$ with $f\in C^\infty_0(M)$ and imposing the following relations \\
\begin{enumerate}
\item \textbf{Linearity} $\phi(c_1 f_1 + c_2 f_2) = c_1 \phi(f_1) + c_2 \phi(f_2)$ for $c_1, c_2 \in \mathbb{C}$
%
\item \textbf{Klein-Gordon equation} $\phi\big( (\Box_g - m^2)f \big) = 0$
%
\item \textbf{Hermitian field} $\phi(f)^* = \phi(\bar{f})$
%
\item \textbf{Commutation relation} $[\phi(f_1), \phi(f_2)] = iE(f_1, f_2) \mathbf{1}$, where $E = E^+ - E^-$ and $E^\pm$ are distributions satisfying $(\Box_g - m^2)E^\pm(x,y) = \delta(x,y)$ with appropriate supports (which define the advanced and retarded operators).
\end{enumerate}
%
We define a \textbf{physical state} $\omega$ as a linear map
$\omega: \mathscr{A}(M,g) \rightarrow \mathbb{C}$ satisfying the normalization condition $\omega(\mathbf{1}) = 1$ and the positivity $\omega(a^*a) \geq 0$ for $a\in\mathscr{A}(M,g)$.
A physical state could simply be considered as an expectation value functional. 
For a given state $\omega$, there is an associated Hilbert space $\mathscr{H}_\omega$, a representation $\pi : \mathscr{A}(M,g)\rightarrow \mathscr{H}_\omega$ and a non-trivial vector $\Psi \in \mathscr{H}_\omega$ such that for $a \in \mathscr{A}(M,g)$
\begin{equation*}
\omega(a) = \frac{\langle \Psi, \pi(a)\Psi\rangle}{\langle \Psi, \Psi \rangle}
\end{equation*}
by the GNS-construction (see \eg the appendix of~\cite{Timmermann2008} for more detail).
For what concerns this projet, the physical state that we will be interested in is the vacuum $(d+1)$-current in a $(d+1)$-dimensional space-time.
The definition of such a current will be given later. \\\\
%
Especially, it is of big interest to study physical states which are \textbf{2-point functions} (or more precisely, 2-point distributions), 
\ie distribution $W$ defined by (smeared) 
\begin{equation*}
W_2(f, g ) \equiv \omega(\phi(f)\otimes \phi(g))
\end{equation*}
It should be noted that $W_2(x,y)$ is defined only in sense of distribution because of the divergence when the coinciding-point limit $x\rightarrow y$ is applied.
In a globally hyperbolic space-time $(M,g)$ (see~\cite{Wald2010} for definition), it is shown (theorem 5.1 of~\cite{Radzikowski1996}) that the singular behaviour of the two-point function takes the same form for the states verifying a certain relation (such states are called \textbf{Hadamard states})\footnote{
To give a mathematically rigorous statement of this relation, 
the notion of \textbf{wave front set} is needed. 
We refer to~\cite{Radzikowski1996} for more details.
}. 
It is proven (cf.~\cite{Fulling1978}) that the singular structure of a Hadamard state is preserved in a globally hyperbolic space-time. 
The singular part at the coinciding-point limit of a Hadamard state is called \textbf{Hadamard parametrix}.
The substruction of the Hadamard parametrix is a method to renormalize the expectation functional $\omega$.
%more explaination ??
%
%continue with something like "the hadamard state of our problem can be characterized by ....
%
%explain what is hadamard parametrix
\paragraph{Computation of the Hadamard parametrix}
In this report, the Hadamard parametrix used for our calculation is obtained in~\cite{Zahn2015}. 
Here, we will just give the main idea of the computation of Hadamard parametrices. 
For more detailed mathematical aspects, one can refer to~\cite{Bar2008}
As we will discuss about Dirac fields which are in the kernel of the operator $i\slashed{\nabla} - m$ (eventually $m=0$ for massless cases), 
we have to calculate the retarded and the advanced propagators of 
\begin{equation*}
P = (i\slashed{\nabla} - m)(-i\slashed{\nabla} -m) 
\end{equation*}
It is a normally hyperbolic differential operator (\cite{Bar2008} for definition).
The retarded and advanced propagators are the distributional solutions $\Delta$ of
\begin{equation*} 
P\Delta = \delta_x
\end{equation*}
with appropriate supports.
Such solutions are given by 
\begin{equation*}
\Delta(x,x') = \sum_{k=0}^\infty V_k(x,x') R_{2k+2}(x,x')
\end{equation*} 
where $V_k$ are \textit{Hadamard coefficients} satisfying certain recursive relation and $R_{2k+2}$ are so-called \textit{Riesz distributions}.
The distributional nature of the Riesz distributions gives raise to singularities when coinciding-point limit is applied. 
We can thus build a Hadamard parametrix for the operator $P$.
However, the differential operator of the problem is the Dirac operator $i\slashed{\nabla} -m$,
the corresponding propagators still remains to be found. \\\\
%
As $\Delta$ is now in the kernel of $P$ in the sense of distribution, 
%$\Delta^{\textrm{ret/adv}}\equiv \Delta^{\textrm{ret}} - \Delta^{\textrm{adv}}$,
we can defined the retarded and advanced propagators as 
\begin{equation*}
S^{\textrm{ret/adv}} = (-i\slashed{\nabla} - m)\Delta^{\textrm{ret/adv}} 
\end{equation*}
with appropriate supports.
One can verify easily that $S$ is in the kernel of the Dirac operator $i\slashed{\nabla} - m$.
We build the Hadamard parametrix $H$ of the Dirac operator by choosing distributions that allow us to get the same singularities at coinciding-point limit as in Riesz distributions such that $H$ satisfied
\begin{equation*}
H^+(x,x') - H^-(x,x') = i\big(S^{\mathrm{ret}}(x,x') - S^{\mathrm{adv}}(x,x')\big)
\end{equation*}
%add something to microlocal stuffs
By consequence, for any state $\omega$ with Hadamard two-point function, the following relations hold
\begin{equation*}
\begin{split}
\omega(\psi^B(x)\bar{\psi}_A(y)) = & H^+(x,y)^B_A + R^B_A(x,y) \\
\omega(\bar{\psi}_A(y)\psi^B(x)) = &- H^-(x,y)^B_A - R^B_A(x,y)
\end{split}
\end{equation*}
where $R$ is smooth and determined up to terms vanishing at coinciding-point limit.
%
%
%
%Kondo
\paragraph{Kondo effect}
We propose to evaluate the vacuum polarization of a system (confined or not) 
%check...
in presence of a Kondo type potential.
This type of potential has been proposed by Jun Kondo in~\cite{Kondo1964} in order to study the so-called \textit{Kondo effect}.
Since the 1930's, people have observed anomalies in electrical resistance for some materials when the temperature decreases. 
Rather than having a decreasing resistance when the temperature gets down,
the resistance of certain types of material increases. 
A satisfactory explanation is that this phenomenon would be due to the existence of magnetic impurities in the material. 
Kondo's paper~\cite{Kondo1964} treats this phenomenon as an interaction between spins of conduction electrons and impurities.
As the impurities are considered as punctual,
the Hamiltonian of the Kondo effect in real space (rather than Fourier mode space) is given by~\cite{Erdmenger2013}
\begin{equation*}
H_K = \psi_\alpha^\dagger \frac{-\nabla^2}{2m}\psi_\alpha +
\frac 1 2\lambda_K \delta(\vec{x})\vec{S}\cdot \psi_{\alpha'}^\dagger  \vec{\sigma}_{\alpha' \alpha} \psi_\alpha
\end{equation*}
where $\psi$ and $\psi^\dagger$ are annihilation and creation operator, 
$\alpha$ is the indice for spin (up or down), 
$\vec{S}$ is the spin of the impurity,
$\delta$ is the Dirac distribution,
$\vec{\sigma}$ represents the vector of Pauli matrices and $\lambda_K$ is the Kondo coupling constant (positive for anti-ferromagnetic and negative for ferromagnetic).
















