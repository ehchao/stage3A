\subsection{Vacuum charge and current in the spatially bounded case}
From now on, we work on the self-adjoint extension of the Hamiltonian of the problem~\cref{nef-Dirac}.
The space of eigenvectors for eigenvalue $k$ is spanned by
\begin{equation}\label{nef-basisSol}
\phi_{L,k} = 
\begin{pmatrix}
1 \\
0
\end{pmatrix} e^{ikx^1} \quad \textrm{and} \quad
\phi_{R,k} = 
\begin{pmatrix}
0 \\
1
\end{pmatrix} e^{-ikx^1}
\end{equation}
%and $\mu(k)$ is the spectral measure for the eigenvalue $k$ (see \eg Chap. 8 of~\cite{Reed1981} for definition).  \\\\
%
Let us look for solutions for the eigenvalue $k$.
Suppose that for the region $x^1<0$, 
 $\phi_L = f e^{-ik(x^0 - x^1)}$ and $\phi_R = g e^{-ik(x^0+x^1)}$, where $f$ and $g$ are complex numbers that we have to determine. 
According to \cref{nef-matching2}, the components of the solution in the region $x^1 > 0$ should be $\phi_L = \frac{1}{D} (Af+Cg) e^{-ik(x^0 - x^1)}$ and $\phi_R = \frac{1}{D}(C^* f + Ag ) e^{-ik(x^0 + x^1)}$. 
Note that the solution on the whole space $x^1 \in [-\frac{L}{2}, \frac{L}{2}] - \{0\}$ is totally determined by $f$ and $g$ due to the matching condition \cref{nef-matching}. \\\\
The boundary condition implies
\begin{equation}
\begin{cases}
-i e^{ik \frac{L}{2}} g = -f e^{-ik \frac{L}{2}}  \quad \textrm{, at $ x^1 = -\frac{L}{2}$}  \\
\frac{A}{D} f e^{ik \frac{L}{2}} + \frac{C}{D} g e^{ik \frac{L}{2}} = -i (\frac{C^*}{D} f e^{-ik \frac{L}{2}} + \frac{A}{D} g e^{-ik \frac{L}{2}})   \quad \textrm{, at $x^1 = \frac{L}{2}$}
\end{cases}
\end{equation}
which can be re-arranged as
\footnote{We can verify that, as $|A|^2 - |C|^2 = |D|^2 > 0$ by assumption, $iA + C$ is always non-vanishing.} 
\begin{equation}\label{nef-boundCond}
\begin{cases}
g = f e^{-i(kL+ \frac{\pi}{2})}  \\
g = \frac{A + iC^* e^{-ikL}}{- C e^{ikL} - iA} f e^{ikL}
\end{cases}
\end{equation}
For a non-vanishing solution, this implies
\begin{equation}\label{nef-boundCond1}
e^{-i(kL + \frac{\pi}{2})} = \frac{A + iC^* e^{-ikL}}{(A + iC^* e^{-ikL})^*} e^{i(kL + \frac{\pi}{2})}
\end{equation}
and 
\begin{equation}\label{nef-boundCond2}
| f | = | g |
\end{equation}
Thus, according to \cref{nef-boundCond1},
\begin{equation}\label{nef-kn1}
kL =  \textrm{Arg}(A - iC e^{ikL}) + \big(n+\frac{1}{2} \big)\pi   \quad \textrm{for n $\in \mathbb{Z}$}
\end{equation}
The case $|C| =0$ is relatively easy to deal with. Let us focus on the cases where $|C| \neq 0$. We should consider separately \cref{nef-kn1} for $n$ odd and $n$ even because of the $2\pi$-periodicity of the exponential term. \\\\
Let us start with the case where $n$ is even. 
For $C = |C| e^{i\eta} \neq 0 $, it follows\footnote{
For $\alpha, \beta, \theta \in \mathbb{R}$, assuming that $\alpha + \beta \cos \theta > 0$, $\alpha + \beta e^{i \theta} = \alpha + \beta \cos \theta + i\beta \sin \theta = (\alpha^2 + \beta^2 + 2\alpha \beta \cos \theta) e^{i \delta}$ with $\delta = \arctan \frac{\beta\sin\theta}{\alpha + \beta\cos\theta}$  
} 
\begin{equation}
\begin{split}
&\textrm{Arg}(A - iC e^{ikL}) \\
= &\textrm{Arg}(A + |C| e^{i(kL - \frac{\pi}{2} + \eta)}) \\
= & \arctan \bigg( \frac{|C| \sin(kL - \frac{\pi}{2} + \eta)}{A + | C| \cos(kL - \frac{\pi}{2} + \eta) }\bigg)
\end{split}
\end{equation}
We want to find $k$ such that $kL \in [0, \pi]$ in order to coincide it with the allowed values of $\arctan$.
Therefore, by \cref{nef-kn1}, $k$ must satisfy
\begin{equation}\label{nef-arctan}
\begin{split}
& \frac{|C| \sin(kL - \frac{\pi}{2} + \eta)}{A + | C| \cos(kL - \frac{\pi}{2} + \eta) } =  - \cot kL  \\
\Leftrightarrow \quad & A \cot kL = |C| \cos(kL + \eta) - |C| \cot kL \sin(kL + \eta)  \\
%
\Leftrightarrow\quad
A \cos kL + |C| \sin\eta= 0
\end{split}
\end{equation}
Hence,
\begin{equation*}
kL = \arccos \big(-\frac{|C|\sin\eta}{A}\big)
\end{equation*}
For all even $n$, the corresponding mode $k_n$ is equal to this value modulo $2 \pi$. \\
For odd $n$, the calculation is similar.
We try to find $k$ such that $kL - \pi \in [0, \pi]$, which gives $kL = 2\pi - \arccos \big(-\frac{|C|}{A}\big)$. 
And thus for all odd $n$, the corresponding mode $k_n$ is equal to this value modulo $2 \pi$.\\
To sum up, the possible values of $k$ are given by
\begin{equation*}
k_{n} = \frac{(-1)^n}{L}\theta  + \frac{\pi}{L}n 
\end{equation*}
where
\begin{equation*}
\theta = \arccos\bigg( \frac{-|C| \sin \eta}{A} \bigg)
\end{equation*}
The coefficients $f_{n}$ and $g_{n}$ for the mode $k_{ n}$ can be determined by using the normalization condition  $\int_{[-\frac{L}{2}, \frac{L}{2}]}\phi^\dagger \phi = 1$. 
In the region $[-\frac{L}{2}, 0)$ , $\phi^\dagger \phi = | f |^2 + | g |^2$. Whereas in the region $(0, \frac{L}{2}]$, 
\begin{equation}\label{nef-norm1}
\begin{split}
\phi^\dagger \phi & = \begin{pmatrix}
\frac{1}{D^*}(Af^* +  C^*g^*)e^{-ikx^1}  & \frac{1}{D^*}(C f^* + Ag^*)e^{ikx^1} 
\end{pmatrix}\begin{pmatrix}
\frac{1}{D}(Af +  Cg)e^{ikx^1}  \\
 \frac{1}{D}(C^* f + Ag)e^{-ikx^1} 
\end{pmatrix}  \\
 & =
\frac{A^2 + | C|^2}{| D |^2}(|f|^2 + |g|^2) + 4\frac{A}{|D|^2}\Re \{C f^* g\}
\end{split}
\end{equation}
By the first equation of \cref{nef-boundCond}, the last term of~\cref{nef-norm1} is 
\begin{equation*}
4\frac{A |C|}{|D|^2}| f|^2\Re\{e ^{-i(kL + \frac{\pi}{2} - \eta)}\} = 
- 4\frac{A |C|}{|D|^2}| f|^2\sin( kL - \eta) 
\end{equation*}
Hence, the normalization condition and \cref{nef-boundCond2} imply
\begin{equation*}
 | f_{n} | =  \sqrt{\frac{1}{L(\alpha - \beta \sin (k_{n} L - \eta))}}  
\end{equation*}
where 
\begin{equation*}
\alpha = 1+\frac{A^2 + |C|^2}{|D|^2} \quad,\quad
\beta = \frac{2 A |C|}{|D|^2}
\end{equation*}
Therefore, we have found an eigenvector for the eigenvalue $k_n > 0$ 
\begin{equation}
\begin{split}
\phi_{k_{n}} = 
& \sqrt{\frac{1}{L(\alpha - \beta \sin (k_{n}L - \eta))}} \Bigg( 
\begin{pmatrix}
1 & 0 \\
0  & e^{-i(kL + \frac{\pi}{2})}
\end{pmatrix}
\Theta(-x^1) + \\
& \begin{pmatrix}
\frac{A}{D}  +  \frac{C}{D} e^{-i(kL + \frac{\pi}{2})} & 0 \\
0  & \frac{C^*}{D}  + \frac{A}{D}e^{-i(kL + \frac{\pi}{2})}
\end{pmatrix}\Theta(x^1)\Bigg)
\begin{pmatrix}
e^{ik_{n} x^1} \\
e^{- ik_{n} x^1}
\end{pmatrix}
\end{split}
\end{equation}
where $\Theta$ is the Heaviside step function.\\\\
Let us compute now the vacuum 1+1 current~\cref{vacuum-currentexpression}.
As we will multiply the two-point function defined in~\cref{vacuum-hadamardstate} by $\gamma^i$, 
only the off-diagonal terms of the two-point function should be considered.
In terms of $\phi$, these terms are expressed as
\begin{equation}\label{vacuum-calculhadamard}
\omega(\psi^B(x) \bar{\psi_A}(y)) = 
(\gamma^0)^B_C \omega(\phi^C(x) \phi^\dagger_A(y)) = (\gamma^0)^B_C
\int_{E_k > 0} \phi^C(x) \phi^\dagger_A(y) e^{-i(x^0 - y^0) E_k} \dd k
%\quad \textrm{for $A = 1,2$}
\end{equation}
Let us begin by considering the region $x^1, y^1 < 0$. 
For $A =1, B= 2$, with $z =x^0 - y^0 - x^1 +y^1$,~\cref{vacuum-calculhadamard} becomes
\begin{equation*}
\begin{split}
& \sum_{2p \geq 0} \frac{e^{-i(\theta + 2p\pi)\frac{z}{L}}}{L(\alpha - \beta \sin (\theta - \eta))} 
+ \sum_{2p+1 \geq 0} \frac{e^{-i(- \theta + (2p+2)\pi)\frac{z}{L}}}{L(\alpha + \beta \sin (\theta + \eta))}\\
%
=& 
\frac{1}{2i L\sin\frac{\pi}{L}z} \bigg( \frac{e^{i(-\theta + \pi)\frac{z}{L}}}{\alpha - \beta \sin (\theta - \eta)}
+ \frac{e^{i(\theta - \pi) \frac{z}{L}}}{\alpha + \beta \sin (\theta + \eta)}
\bigg)
\end{split}
\end{equation*}
Developping the term in the parenthesis up to $\mathcal{O}(1)$, we get
\begin{equation}
\begin{split}
& \frac{1}{\alpha - \beta \sin (\theta - \eta)}
   + \frac{1}{\alpha + \beta \sin (\theta + \eta)} \\
= & \frac{2(\alpha + \beta \sin \eta \cos \theta)}{(\alpha + \beta \sin \eta \cos \theta)^2 - \beta^2 \sin^2 \theta \cos^2 \eta} \\
= & \frac{2(\alpha - \beta \frac{|C|}{A} \sin^2 \eta)}{\alpha^2 - \beta^2 + \beta^2 \sin^2 \eta (1 + \frac{|C|^2}{A^2}) - 2 \alpha \beta \frac{|C|}{A} \sin^2 \eta} \\
\end{split}
\end{equation}
As
\begin{equation*}
\begin{split}
& \alpha^2 - \beta ^ 2 = 2 \alpha \\
&  \beta^2 \big(1 + \frac{|C|^2}{A^2} \big) - 2 \alpha \beta \frac{|C|}{A} \\
= & \big(2\frac{A |C|}{|D|^2} \big)^2 \big( 1+ \frac{|C|^2}{A^2} \big) - 4\big( 1+ \frac{|C|^2}{D^2}))\big(2\frac{A |C|}{|D|^2} \big) \frac{|C|}{A}  \\
= & 4 \frac{A^2 |C|^2}{|D|^4} + 4\frac{|C|^4}{|D|^4} - 8\frac{|C|^2}{|D|^2} - 8\frac{|C|^4}{|D|^4} \\
= & -2 \beta \frac{|C|}{A}
\end{split}
\end{equation*}
we have
\begin{equation}\label{nef-lourdeur}
\frac{1}{\alpha - \beta \sin (\theta - \eta)}
   + \frac{1}{\alpha + \beta \sin (\theta + \eta)} 
= 1
\end{equation}
%Therefore, the singularity of $\mathcal{O}(z^{-1})$ is the same as for the Hadamard parametrix of the vacuum case.\\\\
We calculate now the vacuum polarization in the region $[-\frac{L}{2}, 0)$. Since
\begin{equation*}
\frac{1}{2i \sin \frac{\pi}{L}z } = \frac{-iL}{2 \pi z} - \frac{i \pi z}{12L} + \mathcal{O}(z^3) 
\end{equation*}
using \cref{nef-lourdeur} and denoting
\begin{equation}\label{nef-xi}
\begin{split}
\xi(z) = & \Big( \frac{-i}{2 \pi z} - \frac{i \pi z}{12L^2} + \mathcal{O}(z^3) \Big)
\Big( 1 + \frac{i(-\theta + \pi)\frac{z}{L}}{\alpha - \beta\sin(\theta - \eta)} + \frac{i(\theta - \pi)\frac{z}{L}}{\alpha + \beta\sin(\theta + \eta)}  \\
& - \frac{1}{2}\Big(\frac{(-\theta + \pi)^2}{\alpha - \beta \sin (\theta - \eta)}  
+ \frac{(\theta - \pi)^2}{\alpha + \beta \sin (\theta + \eta)} \Big)\frac{z^2}{L^2}
+  \mathcal{O}(z^3) \Big)  \\
= & \frac{-i}{2 \pi z} + \frac{1}{2\pi L}\Big( \frac{-\theta + \pi}{\alpha - \beta\sin(\theta - \eta)} + \frac{\theta - \pi}{\alpha + \beta\sin(\theta + \eta)} \Big)  
 + \frac{i\pi}{4 L^2}z \big( -\frac{1}{3} + \frac{(\theta - \pi)^2}{\pi^2}\big) + \mathcal{O}(z^2) \\
= &  \frac{-i}{2 \pi z} 
+ \frac{1}{2\pi L}\Big( \frac{\beta \sin \theta \cos \eta}{\alpha + \beta \sin \eta \cos \theta}\Big) (-\theta + \pi) 
+ \frac{i\pi}{4 L^2}\big( -\frac{1}{3} + \frac{(\theta - \pi)^2}{\pi^2}\big)z+ \mathcal{O}(z^2)
\end{split}
\end{equation}
we thus have
\begin{equation*}
\omega(\psi^2(x) \bar{\psi_1}(y)) = \omega(\phi^1(x) \phi^\dagger_1(y)) 
= \xi( x^0 - y^0 - x^1 +y^1)
\end{equation*}
\begin{equation*}
\omega(\psi^1(x) \bar{\psi_2}(y)) =  \omega(\phi^2(x) \phi^\dagger_2(y)) 
= \xi(x^0 - y^0 + x^1 -y^1)
\end{equation*}
We subtract then from the above the Hadamard parametrix~\cref{vacuum-hadamardparametrix} and trace them with the gamma matrices in order to get the vacuum charge and current densities.
In the region $[-\frac{L}{2}, 0)$, the charge density is
\begin{equation}
\rho(x) = \frac{e}{\pi L}\Big( \frac{\beta \sin \theta \cos \eta}{\alpha + \beta \sin \eta \cos \theta}\Big) (-\theta + \pi)
\end{equation}
The same calculation allows us to get the two point function in the region $(0, \frac{L}{2}]$. By denoting
\begin{equation*}
\begin{split} 
\chi(z) = & \omega(\phi^1(x) \phi^\dagger_1(y)) \\
= & \Big(  \frac{-i}{2 \pi z} - \frac{i \pi z}{12L^2} + \mathcal{O}(z^3) \Big)  \bigg( 1 + \frac{i(-\theta + \pi)\frac{z}{L}}{\alpha + \beta\sin(\theta + \eta)}  
+ \frac{ i (\theta - \pi) \frac{z}{L}}{\alpha - \beta\sin(\theta - \eta)}   \\
& - \frac{1}{2}\Big(\frac{(-\theta + \pi)^2}{\alpha + \beta \sin (\theta + \eta)}  
+ \frac{(\theta - \pi)^2}{\alpha - \beta \sin (\theta - \eta)} \Big)\frac{z^2}{L^2}
+ \mathcal{O}(z^3) \bigg)  \\
= & \frac{-i}{2 \pi z} - \frac{1}{2\pi L} \Big( \frac{\beta \sin \theta \cos \eta}{\alpha + \beta \sin \eta \cos \theta}\Big) (-\theta + \pi) 
+ \frac{i\pi}{4 L^2}\big( -\frac{1}{3} + \frac{(\theta - \pi)^2}{\pi^2}\big)z
+ \mathcal{O}(z^2)
\end{split}
\end{equation*}
we find
\begin{equation*}
\omega(\psi^2(x) \bar{\psi_1}(y)) = \chi(x^0 - y^0 - x^1 + y^1)
\end{equation*}
\begin{equation*}
\omega(\psi^1(x) \bar{\psi_2}(y)) = \chi(x^0 - y^0 + x^1 - y^1)
\end{equation*}
Hence, the charge density in the whole space $[-\frac{L}{2}, \frac{L}{2}] - \{0\}$ is
\begin{equation}\label{vacuum-density_without_field}
\begin{split}
\rho(x) = \frac{e}{\pi L}\Big( \frac{\beta \sin \theta \cos \eta}{\alpha + \beta \sin \eta \cos \theta}\Big) (-\theta + \pi) \Big( \Theta(-x^1) - \Theta(x^1)\Big)
\end{split}
\end{equation}
and the current density is zero. 
%%%%%%%%%%%%%%%%%%%%%%%%%%%%%%%%
