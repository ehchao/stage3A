\section{Introduction}
Quantum field theories in external potential have gained importance these years due to the progress in experimental technology on intensive lasers and measurement of higher precision.
Especially, the quantum electrodynamics in such circumstance becomes a great interest (see \eg\cite{Mohr1998}). 
%Quantum electrodynamics has been well tested in weak external fields.
%~\cite{Schwinger1951}
%
%Hadamard parametrix
We calculate the vacuum polarization with the method developed in \textit{quantum field theory in curved space-time} (QFT in curved space-time). 
For a review of this subject, we can refer to~\cite{Hollands2014}. \\\\
%more explaination ??
%some Microlocal analysis stuffs
%According to~\cite{Radzikowski1996}, the Hadamard form of a two-point function is
%
%explain what is hadamard parametrix
\paragraph{Computation of Hadamard parametrix}
In this report, the Hadamard parametrix used for our calculation is obtained in~\cite{Zahn2015}. 
Here, we will just give the main idea of the computation of Hadamard parametrices. 
For more detailed mathematical aspects, one can refer to~\cite{Bar2008}
As we will discuss about Dirac fields which are in the kernel of the operator $i\slashed{\nabla} - m$ (eventually $m=0$ for massless cases), 
we have to calculate the retarded and the advanced propagators of 
\begin{equation*}
P = (i\slashed{\nabla} - m)(-i\slashed{\nabla} -m) 
\end{equation*}
It is a normally hyperbolic differential operator (\cite{Bar2008} for definition).
The retarded and advanced propagators are the distributional solutions $\Delta$ of
\begin{equation*} 
P\Delta = \delta_x
\end{equation*}
with appropriate supports.
Such solutions are given by 
\begin{equation*}
\Delta(x,x') = \sum_{k=0}^\infty V_k(x,x') R_{2k+2}(x,x')
\end{equation*} 
where $V_k$ are \textit{Hadamard coefficients} satisfying certain recursive relation and $R_{2k+2}$ are so-called \textit{Riesz distributions}.
The distributional nature of the Riesz distributions gives raise to singularities when coinciding-point limit is applied. 
We can thus build a Hadamard parametrix for the operator $P$.
However, the differential operator of the problem is the Dirac operator $i\slashed{\nabla} -m$,
the corresponding propagators still remains to be found. \\\\
%
As $\Delta$ is now in the kernel of $P$ in the sense of distribution, 
%$\Delta^{\textrm{ret/adv}}\equiv \Delta^{\textrm{ret}} - \Delta^{\textrm{adv}}$,
we can defined the retarded and advanced propagators as 
\begin{equation*}
S^{\textrm{ret/adv}} = (-i\slashed{\nabla} - m)\Delta^{\textrm{ret/adv}} 
\end{equation*}
with appropriate supports.
One can verify easily that $S$ is in the kernel of the Dirac operator $i\slashed{\nabla} - m$.
We build the Hadamard parametrix $H$ of the Dirac operator by choosing distributions that allow us to get the same singularities at coinciding-point limit as in Riesz distributions such that $H$ satisfied
\begin{equation*}
H^+(x,x') - H^-(x,x') = i\big(S^{\mathrm{ret}}(x,x') - S^{\mathrm{adv}}(x,x')\big)
\end{equation*}
%add something to microlocal stuffs
By consequence, for any state $\omega$ with Hadamard two-point function, the following relations hold
\begin{equation*}
\begin{split}
\omega(\psi^B(x)\bar{\psi}_A(y)) = & H^+(x,y)^B_A + R^B_A(x,y) \\
\omega(\bar{\psi}_A(y)\psi^B(x)) = &- H^-(x,y)^B_A - R^B_A(x,y)
\end{split}
\end{equation*}
where $R$ is smooth and determined up to terms vanishing at coinciding-point limit.


















