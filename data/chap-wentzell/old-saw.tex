We coincide the $r$-dimensional representation space of our Clifford Algebra with $\mathbb{C}^r$.
We denote 
\begin{equation*}
\begin{split}
&E = \bigcup_{p\in M}\{(p, v) \vert v \in\mathbb{C}^r\}  \\
&F=\bigcup_{p\in \partial M}\big\{(p, v) \vert v \in \ran({\mathcal P}_+\vert_p) \subset \mathbb{C}^{r } \big\}
\end{split}
\end{equation*}
What has been called projector $\mathcal P _+$ depends implicitly on the point $p$ where it is constructed, namely, we should denote it as $\mathcal P_+\vert_p \in \en \big(pr_2 \circ \pi^{-1}(p)\big)$ where $\pi$ is the bundle projection between $E$ and $M$, which will be justified in an instant.
For the simplicity, we choose to write it without such a specification.
The same remark is available for the gamma matrices.
We could justify that $E \rightarrow M$ and $F \rightarrow \partial M$ are vector bundles by the proposition 1.2 of chapter III of~\cite{Lang1999}, saying
\begin{proposition}\label{wen-proplang}
Let $X$ be a manifold and $\pi: E\rightarrow X$ a mapping from some set $E$ into $X$. Let $\{ U_i\}$ be an open covering of $X$ and for each $i$, suppose that we are given a Banach space $\mathbf{E}$ and a bijection
\begin{equation*}
\tau_i : \pi^{-1}(U_i) \rightarrow U_i\times \mathbf{E}
\end{equation*}
such that for each pair $i, j $ and $x\in U_i \cap U_j$, the map $(\tau_j\tau_i^{-1})_x$ is a toplinear isomorphism, and the following conditions are satisfied
\begin{enumerate}
\item The mapping $U_i \cap U_j \rightarrow \en(\mathbf{E}, \mathbf{E})$, $x \mapsto(\tau_j\tau_i^{-1})(x, \cdot)$ is a $\mathrm{C}^p$-morphism for some $p\geq 0$
\item The family $(\tau_i)$ verifies the cocycle condition
\end{enumerate}
Then, there exists a unique manifold structure on $E$ such that $\tau_i$ make $\pi$ a vector bundle.
\end{proposition}
\paragraph{Justification of the bundle structures}
The bundle structure of $E \rightarrow M$ is trivial. 
For $F \rightarrow \partial M$,
let $\{\phi_i, U_i\}$ be a chart of $\partial M$ where $\phi_i : \partial M \rightarrow \mathbb{R}^{d}$. 
Define 
\begin{equation*}
\tau_i : \pi^{-1}(U_i) \rightarrow U_i \times \mathbb C^\frac{r}{2} \quad \tau_i(p, v) = (p, v_{inj})
\end{equation*}
where by $v_{inj}$ we mean the corresponding element of $v$ in $\mathbb{C}^{\frac r 2}$ after coinciding $\ran(\mathcal{P}_+\vert_p)$ to it.
To see that $x \mapsto (\tau_j\tau_i^{-1})(x, \cdot)$ is a $C^k$-morphism for some $k$, we proceed as following.
Let $(p, v_i) \in \tau_i (U_i)$ with $v_i$ the coordinates of some $v \in \mathcal{P}_+\vert_p$ in the local coordinate system defined by $\phi_i$ and  $p \in U_i \cap U_j$.
When we change the coordinate system to the one defined by $\phi_j$, 
from the relation between the old gamma matrices $\hat\gamma$ and the new gamma matrices $\gamma$ (Sect. 3.2 of~\cite{Snygg1997}) 
\begin{equation*}
\gamma_\alpha = \hat{\gamma}_\beta\partial_\alpha (\phi_i \phi_j^{-1})^\beta(p)
\end{equation*}
we deduce that $\mathcal{P}_+$ change smoothly \wrt $p$.
By denoting $v'$ the coordinates of $v$ in the system of coordinate defined by $\phi_j$, the mapping 
\begin{equation*}
(p, v) \overset{}{\longrightarrow}(p, v')
\overset{\tau_j}{\longrightarrow} (p, v_j)
\end{equation*}
is clearly a $C^k$-morphism for some $k$.
It is then easy to verify that such a choice verify the rest of the conditions in~\cref{wen-proplang}. \\\\
