\section{Example of generalized solution: shock wave solution in 2+1 dimensional bulk}
We would like to emphasize the importance of the sign of $c$ in \cref{wen-action}. 
The choice of $c>0$ makes the bilinear form \cref{wen-innerpdt} an inner product on $\mathcal{H}$ and gives rise to the well-posedness of the problem discussed in \cref{wen-sect-saw}.
We will see in the following example that $c < 0 $ will lead to some physically unacceptable solutions of~\cref{wen-maineq}. \\\\
The example that we are going to examine here is a shock wave solution of \cref{wen-maineq} in a $2+1$-dimensional bulk $\mathcal{M} = \mathbb{R}^{1,1}\times\mathbb{R}_+$, given by
\begin{equation}\label{wen-shock}
\phi = \begin{pmatrix}
i \big(\delta(x^0 - x^2) + \delta(x^0 + x^2) - 4c^{-1}\Theta(x^0-x^2)e^{- \frac{x^0-x^2}{c}} \big) \\
\delta({x^0 - x^2}) - \delta(x^0 + x^2) - 4c^{-1}\Theta(x^0-x^2)e^{- \frac{x^0-x^2}{c}} \end{pmatrix}
\end{equation}
The boundary is given by $\partial\mathcal{M} = \{x^2 = 0\}$ \\
In this section, we choose the following gamma matrices
\begin{equation*}
\gamma^0 = \begin{pmatrix} 0 & 1 \\ 1 &0 \end{pmatrix} \quad
\gamma^1 = \begin{pmatrix} 0 & -1 \\ 1 &0 \end{pmatrix} \quad
\gamma^2 = \begin{pmatrix} i & 0 \\ 0 &-i \end{pmatrix} \quad
\end{equation*}
One can compute directly the projector $\mathcal{P}_+ = \frac{1}{2}(1+i\gamma^2) 
= \begin{pmatrix}0 & 0 \\ 0 & 1 \end{pmatrix}$ \\
In order to specify in which sense \cref{wen-shock} is a solution of \cref{wen-maineq}, we have to define our distribution space
\begin{definition}\label{wen-distr}
Specifically to this discussion, 
we take
\begin{equation*}
\mathcal{D} = \Big\{\eta \in C^{\infty}_0(\mathcal{M}, \mathbb{C}^2) \enskip\big\vert \enskip -c^{-1}\gamma^0\mathcal{P}_-\eta\vert_{\partial\mathcal{M}} + i\gamma^0\gamma^1\partial_1\mathcal{P}_+\eta\vert_{\partial \mathcal{M}}= i\partial_0\mathcal{P}_+\eta\vert_{\partial\mathcal{M}} \Big\}
\end{equation*}
for the space of test functions. 
Then, for any function $f \in L^{2}(\mathcal{M}, \mathbb{C}^2)$, we define the corresponding distribution $T_f$ by
\begin{equation*}
\forall \psi\in \mathcal{D}, \enskip \langle T_f, \psi\rangle = \int_{\mathcal{M}} f^\dagger \psi
\end{equation*}
where $\langle \cdot , \cdot \rangle$ represents the linear mapping defining the distribution.
\end{definition}
%
\begin{lemma}\label{wen-lem}
Let $f = (-1+ e^{-\frac{x}{c}})c\Theta(x)$. Then \begin{equation*}\frac{d}{d x} f = \Theta(x)e^{-\frac {x}{c}}\end{equation*}
\end{lemma}
\begin{proof}
Let $\eta\in \mathcal{D}$. Then
\begin{equation*}
\int_\mathbb{R} f \eta'  =   c\eta(0) + c\Big[ e^{-\frac{x}{c}} \eta(x) \Big]^\infty_0 - \int_\mathbb{R}\Theta(x)e^{-\frac{x}{c}}\eta(x) = -\int_\mathbb{R}\Theta(x)e^{-\frac{x}{c}}\eta(x)
\end{equation*}
\end{proof}
\begin{proposition}
$\phi$ defined in~\cref{wen-shock} is a solution of the bulk equation of~\cref{wen-maineq} in the sense of distribution defined in~\cref{wen-distr}
\end{proposition}
\begin{proof}
Let 
\begin{equation*}
\begin{split}
& \psi = \big(\Theta(x^0 - x^2) + \Theta(x^0 + x^2) \big) 
\begin{pmatrix} 1 \\ 0 \end{pmatrix} 
+ 4\big(-1+ e^{-\frac{x^0 - x^2}{c}}\big)\Theta(x^0-x^2)\begin{pmatrix} 1 \\ i \end{pmatrix}
\end{split}
\end{equation*}
$\psi$ being a solution of the wave equation $\Box \psi = 0$,
we can construct a solution of the bulk equation of \cref{wen-maineq} by acting $i\partial_0 + i\gamma^0\gamma^j \partial_j$ on it ($j= 1, 2)$.
In effect, it is straightforward to verify that $(\partial_0 - \gamma^j\partial_j)(\partial_0 + \gamma^j\partial_j) = \Box$. 
Then, by taking $\phi = i(\partial_0 + \gamma^0\gamma^j\partial_j)\psi$, we have
\begin{equation*}
\begin{split}
\phi = &
i\Bigg(\mathbb{1}\partial_0 + \begin{pmatrix} 0 &-i \\ i &0\end{pmatrix}\partial_2  \Bigg)\psi \\
\underset{\textrm{\cref{wen-lem}}}{=}&
i\Bigg( \delta(x^0 + x^2)\begin{pmatrix} 1 & -i \\ i & 1 \end{pmatrix} \begin{pmatrix}1 \\ 0 \end{pmatrix}
+ \big( \delta(x^0 - x^2) - 4c^{-1} e^{-\frac{x^0-x^2}{c}}\Theta(x^0-x^2)\big)
\begin{pmatrix} 1 & i \\ -i & 1 \end{pmatrix} \begin{pmatrix}1 \\ 0 \end{pmatrix} \Bigg) \\
= &
\begin{pmatrix}
i \big(\delta(x^0 - x^2) + \delta(x^0 + x^2) - 4c^{-1}e^{- \frac{x^0-x^2}{c}}\Theta(x^0-x^2) \big) \\
\delta({x^0 - x^2}) - \delta(x^0 + x^2) - 4c^{-1}e^{- \frac{x^0-x^2}{c}}\Theta(x^0-x^2) \end{pmatrix}
\end{split}
\end{equation*}
$\phi$ is a weak solution of the first equation of \cref{wen-maineq} if and only if 
\begin{equation*}
\int_{\mathcal{M}} \phi^\dagger (i \mathbb{1}\partial_0 + i\gamma^0\gamma^j\partial_j)\eta = 0  
\end{equation*}
for all $\eta \in \mathcal{D}$.
Let $\eta = \begin{pmatrix} \eta_1 \\ \eta_2\end{pmatrix} \in \mathcal{D}$. 
Meanwhile, because $\phi$ does not depend on $x^1$, the terms in $\phi\partial_1 \eta$ will simply vanish when the integration is applied.
For simplicity, we suppose furthurmore that $\eta$ is only a function of $x^0$ and $x^2$.
Then
\begin{equation*}
-c^{-1 }\gamma^0\mathcal{P}_-\eta\vert_{\partial\mathcal{M}} + i\gamma^0\gamma^1\partial_1\mathcal{P}_+\eta\vert_{\partial\mathcal{M}} = i\mathcal{P}_+\partial_0\eta\vert_{\partial\mathcal{M}}
\quad\Leftrightarrow\quad
-c^{-1}\eta_1\vert_{x^2 = 0} = i\partial_t\eta_2\vert_{x^2 = 0}
\end{equation*}
Since
\begin{equation*}
(\mathbb{1}\partial_0 + \gamma^0\gamma^j\partial_j) \eta=
\begin{pmatrix} \partial_0\eta_1 + i \partial_2 \eta_2 \\
\partial_0\eta_2 - i \partial_2 \eta_1 \end{pmatrix}
\end{equation*}
we have
\begin{equation*}
\begin{split}
 \int_M \phi^\dagger (i \mathbb{1}\partial_0 + i\gamma^0\gamma^j\partial_j)\eta = & 
 %
\int_{\mathbb{R}^2\times\mathbb{R}_+}\Big( (\delta(x^0 - x^2) + \delta(x^0+x^2)-4c^{-1}e^{-\frac{x^0-x^2}{c}}\Theta(x^0-x^2)) (\partial_0\eta_1 + i\partial_2\eta_2) \\
&+ i(\delta(x^0 - x^2) - \delta(x^0+x^2)-4c^{-1}e^{-\frac{x^0-x^2}{c}}\Theta(x^0-x^2)) \big( (\partial_0\eta_2 - i\partial_2\eta_1)\Big) \\
%
\underset{u = \frac{1}{2}(x^0 + x^2), v= \frac{1}{2}(x^0 - x^2)}= &
2 \int_{\mathbb{R}, -\infty<v<u <\infty} \Big( (\delta(v) + \delta(u)-4c^{-1}e^{-\frac{2v}{c}}\Theta(v)) \big( \frac12(\partial_u + \partial_v)\eta_1 + \frac i 2(\partial_u - \partial_v)\eta_2\big) \\
&+ i(\delta(v) - \delta(u)-4c^{-1}e^{-\frac{2v}{c}}\Theta(v)) \big( \frac12(\partial_u + \partial_v)\eta_2 - \frac i 2(\partial_u - \partial_v)\eta_1\big)\Big) \\
%
= &2 \int_{\mathbb{R}, -\infty<v<u <\infty}\Big( \big(\delta(v) \partial_u + \delta(u)\partial_v -4c^{-1}e^{-\frac{2v}{c}}\Theta(v)\partial_u)\eta_1 \big)\\
&+ i (-\delta(u)\partial_v + \delta(v)\partial_u-4c^{-1}e^{-\frac{2v}{c}}\Theta(v)\partial_u)\eta_2
\Big)\\ 
%
=& 2\int_{\mathbb{R}}\Big( -2i\eta_2 (0,0) + 4c^{-1}\int_{\mathbb{R}_+}e^{-\frac{2v}{c} }(\eta_1\vert_{x^2 = 0} +i \eta_2\vert_{x^2 = 0}) \dd v\Big) \\ 
%
\underset{\eta\in \mathcal{D}} =& 2\int_{\mathbb{R}}\Big( -2i\eta_2 (0,0) + 4c^{-1}\int_{\mathbb{R}_+}e^{-\frac{2v}{c} }(-ic \partial_0\eta_2\vert_{x^2 = 0} +i\eta_2\vert_{x^2 = 0}) \dd v\Big) \\ 
% 
=& 2\int_{\mathbb{R}}\Big( -2i\eta_2 (0,0) + 2c^{-1}\int_{\mathbb{R}^2_+}e^{-\frac{x^0-x^2} {c} }\big(-ic \partial_0\eta_2 +i\eta_2\big)\delta(x^2 = 0) \dd x^0 \dd x^2\Big) \\ 
=& 0
\end{split}
\end{equation*}
\end{proof}
As a result, 
\begin{equation*}\Phi = (\phi, \phi_|) \quad \mathrm{with}\quad \phi_| = \mathcal{P}_+ \phi\vert_{\partial \mathcal{M}} = 
\begin{pmatrix} 0 \\ -4c^{-1}e^{- \frac{x^0}{c}} \Theta(x^0) \end{pmatrix}
\end{equation*}
 is a solution of \cref{wen-maineq} in the sense of distribution extended naturally from \cref{wen-distr}. \\\\
However, if $c<0$, the solution on the boundary $\phi_|$ will increase exponentially with time, which is not physically acceptable. 
In effect, the shock wave (singular part of \cref{wen-shock}) reaches at the boundary only at $t = x^0 = 0$.
Afterward, there is no in-coming wave from the bulk to the boundary. 
After absorbing the shock wave $\delta(x^0+x^2)$ entering the boundary,
the boundary can only radiate energy and diminish the amplitude of the boundary field. 
This example shows again why $c$ must be chosen non-negative.












