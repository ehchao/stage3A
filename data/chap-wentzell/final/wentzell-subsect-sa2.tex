\subsection{${M} = \mathbb{R}^{d-1} \times [-L, L]$ }\label{wen-subsect-saw2}
The subtlety that one should beware of in this case is the projectors $\mathcal{P}_+$.
Since $\mathcal{P}_+$ is defined by considering a set of generators of Clifford Algebra on the tangent bundle $T_{ \partial \mathcal{M}}$, 
if we would like to fixe the coordinate system once for all,
we should also take into account the transformation of these generators. \\\\
Let $(x^0, \ldots, x^d)$ be the natural coordinate system of the bulk $\mathcal{M} =\mathbb{R}\times M $. 
Because of the geometric configuration of the bulk, 
this coordinate system is global. 
The boundary is therefore $\partial \mathcal{M} = \{x^d = -L \} \cup \{ x^d = L \}$.
\\\\
We denote $\{\gamma^\mu\}_\mu$ the set of gamma matrices chosen on $\{x^d  = - L \}$. 
We will have then $\mathcal{P}_+\vert_{\{x^d = L\}} = \mathcal{P}_-\vert_{\{x^d = -L\}}$ and $\gamma^\mu\vert_{\{x^d = L\}}=\gamma^\mu\vert_{\{x^d = -L\}}$ for $\mu\in\llbracket 0, d-1 \rrbracket$.
When it is not specified, we refer to the gamma matrices and the projectors on $\{x^d = -L\}$. \\\\
%
As for the boundary component $\phi_|$ of $\Phi= (\phi, \phi_|)$,
we can rewrite it as
\begin{equation*}
\phi_| = \mathbf{1}_{\{x^d = L \}}\phi_+ + \mathbf{1}_{\{x^d = - L \}}\phi_-
\end{equation*}
where $\phi_\pm$ represent the restriction of $\phi_|$ on $\{x^d = \pm L \}$.
Now, the boundary condition becomes
\begin{equation}\label{wen-saw2bound}
\begin{split}
\begin{cases}
\mathcal{P}_+ \phi\vert_{\{x^d = -L\}} = \phi_- \\
c^{-1}\mathcal{P}_-\phi\vert_{\{x^d = -L\}} = \mathcal{P}_-(\partial_d \phi\vert_{x^d = -L}) \\
\mathcal{P}_- \phi\vert_{\{x^d = L\}} =\phi_+ \\
-c^{-1}\mathcal{P}_+\phi\vert_{\{x^d = L\}} = \mathcal{P}_+(\partial_d \phi\vert_{x^d = L}) \\
\end{cases}
\end{split}
\end{equation}
As \cref{wen-subsect1}, the general solution $\phi$ in the bulk can be written as
\begin{equation*}
\phi 
 =\Big( (k \mathbb{1}+ \gamma^0\gamma^a p_a )(A \cos p_d x^d + B \sin p_d x^d)e^{-ip_a x^a}
+ i\gamma^0\gamma^d p_d (-A \sin p_d x^d + B \cos p_d x^d) e^{-ip_a x^a} \Big) \psi
 \end{equation*}
 where $\psi $ is a spinor and the Einstein summation over $a = 1, \ldots, d-1$ is applied. \\
 We compute
\begin{equation*}
\begin{split}
\partial_d \phi\vert_{x^d = \pm L }
= (\gamma^0\gamma^a p_a + k\mathbb{1})(\mp p_d A \sin p_d L + p_d B \cos p_d L)
+ i\gamma^0\gamma^d p_d(-p_d A \cos p_d L \mp p_d B \sin p_d L)\psi e^{-ip_a x^a}
\end{split}
\end{equation*}
Hence, the boundary condition leads to
\begin{equation*}
\begin{split}
0 = &\mathcal{P}_-\Bigg(
(\gamma^0\gamma^a p_a + k\mathbb{1})\Big((p_d A + c^{-1}B) \sin p_d L + (p_d B - c^{-1}A) \cos p_d L\Big) \\
& + i\gamma^0\gamma^d p_d \Big((-p_d A -c^{-1} B) \cos p_d L + (p_d B - c^{-1} A )\sin p_d L\Big) 
\Bigg)\psi
\end{split}
\end{equation*}
\begin{equation*}
\begin{split}
0 = &\mathcal{P}_+\Bigg(
(\gamma^0\gamma^a p_a + k\mathbb{1})\Big((- p_d A + c^{-1}B) \sin p_d L + (p_d B + c^{-1}A) \cos p_d L\Big) \\
& + i\gamma^0\gamma^d p_d \Big((-p_d A + c^{-1} B )\cos p_d L - (p_d B + c^{-1} A )\sin p_d L\Big) 
\Bigg)\psi
\end{split}
\end{equation*}
Once again, there are 4 possible cases to be discussed as in \cref{wen-subsect1}
%
\paragraph{Case 1 : $\psi \in \mathfrak{B}_-$ and $p_d \neq 0$}
The same argument as in~\cref{wen-subsect1} requires
\begin{equation}\label{wen-plates1}
\begin{cases}
(p_d A + c^{-1} B) \sin p_d L + (p_d B - c^{-1} A)\cos p_d L = 0 \\
(- p_d A + c^{-1}B)\cos p_d L - (p_d B + c^{-1} A)\sin p_d L = 0
\end{cases}
\end{equation}
In order to study~\cref{wen-plates1}, we discuss the following four sub-cases according to the coefficients of $\sin$ and $\cos$.
\subparagraph{Sub-case 1: $ p_d A + c^{-1} B = 0$}
\cref{wen-plates1} becomes
\begin{equation}\label{wen-plates1-1}
\begin{cases}
(c^{-2} + p_d ^2)\cos p_d L = 0 \\
(c^{-2} - p_d^2)\sin p_d L = 2 c^{-1}p_d \cos p_d L
\end{cases}
\end{equation}
We can verify easily that supposing $ c^{-2} + p_d^2 = 0 $ will lead to paradoxal situation. 
In effect, under this assumption, the second equation of~\cref{wen-plates1-1} will lead to $e^{\pm c^{-1}L} = 0$. \\
Therefore, the only solutions for this sub-case are $p_d = \pm c^{-1}$ for $L$ satisfying $\cos c^{-1} L = 0$
%
\subparagraph{Sub-case 2: $ p_d B + c^{-1}A = 0$}
By an easy calculation, we obtain the same result as in sub-case 1
%
\subparagraph{Sub-case 3: $\cos p_d L = 0 $}
Idem as in sub-case 1.
%
\subparagraph{Sub-case 4: none of the above statements holds}
\cref{wen-plates1} leads to
\begin{equation*}
\tan p_d L = \frac{-p_d B + c^{-1} A }{p_d A + c^{-1} B} = 
\frac{-p_d A + c^{-1}B}{p_d B + c^{-1} A}
\quad\Rightarrow\quad
(p_d ^2 + c^{-2})(A^2 - B^2 )= 0
\end{equation*}
One can easily verify that $p_d =  \pm i c^{-1}$ can not satisfy the relation. 
Hence, we have $A = \pm B$ and
\begin{equation}\label{wen-tan}
\tan p_d L = \frac{\mp p_d + c^{-1}}{p_d \pm c^{-1}}
\end{equation}
Note that if $p_d$ satifies~\cref{wen-tan},
$p_d$ should be reel.
%%%%%
In effect,~\cref{wen-tan} implies
\begin{equation*}
\Im (\tan p_d L ) = \frac{\Im(c^{-1} p_d^*)}{|p_d \pm c^{-1}|^2}
\end{equation*}
As the \lhs and the \rhs have opposite sign, 
the imaginary part of $p_d$ should be 0.
%
\paragraph{Case 2 : $\psi \in \mathfrak{B}_-$ and $p_d = 0$}
This case implies $A = 0$, which is forbidden because it leads to the trivial solution.
%
\paragraph{Case 3 : $\psi \in \mathfrak{B}_+$ and $p_d \neq 0$}
The relation between $A$ and $B$ becomes
\begin{equation*}
\begin{cases}
(-p_d A - c^{-1} B)\cos p_d L + (p_d B - c^{-1}A)\sin p_d L = 0  \\
(-p_d A + c^{-1}B)\sin p_d L + (p_d B + c^{-1} A)\cos p_d L = 0 
\end{cases}
\end{equation*}
This is strictly equivalent to the case 1 of this discussion.
%
\paragraph{Case 4 : $\psi \in \mathfrak{B}_+$ and $p_d = 0$}
Leads to the trivial solution. \\\\
To conclude, we give the following proposition
\begin{proposition}
In the case where $\cos c^{-1}L \neq 0$,
the eigenvalues $k$ of $\Delta$ are of form $k^2 = q^2 + p^2_d $ where $q\in\mathbb{R}^{d-1}$ and $p_d\in \mathbb{R}$ satisfying~\cref{wen-tan} and $-p^jp_j = k^2$.
Furthermore, the elements of the family $\{(\phi_{n,q,p_d}, \phi_{|n,q,p_d})\}$, where
\begin{equation*}
\phi_k = \Big(\big(i\gamma^0\gamma^a p_a+k\mathbb{1}\big)(\cos p_d x^d + \sin p_d x^d) +
i\gamma^0\gamma^d p_d(-\sin p_d x^d + \cos p_d x^d)\Big) \psi_n e^{-ip_a x^a}
\end{equation*}
, $q=(p_1,\ldots, p_{d-1})\in\mathbb{R}^{d-1}$ and $\psi_n\in\mathfrak{B}_+\cup\mathfrak{B}_-$,
form a generalized orthongonal basis of the eigenspace for eigenvalue $k$.
\end{proposition}



