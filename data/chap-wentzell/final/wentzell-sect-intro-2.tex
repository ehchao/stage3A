%%%%%%%%%%%%%%%%%%
%%%%%%%%%%%%%%%%
\section{The action of the problem}
Combining the boundary actions of~\cite{Henningson1998} and~\cite{Contino2005}, we propose to study the following action
\begin{equation}\label{wen-action}
\mathcal{S} = \frac{1}{2}i\int_{\mathcal{M}} \bar{\psi} \gamma^\mu \partial_\mu \psi - \partial_\mu \bar{\psi} \gamma^\mu \psi 
+ \frac{1}{2}\int_{\partial \mathcal{M}} ic \bar{\psi} \gamma^\alpha \partial_\alpha (1 - i \gamma^\bot) \psi
+ \bar{\psi} \psi
\end{equation}
for a certain constant $c >0$. 
$\gamma^\bot$ here represents the component perpendicular to $\partial \mathcal{M}$ in the incoming direction. 
As we will see later, the supplementary $\frac 1 2 (1-i\gamma^\bot)$ acts as a projector on the boundary field.
\\\\
When $c \rightarrow 0$, the boundary condition becomes the bag condition. 
Furthermore, we suppose $\partial(\partial \mathcal{M}) = \emptyset$.
By variational method, we can derive the equation of motion in the bulk and on the boundary
\begin{equation}\label{wen-motion}
\begin{cases}
i \gamma^\mu \partial_\mu \psi = 0  \quad \textrm{in $\mathcal{M}$}\\
i \gamma^\alpha \partial_\alpha (1 - i\gamma^\bot) \psi = - c^{-1}(1 + i\gamma^{\bot}) \psi \quad \textrm{on $\partial \mathcal{M}$}
\end{cases}
\end{equation}
We define $\phi = \gamma^0 \psi$. 
\cref{wen-motion} can be written as 
\begin{equation}\label{wen-maineq}
\begin{cases}
i \partial_0 \phi = i \gamma^0 \gamma^j \partial_j \phi   \quad \textrm{in $\mathcal{M}$}\\
i \partial_0(1 + i\gamma^\bot) \phi = i\gamma^0 \gamma^a \partial_a (1+ i\gamma^\bot)\phi - c^{-1} \gamma^0(1 - i \gamma^{\bot})\phi \quad \textrm{on $\partial \mathcal{M}$}
\end{cases}
\end{equation}
One can notice that the boundary condition implies constraints on only certain components of $\phi$. 
For instance, for $\dim \mathcal{M} = 3$, we can construct the following gamma matrices as suggested in~\cite{Polchinski1998}
\begin{equation*}
\gamma^0 = i\begin{pmatrix} 0 & 1 \\ -1 & 0 \end{pmatrix}  \quad
\gamma^1 = i\begin{pmatrix} 0 & 1 \\ 1 & 0 \end{pmatrix}  \quad
\gamma^2 = i\begin{pmatrix} 1 & 0 \\ 0 & -1 \end{pmatrix}  
\end{equation*}
Suppose that the inward normal vector of $\partial \mathcal{M}$ is $e_2$ at all point.
We have
\begin{equation*}
1 - i\gamma^\bot = 
\begin{pmatrix} 2 & 0 \\ 0 & 0\end{pmatrix} = 2 \mathcal{P}
\end{equation*}
where $\mathcal{P}$ is one of the chiral projectors on the boundary. 
More generally, for any space-like unit vector $n_j$,
\begin{equation*}
\mathcal{P}_\pm = \frac{1}{2}(1 \pm i n_j\gamma^j) 
\end{equation*}
are Hermitian projectors since 
\begin{equation*}
\mathcal{P}_\pm^\dagger = 
\frac{1}{2}(1 \mp i (n_j \gamma^j)^\dagger)=
\frac{1}{2}(1 \pm i n_j \gamma^j)
\end{equation*}
and
\begin{equation*}
(\mathcal{P}_\pm)^{2} = \frac{1}{4}(2\pm 2i \gamma^\bot) = \mathcal{P}_\pm
\end{equation*}
Furthermore, they have the same rank (equal to the half of the $n$-dimension of the representation space) since $\gamma^0$ is of maximal rank and
\begin{equation*}
\gamma^0\mathcal{P}_\pm = \mathcal{P}_\mp\gamma^0
\end{equation*}
In a more general way, 
we can rewrite~\cref{wen-maineq} as a time evolution problem with Cauchy data
$\Phi = \begin{pmatrix}
\phi, \phi_|
\end{pmatrix}$
\begin{equation*}
i\partial_0 \Phi = \Delta \Phi
\end{equation*}
where $\phi$ is the bulk field, $\phi_|$ is the boundary field and the operator $\Delta$ is defined as
\begin{equation}\label{wen-hamiltonian}
\Delta = \begin{pmatrix}
i \gamma^0 \slashed{\partial}  & 0 \\
-c^{-1} \gamma^0 \mathcal{P}_- \cdot \vert_{\partial M}&  i\gamma^0 \slashed{\partial}_|
\end{pmatrix}
\end{equation}
where $\slashed{\partial} = \gamma^j\partial_j$ for
$j \in \llbracket 1 , d \rrbracket$ and $\slashed{\partial}_| = h^{ab} \gamma_{a} \partial_{b}$ where $h$ is the induced metric on the boundary.
For simplicity, we will work on flat boundaries.
We choose a coordinate system where $n^\perp = n_d$.
Under this choice,
$\partial_|$ can be written in a more concise way
\begin{equation*}
\slashed \partial_| = \gamma^a\partial_a \quad\mathrm{for}\quad 
a \in \llbracket 1, \ldots, d-1 \rrbracket
\end{equation*}
We will take the above choice for the rest of the chapter. 





