\section*{Notations and convention}
For the following studies, we consider a $(d+1)$-dimensional manifold $\mathcal{M}$ for the bulk with a $d$-dimensional boundary $\partial \mathcal{M}$ which is supposed to be static and we use the Minkowski space-time with signature $(+, -,\ldots, -)$.
We denote by $M$ an equal-time hypersurface of $\mathcal{M}$.
%add something for spin manifold
As we work with spinor fields, $\mathcal{M}$ and $\partial \mathcal{M}$ are required to be spin manifolds. 
A good review on the notions of spin structure can be find in~\cite{Trautman2007}.
\footnote{
Such a structure on a given manifold exists if and only if the second Stiefel-Whitney class of its bundle vanishes (see~\eg Chap. 2 of \cite{Lawson1989} or~\cite{Alagia1985} for extension to pseudo-Riemannian manifolds).}
%
The indice 0 will correspond to the time component.
We denote $\gamma^\mu$ for $\mu = 0, \ldots, d$ for the generators of the Clifford algebra\footnote{
For simplicity, we will sometimes call this generators gamma matrices in this report.
} in the bulk, which should satisfy
\begin{equation*}
\{ \gamma^\mu, \gamma^\nu \} = \eta^{\mu\nu}
\end{equation*} 
where $\eta = \mathrm{diag}(1, -1 ,\ldots, -1)$.
Also, we choose a Hermitian representation for $\gamma^0$, 
\ie
\begin{equation*}
(\gamma^0)^\dagger = \gamma^0
\end{equation*}
Like in many literatures, Greek letters are used for both spatial and temporal components, whereas Latin letters are only used for spatial ones.
With shorthand notations, the generators of the induced Clifford algebra on the boundary $\partial \mathcal{M}$ will also be denoted by $\gamma^\alpha$ for $\alpha = 0 ,\ldots, d-1$. 
However, we specify that the Greek letters $\mu$ and $\nu$ will be used for the indices of bulk terms (taking values in $\llbracket 0, d \rrbracket$) and the Greek letters $\alpha$ and $\beta$ will be used for the indices of boundary terms (taking values in $\llbracket 0, d-1 \rrbracket$).
Analogously, the Latin letters $i $ and $j$ will be used for bulk terms and $a$ and $b$ will be used for boundary terms. \\\\
%
We denote the spinor representation spaces of $\mathcal{M}$ and $\partial \mathcal{M}$ by $E$ and $F$ respectively.
For certain manifolds, we can define the Sobolev spaces on them (\cite{Hebey1996}, \cite{Eichhorn1996}).
We suppose that $\mathcal{M}$ and $\partial\mathcal{M}$ belong to these categories of manifold.\footnote{
We might encounter some difficulties defining the Sobolev spaces for the boundaries. Meanwhile, the boundaries of these two cases are in effect "open manifolds", \ie, without boundary and compact connected component. 
According to~\cite{Eichhorn1996}, it is possible to define Sobolev spaces which are also Banach spaces for them.
Then, when working with $L^2$-norm, we have Hilbert space structures since the inner products will be well-defined.  
}
In the following text, if not specified, the Sobolev space $W^{m,n}(\mathcal{M})$ on manifold $\mathcal{M}$ represents $W^{m,n}(\mathcal{M}, E)$.
On the other hand, $W^{m,n}(\partial \mathcal{M})$ represents $W^{m,n}(\partial \mathcal{M}, F)$.
This notation is also applicable to the $L^2$ spaces which appear in this section. \\\\
Finally, when we talk about Hamiltonian, we refer to the Hamiltonian of a dynamical system.
%Furthermore, we require that the dimensions of the representation space of gamma matrices to be the same in the bulk and on the boundary.
%
%section
%%%%%%%%%%%%%%%%%%
%%%%%%%%%%%%%%%%
\section{The action of the problem}
Combining the boundary actions of~\cite{Henningson1998} and~\cite{Contino2005}, we propose to study the following action
\begin{equation}\label{wen-action}
\mathcal{S} = \frac{1}{2}i\int_{\mathcal{M}} \bar{\psi} \gamma^\mu \partial_\mu \psi - \partial_\mu \bar{\psi} \gamma^\mu \psi 
+ \frac{1}{2}\int_{\partial \mathcal{M}} ic \bar{\psi} \gamma^\alpha \partial_\alpha (1 - i \gamma^\bot) \psi
+ \bar{\psi} \psi
\end{equation}
for a certain constant $c >0$. 
$\gamma^\bot$ here represents the component perpendicular to $\partial \mathcal{M}$ in the incoming direction. 
As we will see later, the supplementary $\frac 1 2 (1-i\gamma^\bot)$ acts as a projector on the boundary field.
\\\\
When $c \rightarrow 0$, the boundary condition becomes the bag condition. 
Furthermore, we suppose $\partial(\partial \mathcal{M}) = \emptyset$.
% that $\psi$ and $\partial_\mu \psi$ vanish at $\partial(\partial \M)$. 
By variational method, we can derive the equations of motion in the bulk and on the boundary
\begin{equation}\label{wen-motion}
\begin{cases}
i \gamma^\mu \partial_\mu \psi = 0  \quad \textrm{in $\mathcal{M}$}\\
i \gamma^\alpha \partial_\alpha (1 - i\gamma^\bot) \psi = - c^{-1}(1 + i\gamma^{\bot}) \psi \quad \textrm{on $\partial \mathcal{M}$}
\end{cases}
\end{equation}
We define $\phi = \gamma^0 \psi$. 
\cref{wen-motion} can be written as 
\begin{equation}\label{wen-maineq}
\begin{cases}
i \partial_0 \phi = i \gamma^0 \gamma^j \partial_j \phi   \quad \textrm{in $\mathcal{M}$}\\
i \partial_0(1 + i\gamma^\bot) \phi = i\gamma^0 \gamma^a \partial_a (1+ i\gamma^\bot)\phi - c^{-1} \gamma^0(1 - i \gamma^{\bot})\phi \quad \textrm{on $\partial \mathcal{M}$}
\end{cases}
\end{equation}
One can notice that the boundary condition implies constraints on only a part of the components of $\phi$. 
For instance, for $\dim \mathcal{M} = 3$, we can construct the following generator of Clifford Algebra as suggested in~\cite{Polchinski1998}
\begin{equation*}
\gamma^0 = i\begin{pmatrix} 0 & 1 \\ -1 & 0 \end{pmatrix}  \quad
\gamma^1 = i\begin{pmatrix} 0 & 1 \\ 1 & 0 \end{pmatrix}  \quad
\gamma^2 = i\begin{pmatrix} 1 & 0 \\ 0 & -1 \end{pmatrix}  
\end{equation*}
Suppose that the inward normal vector of $\partial \mathcal{M}$ is $e_2$ at all point.
We have
\begin{equation*}
1 - i\gamma^\bot = 
\begin{pmatrix} 2 & 0 \\ 0 & 0\end{pmatrix} = 2 \mathcal{P}
\end{equation*}
where $\mathcal{P}$ is one of the chiral projectors on the boundary. 
%It seems pertinent to chose a Hermitian representation for $\gamma^0$ to ensure the Hermiticity of the Hamiltonian~\cref{wen-maineq} as proposed in~\cite{Pal2015}.
In order to define the domain of the Hamiltonian of the problem, the following lemma is needed 
\begin{lemma}\label{wen-proj}
Let $\gamma^\mu$ a set of generators of a Clifford Algebra in $d$-dimension and $n_j$ an unit space-like vector.
Then
\begin{equation*}
\mathcal{P}_\pm = \frac{1}{2}(1 \pm i n_j\gamma^j) 
\end{equation*}
are Hermitian projectors (so orthogonal) for the usual inner product $\langle \cdot, \cdot \rangle _{L^2(\Omega)}$, where $\Omega\subseteq M$.
\end{lemma}
\begin{proof}
In the usual case, we require the current $j^\mu = \bar{\psi} \gamma^\mu \psi $, where $\psi$ is the solution of the Dirac equation (the equation in the bulk of \cref{wen-motion}) to be Hermitian.
\ie 
\begin{equation*}
j^\mu 
= \bar{\psi}\gamma^\mu\psi 
= \psi^\dagger\gamma^0 \gamma^\mu \psi
= \psi^\dagger(\gamma^\mu)^\dagger (\gamma^0)^\dagger \psi
=  (j^\mu)^\dagger
\end{equation*}
which implies, for $j \neq 0$
\begin{equation*}
(\gamma^j)^\dagger = \gamma^0\gamma^j\gamma^0 = -\gamma^j
\end{equation*}
For any spatial component $j$, we define $\mathcal{P}_\pm = \frac{1}{2}(1 \pm i\gamma^j)$.
Obviously, $\mathcal{P}_\pm$ are Hermitian since
\begin{equation*}
\mathcal{P}_\pm^\dagger = 
\frac{1}{2}(1 \mp i (n_j \gamma^j)^\dagger)=
\frac{1}{2}(1 \pm i n_j \gamma^j)
\end{equation*}
It is clear that $(\mathcal{P}_\pm)^{2} = \frac{1}{4}(2\pm 2i \gamma^\bot) = \mathcal{P}_\pm$.
Furthermore, they have the same rank (equal to the half of the $n$-dimension of the representation space) since $\gamma^0$ is of maximal rank and
\begin{equation*}
\gamma^0\mathcal{P}_\pm = \mathcal{P}_\mp\gamma^0
\end{equation*}
%Finally, by the Hermicity of $\mathcal{P}_{\pm}$, we have
%\begin{equation*}
%\langle \mathcal{P}_- \phi, \mathcal{P}_+ \psi \rangle = 
%\langle \phi, \mathcal{P}_-\mathcal{P}_+ \psi \rangle = 0
%\end{equation*}
%which shows the two projectors are orthogonal.
\end{proof}
As a result, \cref{wen-proj} ensures that the projectors $\mathcal{P}_\pm$ are not identically null and that the projection spaces by both projectors are of equal dimension. 
Thus, one can define an operator $\Delta$ for the problem for $d+1$-dimensional bulk case
\begin{equation}\label{wen-hamiltonian}
\Delta = \begin{pmatrix}
i \gamma^0 \slashed{\partial}  & 0 \\
-c^{-1} \gamma^0 \mathcal{P}_- \cdot \vert_{\partial M}&  i\gamma^0 \slashed{\partial}_| \mathcal{P}_+
\end{pmatrix}
\end{equation}
where $\slashed{\partial} = \gamma^j\partial_j$ for
$j \in \llbracket 1 , d \rrbracket$ and $\slashed{\partial}_| = h^{ab} \gamma_{a} \partial_{a}$ where $h$ is the second fundamental form.
For simplicity, we will work on flat boundaries.
By denoting $n_j$ the inward unit vector normal to the boundary $\partial M$,
\begin{equation*}
\slashed{\partial}_| = (1 - n_j)\gamma^j\partial_j \quad\mathrm{and}\quad
\mathcal{P}_\pm = \frac 1 2 (1\pm i n_j\gamma^j)
\end{equation*}
One can choose $n_j = \delta_{j,d}$ when defining $\partial_|$ and $\mathcal P _\pm$.
This choice allows us to write down $\slashed{\partial}_|$ in a concise way
\begin{equation*}
\slashed \partial_| = \gamma^a\partial_a \quad\mathrm{for}\quad 
a \in \llbracket 1, \ldots, d-1 \rrbracket
\end{equation*}
We will make the above choice for the rest of the chapter. 
However, it would also be possible to show that the results that we will obtain are valid in more general case of time-like boundary in Minkowski space-time.





