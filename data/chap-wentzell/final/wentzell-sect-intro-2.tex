%%%%%%%%%%%%%%%%%%
%%%%%%%%%%%%%%%%
\section{The action of the problem}
Combining the boundary actions of~\cite{Henningson1998} and~\cite{Contino2005}, we propose to study the following action
\begin{equation}\label{wen-action}
\mathcal{S} = \frac{1}{2}i\int_{\mathcal{M}} \bar{\psi} \gamma^\mu \partial_\mu \psi - \partial_\mu \bar{\psi} \gamma^\mu \psi 
+ \frac{1}{2}\int_{\partial \mathcal{M}} ic \bar{\psi} \gamma^\alpha \partial_\alpha (1 - i \gamma^\bot) \psi
+ \bar{\psi} \psi
\end{equation}
for a certain constant $c >0$. 
$\gamma^\bot$ here represents the component perpendicular to $\partial \mathcal{M}$ in the incoming direction. 
As we will see later, the supplementary $\frac 1 2 (1-i\gamma^\bot)$ acts as a projector on the boundary field.
\\\\
When $c \rightarrow 0$, the boundary condition becomes the bag condition. 
Furthermore, we suppose $\partial(\partial \mathcal{M}) = \emptyset$.
% that $\psi$ and $\partial_\mu \psi$ vanish at $\partial(\partial \M)$. 
By variational method, we can derive the equations of motion in the bulk and on the boundary
\begin{equation}\label{wen-motion}
\begin{cases}
i \gamma^\mu \partial_\mu \psi = 0  \quad \textrm{in $\mathcal{M}$}\\
i \gamma^\alpha \partial_\alpha (1 - i\gamma^\bot) \psi = - c^{-1}(1 + i\gamma^{\bot}) \psi \quad \textrm{on $\partial \mathcal{M}$}
\end{cases}
\end{equation}
We define $\phi = \gamma^0 \psi$. 
\cref{wen-motion} can be written as 
\begin{equation}\label{wen-maineq}
\begin{cases}
i \partial_0 \phi = i \gamma^0 \gamma^j \partial_j \phi   \quad \textrm{in $\mathcal{M}$}\\
i \partial_0(1 + i\gamma^\bot) \phi = i\gamma^0 \gamma^a \partial_a (1+ i\gamma^\bot)\phi - c^{-1} \gamma^0(1 - i \gamma^{\bot})\phi \quad \textrm{on $\partial \mathcal{M}$}
\end{cases}
\end{equation}
One can notice that the boundary condition implies constraints on only a part of the components of $\phi$. 
For instance, for $\dim \mathcal{M} = 3$, we can construct the following generator of Clifford Algebra as suggested in~\cite{Polchinski1998}
\begin{equation*}
\gamma^0 = i\begin{pmatrix} 0 & 1 \\ -1 & 0 \end{pmatrix}  \quad
\gamma^1 = i\begin{pmatrix} 0 & 1 \\ 1 & 0 \end{pmatrix}  \quad
\gamma^2 = i\begin{pmatrix} 1 & 0 \\ 0 & -1 \end{pmatrix}  
\end{equation*}
Suppose that the inward normal vector of $\partial \mathcal{M}$ is $e_2$ at all point.
We have
\begin{equation*}
1 - i\gamma^\bot = 
\begin{pmatrix} 2 & 0 \\ 0 & 0\end{pmatrix} = 2 \mathcal{P}
\end{equation*}
where $\mathcal{P}$ is one of the chiral projectors on the boundary. 
More generally, for any space-like unit vector $n_j$,
\begin{equation*}
\mathcal{P}_\pm = \frac{1}{2}(1 \pm i n_j\gamma^j) 
\end{equation*}
are Hermitian projectors since 
\begin{equation*}
\mathcal{P}_\pm^\dagger = 
\frac{1}{2}(1 \mp i (n_j \gamma^j)^\dagger)=
\frac{1}{2}(1 \pm i n_j \gamma^j)
\end{equation*}
and
\begin{equation*}
(\mathcal{P}_\pm)^{2} = \frac{1}{4}(2\pm 2i \gamma^\bot) = \mathcal{P}_\pm
\end{equation*}
Furthermore, they have the same rank (equal to the half of the $n$-dimension of the representation space) since $\gamma^0$ is of maximal rank and
\begin{equation*}
\gamma^0\mathcal{P}_\pm = \mathcal{P}_\mp\gamma^0
\end{equation*}
Thus, one can define an operator $\Delta$ for the problem for $d+1$-dimensional bulk case
\begin{equation}\label{wen-hamiltonian}
\Delta = \begin{pmatrix}
i \gamma^0 \slashed{\partial}  & 0 \\
-c^{-1} \gamma^0 \mathcal{P}_- \cdot \vert_{\partial M}&  i\gamma^0 \slashed{\partial}_|
\end{pmatrix}
\end{equation}
where $\slashed{\partial} = \gamma^j\partial_j$ for
$j \in \llbracket 1 , d \rrbracket$ and $\slashed{\partial}_| = h^{ab} \gamma_{a} \partial_{b}$ where $h$ is the induced metric on the boundary.
For simplicity, we will work on flat boundaries.
By denoting $n_j$ the inward unit vector normal to the boundary $\partial M$,
\begin{equation*}
\slashed{\partial}_| = (1 - n_j)\gamma^j\partial_j \quad\mathrm{and}\quad
\mathcal{P}_\pm = \frac 1 2 (1\pm i n_j\gamma^j)
\end{equation*}
One can choose $n_j = \delta_{j,d}$ when defining $\partial_|$ and $\mathcal P _\pm$.
This choice allows us to write down $\slashed{\partial}_|$ in a concise way
\begin{equation*}
\slashed \partial_| = \gamma^a\partial_a \quad\mathrm{for}\quad 
a \in \llbracket 1, \ldots, d-1 \rrbracket
\end{equation*}
We will make the above choice for the rest of the chapter. 
However, it would also be possible to show that the results that we will obtain are valid in more general case of time-like boundary in Minkowski space-time.





