\section{Solution for $M = \mathbb R \times [-L, L]$ }
We give an explicit example to see how this boundary condition can affect the solution. 
In particular, we calculte the vacuum polarization by using the method introduced in~\cref{chap-vacuum}\\\\
We choose for this example
\begin{equation*}
\gamma^0 = \begin{pmatrix} 0 & 1 \\ 1 & 0 \end{pmatrix} \quad
\gamma^1 = \begin{pmatrix} 0 & -1 \\ 1 & 0 \end{pmatrix}
\end{equation*}
We then have 
\begin{equation*}
\mathcal{P}_\pm = \begin{pmatrix} 1 & \mp i \\ \pm i & 1 \end{pmatrix}
\end{equation*}
Let $\Phi = (\phi, \phi_|)$ where $\phi = \begin{pmatrix} \phi_L \\ \phi_R \end{pmatrix}$ be a solution of~\cref{wen-maineq}. 
It is easy to verify that $\phi_L$ and $\phi_R$ are of form
\begin{equation*}
\phi_L = f e^{ik(x^0 - x^1)} \quad
\phi_R = g e^{ik(x^0 + x^1)}
\end{equation*}
where $f, g\in\mathbb C$ are constants which have to be determined. \\\\
By the discussion in~\cref{wen-subsect-saw2} and~\cref{wen-saw2bound}, $\phi$ must satisfy
\begin{equation*}
\begin{cases}
\begin{pmatrix} 1 & i \\ -i & 1 \end{pmatrix}(c^{-1} \phi - \partial_1 \phi)\vert_{x^1 = -L} = 0 \\
%
\begin{pmatrix} 1 & -i \\ i & 1 \end{pmatrix}(c^{-1} \phi + \partial_1 \phi)\vert_{x^1 = L} = 0
\end{cases}
\end{equation*}
In terms of components, this system of equations is equivalent to
\begin{equation*}
\begin{cases}
f(c^{-1} + ik)e^{ikL} + ig(c^{-1} - ik)e^{-ikL} = 0 \\
f(c^{-1} - ik)e^{-ikL} - ig(c^{-1}+ ik) e^{ikL} = 0
\end{cases}
\end{equation*}
\ie
\begin{equation}\label{wen-bound1d}
\begin{cases}
f = -ig \frac{(c^{-1} - ik)e^{-ikL}}{(c^{-1} + ik) e^{ikL}} \\
%
(c^{-1} - ik)^2 e^{-2ikL} + (c^{-1}+ik)^2 e^{2ikL} = 0
\end{cases}
\end{equation}
The second equation of~\cref{wen-bound1d} implies
\begin{equation}\label{wen-k1d1}
2kL + 2\arctan{ck} = \big( n +\frac 1 2 \big) \pi \quad\textrm{for $n \in \mathbb Z$}
\end{equation}
A solution of~\cref{wen-k1d1} should satisfy
\begin{equation}\label{wen-mode1d}
\exists n\in \mathbb{Z}\quad
kL - \big(\frac{n}{2} + \frac 1 4 \big) \pi \in \big[-\frac{\pi}{2}, \frac{\pi}{2}\big] \quad
\tan\Big( \big(\frac{n}{2}+\frac 1 4 \big)\pi -kL \Big) = ck
\end{equation}
It is easy to show that this correspond to~\cref{wen-tan}.
Since for a given $n$, we can prove that there exists a unique $k$ verifying~\cref{wen-mode1d}\footnote{
For a given $n$, 
\begin{equation*}
\tan \theta = \frac c L \Big( -\theta + \big( \frac n 2 + \frac 1 4 \big)\Big)
\end{equation*}
has a unique solution for $\theta \in \enskip] -\frac{\pi}{2}, \frac{\pi}{2}[$
}
, we denote $k_n$ for the mode corresponding to $n$.







