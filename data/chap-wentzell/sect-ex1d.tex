\section{Solution for $M = \mathbb R \times [-L, L]$ }
We give an explicit example to see how this boundary condition can affect the solution. 
In particular, we calculte the vacuum polarization by using the method introduced in~\cref{chap-vacuum}\\\\
We choose for this example
\begin{equation*}
\gamma^0 = \begin{pmatrix} 0 & 1 \\ 1 & 0 \end{pmatrix} \quad
\gamma^1 = \begin{pmatrix} 0 & -1 \\ 1 & 0 \end{pmatrix}
\end{equation*}
We then have 
\begin{equation*}
\mathcal{P}_\pm = \begin{pmatrix} 1 & \mp i \\ \pm i & 1 \end{pmatrix}
\end{equation*}
Let $\Phi = (\phi, \phi_|)$ where $\phi = \begin{pmatrix} \phi_L \\ \phi_R \end{pmatrix}$ be a solution of~\cref{wen-maineq}. 
It is easy to verify that $\phi_L$ and $\phi_R$ are of form
\begin{equation*}
\phi_L = f e^{ik(x^0 - x^1)} \quad
\phi_R = g e^{ik(x^0 + x^1)}
\end{equation*}
where $f, g\in\mathbb C$ are constants which have to be determined. \\\\
By the discussion in~\cref{wen-subsect-saw2} and~\cref{wen-saw2bound}, $\phi$ must satisfy
\begin{equation*}
\begin{cases}
\begin{pmatrix} 1 & i \\ -i & 1 \end{pmatrix}(c^{-1} \phi - \partial_1 \phi)\vert_{x^1 = -L} = 0 \\
%
\begin{pmatrix} 1 & -i \\ i & 1 \end{pmatrix}(c^{-1} \phi + \partial_1 \phi)\vert_{x^1 = L} = 0
\end{cases}
\end{equation*}
In terms of components, this system of equations is equivalent to
\begin{equation*}
\begin{cases}
f(c^{-1} + ik)e^{ikL} + ig(c^{-1} - ik)e^{-ikL} = 0 \\
f(c^{-1} - ik)e^{-ikL} - ig(c^{-1}+ ik) e^{ikL} = 0
\end{cases}
\end{equation*}
\ie
\begin{equation}\label{wen-bound1d}
\begin{cases}
f = -ig \frac{(c^{-1} - ik)e^{-ikL}}{(c^{-1} + ik) e^{ikL}} \\
%
(c^{-1} - ik)^2 e^{-2ikL} + (c^{-1}+ik)^2 e^{2ikL} = 0
\end{cases}
\end{equation}
The second equation of~\cref{wen-bound1d} implies
\begin{equation}\label{wen-k1d1}
2kL + 2\arctan{ck} = \big( n +\frac 1 2 \big) \pi \quad\textrm{for $n \in \mathbb Z$}
\end{equation}
A solution of~\cref{wen-k1d1} should satisfy
\begin{equation}\label{wen-mode1d}
\exists n\in \mathbb{Z}\quad
kL - \big(\frac{n}{2} + \frac 1 4 \big) \pi \in \big[-\frac{\pi}{2}, \frac{\pi}{2}\big] \quad
\tan\Big( \big(\frac{n}{2}+\frac 1 4 \big)\pi -kL \Big) = ck
\end{equation}
It is easy to show that this correspond to~\cref{wen-tan}.
Since for a given $n$, we can prove that there exists a unique $k$ verifying~\cref{wen-mode1d}\footnote{
For a given $n$, 
\begin{equation*}
\tan \theta = \frac c L \Big( -\theta + \big( \frac n 2 + \frac 1 4 \big)\Big)
\end{equation*}
has a unique solution for $\theta \in \enskip] -\frac{\pi}{2}, \frac{\pi}{2}[$
}
, we denote $k_n$ for the mode corresponding to $n$. \\\\
Let us study the asymptotic behavior of $k_n$. 
Denoting $\theta_n = \big( \frac n 2 + \frac 1 4 \big)\pi - kL$ and $C = \frac c L$,~\cref{wen-mode1d} becomes
\begin{equation}\label{wen-modetheta}
\tan \theta_n = C \Big( \big(\frac n 2 +\frac 1 4 \big) - \theta_n \Big)
\end{equation} 
It is not difficult to see that 
\begin{equation*}
\lim_{n\rightarrow+\infty}\theta_n = \frac \pi 2
\end{equation*}
By expanding
\begin{equation*}
\begin{split}
\tan\Big(\frac{\pi}{2} + \big(\theta - \frac \pi 2 \big) \Big) = &
- \frac{\cos\big(\theta-\frac{\pi}{2}\big)}{\sin\big(\theta-\frac{\pi}{2}\big)} \\
= & 
-\frac{1}{\theta - \frac \pi 2 } - \frac{1}{2}\big(\theta - \frac{\pi}{2}\big) + \mathcal{O}\Big(\big(\theta - \frac \pi 2 \big)^2\Big)
\end{split}
\end{equation*}
\cref{wen-modetheta} implies
\begin{equation}\label{wen-modetheta2}
-\frac{1}{\big(\theta - \frac \pi 2\big)} = 
C\Big( \big(\frac n 2 -\frac 1 4 \big) - \theta \Big) +  \frac{1}{2}\big(\theta - \frac{\pi}{2}\big) + \mathcal{O}\Big(\big(\theta - \frac \pi 2 \big)^2\Big)
\end{equation}
The following lemma will be useful
\begin{lemma}\label{wen-mode1dlem}
Let $a$ be a function of $x$ and $b \in\mathbb{R}$.
Suppose that $a $ satisfies
\begin{equation*}
-\frac 1 x = a + bx + \mathcal{O}(x^2)
\end{equation*}
Then, the following holds
\begin{equation*}
x = -\frac{1}{a-\frac 1 a}\Big( 1 - \frac{b}{a^2}\big(a-\frac 1 a\big)^{-1}x + \mathcal{O}(x^2)\Big)
\end{equation*}
\end{lemma}
\begin{proof}
\begin{equation*}
\begin{split}
-\frac 1 x  = & a + bx + \mathcal{O}(x^2) \\ 
=& a - \frac{1}{a+bx + \mathcal{O}(x^2)} + \mathcal{O}(x^2) \\
= & a - \frac 1 a + \frac{b}{a^2}x + \mathcal{O}(x^2)
\end{split}
\end{equation*}
So we have
\begin{equation*}
\begin{split}
x = & -\frac{1}{a - \frac 1 a + \frac{b}{a^2}x + \mathcal{O}(x^2)} \\
=& -\frac{1}{a-\frac 1 a}\Big( 1 - \frac{b}{a^2}\big(a-\frac 1 a\big)^{-1}x + \mathcal{O}(x^2)\Big)
\end{split}
\end{equation*}
\end{proof}
Therefore, by~\cref{wen-mode1dlem},~\cref{wen-modetheta2} reads\footnote{
Before passing to the last line of the following equation, one can easily show from~\cref{wen-modetheta2} that 
\begin{equation*}
\theta_n - \frac \pi 2 = \mathcal{O}\big(\frac{1}{n}\big)
\end{equation*}
when $n\rightarrow +\infty$
}
\begin{equation*}
\begin{split}
\theta - \frac{\pi}{ 2} = & 
-\bigg(\frac{\pi}{2} C \big( n- \frac 1 n \big) - \frac{ 2}{ \pi C \big(n - \frac{ 1 }{2} \big)} \bigg)^{-1}
\bigg( 1 - \big(\frac{ 1}{ 2} - C \big)\Big( \frac C 2 \big( n - \frac{ 1}{ 2} \big) \pi - \frac{2}{\pi C\big(n- \frac{1}{2}\big)} + \mathcal{O}\Big(\big(\theta - \frac \pi 2 \big)^2\Big)\bigg) \\
%
=& -\frac{2}{\pi C n}\Big( 1 + \mathcal{O}\big( \frac 1 n\big) \Big)
\end{split}
\end{equation*}
Hence, when $n\rightarrow +\infty$, 
\begin{equation*}
k_n = \frac 1 L \Big( \big( \frac n 2 + \frac 1 4 \big)\pi + \frac{2}{\pi C n}+ \mathcal{O}\big(\frac{1}{n^2}\big) \Big)
\end{equation*}
Two more lemmas will be needed for our numeric work.
\begin{lemma}
Let $b\in\mathbb{R}$
\begin{equation*}
\sum_{n=0}^\infty e^{i(n+\frac b n)z} = \frac i z - \frac 1 2 + \mathcal{O}(z^{\frac 1 2})
\end{equation*}
when $z\rightarrow 0 $
\end{lemma}
\begin{proof}
\begin{equation*}
\begin{split}
\sum_{n=0}^{+\infty} e^{i(n+\frac b n)z} = & 
\sum_{n=0}^{+\infty} \sum_{k=0}^{+\infty} e^{inz} \frac{1}{k!}\big(\frac{ib}{n}z\big)^k \\
%
= & \sum_n e^{inz} + \sum_n \frac{ibz}{n}e^{inz} + \sum_{n=0}^{+\infty} \sum_{k=2}^{+\infty} e^{inz} \frac{1}{k!}\big(\frac{ib}{n}z\big)^k  \\
%
= & \sum_n e^{inz} + \sum_n \frac{ibz}{n}e^{inz} - 
\bigg(\sum_{n=0}^{+\infty} \sum_{k=0}^{+\infty} 
e^{inz} \frac{1}{n^2(k+2)!}\big(\frac{ib}{n}z\big)^k 
\bigg)b^2z^2
\end{split}
\end{equation*}
The last term of the last line is normally convergent.
We conclude the proof with the known equalities
\begin{equation*}
\begin{split}
& \sum_{n=0}^{+\infty}e^{inz} = 
\frac i z - \frac 1 2 - \frac{iz}{12} + \mathcal{O}(z^2) \\
%
& \sum_{n=0}^{+\infty}\frac 1 n e^{inz} =
\frac{i\pi}{2} - \ln z - \frac{iz}{2} + \mathcal{O}(z^2)
\end{split}
\end{equation*}
\end{proof}
%
\begin{lemma}
Let $(a_n)$ be a sequence such that $a_n = \mathcal{O}\big(\frac{1}{n^2})$ when $n\rightarrow+\infty$ and $b\in\mathbb{R}$. Then
\begin{equation*}
\sum_{n=0}^{+\infty} e^{i(n+\frac{b}{n}+a_n)z } - e^{i(n+\frac{b}{n})z }
\end{equation*}
converges uniformly.
\end{lemma}
\begin{proof}
Since $a_n = \mathcal{O}\big(\frac{1}{n^2}\big)$, 
there exists $ K > 0$ such that $ \exists N \in\mathbb{N}, \forall n\geq N, \| a_n\| \leq \frac{K}{n^2}$ \\\\
Let $m,l\in\mathbb{N}$ such that $l > m \geq N$
\begin{equation*}
\begin{split}
\sum_{n=m}^{l}  | e^{i(n+\frac{b}{n}+a_n)z } - e^{i(n+\frac{b}{n})z } | = & 
\sum_{n=m}^{l} | e^{ia_n z} - 1 | \\
%
\leq & \sum_{n=m}^{l}\sum_{k=1}^{+\infty}\frac{1}{k!}| a_n z |^k \\
%
\leq & \sum_{n=m}^l \sum_{k=1}^{+\infty}\frac{1}{k!n^2}(Kz)^k
\end{split}
\end{equation*}
Since the last line is a normally convergent series for $z \in \mathbb{R}$, it could be made small by taking $m$ and $l$ big enough.
Hence, the series of our lemma converges normally, which implies that it converges uniformly.
\end{proof}
Now, let us compute the vacuum polarization, given by~\cref{vacuum-nefvacpol}.
As one can find in the discussion in~\cref{chap-vacuum}, the Hadamard parametrix of in the bulk remains the same. 
The divergent terms when calculating the vacuum polarization are given by~\cref{vacuum-nefhadamard}. \\\\
From \cref{wen-bound1d} and the normalization condition, we deduce $|f| = |g| = \frac 1 2$.
By adapting time-splitting $x^0 - y^0 = t$, the Hadamard states are
\begin{equation*}
\omega(\phi^1(x)\phi^\dagger_1(y)) = \omega(\phi^2(x)\phi^\dagger_2(y)) = 
\frac{1}{4L}\sum_{k_n>0} e^{-ik_nt}
\end{equation*} 
%where to begin?
%plus 1/2 in the two previous lemmas and review the constant term
Noticing that 
\begin{equation*}
\frac{ 1}{ 4L}\sum_{n=0}^{+\infty} e^{-i\frac{1}{2L}(n+\frac 1 2 ) \pi t} 
= \frac{-i}{2\pi t} + \frac{1}{4L} + \mathcal{O}(t)
\end{equation*}
has the same divergence as the Hadamard parametrix~\cref{chap-nefhadamard}, the renormalized vacuum polarization can be given by
\begin{equation*}
\rho =\lim_{t\rightarrow 0} e \bigg(\frac{1}{4L}\sum_{k_n>0} e^{-ik_nt} 
- \frac{ 1}{ 4L}\sum_{n=0}^{+\infty} e^{-i\frac{1}{2L}(n+\frac 1 2 ) \pi t} + 1 \bigg)
\end{equation*}















