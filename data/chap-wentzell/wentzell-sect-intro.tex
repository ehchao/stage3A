\section{Introduction}
The AdS (anti-deSitter)/CFT (conformal field theory) correspondance has been a very active field of research in theoretical physics these years. 
This principle is also called holography.
%add ref??
%Maldacena's paper?
The main idea of the theory is to study the duality between the string theory in an AdS space-time bulk and the quantum field theory living on the boundary of the AdS space-time.
Another interesting application of the AdS/CFT correspondance is to study the strong/weak coupling duality, \ie
a strong coupling of the quantum field theory on the boundary corresponds to a weak coupling of the string theory in the bulk.
For instance,~\cite{Skenderis2002} gives examples to calculate renormalized correlation functions\footnote{
See~\eg\cite{Peskin1995} for the definition.
} of the quantum field theory by performing computations on the gravity side.
The strong couplings in QCD might be better understood in using the holography principal. 
Based on the observation of the IR-UV connection~\cite{Susskind1998}, 
\ie the fact that the ultraviolet divergence of the boundary theory corresponds to the infrared divergence of the bulk theory, 
we can renormalize a theory by undergoing the holographic renormalization procedure~\cite{Skenderis2002}. \\\\
%
However, one can wonder how the boundary field should be built in order to ensure that the two-point function constructed on the boundary has the behaviors that correspond to the bulk field. 
\cite{Zahn2016} proposes thus an additional boundary action for a scalar field, namely, by considering the following action
\begin{equation*}
\mathcal{S} = \mathcal{S}_{\mathrm{bulk}} + \mathcal{S}_{\mathrm{bdy}} = 
-\frac 1 2 \int_M g^{\mu\nu} \partial_\mu \phi \partial_{\nu} + 
\mu^2\phi^2 - \frac c 2 \int_{\partial M}h^{\mu\nu}\partial_\mu\phi\partial_\nu\phi + \mu^2\phi^2
\end{equation*}
and studies the temporal evolution and the quantization of the field.
Indeed, the boundary term appears as a counter-term in the holographic renormalization theory~\cite{Skenderis2002}.
In this report, we will try to extend this kind of studies to the Dirac field case by proposing an action and studying the dynamics of the Dirac field under the induced boundary condition.
Before showing the main studies, 
we give a brief description of the AdS/CFT correspondance by using the example of a massless scalar field from~\cite{Witten1998} and~\cite{Skenderis2002}. 
%
\paragraph{AdS/CFT correspondance in scalar field case}
The metric of the $(d+1)$-AdS space-time can be identified as the open unit ball $B_{d+1}$ in a Euclidean space $\mathbb{R}^{d+1}$ with coordinates $y_0, \ldots, y_d$ such that $\sum_{i=0}^d y_i^2 <1$ with the metric
\begin{equation*}
ds^2 = \frac{4\sum_{i=0}^d dy_i^2}{(1 - |y|^2)^2}
\end{equation*}
We can include the infinity boundary by taking the closure of the unit ball $B_{d+1}$. 
As one can notice, the singularity of the metric at infinity ($|y|^2 = 1$) can be resolved by studying the conformal transformation of the metric, 
\ie by replacing $ds^2$ by $d\tilde{s}^2$ which is defined by
\begin{equation*}
d\tilde{s}^2 = (1 - |y|^2) ds^2
\end{equation*}
We consider a field theory with the action of a massless scalar field $\phi$
\begin{equation}\label{wen-adscft1}
\mathcal{S}[\phi] = \frac 1 2 \int_{B_{d+1}} \dd^{d+1} y \sqrt{g} |\dd \phi|^2
\end{equation}
where $g$ is the absolute value of the determinant of the metric tensor.
By doing integration by part to~\cref{wen-adscft1}, we find
\begin{equation*}
\mathcal{S}[\phi] = -\int_{B_{d+1}} \sqrt{g} \phi D_i D^i \phi + 
\lim_{\epsilon\rightarrow 0}\int_{T_\epsilon}  \sqrt{h} \phi (\vec{n}\cdot\vec{\nabla})\phi
\end{equation*}
where $T_\epsilon$ is a family of surfaces parametrized by $\epsilon$ and converges to the boundary of $B_{d+1}$ when $\epsilon\rightarrow 0$ and $h$ is the absolute value of the determinant of the induced metric on the boundary.
The first term of the \rhs vanishes on-shell and the action can be written in terms of the boundary field $\phi_0$
\begin{equation*}
\mathcal{S}[\phi] \sim \int \dd \mathbf{x} \dd \mathbf{x}' 
\frac{\phi_0(\mathbf{x})\phi_0(\mathbf{x}')}{|\mathbf{x} - \mathbf{x}'|^{2d}}
\end{equation*}
Consider a field $\mathcal{O}$ whose source is the boundary field $\phi$. 
It is defined by~\cite{Gubser1998}
\begin{equation*}
\mathcal{S}[\phi] = \int \dd \mathbf{x} \phi_0(\mathbf{x})\mathcal{O}(\mathbf{x})
\end{equation*}
It could then be shown by the prescription of the section 2 of~\cite{Skenderis2002} that the two-point function of the operator $\mathcal{O}$ is a multiple of $|\mathbf{x} - \mathbf{x}'|^{-2d}$, which agrees with the two-point function of the $d$-dimensional conformal field theory~\cite{Qualls2015}. \\\\
%
%











