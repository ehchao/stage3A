\subsection{Bulk $M = \mathbb{R}^{d-1} \times [-L, L]$ }
The subtlety that one should beware of in this case is the projectors $\mathcal{P}_+$.
Since $\mathcal{P}_+$ is defined by considering a set of generators of Clifford Algebra on the tangent bundle $T_{ \partial M}$, 
if we would like to fixe the system of coordinates once for all,
we should also take into account the transformation of these generators. \\
Let $(x^0, \ldots, x^d)$ a system of coordinate of the bulk $M$. 
Because of the geometric configuration of the bulk, 
this system of coordinate is global. 
The boundary is therefore $\partial M = \{x^d = -L \} \cup \{ x^d = L \}$.
Furthermore, we  \\
We denote $(\gamma^\mu)_\mu$ the set of gamma matrices chosen on $\{x^d  = - L \}$. 
We will have then $\mathcal{P}_+\vert_{\{x^d = L\}} = \mathcal{P}_-\vert_{\{x^d = -L\}}$ and $\gamma^\mu\vert_{\{x^d = L\}}=\gamma^\mu\vert_{\{x^d = -L\}}$ for $\mu\in\llbracket 0, d-1 \rrbracket$
When it is not specified, we refer to the gamma matrices and the projectors on $\{x^d = -L\}$ \\
Now, for $\Phi = (\phi, \phi_|)$ the boundary condition becomes
\begin{equation*}
\begin{split}
\begin{cases}
\mathcal{P}_+ \phi\vert_{\{x^d = -L\}} = \mathcal{P}_+\phi_| \\
c^{-1}\mathcal{P}_-\phi\vert_{\{x^d = -L\}} = \mathcal{P}_-(\partial_d \phi\vert_{x^d = -L}) \\
\mathcal{P}_- \phi\vert_{\{x^d = L\}} = \mathcal{P}_-\phi_| \\
-c^{-1}\mathcal{P}_+\phi\vert_{\{x^d = L\}} = \mathcal{P}_+(\partial_d \phi\vert_{x^d = L}) \\
\end{cases}
\end{split}
\end{equation*}
As \cref{wen-subsect1}, the general solution $\phi$ in the bulk can be written as
\begin{equation*}
\phi 
 =\Big( (k \mathbb{1}+ \gamma^0\gamma^a p_a )(A \cos p_d x^d + B \sin p_d x^d)e^{-ip_a x^a}
+ i\gamma^0\gamma^d p_d (-A \sin p_d x^d + B \cos p_d x^d) e^{-ip_a x^a} \Big) \psi
 \end{equation*}
 where $\psi $ is a spinor and the Einstein summation over $a = 1, \ldots, d-1$ is applied. \\
 We compute
\begin{equation*}
\begin{split}
\partial_d \phi\vert_{x^d = \pm L }
= (\gamma^0\gamma^a p_a + k\mathbb{1})(\mp \sin p_d L + p_d B \cos p_d L)
+ i\gamma^0\gamma^d p_d(-p_d A \cos p_d L \mp p_d B \sin p_d L)\psi
\end{split}
\end{equation*}
Hence, the boundary condition leads to
\begin{equation*}
\begin{split}
0 = &\mathcal{P}_-\Bigg(
(\gamma^0\gamma^a p_a + k\mathbb{1})\Big((p_d A + c^{-1}B) \sin p_d L + (p_d B - c^{-1}A) \cos p_d L\Big) \\
& + i\gamma^0\gamma^d p_d \Big((-p_d A -c^{-1} B) \cos p_d L + (p_d B -c^{-1} A )\sin p_d L\Big) 
\Bigg)\psi
\end{split}
\end{equation*}
\begin{equation*}
\begin{split}
0 = &\mathcal{P}_+\Bigg(
(\gamma^0\gamma^a p_a + k\mathbb{1})\Big((- p_d A - c^{-1}B) \sin p_d L + (p_d B + c^{-1}A) \cos p_d L\Big) \\
& + i\gamma^0\gamma^d p_d \Big((-p_d A + c^{-1} B )\cos p_d L - (p_d B + c^{-1} A )\sin p_d L\Big) 
\Bigg)\psi
\end{split}
\end{equation*}
Once again, there are 4 possible cases to be discussed as in \cref{wen-subsect1}










