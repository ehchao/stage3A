\section{Well-posedness of the problem}\label{wen-sect-saw}
We define the inner product for our Hilbert space $\mathcal{H} = W^{1,2}(M)\times W^{1,2}(\partial M)$,
where $M$ is an equal-time hypersurface of $\mathcal{M}$ and $\partial M = M\cap \partial \mathcal{M}$
\begin{equation}\label{wen-innerpdt}
\langle \cdot, \cdot \rangle _\mathcal{H} = \langle \cdot, \cdot \rangle _{L^2(M)} + c \langle \cdot, \cdot \rangle _{L^2(\partial M)}
\end{equation}
In order to have a self-adjoint operator, we define $\Delta$ on the domain
\begin{equation*}
\dom( \Delta) = \{ \Phi = (\phi, \phi_|) \in \mathcal{H} \enskip | \enskip \mathcal{P}_+\big(\phi \vert_{\partial M} - \phi_|\big) = 0 \}
\end{equation*}  
For $\Phi = (\phi, \phi_|) \in \mathcal{H}$ and $ \Psi = (\psi, \psi_|)\in\dom(\Delta)$
\begin{equation}
\begin{split}
\langle \Phi, \Delta \Psi \rangle _\mathcal{H}
 = & \int_M \phi^\dagger i \gamma^0 \gamma^j \partial_j \psi 
+ \int_{\partial M} \phi^\dagger_|(  -\gamma^0\mathcal{P}_- \psi\vert_{\partial M} + ic \gamma^0 \gamma^a \partial_a \mathcal{P}_+ \psi_|)   \\
 = & - \int_M \partial_j \phi^\dagger i \gamma^0 \gamma^j \psi 
+ \int_{\partial M} \phi^\dagger_|(-\gamma^0 \mathcal{P}_- \psi\vert_{\partial M} + ic \gamma^0 \gamma^a \partial_a \mathcal{P}_+ \psi_|) 
- i\phi\vert_{\partial M}^\dagger \gamma^0 \gamma^\bot \psi\vert_{\partial M}   \\
= &
- \int_M \partial_j \phi^\dagger i \gamma^0 \gamma^j \psi 
+ \int_{\partial M} - \phi^\dagger\vert_{\partial M}\mathcal{P}_- \gamma^0 \psi_| + ic \phi^\dagger_|\gamma^0 \gamma^a \partial_a \mathcal{P}_+ \psi_|  \\
& - i \phi\vert_{\partial M}^\dagger \gamma^0 \gamma^\bot \psi\vert_{\partial M} 
-\phi_|^\dagger \gamma^0 \mathcal{P}_- \psi_{\partial M} 
+ \phi^\dagger\vert_{\partial M}\mathcal{P}_- \gamma^0 \psi_| \\
\underset{(\mathcal{P}_-)^\dagger = \mathcal{P}_-}{=} 
& \langle \Delta\Phi, \Psi \rangle_\mathcal{H}
+\phi^\dagger\vert_{\partial M} \gamma^0 (-i \gamma^\bot \psi\vert_{\partial M} + \mathcal{P}_+ \psi_|)
- \phi_|^\dagger \gamma^0 \mathcal{P}_- \psi\vert_{\partial M} \\
= & \langle \Delta\Phi, \Psi \rangle_\mathcal{H}
+ (\phi^\dagger\vert_{\partial M} - \phi_|^\dagger)\mathcal{P}_+ \gamma^0 \psi\vert_{\partial M}
\end{split}
\end{equation}
The residual term of the last line vanishes if and only if $\mathcal{P}_+\big(\phi \vert_{\partial M} - \phi_|\big) = 0$, which shows the self-adjointness of $\Delta$. \\\\
One can wonder if the problem has unique smooth solution which propagates causally, \ie depending smoothly on the initial data of a Cauchy surface. \\
Let us determine $\dom (\Delta^k)$ now. 
To start, for $k =2$, 
\begin{equation}\label{wen-deltak}
\begin{split}
\Delta^2 \Phi & =  
\begin{pmatrix} i \gamma^0 \gamma^j \partial_j & 0 \\
- c^{-1} \gamma^0 \mathcal{P}_- \cdot \vert_{\partial M} & i \gamma^0 \gamma^a \partial_a \mathcal{P}_+ \end{pmatrix}
\begin{pmatrix}   i \gamma^0 \gamma^j \partial_j \phi \\
- c^{-1} \gamma^0 \mathcal{P}_- \phi\vert_{\partial M} + i \gamma^0 \gamma^a \partial_a \mathcal{P}_+ \phi_| \end{pmatrix} \\
&= 
\begin{pmatrix} (i\gamma^0\gamma^j\partial_j)^2 \phi \\
-c^{-1}\gamma^0\mathcal{P}_- \big( (i\gamma^0\gamma^j\partial_j \phi)\vert_{\partial M}\big)
-c^{-1}i \gamma^0\gamma^a\partial_a \mathcal{P}_+ \gamma^0 \mathcal{P}_-(\phi\vert_{\partial M})
+ (i\gamma^0\gamma^a\partial_a\mathcal{P}_+)^2\phi_| \end{pmatrix}
\\
\end{split}
\end{equation}
One finds that for $\Phi \in \dom(\Delta)$, $\Delta \Phi \in \dom \Delta$ if and only if
\begin{equation*}
\begin{split}
&\mathcal{P}_+\Big( ( i\gamma^0\gamma^j \partial_j \phi ) \vert_{\partial M} +
 c^{-1}\gamma^0 \mathcal{P}_- ( \phi \vert_{\partial M})
- i \gamma^0 \gamma^a \partial_a \mathcal{P}_+\phi _| \Big) \\
\underset{\Phi\in \dom(\Delta)}{ = }& 
\mathcal{P}_+\Big( (i\gamma^0\gamma^\bot \partial_\bot\phi)\vert_{\partial M} +
c^{-1}\gamma^0 \mathcal{P}_- ( \phi \vert_{\partial M}) \Big) \\ 
= & 
\gamma^0\mathcal{P}_-\big(- \partial_\bot \phi \vert_{\partial M} + c^{-1} \phi\vert_{\partial M}\big)
\\ = & 0
\end{split}
\end{equation*}
The fact that $\partial M = \{x^\bot = 0\}$ is used to coincide the tangential derivatives evaluated on the static boundary.
Continuing the calculation of \cref{wen-deltak}, we get
\begin{equation}\label{wen-delta2}
\Delta^2 \Phi = \begin{pmatrix} (i \gamma^0 \gamma^j\partial_j)^2 & 0 \\ 
- c^{-1}\mathcal{P}_+( \partial_\bot \cdot\vert_{\partial M} ) & 
(i \gamma^0 \gamma^a \partial_a)^2 \mathcal{P}_+ \end{pmatrix}
\begin{pmatrix} \phi \\ \phi_| \end{pmatrix}
\end{equation}
with domain 
\begin{equation*}
\dom(\Delta^2) = \Big\{ (\phi, \phi_|) \in W^{2,2}(M)\times W^{2,2}(\partial M) \big\vert\enskip \mathcal{P}_- \Big(\partial_\bot\phi\vert_{\partial M} - c^{-1} (\phi\vert_{\partial M}) \Big) = 0 \Big\} \cap \dom(\Delta)
\end{equation*}
For $k = 3$, 
we check the domain
\begin{equation*}
\dom(\Delta^3) = \big\{ (\phi, \phi_|)\in W^{3,2}(M)\times W^{3,2}(\partial M) \big\vert \enskip \mathcal{P}_+(\partial_\bot^2 \phi\vert_{\partial M} - c^{-1}\partial_\bot \phi\vert_{\partial M} ) = 0 \big\} 
\cap \dom(\Delta^2)
\end{equation*}
We compute (with our signature convention)
\begin{equation*}
\begin{split}
\Delta^3 \Phi  = &
\begin{pmatrix} i\gamma^0\gamma^j\partial_j & 0 \\ 
-c^{-1}\gamma^0 \mathcal{P}_- \cdot \vert_{\partial M} & i\gamma^0\gamma^a\partial_a \mathcal{P}_+ \end{pmatrix} 
\begin{pmatrix} (i\gamma^0\gamma^j\partial_j)^2 \phi \\
-c^{-1}\mathcal{P}_+ (\partial_\bot \phi\vert_{\partial M}) + (i\gamma^0\gamma^a\partial_a)^2 \mathcal{P}_+ \phi_| \end{pmatrix} \\ 
= & \begin{pmatrix} i\gamma^0\gamma^j \partial_j& 0 \\ 
-c^{-1}\gamma^0 \mathcal{P}_- \cdot \vert_{\partial M} & i\gamma^0\gamma^a\partial_a \mathcal{P}_+ \end{pmatrix} 
\begin{pmatrix} -(\partial_j)^2 \phi \\
-c^{-1}\mathcal{P}_+ (\partial_\bot \phi\vert_{\partial M}) - (\partial_a)^2 \mathcal{P}_+ \phi_| \end{pmatrix} \\ 
= &
\begin{pmatrix} -  i\gamma^0\gamma^j\partial_j(\partial_l^2 \phi ) \\
c^{-1}\gamma^0\mathcal{P}_-(\partial_j^2\phi\vert_{\partial M}) - ic^{-1}\gamma^0\gamma^a\partial_a \mathcal{P}_+(\partial_\bot \phi\vert_{\partial M})  -i\gamma^0\gamma^a\partial_a(\partial_b^2 \mathcal{P}_+\phi_| )\end{pmatrix} \\ 
\underset{\Phi \in \dom(\Delta^3)}{= }&
 \begin{pmatrix} - i\gamma^0\gamma^j\partial_j(\partial_l^2 \phi ) \\
c^{-1}\gamma^0\mathcal{P}_-(\partial_j^2\phi\vert_{\partial M}) 
- i\gamma^0\gamma^a\partial_a \mathcal{P}_+(\partial^2_\bot \phi\vert_{\partial M}) - i\gamma^0\gamma^a\partial_a(\partial_b^2 \mathcal{P}_+\phi_| )\end{pmatrix} \\ 
& = \begin{pmatrix} -i\gamma^0\gamma^j\partial_j (\partial_l)^2  & 0\\
c^{-1}\gamma^0 \mathcal{P}_- \big( (\partial_j^2 \cdot\vert_{\partial M}) - i\gamma^0\gamma^a\partial_a\partial_\bot\cdot\vert_{\partial M} \big)
 & -i\gamma^0\gamma^a\partial_a(\partial_b^2 \mathcal{P}_+)\end{pmatrix}
\begin{pmatrix}
\phi \\ \phi_|
\end{pmatrix}
\end{split}
\end{equation*}
In fact, we have the following proposition 
\begin{proposition}
Let $k \in \mathbb{N}^*$. Then
\begin{equation*}
\begin{split}
& \dom (\Delta^{2k} )  =
\big\{ (\phi, \phi_|)\in W^{2k,2}(M)\times W^{2k,2}(\partial M) \big\vert \enskip \mathcal{P}_-\big( \partial_\bot^{2k-1}\phi\vert_{\partial M} - c^{-1}\partial_\bot^{2k-2}\phi\vert_{\partial M}\big) \big\} 
\cap \dom(\Delta^{2k - 1})  \\
& \dom (\Delta^{2k+1} )  =
\big\{ (\phi, \phi_|)\in W^{2k+1,2}(M)\times W^{2k+1,2}(\partial M) \big\vert \enskip \mathcal{P}_+\big( \partial_\bot^{2k}\phi\vert_{\partial M} - c^{-1}\partial_\bot^{2k-1}\phi\vert_{\partial M}\big) \big\}  \cap \dom(\Delta^{2k })
\end{split}
\end{equation*}
\end{proposition}
\begin{proof}
Prove the proposition by induction.
The claim holds for $k =1$ as we have already shown. 
Let $k \in \mathbb{N}$ and suppose that the claim holds for $\forall n \in \llbracket 1, k-1 \rrbracket $. 
It is easy to show that $\Delta^{2m}$ takes the form
\begin{equation*}
 \Delta^{2m}  = 
\begin{pmatrix} (-\partial_j^2)^m & 0 \\
\star & (-\partial_a^2)^m\mathcal{P}_+ \end{pmatrix}
\end{equation*}
For $\Phi = (\phi, \phi_|)$
\begin{equation*}
\begin{split}
\Phi \in \dom(\Delta^{2k}) \quad &\Leftrightarrow \quad
\Phi \in W^{2k,2}(M)\times W^{2k,2}(\partial M) \enskip \mathrm{and}\enskip \Delta^{2k-2}\Phi \in \dom(\Delta^{2}) \\
&\Leftrightarrow \quad 
\mathcal{P}_{-}\big(\partial_\bot(-\partial_j^2)^{k-1}\phi\vert_{\partial M} - c^{-1}(-\partial_j^2)^{k-1}\phi\vert_{\partial M} \big) = 0
\end{split}
\end{equation*}
and
\begin{equation*}
\begin{split}
\Phi \in \dom(\Delta^{2k+1}) \quad &\Leftrightarrow \quad
\Phi \in W^{2k+1,2}(M)\times W^{2k+1,2}(\partial M) \enskip \mathrm{and}\enskip \Delta^{2k-2}\Phi \in \dom(\Delta^{3}) \\
&\Leftrightarrow \quad 
\mathcal{P}_{+}\big(\partial_\bot^2(-\partial_j^2)^{k-1}\phi\vert_{\partial M} - c^{-1}\partial^\bot(-\partial_j^2)^{k-1}\phi\vert_{\partial M} \big) = 0
\end{split}
\end{equation*}
As the lower powers coincide by hypothesis, 
the claim still holds for $k$.
\end{proof}
Consider now a subspace
\begin{equation*}
\mathcal{K} = \cap_{k\in \mathbb{N}^*} \dom(\Delta^k)
\end{equation*}
and an element $\Phi \in \mathcal{K}$.
The time evolution of the system is given by
\begin{equation*}
i \frac{\partial }{\partial t} \Phi = \Delta \Phi 
\end{equation*}
Taking into account the fact that $\Phi\in\dom(\Delta^2)$ one gets
\begin{equation*}
\begin{split}
\big\| \frac{\partial }{\partial t} \Phi \big\|^2_\mathcal{H} = \| \Delta \Phi \|^2_{\mathcal{H}}  =
& - \int_M \partial^j \phi^\dagger \partial_j \phi 
+ \int_{\partial M}\phi\vert_{\partial M} \partial_\bot \phi\vert_{\partial M}
- \int_{\partial M} \phi_|^\dagger \mathcal{P}_+(\partial_\bot \phi\vert_{\partial M})
- c\int_{\partial M} \partial^a \mathcal{P}_+ \phi^\dagger_| \partial_a \phi_| \\
\underset{\Phi \in \dom(\Delta^2)}{=} & - \int_M \partial^j \phi^\dagger \partial_j \phi
+c^{-1}\int (\mathcal{P}_- \phi\vert_{\partial M})^\dagger \mathcal{P}_- \phi\vert_{\partial M} 
- c\int_{\partial M} \partial^a \mathcal{P}_+ \phi^\dagger_| \partial_a \phi_| \\
= & \|\partial_j \phi \|^2_{L^2 (M)} + c^{-1} \| \mathcal{P}_- \phi\vert_{\partial M} \|^2_{L^2(\partial M)}
+ c \| \mathcal{P}_+ \partial_a \phi\vert_{\partial M} \|^2_{L^2(\partial M)}
\end{split}
\end{equation*}
and
\begin{equation*}
\begin{split}
\big\| \frac{\partial^2 }{\partial t^2} \Phi \big\|^2_\mathcal{H} = 
\| \Delta^2 \Phi \|^2_{\mathcal{H}} 
= &  \int_M (\partial_j^2) \phi^\dagger (\partial_j^2) \phi
+ c \int_{\partial M} \big| -c^{-1} \mathcal{P}_+\partial_\bot \phi\vert_{\partial M} - (\partial_a^2) \mathcal{P}_+\phi_| \big|^2  \\
\underset{\Phi \in \dom(\Delta^3)}{=} & \int_M (\partial_j^2) \phi^\dagger (\partial_j^2) \phi
+ c \int_{\partial M} \big|  \mathcal{P}_+\partial^2_\bot \phi\vert_{\partial M} + (\partial_a^2) \mathcal{P}_+\phi_| \big|^2 \\
= & \|(\partial^2_j) \phi \|^2_{L^2 (M)} 
+ c \| \mathcal{P}_+ (\partial_j^2) \phi\vert_{\partial M} \|^2_{L^2(\partial M)}
\end{split}
\end{equation*}
Hence, we have the following relation for high order time derivatives for $m\in \mathbb{N}^*$
\begin{equation*}
\begin{split}
& \big\|\frac{\partial^{2m+1}}{\partial t^{2m+1}} \Phi \big\|^2 = 
\|\partial_j (\partial_l^2)^{m}\phi \|^2_{L^2 (M)} 
+ c^{-1} \| \mathcal{P}_- (\partial_l^2)^{m}\phi\vert_{\partial M} \|^2_{L^2(\partial M)}
+ c \| \mathcal{P}_+ \partial_a (\partial_l^2)^{m}\phi\vert_{\partial M} \|^2_{L^2(\partial M)}
\\
& \big\|\frac{\partial^{2m+2}}{\partial t^{2m+2}} \Phi \big\|^2 = 
 \|(\partial^2_j)(\partial_l^2)^{m} \phi \|^2_{L^2 (M)} 
+ c \| \mathcal{P}_+ (\partial_j^2) (\partial_l^2)^{m}\phi\vert_{\partial M} \|^2_{L^2(\partial M)}
\end{split}
\end{equation*}
Noticing that
\begin{equation*}
\big\|\partial_t^m \Phi(t) \big\|= 
\big\|\Delta^m\Phi(t) \big\| =
\big\|\Delta^m \Phi(0) \big\|
\end{equation*}
we would like to prove the following proposition as in~\cite{Zahn2016}
\begin{proposition}\label{wen-propcau}
Let $\Sigma_0$ and $\Sigma_1$ two equal-time surfaces. 
Then, for $\Phi = (\phi, \phi_|) \in \mathcal{K}$ any smooth solution of \cref{wen-maineq} for smooth initial data $(\phi_0, \phi_{|0}) \in C^\infty(M) \times C^\infty(\partial M)$ on a Cauchy surface $\mathcal{S}_0 \in \Sigma_0$,
the following relation holds
\begin{equation}\label{wen-causal}
\big\| \frac{\partial}{\partial t} \Phi \big\|_{\mathcal{H}(\mathcal{S}_1)}^2
\leq 
\big\| \frac{\partial}{\partial t} \Phi \big\|_{\mathcal{H}(\mathcal{S}_0)}^2
\end{equation}
for any $\mathcal{S}_1 \in D^+(\mathcal{S}_0)\cap\Sigma_1$, where $D^+(\mathcal{S}_0)$ is the future domain of dependence of $\mathcal{S}_0$.
\end{proposition}
\begin{proof}
As in~\cite{Zahn2016} we introduce the following quantities
\footnote{
These quantities are not tensors because the $\phi$s are spinor fields. 
Under all change of frame of reference
\begin{equation*}
T_{\mu'\nu'} \neq \Lambda^\mu_{\enskip \mu'}\Lambda^\nu_{\enskip \nu'}T_{\mu \nu} 
\end{equation*}
where $\Lambda$ is the element of the Lorentz group corresponding to the change of frame.
However, 
this will not affect our proof since we are not going to use tensor properties directly in the following.
}
\begin{equation*}
\begin{split}
& T_{\mu\nu} = \partial_{(\mu} \phi^\dagger \partial_{\nu)} \phi - \frac{1}{2}g_{\mu\nu} \partial_\lambda\phi^\dagger\partial^\lambda\phi  \\
& T_{|\alpha\beta} = c\Big( \partial_{(\alpha}\mathcal{P}_+\phi^\dagger_| \partial_{\beta)}\mathcal{P}_+\phi^\dagger_| - 
\frac{1}{2}h_{\alpha\beta}\big( \partial_\gamma\mathcal{P}_+\phi^\dagger_| \partial^\gamma\mathcal{P}_+\phi_|
 - c^{-2}|\mathcal{P}_- \phi\vert_{\partial M}|^2 \big)\Big) 
\end{split}
\end{equation*}
As $\phi$ is a solution for the first equation of \cref{wen-maineq},
\begin{equation*}
(i\partial_0 + i\gamma^0\gamma^j\partial_j)( i\partial_0 -i\gamma^0\gamma^j\partial_j)\phi  = 
\Box \phi= 0
\end{equation*}
where $\Box = \partial^\mu\partial_\mu$ is the d'Alembertian.
On the other hand, 
\begin{equation*}
\begin{split}
&0 = (i\gamma^\mu\gamma^0\partial_\mu\phi)^\dagger
= - i\partial_j\phi^\dagger\gamma^j\gamma^0 - i\partial_0\phi^\dagger. \\
\Rightarrow \quad &
\phi^\dagger(i\overleftarrow{\partial}_j\gamma^j\gamma^0 - i\overleftarrow{\partial}_0)
(i\overleftarrow{\partial}_j\gamma^j\gamma^0 + i\overleftarrow{\partial}_0)
= \phi^\dagger \overleftarrow{\Box} = 0
\end{split}
\end{equation*}
$T_{\mu\nu}$ is thus conserved on shell. \\\\
From the expression of $\Delta^2$ \cref{wen-delta2}, 
\begin{equation*}
\Box\mathcal{P}_+ \phi_| = c^{-1}\mathcal{P}_+\phi\vert_{\partial M}
\end{equation*}
which leads to
\begin{equation*}
\begin{split}
\partial^\alpha T_{|\alpha \beta} = & 
 \mathcal{P}_+(\partial_{(\bot} \phi^\dagger\vert_{\partial M})\partial_{\beta)}\mathcal{P}_+ \phi_| + 
c^{-1}(\mathcal{P}_-\phi^\dagger\vert_{\partial M}) \partial_\beta\mathcal{P}_- \phi\vert_{\partial M}
+c^{-1}(\mathcal{P}_-\partial_\beta\phi^\dagger\vert_{\partial M}) \mathcal{P}_- \phi\vert_{\partial M} \\ 
\underset{\Phi \in \dom(\Delta^2)}{=} & 
\partial_{(\bot}\phi^\dagger\vert_{\partial M} \partial_{\beta)}\phi\vert_{\partial M} \\
= & T\vert_{\partial M,{\bot\beta}}
\end{split}
\end{equation*}
on-shell. \\\\
One can easily verify that $T_{\mu\nu}$ and $T_{| \alpha\beta}$ satisfy the dominant energy condition, \ie for a time-like vector $\xi$ \footnote{
Even though these are not tensors, their positivities could still be ensured by the presence of the products of conjugate quantities.
}
\begin{equation*}
\begin{split}
T_{\mu\nu} \xi^\mu \xi^\nu \geq 0 \quad \mathrm{ and }\quad
T_{\alpha\beta} \xi^\mu \xi^\nu \geq 0 
\end{split}
\end{equation*}
Let $\xi = e_0$, which is a Killing vector field in the static case.
Then we have $\partial^\mu T_{\mu\nu}\xi^\nu = 0$.
Using Stokes' theorem, the integral of this quantity over (cf.~\cite{Zahn2016} for the figure for the moment) %TODO add picture
gives
\begin{equation*}
\begin{split}
0 \underset{\partial^\bot = -\partial_\bot}{=} &
- \int_{\mathcal{S}_0}T_{00} + \int_{\mathcal{S}_1} T_{00} + \int_{\mathcal{S}_2} n^\mu T_{\mu 0} + \int_{\partial M} T \vert_{\partial M, \bot 0}   \\
= & - \int_{\mathcal{S}_0} T_{00} + \int_{\mathcal{S}_1} T_{00} + 
 \int_{\mathcal{S}_2} n^\mu T_{\mu 0} -
  \int_{\partial M \cap \mathcal{S}_0} T_{|00}  +\int_{\partial M \cap \mathcal{S}_1} T_{|00}
 + \int_{\partial M \cap \mathcal{S}_2} s^\alpha T_{\alpha 0}
\end{split}
\end{equation*}
where $n^\mu$ and $s^\alpha$ are future-directed unit vectors normal to $\mathcal{S}_2$ and $\partial M \cap \mathcal{S}_2$ respectively.
By the dominant energy condition, 
$T_{\mu 0 }\geq 0$ and $T_{\alpha 0 }\geq 0$. 
We obtain \cref{wen-causal} by replacing $T_{00}$ and $T_{| 00}$ by their expression in terms of $\phi$
\end{proof}
We introduce the norm
\begin{equation*}
\| \cdot \|_{\mathcal{H}^k(M)} = \| \Delta^k \cdot \|_{\mathcal{H}(M)}
\end{equation*}
which enables us to conclude on the causal propagation.
\begin{proposition}
Let $k \in \mathbb{N}^*$
With the same notations and the same conditions as in \cref{wen-propcau}, 
%the smooth solution $\Phi$ exists and is unique. 
%In addition, 
the smooth solution $\Phi$ depends continuously on the initial data $(\phi, \phi_|)$ in the sense that
\begin{equation*}
\big\| \frac{\partial^k}{\partial t^k} \Phi\big\|_{\mathcal{H}(\mathcal{S}_1)}
\leq
\big\| \Phi\big\|_{\mathcal{H}^{k}(\mathcal{S}_0)}
\end{equation*}
\end{proposition}
We will discuss two simple cases, $M = \mathbb{R}^{d-1} \times \mathbb{R}_+$ at first and $M = \mathbb{R}^{d-1} \times [-L, L]$ later in this section. 
%%%%%%%
\subsection{$M = \mathbb{R}^{d-1} \times \mathbb{R}_+$}\label{wen-subsect1}
\begin{proposition}
Let $M = \mathbb{R}^{d-1} \times \mathbb{R}_+$. $\Delta$ is a self-adjoint operator with spectrum $\mathbb{R}$
\end{proposition}
\begin{proof}
The self-adjointness being proven, let us find eigenvectors and eigenvalues of $\Delta$. 
Let $k$ be an eigenvalue of $\Delta$ and $\Phi_k = (\phi_k, \phi_{| k})$. Then
\begin{equation}\label{wen-motion2}
\begin{cases}
i \gamma^0 \gamma^j \partial_j \phi_k = k \phi_k \\
-c^{-1} \gamma^0 \mathcal{P}_- \phi\vert_{\partial M} + i \gamma^0 \gamma^a \partial_a \phi_{| k} = k \phi_{| k}
\end{cases}
\end{equation}
Since for all vector $\psi$,
\begin{equation*}
(i\gamma^0 \gamma^j\partial_j - k )(i\gamma^0 \gamma^j\partial_j + k )\psi = 
(- \partial^j\partial_j - k^2) \psi = 0
\end{equation*}
has plane wave solutions, 
all vectors of the form
\begin{equation*}
\begin{split}
\phi_n = & \Big((i\gamma^0\gamma^j\partial_j + k \mathbb{1}) e^{-ip_a x^a }(A\cos p_\bot x^\bot + B \sin p_\bot x^\bot) \psi_n \\
 = &\gamma^0\gamma^a p_a (A \cos p_\bot x^\bot + B \sin p_\bot x^\bot)e^{-ip_a x^a}
+ i\gamma^0\gamma^\bot p_\bot (-A \sin p_\bot x^\bot + B \cos p_\bot x^\bot) e^{-ip_a x^a} \\
& + k \mathbb{1} e^{-ip_a x^a}(A\cos p_\bot x^\bot + B \sin p_\bot x^\bot)\Big)\psi_n  \\
\end{split}
\end{equation*}
with $p_j$ satisfying
\begin{equation*}
- p^j p_j = k^2
\end{equation*}
are solutions of the bulk equation of~\cref{wen-motion}. 
The constants $A$ and $B$ will be determined by the boundary equation of \cref{wen-motion} and the boundary condition. 
The only constraint on $\psi_n$ is that $\psi_n$ should not be in $\ker( \gamma^0 \gamma^j p_j + k \mathbb{1})$. 
\footnote{
As 
$(\gamma^0\gamma^j p_j + k \mathbb{1})
(-\gamma^0\gamma^j p_j + k \mathbb{1})  
= k^2 - (p_j)^2= 0$, 
the kernel of $\gamma^0 \gamma^j p_j + k \mathbb{1}$ is not reduced to $\{ 0 \}$
} 
Since we have projectors $\mathcal{P}_\pm$, 
it is natural to take the basis of the total representation space as the union of an orthonormal basis $\mathfrak{B}_+$ of $\ran(\mathcal{P}_+)$ and an orthonormal basis $\mathfrak{B}_-$ of $\ran(\mathcal{P}_-)$. \\\\
Considering the boundary equation of~\cref{wen-motion} and the condition that an element of $\dom(\Delta)$ should satisfy, one gets
\begin{equation}\label{wen-boundary}
\begin{split}
& \mathcal{P}_+\phi\vert_{\partial M} =  \phi_| \\
& -c^{-1} \gamma^0 \mathcal{P}_- \phi\vert_{\partial M} + i\gamma^0\gamma^a\partial_a \mathcal{P}_+\phi\vert_{\partial M} = k \phi_| 
\end{split}
\end{equation}
which implies
\begin{equation}\label{wen-boundary2}
-c^{-1} \gamma^0 \mathcal{P}_-(\phi\vert_{\partial M}) = 
\mathcal{P}_+(k\phi\vert_{\partial M} - i\gamma^0\gamma^a\partial_a\mathcal{P}_+(\phi\vert_{\partial M})) = 
i\gamma^0\mathcal{P}_-(\gamma^\bot\partial_\bot \phi\vert_{\partial M})
\end{equation}
We compute
\begin{equation*}
\begin{split}
\partial_\bot \phi \vert_{\partial M} = 
\big((\gamma^0\gamma^a p_a + k\mathbb{1})e^{-i_a x^a} p_\bot B - i\gamma^0\gamma^\bot p_\bot^2 A \big) \psi
\end{split}
\end{equation*}
Hence, \cref{wen-boundary} implies
\begin{equation}\label{wen-projected}
\mathcal{P}_- \Big( (\gamma^0 \gamma^a p_a + k\mathbb{1})(c^{-1} A - p_\bot B)
+i \gamma^0 p_\bot(c^{-1} B + p_\bot A) \Big) \psi = 0
\end{equation}
To show that the spectrum of $\Delta$ is $\mathbb{R}$, 
we construct a generalized orthonormal basis for the eigenspace consisting of eigenvectors of eigenvalue $k{\in}\mathbb{R}$.
We observe that, 
for $\psi = e_+ + e_-$ where $e_\pm \in \ran({P}_\pm)$,
\cref{wen-projected} implies
\begin{equation}\label{wen-projected2}
(\gamma^a p_a + k\mathbb{1})(c^{-1}A - p_\bot B) e_- +\gamma^0 p_\bot(c^{-1}B + p_\bot A) e_+ = 0
\end{equation}
In order to construct a generalized orthonormal basis (normalizable to the $\delta$-function),
we can take elements of $\mathfrak{B}_+ \cup \mathfrak{B}_-$ for $\psi$ in~\cref{wen-projected}.
However, as previously mentioned, we should choose $\psi$ such that $\psi\slashed{\in}\ker(\gamma^0\gamma^j p_j + k\mathbb{1})$.
As a result, there are four possible cases to be discussed: 
\paragraph{Case 1 : $\psi \in \mathfrak{B}_-$ and $p_\bot \neq 0$} 
By \cref{wen-projected2}
\begin{equation*}
(\gamma^a p_a + k\gamma^0)(c^{-1}A - p_\bot B) \psi = 0
\end{equation*}
Now, since 
\begin{equation*}
(\gamma^a p_a + k\gamma^0)^2 = ( k^2 - p^a p_a ) \mathbb{1}= p^\bot p_\bot \mathbb{1} \neq 0
\end{equation*}
we must have 
\begin{equation*}
c^{-1} A = p_\bot B
\end{equation*}
\paragraph{Case 2 : $\psi \in \mathfrak{B}_-$ and $p_\bot = 0$}
In order to have non-trivial solution, 
we must have $A \neq 0$. 
In this case, 
\cref{wen-projected2} implies 
\begin{equation*}
(\gamma^a p_a + k \mathbb{1})\psi = 0
\end{equation*}
which is not allowed because $\psi\in\ker(\gamma^0\gamma^jp_j + k\mathbb{1})$ in this case. 
This case should thus be discarded.
\paragraph{Case 3 : $\psi \in \mathfrak{B}_+$ and $p_\bot \neq 0$}
$A$ and $B$ should verify 
\begin{equation*}
c^{-1} B + p_\bot A = 0
\end{equation*}
\paragraph{Case 4 : $\psi \in \mathfrak{B}_+$ and $p_\bot = 0$}
The terms in $B$ in the expression of $\phi$ vanish because  $p_\perp = 0$ and $A$ is determined by the normalization condition.
\\\\
%The above discussion indicates that for any $k \in \mathbb{R}$, 
Therefore, for any $k\in\mathbb{R}$, 
we can find eigenvectors of $\Delta$ with eigenvalue $k$ and a corresponding generalized orthonormal basis for the eigenspace.
\end{proof}
\subsection{${M} = \mathbb{R}^{d-1} \times [-L, L]$ }\label{wen-subsect-saw2}
The subtlety that one should beware of in this case is the projectors $\mathcal{P}_+$.
Since $\mathcal{P}_+$ is defined by considering a set of generators of Clifford Algebra on the tangent bundle $T_{ \partial \mathcal{M}}$, 
if we would like to fixe the coordinate system once for all,
we should also take into account the transformation of these generators. \\\\
Let $(x^0, \ldots, x^d)$ be the natural coordinate system of the bulk $\mathcal{M} =\mathbb{R}\times M $. 
Because of the geometric configuration of the bulk, 
this coordinate system is global. 
The boundary is therefore $\partial \mathcal{M} = \{x^d = -L \} \cup \{ x^d = L \}$.
\\\\
We denote $\{\gamma^\mu\}_\mu$ the set of gamma matrices chosen on $\{x^d  = - L \}$. 
We will have then $\mathcal{P}_+\vert_{\{x^d = L\}} = \mathcal{P}_-\vert_{\{x^d = -L\}}$ and $\gamma^\mu\vert_{\{x^d = L\}}=\gamma^\mu\vert_{\{x^d = -L\}}$ for $\mu\in\llbracket 0, d-1 \rrbracket$.
When it is not specified, we refer to the gamma matrices and the projectors on $\{x^d = -L\}$. \\\\
%
As for the boundary component $\phi_|$ of $\Phi= (\phi, \phi_|)$,
we can rewrite it as
\begin{equation*}
\phi_| = \mathbf{1}_{\{x^d = L \}}\phi_+ + \mathbf{1}_{\{x^d = - L \}}\phi_-
\end{equation*}
where $\phi_\pm$ represent the restriction of $\phi_|$ on $\{x^d = \pm L \}$.
Now, the boundary condition becomes
\begin{equation}\label{wen-saw2bound}
\begin{split}
\begin{cases}
\mathcal{P}_+ \phi\vert_{\{x^d = -L\}} = \phi_- \\
c^{-1}\mathcal{P}_-\phi\vert_{\{x^d = -L\}} = \mathcal{P}_-(\partial_d \phi\vert_{x^d = -L}) \\
\mathcal{P}_- \phi\vert_{\{x^d = L\}} =\phi_+ \\
-c^{-1}\mathcal{P}_+\phi\vert_{\{x^d = L\}} = \mathcal{P}_+(\partial_d \phi\vert_{x^d = L}) \\
\end{cases}
\end{split}
\end{equation}
As \cref{wen-subsect1}, the general solution $\phi$ in the bulk can be written as
\begin{equation*}
\phi 
 =\Big( (k \mathbb{1}+ \gamma^0\gamma^a p_a )(A \cos p_d x^d + B \sin p_d x^d)e^{-ip_a x^a}
+ i\gamma^0\gamma^d p_d (-A \sin p_d x^d + B \cos p_d x^d) e^{-ip_a x^a} \Big) \psi
 \end{equation*}
 where $\psi $ is a spinor and the Einstein summation over $a = 1, \ldots, d-1$ is applied. \\
 We compute
\begin{equation*}
\begin{split}
\partial_d \phi\vert_{x^d = \pm L }
= (\gamma^0\gamma^a p_a + k\mathbb{1})(\mp p_d A \sin p_d L + p_d B \cos p_d L)
+ i\gamma^0\gamma^d p_d(-p_d A \cos p_d L \mp p_d B \sin p_d L)\psi e^{-ip_a x^a}
\end{split}
\end{equation*}
Hence, the boundary condition leads to
\begin{equation*}
\begin{split}
0 = &\mathcal{P}_-\Bigg(
(\gamma^0\gamma^a p_a + k\mathbb{1})\Big((p_d A + c^{-1}B) \sin p_d L + (p_d B - c^{-1}A) \cos p_d L\Big) \\
& + i\gamma^0\gamma^d p_d \Big((-p_d A -c^{-1} B) \cos p_d L + (p_d B - c^{-1} A )\sin p_d L\Big) 
\Bigg)\psi
\end{split}
\end{equation*}
\begin{equation*}
\begin{split}
0 = &\mathcal{P}_+\Bigg(
(\gamma^0\gamma^a p_a + k\mathbb{1})\Big((- p_d A + c^{-1}B) \sin p_d L + (p_d B + c^{-1}A) \cos p_d L\Big) \\
& + i\gamma^0\gamma^d p_d \Big((-p_d A + c^{-1} B )\cos p_d L - (p_d B + c^{-1} A )\sin p_d L\Big) 
\Bigg)\psi
\end{split}
\end{equation*}
Once again, there are 4 possible cases to be discussed as in \cref{wen-subsect1}
%
\paragraph{Case 1 : $\psi \in \mathfrak{B}_-$ and $p_d \neq 0$}
The same argument as in~\cref{wen-subsect1} requires
\begin{equation}\label{wen-plates1}
\begin{cases}
(p_d A + c^{-1} B) \sin p_d L + (p_d B - c^{-1} A)\cos p_d L = 0 \\
(- p_d A + c^{-1}B)\cos p_d L - (p_d B + c^{-1} A)\sin p_d L = 0
\end{cases}
\end{equation}
In order to study~\cref{wen-plates1}, we discuss the following four sub-cases according to the coefficients of $\sin$ and $\cos$.
\subparagraph{Sub-case 1: $ p_d A + c^{-1} B = 0$}
\cref{wen-plates1} becomes
\begin{equation}\label{wen-plates1-1}
\begin{cases}
(c^{-2} + p_d ^2)\cos p_d L = 0 \\
(c^{-2} - p_d^2)\sin p_d L = 2 c^{-1}p_d \cos p_d L
\end{cases}
\end{equation}
We can verify easily that supposing $ c^{-2} + p_d^2 = 0 $ will lead to paradoxal situation. 
In effect, under this assumption, the second equation of~\cref{wen-plates1-1} will lead to $e^{\pm c^{-1}L} = 0$. \\
Therefore, the only solutions for this sub-case are $p_d = \pm c^{-1}$ for $L$ satisfying $\cos c^{-1} L = 0$
%
\subparagraph{Sub-case 2: $ p_d B + c^{-1}A = 0$}
By an easy calculation, we obtain the same result as in sub-case 1
%
\subparagraph{Sub-case 3: $\cos p_d L = 0 $}
Idem as in sub-case 1.
%
\subparagraph{Sub-case 4: none of the above statements holds}
\cref{wen-plates1} leads to
\begin{equation*}
\tan p_d L = \frac{-p_d B + c^{-1} A }{p_d A + c^{-1} B} = 
\frac{-p_d A + c^{-1}B}{p_d B + c^{-1} A}
\quad\Rightarrow\quad
(p_d ^2 + c^{-2})(A^2 - B^2 )= 0
\end{equation*}
One can easily verify that $p_d =  \pm i c^{-1}$ can not satisfy the relation. 
Hence, we have $A = \pm B$ and
\begin{equation}\label{wen-tan}
\tan p_d L = \frac{\mp p_d + c^{-1}}{p_d \pm c^{-1}}
\end{equation}
Note that if $p_d$ satifies~\cref{wen-tan},
$p_d$ should be reel.
%%%%%
In effect,~\cref{wen-tan} implies
\begin{equation*}
\Im (\tan p_d L ) = \frac{\Im(c^{-1} p_d^*)}{|p_d \pm c^{-1}|^2}
\end{equation*}
As the \lhs and the \rhs have opposite sign, 
the imaginary part of $p_d$ should be 0.
%
\paragraph{Case 2 : $\psi \in \mathfrak{B}_-$ and $p_d = 0$}
This case implies $A = 0$, which is forbidden because it leads to the trivial solution.
%
\paragraph{Case 3 : $\psi \in \mathfrak{B}_+$ and $p_d \neq 0$}
The relation between $A$ and $B$ becomes
\begin{equation*}
\begin{cases}
(-p_d A - c^{-1} B)\cos p_d L + (p_d B - c^{-1}A)\sin p_d L = 0  \\
(-p_d A + c^{-1}B)\sin p_d L + (p_d B + c^{-1} A)\cos p_d L = 0 
\end{cases}
\end{equation*}
This is strictly equivalent to the case 1 of this discussion.
%
\paragraph{Case 4 : $\psi \in \mathfrak{B}_+$ and $p_d = 0$}
Leads to the trivial solution. \\\\
To conclude, we give the following proposition
\begin{proposition}
In the case where $\cos c^{-1}L \neq 0$,
the eigenvalues $k$ of $\Delta$ are of form $k^2 = q^2 + p^2_d $ where $q\in\mathbb{R}^{d-1}$ and $p_d\in \mathbb{R}$ satisfying~\cref{wen-tan} and $-p^jp_j = k^2$.
Furthermore, the elements of the family $\{(\phi_{n,q,p_d}, \phi_{|n,q,p_d})\}$, where
\begin{equation*}
\phi_k = \Big(\big(i\gamma^0\gamma^a p_a+k\mathbb{1}\big)(\cos p_d x^d + \sin p_d x^d) +
i\gamma^0\gamma^d p_d(-\sin p_d x^d + \cos p_d x^d)\Big) \psi_n e^{-ip_a x^a}
\end{equation*}
, $q=(p_1,\ldots, p_{d-1})\in\mathbb{R}^{d-1}$ and $\psi_n\in\mathfrak{B}_+\cup\mathfrak{B}_-$,
form a generalized orthongonal basis of the eigenspace for eigenvalue $k$.
\end{proposition}








