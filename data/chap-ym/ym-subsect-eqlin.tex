\subsection{Linearized system}
Let us begin by studying~\cref{ym-dyn} up to first order in $A$ and $\bar{A}$. 
We define the operator $G_L$ with
\begin{equation}
G_L (A, E, \bar{A}, \bar{E}) = (E, -\dd^c \dd A, \bar{E}, -\dd^c \dd \bar{A} + c^{-1} dA_\bot)
\end{equation}
where $dA_\bot$ is the 1-form with components $(dA_\bot)_j = (dA)_{kj}n^k\vert_{\partial \Sigma}$ with $n$ unit vector normal to the boundary (inward directed). \\\\
%
%\color{red} Because of the complexity of YM, I decided to work on U(1) gauge rather than non-abelian cases for this report. The system is thus linearized \\\\
%\color{black}
%
We denote for the Hilbert space\footnote{
The space $\Omega^k_{p,q}(M)$ defined in~\cref{ym-pw} is a Banach space (see~\cite{Mazya2009} p. 200)
}
\begin{equation*}
\mathcal{H} = \Omega^{1}_{2,2}(\Sigma)\times L^2(\Sigma) \times \Omega^{1}_{2,2}(\partial \Sigma) \times L^2(\Sigma) 
\end{equation*}
%Naturally, $\mathcal{H}$ is equipped 
with the norm associated to the inner product
\begin{equation}\label{ym-norm1}
\begin{split}
\| (A,E,\bar{A},\bar{E}) \|^2_{\mathcal{H}} = &
\| A\|^2_{L^2(\Sigma)} + \|\dd A\|^2_{L^2(\Sigma)}  + \| E \|^2_{L^2(\Sigma)} \\ & +
\| \bar{A}\|^2_{L^2(\partial \Sigma)} + \|\dd \bar{A}\|^2_{L^2(\partial \Sigma)}  + \| \bar{E} \|^2_{L^2(\partial \Sigma)}
\end{split}
\end{equation}
where $\Sigma$ is an equal-time hypersurface of $M$. 
%Meanwhile, for our convenience, another norm (associated to an inner product) will be given in the following. 
\\\\
%
As we assume that the torsion of $M$ is reduced to $\{0\}$, we can apply the Hodge-Kodaira decomposition~\cref{ym-kodaira}, \ie for a 1-form $B$, we have (analogous for $\partial M$)
\begin{equation*}
B = B_c + B_d
\quad\textrm{where}\quad
B_d\in\im d
\quad\textrm{and}\quad
B_c\in\im d^c\oplus\big(L^2(M,\Lambda^k)\cap\ker(d d^c+d^c d)\big)
\end{equation*}
In particular, $d B_d = 0$. 
We thus have 
\begin{equation*}
\begin{split}
\langle (A_1,E_1,\bar{A}_1,\bar{E}_1), G_L(A_2,E_2,\bar{A}_2,\bar{E}_2) \rangle_{\mathcal{H}} = &
\langle A_{1c}, E_{2c} \rangle_{L^2(\Sigma)} + \langle A_{1d}, E_{2d} \rangle_{L^2(\Sigma)} \\
&+ \langle d A_{1c}, d E_{2c} \rangle_{L^2(\Sigma)}
+ \langle E_{1c}, -d^c d A_{2c}\rangle_{L^2(\Sigma)} 
 + \textrm{boundary terms}
\end{split}
\end{equation*}
$G_L$ does not seems to be symmetric in $\mathcal{H}$ with the corresponding inner product. 
However, we can construct a suitable Hilbert space by imposing conditions on the regularity of forms. 
The approach which is going to be introduced might seem very restrictive but can be easily adapted to the initial data.\\\\
The main idea is to find a Hilbert space in which $\| \cdot \|_{L^2(\Sigma)}$ and $\| d \cdot \|_{L^2(\Sigma)}$ are equivalent. 
By our assumption on the torsion and~\cref{ym-pw}, we know that
\begin{equation*}
\exists C' \quad
\forall \theta \in \Omega^1_{2,2}(\Sigma) \quad
\exists\xi\in\ker d\quad
\| \theta - \xi \|_{L^2(\Sigma)}\leq C' \| d \theta \|_{L^2(\Sigma)}
\end{equation*}
Therefore, for a constant $C_1 > C'$, the open set $\big\{\theta\in\Omega^1_{2,2}(\Sigma) \enskip \big/\enskip \|\theta \|_{L^2(\Sigma)} < C_1\|d \theta \|_{L^2(\Sigma)} \big\} \neq \emptyset$.\footnote{ 
It should be noted that $C_1 \geq C'$ is only a sufficient condition here. 
The point here is that one should choose a constant $C_1$ such that we have a non-empty set for the wanted estimation.
}
On the other hand, since the exterior derivative $d$ is a continuous map, in any compact of $\Sigma$, there exists a constant $C_2 >0$, $\| d \theta \|_{L^2(\Sigma)} \leq C_2 \| \theta \|_{L^2(\Sigma)}$. 
By these observations, we define 
\begin{equation*}
\begin{split}
\mathcal{G} = \mathcal{H}_{C_1, C_2} \equiv\big\{ (A,E,\bar{A},\bar{E}) \in \mathcal{H} \enskip/\enskip &
C_2 \|A\|_{L^2(\Sigma)} \leq \| d A \|_{L^2(\Sigma)} \leq C_1\| A\|_{L^2({\Sigma})} \\
& C_2 \|\bar{A}\|_{L^2(\partial\Sigma)} \leq \| d \bar{A} \|_{L^2(\partial\Sigma)} \leq C_1\| \bar{A}\|_{L^2({\partial\Sigma})} \big\}
\end{split}
\end{equation*}
$\mathcal{G}$ is a Hilbert space because it is a closed sub-space of $\mathcal{H}$. 
With this definition, we can see that the norms $\|\cdot\|_{L^2(\Sigma)}$ and $\|d\cdot\|_{L^2(\Sigma)}$ are equivalent (analogous for the boundary norms).
Hence, we will work on the Hilbert space $(\mathcal{G}, \langle\cdot , \cdot\rangle_{\mathcal{G}})$, where the norm associated to the inner product is
\begin{equation*}
\|(A,E,\bar{A},\bar{E})\|^2_{\mathcal{G}} = 
\| dA\|^2_{L^2(\Sigma)} + \|E\|^2_{L^2(\Sigma)} +c\| dA\|^2_{L^2(\partial \Sigma)} + c\|E\|^2_{L^2(\partial \Sigma)}
\end{equation*}
Now, we define
\begin{equation*}
\dom G_L = \{ (A,E,\bar{A},\bar{E}) \in \mathcal{G} / A\in W^{2,2}(\Sigma) \enskip
\bar{A}\in W^{2,2}(\partial \Sigma) \enskip
E \in W^{1,2}(\Sigma) \enskip
\bar{E} \in W^{1,2}(\partial \Sigma)
E = \bar{E}\}
\end{equation*}
Let $(A,E,\bar{A},\bar{E}) \in \dom G_L$, we compute
\begin{equation*}
\begin{split}
\langle (A_1,E_1,\bar{A}_1,\bar{E}_1), & G_L(A_2,E_2,\bar{A}_2,\bar{E}_2) \rangle_{\mathcal{G}} \\ =& 
\langle d A_{1c}, d E_{2c} \rangle_{L^2(\Sigma)} + \langle E_{1c}, -d^c dA_{2c}\rangle_{L^2(\Sigma)} \\
& + c \langle d\bar{A}_{1c}, d\bar{E}_{2c}\rangle_{L^2(\partial \Sigma)} 
+ c \langle \bar{E}_{1c}, -d^c d \bar{A}_{2c} + c^{-1}d A_{2,\bot} \rangle_{L^2(\partial \Sigma)} \\
%
= & \langle d^cdA_{1c}, E_{2c} \rangle_{L^2( \Sigma)} - \langle d E_{1c}, dA_{2c} \rangle_{L^2(\Sigma)}
+c \langle d^cd\bar{A}_{1c}, \bar{E}_{2c} \rangle_{L^2(\partial \Sigma)} - c \langle d \bar{E}_{1c}, d\bar{A}_{2c} \rangle_{L^2(\partial\Sigma)} \\
& -\langle (dA_{1c})_\bot , E_{2c} \rangle_{L^2(\partial \Sigma)} 
-\langle (dA_{2c})_\bot , {E}_{1c} \rangle_{L^2(\partial \Sigma)} 
+\langle (dA_{2c})_\bot , \bar{E}_{1c} \rangle_{L^2(\partial \Sigma)}  \\
%
\underset{\dom G_L}{=} &
\langle d^cdA_{1c}, E_{2c} \rangle_{L^2( \Sigma)} - \langle d E_{1c}, dA_{2c} \rangle_{L^2(\Sigma)}
+c \langle d^cd \bar{A}_{1c}, \bar{E}_{2c} \rangle_{L^2(\partial \Sigma)} - c \langle d \bar{E}_{1c}, d\bar{A}_{2c} \rangle_{L^2(\partial\Sigma)} \\
& -\langle (dA_{1c})_\bot , E_{2c} \rangle_{L^2(\partial \Sigma)} \\
%
= &
-\langle G_L(A_1,E_1,\bar{A}_1,\bar{E}_1), (A_2,E_2,\bar{A}_2,\bar{E}_2) \rangle_{\mathcal{G}} 
+ \langle (dA_{1c})_\bot , \bar{E}_{2c}-E_{2c} \rangle_{L^2(\partial \Sigma)} 
\end{split}
\end{equation*}
%
\color{red}
Claim: 
\begin{equation*}
L^2(\partial \Sigma) \in \im\Big(
n^j\big(d \cdot \big)_j\big\vert_{L^2(M)} \Big)
\end{equation*}
\begin{proof}
Idea ($\mathbb{R}^d$ case) : 
Let $f\in L^2(\partial \Sigma)$. 
Find $A \in L^(\Sigma)$ such that 
\begin{equation*}
f_a + \partial_a A_\bot\vert_{\partial \Sigma} = \partial_\bot A_j\vert_{\partial \Sigma}
\end{equation*}
One possible choice is 
\begin{equation*}
A_a = 0 \quad,\quad
A_\bot = i e^{i\Pi_b\int_0^{x_b} f_a} e^{-(x^{d})^{2}}
\end{equation*}
How to generalize to arbitrary manifolds under our assumptions?
\end{proof}
\color{black}
%
As a result, $G_L$ is skew-adjoint with the given domain.























