\subsection{Non-linear equation}
We put the gauge field part into the action~\cref{wen-action} and use covariant derivative instead. 
The total action is now
\begin{equation*}
\begin{split}
\mathcal{S} = & \frac{1}{2}i\int_M \bar{\psi} \gamma^\mu (\partial_\mu+i A_\mu) \psi - (\partial_\mu - iA_\mu) \bar{\psi} \gamma^\mu \psi 
+ \frac{1}{2}\int_{\partial M} ic \bar{\psi} \gamma^\alpha (\partial_\alpha+iA_\alpha) (1 - i \gamma^\bot) \psi
+ \bar{\psi} \psi \\ 
%
& - \frac 1 4 \int_M F^{\mu\nu}F_{\mu\nu} - \frac 1 4 \int_{\partial M} F^{\alpha\beta}F_{\alpha\beta}
\end{split}
\end{equation*}
From this action, we can derive a system of equations containing the equations of motion of the Dirac fields as~\cref{wen-motion} and the equations coupling the matter field to the gauge field.
The coupling equations of motion are
\begin{equation*}
\begin{cases}
\nabla^A_\mu F^{\mu\nu} = i\phi^\dagger\gamma^\nu\gamma^0\phi \\
%
\nabla^{\bar{A}}_\alpha \bar{F}^{\alpha\beta} = i\phi_|^\dagger\gamma^\beta\gamma^0\phi_| -c^{-1}\bar{F}_{\nu\beta}n^\nu 
\end{cases}
\end{equation*}
It could be shown that the tangential component of the current term in the bulk is equal to the current term on the boundary as long as we have $\mathcal{P}_+\phi\vert_{\partial M} = \phi_|$. 
In effect, this condition implies
\begin{equation}\label{ym-current}
\phi = 2 \phi_| - i\gamma^\bot\phi
\end{equation}
Multiplying~\cref{ym-current} by $(\gamma^j\phi)^\dagger\gamma^0$ on the left, we have
\begin{equation}\label{ym-current1}
(\gamma^j\phi)^\dagger\gamma^0\phi = 2(\gamma^j\phi)^\dagger\gamma^0\phi_| - i(\gamma^j\phi)^\dagger\gamma^0\gamma^\bot\phi
\end{equation}
Taking the adjoint of~\cref{ym-current} multiplied by $\gamma^0$ and multiplying by $\gamma^j\phi$ on the right, we get
\begin{equation}\label{ym-current2}
(\gamma^0\phi)^\dagger\gamma^j\phi = 2\phi^\dagger_|\gamma^0\gamma^j\phi - i\phi^\dagger\gamma^\bot\gamma^0\gamma^j\phi
\end{equation}
%perhaps put earlier?
Choosing $j = \bot$, one gets from~\cref{ym-current1} and~\cref{ym-current2}
\begin{equation}\label{ym-current3}
\phi^\dagger\gamma^0\gamma^\bot\phi = i\phi^\dagger(\mathcal{P}_+\gamma^0 - \gamma^0\mathcal{P}_+)\phi  
\end{equation}
By comparing~\cref{ym-current3} to its adjoint, we find
\begin{equation*}
\phi^\dagger\gamma^0\gamma^\bot\phi\vert_{\partial M} = 0
\end{equation*}
As in the normal bag boundary condition, there is no current in the direction perpendicular to the boundary. \\\\
We can also check that the current on the boundary is conserved. 
The boundary current density can be expressed as
\begin{equation*}
J_|^\alpha = i \phi_|\gamma^\alpha\gamma^0\phi_|
\end{equation*}
Hence, the on-shell divergence of the current density is
\begin{equation*}
\begin{split}
\nabla_\alpha J_|^\alpha  = & (\nabla_\alpha\phi_|^\dagger)\gamma^\alpha\gamma^0\phi_| + \phi_|^\dagger\gamma^\alpha\gamma^0(\nabla_\alpha\phi) \\
= &
-ic^{-1}\mathcal{P}_+\phi^\dagger\vert_{\partial M} \gamma^\alpha\gamma^0\phi_| + i\phi_|^\dagger\gamma^\alpha\gamma^0 c^{-1}\mathcal{P}_+\phi\vert_{\partial M} \\
= & 0
\end{split}
\end{equation*}
This result is expected as there is no current coming from the bulk.















