\subsection{Non-linear equation}
We put the gauge field part into the action~\cref{wen-action} and use covariant derivative instead. 
The total action is now
\begin{equation*}
\begin{split}
\mathcal{S} = & \frac{1}{2}i\int_M \bar{\psi} \gamma^\mu (\partial_\mu+i A_\mu) \psi - (\partial_\mu - iA_\mu) \bar{\psi} \gamma^\mu \psi 
+ \frac{1}{2}\int_{\partial M} ic \bar{\psi} \gamma^\alpha (\partial_\alpha+iA_\alpha) (1 - i \gamma^\bot) \psi
+ \bar{\psi} \psi \\ 
%
& - \frac 1 4 \int_M F^{\mu\nu}F_{\mu\nu} - \frac 1 4 \int_{\partial M} F^{\alpha\beta}F_{\alpha\beta}
\end{split}
\end{equation*}
From this action, we can derive a system of equations containing the equations of motion of the Dirac fields as~\cref{wen-motion} and the equations coupling the matter field to the gauge field.
The coupling equations of motion are
\begin{equation*}
\begin{cases}
\nabla^A_\mu F^{\mu\nu} = i\phi^\dagger\gamma^\nu\gamma^0\phi \\
%
\nabla^{\bar{A}}_\alpha \bar{F}^{\alpha\beta} =2 i\phi_|^\dagger\gamma^\beta\gamma^0\phi_| -c^{-1}\bar{F}_{\nu\beta}n^\nu 
\end{cases}
\end{equation*}
It could be shown that the tangential component of the current term in the bulk is equal to the current term on the boundary as long as we have $\mathcal{P}_+\phi\vert_{\partial M} = \phi_|$. 
In effect, this condition implies
\begin{equation}\label{ym-current}
\phi = 2 \phi_| - i\gamma^\bot\phi
\end{equation}
By taking the adjoint of~\cref{ym-current}, we calculate
\begin{equation*}
\phi^\dagger\gamma^j\gamma^0\phi = 2 \phi_|^\dagger\gamma^j\gamma^0\phi - 
\phi^\dagger i \gamma^\bot\gamma^j\gamma^0\phi
\end{equation*}
On the other hand, when we multiply~\cref{ym-current} by $(\gamma^0\phi)^\dagger\gamma^j$ on the right
\begin{equation*}
(\gamma^0\phi)^\dagger\gamma^j\phi = 2\phi^\dagger\gamma^0\gamma^j\phi_| - 
\phi^\dagger i \gamma^\bot\gamma^0\gamma^j\phi
\end{equation*}












