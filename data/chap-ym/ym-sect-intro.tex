\paragraph{Convention} 
For this chapter, we work on Minkowski space-time with metric signature $(-, \ldots, +)$
\section{Motivation}
It is curious to know if the system studied in~\cref{chap-wentzell} can be coupled with a gauge field by Yang-Mills theory. 
Studies on the well-posedness of the problem derived from the Yang-Mills' action have been done in many papers in detailed mathematical aspects. 
The well-posedness of Yang-Mills field coupled with Dirac field satisfying the bag boundary condition has been presented in~\cite{Sinatycki1993}, with certain boundary condition on the gauge fields.
In this work, I will use certain method from~\cite{Sinatycki1993} to study another type of possible boundary condition. \\\\
%
We start by investigate the equations of motion (field part) derived from the following action
\begin{equation}\label{ym-action}
\mathcal{S}_{YM} = -\frac 1 4 \int_M F^{\mu\nu} F_{\mu\nu} 
-\int_{\partial M} \frac 1 4 c\ F^{\alpha\beta} F_{\alpha\beta}
\end{equation}
where 
\begin{equation*}
F_{\mu\nu} = \partial_\mu A_\nu - \partial_\nu A_\mu + [A_\mu, A\nu]
\end{equation*}
for a certain potential $A$. $[\cdot, \cdot]$ is the anti-commutator. The coupling constant is reduced to 1 here. The $2$-form $F$ can also be expressed in a more compact way
\begin{equation*}
F = \dd A + A \wedge A
\end{equation*}
where $\dd$ is the exterior derivative on $M$ and $\wedge$ is the exterior product.\\\\
As suggested in the litterature~\cite{Tao2003}, 
we choose to work with the temporal gauge ($A_0 = 0$). 
We get the following system of linearized equations by applying variational method to the action
\begin{equation}
\begin{split}
& \partial_0 A = E, \quad \partial_0 E = \delta A_{df} \\
& \partial_0 \bar{A} = \bar{E} , \quad \partial \bar{E} = \delta \bar{A}_{df} - c^{-1}\partial_\bot A\vert_{\partial M}
\end{split}
\end{equation}
where the Helmholtz-Hodge decomposition is used and $df$ means the divergence-free part.
%TODO add reference and more precisions.
