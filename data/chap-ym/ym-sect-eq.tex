\section{Equation of motion}
Given $M$ a $(d +1)$-dimensional manifold. 
Let us now derive the equations of motion from the action~\cref{ym-action}.
A variation $A \rightarrow A + \delta A$ leads to $\mathcal{S}_{YM} \rightarrow \mathcal{S}_{YM} + \delta\mathcal{S}_{YM}$ where
%
\begin{equation*}
\begin{split}
\delta \mathcal S_{YM} = & -\frac 1 2 \Big(
 \langle \dd \delta A, \dd A +A\wedge A  \rangle_{L^2(M)} +  
 \langle \delta A \wedge A, \dd A + A\wedge A \rangle_{L^2(M)} + 
 \langle A \wedge \delta A , \dd A + A\wedge A \rangle_{L^2(M)}  \\ & + 
 c (\langle \dd \delta A, \dd A +A\wedge A\rangle_{L^2(\partial M)}+
 \langle \delta A \wedge A, \dd A + A\wedge A  \rangle_{L^2(\partial M)} + 
 \langle A \wedge \delta A , \dd A + A\wedge A \rangle_{L^2(\partial M)} 
 )
 \Big) \\
 %
 \underset{\textrm{\cref{ym-stokes}}}{=} & - \frac{1}{2} \Big(
 \langle \delta A_t, (\dd A + A\wedge A)_n \rangle_{L^2(\partial M)} +
 \langle \delta A, \dd^c (\dd A + A\wedge A) \rangle_{L^2(M)} + 
 \langle \delta A \wedge A,  \dd A + A \wedge A \rangle_{L^2(M)} \\ &+
 \langle A \wedge \delta A,  \dd A + A \wedge A \rangle_{L^2(M)}  +
 \langle \delta A, \dd^c (\dd A + A\wedge A) \rangle_{L^2(\partial M)} + 
 \langle \delta A \wedge A, \dd A + A \wedge A \rangle_{L^2(\partial M)} \\&+
 \langle A \wedge \delta A, \dd A + A \wedge A \rangle_{L^2(\partial M)}
 \Big) \\
 %
\underset{\textrm{\cref{ym-wedge}}}{=} & - \frac 1 2 \Big(
\langle \delta A, (\dd^c + [A, \cdot ])(\dd A+ A \wedge A) \rangle_{L^2(M)} \\ &+
c \big( c^{-1}\langle \delta A_t, (\dd A + A\wedge A)_n \rangle_{L^2(\partial M)} +
\langle \delta A, (\dd^c + [A, \cdot ])(\dd A+ A \wedge A) \rangle_{L^2(\partial M)} \big)
\Big)
\end{split}
\end{equation*}
To compute the codifferential of 2-forms, the dimension of the space should be taken into account. In the bulk, we have
\begin{equation*}
\begin{split}
\dd^c F_{\mu\nu} \dd x^\mu \wedge \dd x^\nu = & (-1)^{d+3} \star \dd \star (F_{\mu\nu} \dd x^\mu \wedge \dd x^\nu ) \\
%
= & \frac{(-1)^{d+3}}{2 d!} \star \dd \Big( F_{\mu\nu} \varepsilon^{\mu\nu}_{\enskip \lambda_1\cdots \lambda_{d}} \bigwedge_{i} \dd x^{\lambda_i} \Big) \\
%
= & \frac{(-1)^{d+1}}{2d !} \star \Big( \partial_\sigma F_{\mu\nu} \varepsilon^{\mu\nu}_{\enskip \lambda_1\cdots \lambda_{d}} \enskip \dd x^\sigma\wedge \big(\bigwedge_{i} \dd x^{\lambda_i} \big)  \Big) \\
%
= & \frac{1}{2d !} \star \Big( \partial_\sigma F_{\mu\nu} \varepsilon^{\mu\nu}_{\enskip \lambda_1\cdots \lambda_{d}} \enskip \big(\bigwedge_{i} \dd x^{\lambda_i} \big) \wedge \dd x^\sigma \Big) \\
%
= & \frac{1}{2(d+1)!d !} \enskip  \partial^\sigma F_{\mu\nu} \varepsilon^{\mu\nu}_{\enskip \lambda_1\cdots \lambda_{d}} \varepsilon^{\lambda_1 \cdots\lambda_d}_{\enskip\enskip\enskip\enskip \sigma\eta} \enskip \dd x^\eta \\
%
= & \partial^\mu F_{\mu\nu} \dd x^\nu
\end{split}
\end{equation*}
Motivated by this computation, we propose to study the following system of equations of motion
\begin{equation}\label{ym-motion1}
\begin{cases}
\partial^\mu F_{\mu\nu} +  [A^\mu, F_{\mu\nu}] = 0 \quad\textrm{ in $M$}\\
%
\partial^\alpha \bar F_{\alpha\beta} +  [\bar A^\alpha, \bar F_{\alpha\beta}] = 
 c^{-1} F_{\nu\beta}n^\nu
\quad\textrm{ on $\partial M$}
\end{cases}
\end{equation}
where the quantities with bar live on the boundary and $n$ is the unit \textbf{in-coming} vector normal to $\partial M$\\
Compared to the original work of C.N. Yang and R. L. Mills~\cite{Yang1954}, there is a supplementary source term in the equation of motion on the boundary due to the field of the bulk.\\\\
Denote $E_i = F_{0i}$ and $\bar E_a = \bar F_{0a} $. 
We choose the temporal gauge for the following calculation, \ie $A_0 = 0$
We can split~\cref{ym-motion1} into the dynamical system 
\begin{equation}\label{ym-dyn}
\begin{cases}
\partial_0 A_i = E_i \\
\partial_0 E_i =  \partial_j F_{ji} - [A_j, F_{ji}] \\
\partial_0 \bar A_a = \bar E_a \\
\partial_0 \bar E_a =  \partial_b \bar F_{ba} -  [\bar A_b, \bar F_{ba}] + c^{-1} F_{\nu a} n^\nu \\
\end{cases}
\end{equation}
and the constraint equation
\begin{equation}\label{ym-const}
\begin{cases}
\partial_i E_i  = - [A^i, E_i] \\
\partial_a \bar E_a = - [\bar A_a , \bar E_a]  + c^{-1} E_{b}n^b
\end{cases}
\end{equation}
Before defining the Hamiltonian of the system, we have to ensure that if a data $(A, E, \bar A, \bar E)$ satisfies~\cref{ym-const} at a given moment and is a solution of~\cref{ym-dyn}, then its evolution will always satisfy~\cref{ym-const}.
This consequence is general enough. 
%add explanation
\begin{equation*}
\begin{split}
\nabla^A_0\nabla^A_j E^j = & \partial_0 \nabla^A_j E^j  \\
= & -\partial_0(\partial_j E^j + [A_j , E^j]) + [A_j, \partial_0 E^j] \\
= &- \partial_j (\partial^i F_{ij} + [A^i, F_{ij}]) + [E_j, E^j] + [A_j, \partial^i F_{ij} + [A^i, F_{ij}]] \\
= & \partial^j\partial^i F_{ij}  + [A^i, \partial^j F_{ij}] + [A^j, \partial^i F_{ij}] + [\partial^j A^i, F_{ij}]+ [A^j, [A^i, F_{ij}]]
\end{split}
\end{equation*}
Since $F$ is anti-symmetric, the first three terms vanish. 
And we have for the last term
\begin{equation*}
[A^j, [A^i, F_{ij}]] = - [A^i, [A^j, F_{ij}]]
= [A^j, [F_{ij}, A^i]] + [F_{ij}, [A^i, A^j]]
\end{equation*}
Then, by a simple calculation, we find that the sum of the last two terms vanishes.









