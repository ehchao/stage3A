%\paragraph{Convention} In this section, we will work on Minkowski space-time with signature $(-, +,\ldots,+)$.
\section{Preliminary}
We recall briefly some mathematical notions which will be used in this section (see~\cite{Zeidler1995} for more details).
Let $M$ be a $n$-dimensional oriented manifold with boundary $\partial M$.
We denote $\Omega^p(U)$ the set of $p$-forms in $U\subseteq M$. 
\begin{definition}
The \textbf{Hodge star operator} $\star : \Omega^p(U) \mapsto \Omega^{p+1}(U)$ for all $p\in\mathbb{N}$ is defined by the following mapping
\begin{equation*}
\star ( \dd x^{\mu_1}\wedge\ldots\wedge\dd x^{\mu_p}) = 
\frac{1}{(n-p)!}\varepsilon^{\mu_1\ldots\mu_p}_{\quad\mu_{p+1}\ldots\mu_n}(\dd x^{\mu_{p+1}}\wedge\ldots\wedge\dd x^{\mu_n})
\end{equation*}
where $\varepsilon$ is the Levi-Civita tensor where $\varepsilon_{0\ldots p} = 1$ \\\\
For a $p$-form $F_{(p)}$, the mapping of $\star$ is given by
\begin{equation*}
\star F_{(p)} = \frac 1 {p !} F_{\mu_1\ldots\mu_p} \star(\dd x^{\mu_1}\wedge\ldots\wedge \dd x^{\mu_p})
\end{equation*}
\end{definition}
%
\begin{definition}
Let $d$ be the exterior differential operator on $\Omega^p(M)$. We define the \textbf{codifferential operator} $\dd^c: \Omega^p(M) \mapsto \Omega^{p-1}(M)$ by the mapping
\begin{equation*}
\dd^c \omega = (-1)^{n(p+1)+1}\star d \star \omega
\end{equation*}
for all $\omega\in\Omega^p(M)$ where $1 \leq p \leq n$
\end{definition}
%
One can get the mapping $\star^2$ by simple calculation
\begin{proposition}
Let $F_{(p)}$ be a $p$-form on $\mathbb{R}^{n-m, m}$. Then
\begin{equation*}
\star^2 F_{(p)} = (-1)^{p(n-p)+m}F_{(p)}
\end{equation*}
\end{proposition}
%
We can define an inner product $\langle \cdot, \cdot \rangle_{L^2(M)}$ on $\Omega^p(M)\cap L^2(M)$ by
\begin{equation*}
\langle w, v \rangle_{L^2(M)} = \int_M w \wedge \star v
\end{equation*}
We derive the following version of Stokes' theorem, which will be useful for our further discussion.
\begin{proposition}
Let $w$ and $v$ be two $1$-forms. Then
\begin{equation}\label{ym-stokes}
\langle \dd w,  \dd v \rangle_{L^2(M)} =
\int_{\partial M} w_t\wedge \star \dd v_n  + (-1)^s
\langle w, \dd^c \dd v \rangle_{L^2(M)}
\end{equation}
where the indices $t$ and $n$ are tangential and normal outward component to $\partial M$ and $s$ is the signature of the metric ($s=0$ for Riemannian and $s = 1$ for Lorentzian).
\end{proposition}
%
\begin{proof}
For $w$ and $v$ $1$-forms, we have by exterior derivation 
\begin{equation*}
\begin{split}
\dd(w \wedge \star \dd v) = & \dd w \wedge \star \dd v - w \wedge \dd \star \dd v \\
= & \dd w \wedge  \star \dd v - (-1)^s w \wedge \star \dd^c \dd v
\end{split}
\end{equation*}
We can then obtain~\cref{ym-stokes} by applying Stokes' theorem on differential forms.
\end{proof}
We introduce the following relation which will be useful when deriving the equation of motion
\begin{proposition}\label{ym-wedge}
Let $A$ be a 2-form and $B$ and $C$ be two 1-forms. Then
\begin{equation}
\langle A, B\wedge C \rangle_{L^2(M)} + \langle A, C\wedge B \rangle_{L^2(M)} =
2 \langle B, C\ast A \rangle_{L^2(M)}
\end{equation}
where $C\ast A$ is a 1-form defined by
\begin{equation*}
(C\ast A)_{i} \equiv C^j A_{ij} - A_{ij}C^j
\end{equation*}
where the summation over $\nu$ is understood.
\end{proposition}
\begin{proof}
In terms of components, we have (repeated indices are summed)
\begin{equation*}
\begin{split}
\langle A, B\wedge C \rangle_{L^2(M)} + \langle A, C\wedge B \rangle_{L^2(M)} = & 
A_{ij}(B^i C^j - C^j B^i) + A_{ij}(C^i B^j - B^j C^i) \\
\underset{A_{ij} = -A_{ji}}{=} & 2 A_{ij}(B^i C^j - C^j B^i)\\
\underset{\textrm{symmetry of }\langle \cdot, \cdot \rangle}{ = } &2 (B^i C^j - C^j B^i)A_{ij}
\end{split}
\end{equation*}
Consider the 1-forms $D$ and $E$ where $D_i = A_{ij}C^j$ and $E_i = B^j A_{ji}$, we can show that 
\begin{equation*}
\langle B, D\rangle_{L^2{(M)}} = \langle E, C \rangle_{L^2(M)}
\end{equation*}
which allows us to conclude.
\end{proof}









