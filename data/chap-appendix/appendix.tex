\chapter{QFT in curved space-time}
QFT in curved space-time is a semi-classical theory taking the curvature of space-time in quantum field theory into account.
A review on the necessary notions for this report will be given in this paragraph.
For a more complete review of this subject, we can refer to~\cite{Hollands2014}. \\\\
%
For a given curved Lorentzian space-time $(M,g)$,
the construction of the theory is based on algebraic approach.
Instead of considering a given Hilbert space, 
the theory is formulated with an algebra of quantum observables $\mathscr{A}(M,g)$.
As done in the section 2 of~\cite{Hollands2014}, we give here a brief review of QFT in curved space-time for quantum algebra of observables generated by a Klein-Gordon field $\phi$. 
However, the field $\phi$ should be treated as general function, \ie  distribution.
As a quantized field, the contribution of high frequency modes makes it difficult to define $\phi$ at a precise point $x$.
Hence, we sometimes write the smearing of $\phi$ with some test function $f\in C^\infty_0(M)$, given by
\begin{equation*}
\phi(f) = \int_M \phi f
\end{equation*}
We can then construct $\mathscr{A}(M,g)$ by starting with the free *-algebra generated by unit element $\mathbf{1}$ and elements $\phi(f)$ with $f\in C^\infty_0(M)$ and imposing the following relations \\
\begin{enumerate}
\item \textbf{Linearity} $\phi(c_1 f_1 + c_2 f_2) = c_1 \phi(f_1) + c_2 \phi(f_2)$ for $c_1, c_2 \in \mathbb{C}$
%
\item \textbf{Klein-Gordon equation} $\phi\big( (\Box_g - m^2)f \big) = 0$
%
\item \textbf{Hermitian field} $\phi(f)^* = \phi(\bar{f})$
%
\item \textbf{Canonical commutation relation (CCR)} $[\phi(f_1), \phi(f_2)] = iE(f_1, f_2) \mathbf{1}$, where $E = E^+ - E^-$ and $E^\pm$ are distributions satisfying $(\Box_g - m^2)E^\pm(x,y) = \delta(x,y)$ with appropriate supports (which define the advanced and retarded operators).
\end{enumerate}
%
We define a \textbf{physical state} $\omega$ as a linear map
$\omega: \mathscr{A}(M,g) \rightarrow \mathbb{C}$ satisfying the normalization condition $\omega(\mathbf{1}) = 1$ and the positivity $\omega(a^*a) \geq 0$ for $a\in\mathscr{A}(M,g)$.
A physical state could simply be considered as an expectation value functional. 
For a given state $\omega$, there is an associated Hilbert space $(\mathscr{H}_\omega, \langle \cdot, \cdot \rangle)$, a representation $\pi : \mathscr{A}(M,g)\rightarrow \mathrm{End}(\mathscr{H}_\omega)$ and a non-trivial vector $\Psi \in \mathscr{H}_\omega$ such that for $a \in \mathscr{A}(M,g)$
\begin{equation*}
\omega(a) = \frac{\langle \Psi, \pi(a)\Psi\rangle}{\langle \Psi, \Psi \rangle}
\end{equation*}
by the GNS-construction (see \eg\cite{bar2009quantum}).
For what concerns this project, we consider the expectation value the vacuum $(1+1)$-current as a physical state.
The definition of such a state will be given later. \\\\
%
In order to adopt the point-splitting formalism, 
we consider the \textbf{2-point function} (or more precisely, 2-point distribution) constructed with the physical state $\omega$, 
\ie the distribution $W$ defined by 
\begin{equation*}
W_2(f, g ) \equiv \omega(\phi(f) \phi(g))
\end{equation*}
It should be noted that $W_2(x,y)$ is defined only in sense of distribution because of the divergence when the coinciding-point limit $y\rightarrow x$ is applied.
